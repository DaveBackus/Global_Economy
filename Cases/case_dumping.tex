\documentclass[letterpaper,12pt]{article}

\RequirePackage{GE05}
% this inputs graphicx, too
\RequirePackage{hyperref}
\RequirePackage{comment}

\def\ClassName{The Global Economy}
\def\Category{Mini-Case}
\def\HeadName{Dumping Chinese Furniture in the US}

\begin{document}
\parindent = 0.0in
\parskip = \bigskipamount
\thispagestyle{empty}%
\Head

\centerline{\large \bf \HeadName}%
\centerline{Revised:  \today}

\bigskip
You are the CEO of a Chinese furniture manufacturer 
with a growing business selling to US retailers.  
Suddenly you face a new obstacle:  several US producers 
have filed suit against you in US courts accusing you of dumping.  
You wonder:  What is dumping?  
How could the suits affect your business?  
What should you do?  



%\vfill \centerline{\it \copyright \ \number\year \ 
%NYU Stern School of Business}

% --------------------------------------------------------------------
\pagebreak%
{\bf Exhibit 1.} {\bf The World Trade Organization and Anti-Dumping Actions.} 

The World Trade Organization (WTO) is the only global international organization dealing with the rules of trade between nations. At its heart are the WTO agreements, negotiated and signed by the bulk of the world's trading nations and ratified in their parliaments.  The goal is to help producers of goods and services, exporters, and importers conduct their business.

Anti-dumping actions.  
If a company exports a product at a price lower than the price it normally charges on its own home market, it is said to be ``dumping'' the product. Is this unfair competition? Opinions differ, but many governments take action against dumping in order to defend their domestic industries. The WTO agreement does not pass judgement. Its focus is on how governments can or cannot react to dumping -- it disciplines anti-dumping actions, and it is often called the ``Anti-Dumping Agreement''. 

The legal definitions are more precise, but broadly speaking the WTO agreement allows governments to act against dumping where there is genuine (``material'') injury to the competing domestic industry. In order to do that the government has to be able to show that dumping is taking place, calculate the extent of dumping (how much lower the export price is compared to the exporter's home market price), and show that the dumping is causing injury or threatening to do so. ... [T]ypically anti-dumping action means charging extra import duty on the particular product from the particular exporting country in order to bring its price closer to the ``normal value'' or to remove the injury to domestic industry in the importing country.

Source:  WTO website, 
\href{http://www.wto.org/}{http://www.wto.org/}.

% --------------------------------------------------------------------
\pagebreak%
{\bf Exhibit 2.} {\bf There are wise and less wise ways to control cheap imports.} 

Last week, President Clinton promised to stop the ``flooding of our markets" with low-priced steel and other imports that he said threatened American workers. This week, he sent Vice President Al Gore to Malaysia to preach open borders to a gathering of Asian-Pacific countries, many of which are clawing their way out of recession.

Apparently it is acceptable for Russia and other countries to export their way to recovery as long as those exports do not land here. Mr. Clinton is right to seek protection from a deluge of temporarily cheap steel imports. But there are good and bad ways to go about this task. So far the Administration has not chosen the best course of action. ...

\begin{comment}
American steel manufacturers have accused Japan, Russia and Brazil of ``dumping" steel in the United States, or selling it for less than it costs them to make it at home. But anti-dumping rules too often amount to protectionism, shutting out legitimate exports and provoking trading partners to retaliate. The better course would be for the Administration to exploit a different provision of international trade rules, known as the safeguard clause, which permits countries to impose quotas or tariffs in order to block a sudden rush of cheap imports. That would give the United States a principled and constructive way to insulate workers from the temporary ravages of trade.
\end{comment}

The steel industry's complaints center on imports of hot-rolled steel, which account for more than half of all steel products. These imports are soaring in part because Japanese and Brazilian mills are selling steel to the United States because buyers are evaporating in neighboring countries mired in recession. Russian steel is flooding America in part because the ruble has collapsed, cutting the dollar price of Russian goods. The price of steel in the United States has fallen by up to a third, and imports have risen about 80 percent since last year. Layoffs have already started.

The steel industry's answer is to accuse Japan, Russia and Brazil, which account for all of the explosive rise this year, of dumping. The one-third price drop ``is out of bounds," said Keith Busse, president of of Steel Dynamics Inc. of Indiana, one of the 12 companies that brought the dumping charges. ``Right is right and wrong is wrong," he said, and ``import prices are crazy and will take out not just inefficient producers."

But to many economists, invoking international anti-dumping rules lies somewhere between a joke and protectionist fraud. The Commerce Department, which determines whether dumping has occurred, not only gets to decide what another country's costs are but also gets to tack on a nifty profit. In the case of Russia, the Commerce Department can determine that its markets are too primitive to yield reliable cost estimates and substitute figures based on higher production costs in, say, Turkey. The process is designed to help the domestic industry win.

An independent agency then decides whether the dumped imports have caused the aggrieved industry ``material harm," a low threshold. If the steel industry gets its way, the United States will then impose prohibitive tariffs.

Americans' overuse of dumping charges is now taking a large toll. The rest of the world is catching on to the charade. The dumper now often claims to be the dumpee -- last year the United States was a defendant in about as many cases as it was the plaintiff.

Start from first principles. Low-cost imports are a gift to American consumers if they can be expected to last. If foreigners want to give us \$2 worth of steel for \$1 year in and year out, the country would be foolish to turn down the largess. But if the something-for-nothing is only temporary -- as is almost certain in the current case -- then the gift can turn sour. The temporary influx would trifle with the livelihoods of ordinary steelworkers, throwing them out of work one year only to lure them back in another.

That is why international trade rules include a safeguard clause. It allows nations to impose quotas or tariffs in an effort to hold off a sudden rush of cheap imports. But the safeguard clause is better tailored than anti-dumping rules to prevent abuse. The safeguard clause can only be invoked temporarily and must be phased out. An independent agency must determine that imports impose ``serious harm" to the domestic industry rather than merely ``material harm" under the anti-dumping rules. And if countermeasures are left in place for an extended period, then the United States, in the steel case, would be required to compensate Russia, Japan and Brazil.

That is the right principle. Americans protect their steelworkers from needless ravage but do so in a way that minimizes harm to suffering workers in Russia and Brazil.

The Administration is right to pursue relief for the domestic steel industry. But how it pursues that relief also matters. If the domestic industry successfully exploits anti-dumping rules in this case, the semiconductor, apparel and scores of other industries will follow suit. Then watch retaliatory measures take hold by America's trading partners around the globe.

The steel mills that brought dumping charges cannot be blamed for resorting to a legal device stacked in their favor. But Congress and the President are obligated to take a wider view. Under the law, each can initiate use of the safeguard clause on behalf of steelmakers. There is, despite the continuing dumping case, still time.


Source:  New York Times, November 19, 1998.  


% --------------------------------------------------------------------
\pagebreak%
{\bf Exhibit 3.} {\bf Byrd Brains.} 

Vietnam's Binh Duong Province may not sound like the home of traditional American furniture, but that's where Stickley, a classic American name, is building a new \$6 million plant. According to Stickley partner Edward Audi, ``The new 187,000 square-foot factory will provide an opportunity for Stickley to offer a more diversified product line and to open new markets globally."

But the Vietnam factory is only one part of Stickley's competitive strategy. The other is an anti-dumping suit -- filed in Washington along with 25 other US furniture manufacturers -- against Chinese- made bedroom furniture. The petition is ostensibly about saving US jobs, but knocking off the competition that makes furniture in China is what this high-profile dumping case is really about.

US furniture producers have been importing from China for at least a decade, using a combination of imported and US-made furniture to complete their lines. The Cato Institute's Dan Ikenson cites the testimony of Jeffrey Seaman, CEO of Rooms to Go, before the International Trade Commission (ITC): US furniture makers ``knew after traveling to China and seeing the infrastructure there that they could make certain bedrooms in China, bring it here, mark it up 30\% to 40\% to a retailer and still sell it for less than they could have made it for."

The ITC's preliminary report on the case says that 20 of the 40 producers it surveyed ``imported Chinese merchandise during the period and that the 12 largest domestic producers of wooden bedroom furniture all imported reasonably substantial and increasing volumes of merchandise from China." What happened then is that US retailers caught on and began importing Chinese-made furniture themselves, cutting out the middleman US producers and their markup. The dumping suit is about raising prices for those importers.

The furniture makers are also hoping to cash in on what is known as the Byrd Amendment, which mandates that the proceeds of anti-dumping tariffs be dumped into the pockets of the ``injured" parties -- ie, the same companies that support the dumping suits. The Byrd maneuver has been ruled a trade violation by the World Trade Organization, and that body will soon decide on retaliation since Uncle Sam refuses to repeal it.

The Byrd Amendment promises to become a windfall for Stickley of New York and Vietnam and its dumping-suit partners. China annually sends \$1 billion worth of headboards, beds, night tables, dressers, bureaus, armoires and some book cases and writing tables to the U.S., and the dumping tariffs can run as high as 440\%. If you can't beat a competitor, get the government to tax his goods for you. This is trade law as an income redistribution scheme.

The ITC has already made a preliminary decision that there is a reasonable indication that the US industry has been ``harmed" by incoming Chinese beds. Commerce will make a preliminary determination on June 17 whether unfair pricing is the cause of that harm. If Commerce rules in favor of the petitioners, a final ruling would be due before the end of the year.

The economic harm would start immediately, however, as importers scramble to calculate catalog prices in an uncertain environment, insecurity rises among the 200,000 Americans who work in the furniture-retail industry, and consumers pay higher prices that pad the bottom lines of Stickley and friends.

In other words, all of this has nothing to do with bringing low-wage factory jobs back to the US. Those jobs are gone for good -- to Vietnam, if not to China. The furniture dumping suit is a classic Beltway game of greasing a squeaky wheel. Let's hope that between now and December the Bush Administration decides it doesn't want to play along.

Source:  Wall Street Journal, June 9, 2004.  


% --------------------------------------------------------------------
\pagebreak%
{\bf Exhibit 4.} {\bf Chinese Furniture Faces US Tariffs.} 

The US government is expected to propose new tariffs tomorrow on an array of Chinese-made furniture imports -- the latest turn in a trade battle that has split a US industry and begun to remap a global supply chain.

Regulators could propose as many as nine different tariff rates -- some for individual Chinese manufacturers and some for groups of manufacturers, according to officials at the US Department of Commerce. The dumping investigation that accompanies tariffs covers \$1.2 billion of annual imports of dressers, bed frames and nightstands, making it the largest-ever antidumping probe that the US has conducted against China.

The full impact of the new duties on consumers and the US furniture industry will depend on the tariffs' size, which could range from near zero to more than 50\%. Any duties would take preliminary effect next week. US investigators then will travel to China to verify the numbers they already have collected through written communication. Appeals from both sides can take place until early next year, when any tariffs could be adjusted.

If the new duties are high, retail prices of bedroom furniture could rise significantly. The higher cost of the Chinese products would allow US manufacturers to impose their own price increases. The threat of tariffs has already apparently led to a sharp drop in Chinese production in recent weeks as manufacturers wait to see what they're up against. A sustained drop in China's output could free up floor space in American furniture stores that have relied heavily on imports, opening the door for US manufacturers to reclaim lost market share.

That would be a major victory for the more than two dozen US furniture makers that requested the investigation last October, charging that Chinese factories are flooding the US with artificially low-priced goods. The US furniture makers say imports from China have battered their sales and operating income. Total annual retail sales of all wooden bedroom furniture in the U.S. is about \$9 billion.

But if the duties imposed are as little as 10\%, for instance, then US factories are unlikely to feel much relief, and consumers might not even notice. At that level, US retailers, import agents and Chinese factories can absorb much of the additional charge.

Potentially most threatening for US factory owners is that the debate over the tariffs that has roiled the industry and embittered many retail customers already is spurring furniture-making in other low-labor countries, such as Vietnam and Brazil. Many economists say that phenomenon underscores how the cumbersome US dumping laws are out of date and ineffective, particularly in the case of China's fast-evolving economy.

The case also touches on a delicate political issue in the presidential campaign. The tariffs on furniture would increase the number of Chinese-made products with antidumping duties to 57, ranging from paintbrushes and brake rotors to crawfish-tail meat and television sets. Those measures are popular with factory workers and opponents of global outsourcing, but they also risk irritating voters as prices on popular consumer goods rise.

Commerce Department officials, who are leading this phase of the probe, declined to comment on what tomorrow's announcement will be.

US bedroom-furniture makers that support the tariffs say they have no choice but to pursue the dumping case. They say that at least some Chinese factories, fueled by government subsidies and a devalued currency, are selling products at or below what it costs to make them.

Since a wave of Chinese-made wooden furniture swamped the U.S. market beginning in 2000, about 35,000 employees, or 27\% of the total work force at US woodworking factories, have lost their jobs.

``As American citizens, don't we at least have the right to ask for an investigation?" said John Bassett, president and chief executive of Vaughan-Bassett Furniture Co. in Galax, Va., who is spearheading the campaign for tariffs and taking withering criticism from big US retailers for doing so.

The mere specter of tariffs against China is accelerating development of furniture production in low-wage countries that aren't under attack for unfair pricing. During the first three months of the year, bedroom-furniture imports to the US from Vietnam tripled from a year earlier, while those from Brazil grew by about 27\%, according to the U.S. Census Bureau.

Taiwan entrepreneurs who revolutionized furniture making in China by building huge, state-of-the-art factories there are dipping into Vietnam. Art Heritage International Ltd., based in Dongguan, China, just bought a factory in Ho Chi Minh City that is about the size of four football fields, plus a parcel of land in southern Vietnam that is seven times bigger.

Jack Wang, Art Heritage's general manager, said the company plans to immediately start building a factory on that tract if the proposed tariffs are steep. ``We're pretty much right on schedule," he said.

The owner of a factory in the northern Chinese city of Yichun that makes bedroom furniture for Global Furniture Inc. of Mooresville, N.C., recently moved part of those operations across the Russian border about 60 miles away to escape any trade duties on Chinese-made products. Bedroom sets to be churned out at the new Russian plant by workers brought in from Yichun were being touted two months ago at the furniture industry's big spring buyers' market in High Point, NC.

Still, China remains by far the biggest exporter of bedroom furniture to the US, accounting for about as many furniture imports as the next 10 countries combined. It would probably take years before these growing rivals could compete significantly with China's huge industry.

During a trip to China last week, Keith Koenig, president of City Furniture, a big retailer based outside Fort Lauderdale, Fla., said bedroom-furniture production was going silent. ``Production, the pipeline, has all but dried up," he said. ``I think everybody is preparing for some level of tariffs."

Within the US furniture industry, the trade battle has split a normally tightknit group that during the past settled most disagreements in private.

Furniture stores that buy heavily from China have waged a counterattack against US furniture makers that support the trade duties. At least 15 retailers across the US have removed Mr. Bassett's products from their stores or have stopped ordering new styles from him, Mr. Bassett said, wiping out \$8 million in orders.

Retailers have targeted other manufacturers as well, though total losses aren't clear.

``Why support anyone who is trying to hurt me?" said Jake Jabs, owner of American Furniture Warehouse, a nine-store chain based in Englewood, Colo., that has banished products from four US suppliers that are pushing for the tariffs. About two-thirds of the wooden furniture he sells is imported. Of those imports, two-thirds comes from China.

Lobbyists for the retailers, importers and Chinese factories have denounced the US furniture makers as selfish opportunists looking to cash in. Under a U.S. measure known as the ``Byrd Amendment," the duties collected as a result of a dumping case can be paid to the US companies hurt by the dumping. (Mr. Bassett said that wasn't the motivation of companies in his coalition, and he expects the Byrd measure to be eliminated soon.)

Some retailers helping bankroll or lending their name to the lobbying and legal campaign are household names with deep pockets, including Crate \& Barrel, JC Penney Co. and Berkshire Hathaway Inc.'s Nebraska Furniture Mart and RC Willey Home Furnishings units. Big retailers are also concerned that large duties on bedroom furniture could lead to a push for tariffs against other countries or tariffs on other types of furniture.

A few of the US manufacturers that initially sought the investigation have dropped out of the effort in the face of pressure from retailers. Paul Toms, chairman and CEO of Hooker Furniture Corp., said the Martinsville, Va., company decided to abandon its support, in part, after three of its five largest retail customers ``expressed displeasure." Mr. Toms said he also didn't want to alienate any retailers that haven't called Hooker to complain.

Furniture Retailers of America, an industry group formed to oppose the trade duties, says it didn't call for a boycott of manufacturers supporting the tariffs. But promotional material that the group distributed named 26 companies that joined the tariff petition filed with the U.S. government.

``We wanted to make this as painful as we could for the petitioners," said Michael Veitenheimer, the retail group's spokesman and general counsel of Bombay Co., a home-furnishings and accessories retailer based in Fort Worth, Texas.


Source:  Wall Street Journal, June 17, 2004.  


% --------------------------------------------------------------------
\pagebreak%
{\bf Exhibit 5.} {\bf China Accuse Corning of ``Dumping.''}  

Corning Inc., the big US fiber-optic and glass maker, said the Chinese government has charged it with selling optical-fiber products in China at an unfairly low price that damaged Chinese producers, a practice known as dumping.

Corning denied the charge, which followed a nearly yearlong investigation by China's Ministry of Commerce after two Chinese companies alleged that optical-fiber imports were priced below what market conditions justified. Corning said it hopes to settle the matter when Chinese officials visit the Corning, NY, company within the next several weeks.

Eight other makers of the products, mainly from South Korea and Japan, were also found guilty of dumping by the ministry.

``We are extremely disappointed with this preliminary determination," said Robert Brown, a senior vice president at Corning. If the ruling holds, it ``could have a significant negative impact on the company's ability to export fiber into China," he said.

While dumping cases against China, whose companies benefit from low- cost labor and easy credit, are well known, cases brought by China against foreign companies had long been rare.

But since China joined the World Trade Organization in late 2001, China has become ``very aggressive in using the antidumping weapon against the US and other countries," said Joseph Dorn, a partner in the Washington office of law firm King \& Spalding. ``No one seems to realize this."

Since it joined the WTO, China has brought about 25 dumping cases against foreign companies, according to a King \& Spalding estimate. In that same period, US companies have brought 24 dumping cases against China, according to the International Trade Commission.

Separately, two Japanese computer chip makers asked the Japanese government yesterday to impose import duties on random-access-memory chips from South Korea's Hynix Semiconductor Inc. Hynix said the action amounted to harassment.

If the Japanese government complies with the request, from Elpida Memory Inc. and Micron Japan Ltd., it would mark the first time that Japan has used tariffs to counter alleged trade subsidies from another government. The US and Europe decided to take similar tariff actions against Hynix last year, which Hynix is challenging.

The Corning dispute comes as trade frictions between China and the US continue to simmer, exacerbated by concerns over the growing US trade deficit with China, which reached \$42.2 billion in the year through April, and by election-year political pressures in the US.

Corning derived 6\%, or \$46 million, of its \$760 million of global sales in optical fiber and cable last year from China, according to the company. It estimates that any damage to its bottom line from the dispute should be less than a penny a share in the second half of this year. A final decision by the Chinese government is expected by the end of the year.

Recent US trade actions against China, most notably an antidumping case launched in October against \$1 billion worth of Chinese wood and bedroom furniture imports, have likely played a role, too, according to trade experts.

The high-profile US furniture case against China and China's charge against fiber makers such as Corning also exemplify the chief economic concerns in each economy: The US is preoccupied with protecting workers in its hard-hit manufacturing sector, while China is interested in nurturing its technology industry.

The low prices from a capacity glut in the US tech sector could be another factor. ``Chinese producers could be feeling heavy price pressures from the U.S.," said Gary Hufbauer, a trade expert at the Institute of International Economics, a Washington think tank.

Corning made a disastrous foray into fiber-optic cable in the late 1990s that forced it to halve its work force. The company is now betting on high-tech glass used in flat-panel computer monitors and television sets.

Corning's stock fell 28 cents to \$12.28 in 4 p.m. New York Stock Exchange composite trading yesterday.

With the filing of the Chinese charges, Corning customers in China will have to pay a 16\% deposit on the purchase price of the company's products, starting immediately. That money will be held in an escrow account until the matter is resolved.

Source:  Wall Street Journal, June 17, 2004.  


% --------------------------------------------------------------------
\pagebreak%
{\bf Exhibit 6.} {\bf When it comes to Law, China Buys American.}  

Markor International Furniture Manufacture Co., a Chinese company, was charged by US furniture producers in 2003 with harming their businesses by selling products below the fair cost of production. That's when the company hired the respected US law firm of Wilmer Cutler Pickering Hale \& Dorr to defend itself.

The result, announced in January of last year by the US International Trade Commission of the Commerce Department, amounted to a big victory for Markor. The company escaped with a 0.83\% tariff on its products. More than 100 Chinese furniture companies that, like Markor, hired American lawyers, also received relatively light tariffs. Thousands of other companies that didn't respond to the accusations were levied devastating tariffs of 198\%, essentially putting them out of that business.

``We paid a lot of money in this case, but I think it is deserved," says Steven Wu, international-affairs manager of Markor, which is based in the eastern Chinese city of Tianjin. Mr. Wu won't say how much the law firm billed his company, but lawyers familiar with such cases estimate that they typically cost at least several hundred thousand dollars.

Chinese firms are going to war against so-called dumping charges -- and US  lawyers are a chief beneficiary of the fighting.

As Chinese exports surge, U.S. producers have brought an increasing number of antidumping suits against Chinese exporters of a wide range of products, such as steel rods, lined paper and artists' canvases. They argue that Chinese factories are able to sell products at unfairly low prices, thanks to an undervalued currency and government subsidies.

The number of cases against Chinese exporters has grown to roughly half of the total US antidumping probes launched against foreign firms. After long ignoring trade investigations or trying to challenge them on the cheap -- and getting slammed with huge penalties -- officials in Beijing say more Chinese corporations are looking for help from top U.S. lawyers. ``Big Chinese companies want to make sure they hire the best legal firms [because this gives them] the highest chance of success in an investigation," says Wang Shouwen, deputy director general of the Chinese Commerce Ministry's Bureau of Fair Trade for Imports and Exports. ``If they don't defend themselves forcefully . . . their market could be impaired."

US law firms are learning that there's money to be made in representing Chinese companies. ``When antidumping cases are announced, you can see companies receiving hundreds of pages of faxes pitching different law firms," says US-trained lawyer Zhang Yuqing. Mr. Zhang headed the legal department in China's Commerce Ministry before setting up his own firm in Beijing several years ago.

The rise in cases has prompted US law firms to expand their rosters of antidumping experts. James Jochum joined the law firm Mayer Brown Rowe \& Maw last year after stepping down as the US Commerce Department's top antidumping official.

``Companies have begun to realize that they can get better results if they hire the right lawyers," says Mr. Jochum, who sees the furniture case as a "turning point" for Chinese firms. A Chinese company hired Mr. Jochum to fight a lined-paper antidumping suit brought by US paper producers. Mayer Brown Rowe also recently hired Linda Chang, a former Commerce Department attorney who speaks Chinese; she now works in Beijing on antidumping cases.

The big-name US lawyers retained by Chinese companies help them navigate the paper labyrinth surrounding the antidumping cases. ``Only American firms know the rules of the game, so who else should we hire?" asks Liao Yuanhuang, deputy manager of Lacquer Craft Manufacturing Co., another furniture firm that hired Wilmer Hale.

But retaining lawyers to resolve legal disputes is still a fairly new concept in China. ``Culturally in China people used to try to avoid litigation," says Mr. Wang, the Commerce Ministry official.

Another factor complicating Chinese cases is China's designation as a ``nonmarket" economy, which was a condition for it joining the World Trade Organization in 2001. This designation allows the US and other countries to use estimates of production costs in a surrogate third country -- often India -- to determine how much it should cost China to produce an item.

As a result, antidumping margins levied against Chinese firms are often much higher than they would be if China were treated as a market economy.

In the furniture case that included Markor, Wilmer Hale lawyer John Greenwald sought to use Indonesia as the surrogate instead of India, which doesn't have a strong wooden furniture industry. Because such items cost less to produce in Indonesia, that country's prices make China's seem more justified. But Mr. Greenwald failed and is now appealing the case to the Court of International Trade in New York.

One of the Chinese companies that has enlisted U.S. help to fight the current US suit over lined paper is Zhejiang Guangbo Group. Lin Xiaofan, assistant to the firm's general manager, says that its Washington-based lawyers, Miller \& Chevalier, have helped it get quick updates in the case.

Augustine Tantillo, executive director of the American Manufacturing Trade Action Coalition, a textile trade group, doesn't begrudge Chinese companies their representation. ``It's part of our system of allowing interested parties to have their say," he says. But he is concerned that ``many high-level former US government trade officials" end up ``working for offshore entities." Says Mr. Tantillo, ``Where I would draw the line is that there should be prohibitions on US government officials leaving government and immediately going to represent Chinese interests here in Washington."

Mr. Lin says that hiring US lawyers is now ``a must" even if they are expensive. ``This is money worth spending, if you want to continue exporting to the US market," says Mr. Lin. ``And if we win a favorable tariff [rate], it will help us gain some edge against our rivals."

Source:  Wall Street Journal, February 17, 2006.  

\end{document}

% --------------------------------------------------------------------
\pagebreak%
{\bf Exhibit 7.} {\bf Polish Firms Fend Off Rivals in China.}  

As a boy, Grzegorz Darlak played cowboys and Indians astride his make-believe horse -- a rocket-shaped vacuum cleaner made by Polish company Zelmer SA.

Now 41 years old, he is the sales director for Zelmer, Poland's largest manufacturer of household appliances, which is circling the wagons to fend off stiff competition from China.

Once secure in their niche as providers of low-cost products across Europe, Polish companies, like their Western peers, are getting squeezed by cheaper Chinese imports. Production costs are 20\% to 30\% lower in China, and Poland's entry into the European Union last year is expected to gradually push up its wages and widen the gap.

To stay competitive, Zelmer is stressing its proximity and flexibility in serving loyal Polish customers and making private-label appliances for Western Europe's hypermarkets, while increasing production of higher-priced goods less vulnerable to Chinese competition.

Another Polish company, metal-accessories maker Gamet SA, is trying a different approach. Five years ago, it bought a stake in a Chinese manufacturer and now is splitting production between China and Poland -- both to lower costs and begin the lengthy process of nurturing relationships with Chinese suppliers, customers and government officials.

Both companies claim some success with their strategies -- and, in a sign that no single strategy is right or wrong, Zelmer and Gamet have attracted some of the same investors. But both concede it is too early to rely on a single tactic to meet the rising threat from Chinese competition. Here is a look at what they have done so far:

Located in Rzeszow in southeastern Poland close to the Ukrainian and Slovakian borders, Zelmer has supplied Poles with vacuum cleaners since 1951. The former state-owned monopoly attracted venture-capital and pension-fund investors when its shares were floated on the Warsaw Stock Exchange this year.

Mr. Darlak, who joined the company in 1988, concedes that competing with Chinese products on price is impossible. For example, a Chinese-made vacuum cleaner called the Shark was selling for 89.99 zlotys, or about \$28, at one Carrefour hypermarket in Warsaw in mid-September, while Zelmer's least-expensive Elf model was priced at 139 zlotys.

Zelmer tries to make up the difference by stressing its just-in-time delivery and flexibility in meeting customer demands. The company needs no more than 10 days to make and deliver a standard product to Western markets, and about three weeks for new production or to fill a special order. With Chinese products, it takes one month to make and deliver a standard product and about six weeks to make and deliver a new one, says Andrzej Musialowski, the head of Zelmer Trading, an export-sales unit.

Even though Mr. Darlak acknowledges that the Elf is sold at ``almost no margin at all," he says Zelmer continues to make lower-priced models ``to block the competition."

Krzysztof Zak, general director at Poland trading company Tako, which has imported the Shark, says the Chinese vacuum cleaner offers a decent margin both for the producer and importer, with unit production costs of about 50 zlotys. Both Mr. Darlak and China's Jiuchang Electrical Equipment Co., which makes the Shark, declined to comment on their respective production costs per unit.

So far, though, Zelmer's strategy seems to be working. ``After importing about 30,000 [Shark] units during last year, I'm giving up," Mr. Zak says. While Sharks sell well, "the Zelmer brand is too strong. And consumers tend to buy what they know."

In a report last month, Polish market-research firm MillwardBrown SMG/KRC said 55\% of about 1,000 respondents in Poland said Zelmer was the first brand they associate with vacuum cleaners.

Still, Zelmer -- with 42\% of Poland's vacuum-cleaner market -- concedes losing several percentage points of its market share to Chinese rivals that first appeared in Poland during the late 1990s and sell under local distributors' brands.  

And Chinese producers don't intend to give up. According to Jiuchang's customer-service manager, who declines to give his name, the company will continue to sell its products in Poland through trading companies.

Meanwhile, Zelmer is bolstering its brand -- spending 3\% of annual revenue on marketing and advertising -- and pushing its higher-priced models. Mr. Darlak says the best-selling model is the Magnat, priced at 339 zlotys, while the most sophisticated is the Wodnik Trio, at 749 zlotys. ``We can't go below [a] 100 zlotys price tag," he says. ``That would mean depreciating the product."

In addition to its own brand, Zelmer makes private-label appliances for seven European and international hypermarket networks, including France's Carrefour SA. Export sales amount to 30\% of total sales, which this year Zelmer forecasts will be 303.7 million zlotys, up 7.2\% from 2004.

In January, a 300 million euros (about \$360 million) equity fund, managed by Warsaw-based venture-capital firm Enterprise International and backed by US and European investors, increased its stake in Zelmer to 47\% from 25\% and said it intends to remain an investor for three to five years.

Aleksander Kacprzyk, an Enterprise International executive who serves on Zelmer's supervisory board, says Zelmer could become strong enough to seek acquisitions -- or, if targeted for a takeover, the price tag for a potential buyer should be high enough to discourage buying the brand name alone and closing down production in Poland.

Meanwhile, China remains very much on the company's mind. ``You can't think about the future without taking Far Eastern producers into account," says Janusz Plocica, a Zelmer management-board member. ``We have already understood it, even here in Rzeszow."

Gamet's decision to split production between China and Poland has been a gradual one.

Polish entrepreneur Andrzej Gajek founded Gamet in 1988, employing about 20 workers making handles for an Austrian furniture company. As it broadened its product line into a variety of metal furniture accessories, Gamet started exploring China for reliable and low-cost suppliers. Small orders evolved into increased design work in China and the purchase in 2000 of a 24\% stake in Chinese company Gastar, located in the Shenzhen Special Economic Zone near Hong Kong, for an undisclosed price. Mr. Gajek in 2002 sold the company to the same Enterprise-managed fund that owns 47\% of Zelmer.

Gamet now makes 8,500 different products -- ranging in price from less than one zloty to more than 23 zlotys -- and employs 400 in its plant in the northern Poland city of Torun and 400 in Shenzhen. It also buys from 10 other suppliers from Shenzhen and Shanghai.

Accessories made in China tend to be more simply designed and purchased in large quantities. ``To us, production in China is a tool to reach a mass market with simpler product," Gamet Chief Executive Krzysztof Pioro says. ``Here we can be very flexible with the price."

In Poland, Gamet makes smaller batches of higher-quality accessories tailored to customers' needs. The company is investing 20 million zlotys in a new computerized galvanic production line that will be operated by three persons rather than the 30 that work on the current manual line. No layoffs are anticipated because of planned expansion.

Gamet has built close relations with Chinese suppliers, a move that Mr. Pioro says also acts as a barrier to entry for potential rivals or customers that might try to bypass Gamet and develop their own connections with Chinese suppliers. ``You don't establish good business relations with Chinese partners within a few weeks," he says. ``It takes years, not to mention lengthy bureaucratic procedures. So we tell our customers, `Why reinvent the wheel?' "

That logic has resonated with Warsaw-based wholesaler Juan Sp.J. ``We buy only from Gamet," says Magdalena Woznicka, a representative for Juan Sp.J. ``They have all [the] China-made stuff we need, so why bother?"

Source:  Wall Street Journal, October 12, 2005.  

\end{document}

