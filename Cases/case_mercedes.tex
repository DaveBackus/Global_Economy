\documentclass[letterpaper,12pt]{article}

\usepackage{../LaTeX/ge14}
% spacing on page
\oddsidemargin=0.25truein \evensidemargin=0.25truein
\topmargin=-0.5truein \textwidth=6.0truein \textheight=8.75truein

\usepackage{comment}
\usepackage{graphicx}
\usepackage{amssymb}
\usepackage{amsmath}

% layout of figures and tables
\usepackage[margin=0pt, labelsep=period, labelfont=bf]{caption}
%\usepackage{float}

\usepackage{hyperref}
\urlstyle{rm}   % change fonts for url's (from Chad Jones)
\hypersetup{
    colorlinks=true,        % kills boxes
%    allcolors=blue,
    pdfsubject={ECON-UB233, Macroeconomic foundations for asset pricing},
    pdfauthor={Dave Backus @ NYU},
    pdfstartview={FitH},
    pdfpagemode={UseNone},
%    pdfnewwindow=true,      % links in new window
    linkcolor=blue,         % color of internal links
%    citecolor=blue,         % color of links to bibliography
    filecolor=blue,         % color of file links
    urlcolor=blue           % color of external links
% see:  http://www.tug.org/applications/hyperref/manual.html
}

% for listing code in tt font
\usepackage{verbatim}

% for table spacing
\usepackage{booktabs}

% section headers and spacing
\usepackage[tiny, compact]{titlesec}

% list spacing
\usepackage{enumitem}
\setitemize{leftmargin=*, topsep=0pt}
\setenumerate{leftmargin=*, topsep=0pt}

% attach files to the pdf
\usepackage{attachfile}
    \attachfilesetup{color=0.75 0 0.75}

\usepackage{needspace}
% \needspace{4\baselineskip} makes sure we have four lines available before a pagebreak

%\newcommand{\phm}{\phantom{--}}
\newcommand{\NX}{\mbox{\it NX\/}}
\newcommand{\POP}{\mbox{\it POP\/}}
\renewcommand{\log}{\ln}

\renewcommand{\thefootnote}{\fnsymbol{footnote}}

\def\ClassName{The Global Economy}
\def\Category{Mini-Case}
\def\HeadName{Mercedes-Benz USA}

\begin{document}
\parindent = 0.0in
\parskip = \bigskipamount
\thispagestyle{empty}%
\Head

\centerline{\large \bf \HeadName}%
\centerline{Revised:  \today}

\medskip
It's the first day of your summer internship at Mercedes-Benz USA.
You are thrilled by the opportunity to market Mercedes
to American consumers --- and perhaps to get one to drive yourself ---
but somewhat nervous about your new job.
Your boss enters your office with your first assignment:
(i)~summarize the likely cyclical sensitivity of US Mercedes sales
and (ii)~describe how current macroeconomic conditions are likely to affect them.

She explains that a formal macroeconomic analysis
typically guides their decisions on production,
pricing, and product positioning.
Your overview is the first input into that analysis.

%She asks specifically that you
%%
%\begin{itemize}
%\item Describe the state of the US economy.
%\item Speculate about its impact on the US market
%for Mercedes-Benz automobiles.
%%\item Describe some of the key indicators of the US economy:
%%Which indicators suggest faster growth in the near future?
%%Slower?
%%What is your overall assessment?
%%\item Decide whether Mercedes should increase production for the US market.
%%\item Determine the market segments likely to be strongest.
%%\item Decide how to price cars, and whether to continue
%%the attractive financing programs currently on offer.
%\end{itemize}


You check with your friends and collect the following information
while you think about what to tell your boss.



\section{GDP growth and car sales}

Automobiles --- and other durable goods --- are more sensitive to
economic fluctuations than nondurable goods and services.
You can see that in the growth rates over the last few years:
as GDP growth fell, consumption of durables and sales of cars
fell sharply.
The following numbers are
from the Bureau of Economic Analysis and the Federal Reserve.


\begin{center}
\begin{tabular}{lrrrrrr}
\multicolumn{6}{l}{Table 1.  Growth rates (percentages)} \\
\toprule
        & 2008 & 2009 & 2010 & 2011 & 2012 & 2013 \\

\midrule
Real GDP
        & --0.3 & --2.8 & 2.5 & 1.8 & 2.8 & 1.9\\
Real consumption (total)
        & --0.4 & --1.6 & 2.0 & 2.5 & 2.2 & 2.0 \\
Real consumption (durables)
        & --5.1 & --5.5 & 6.1 & 6.6 & 7.7 & 6.9 \\
Real consumption (motor vehicles)
        & --19.3 & --11.5& 7.9 & 10.9 & 12.0 & 4.0 \\
\bottomrule
\end{tabular}
\end{center}

%\bigskip
\section{Major product lines}

Mercedes-Benz USA sells a broad range of automobiles
designed to appeal to a similarly broad range of income segments.
Sedans include the C-Class (starting at \$29,900),
the C-Class (\$40,400),
the E-Class (\$51,400), and the CL-Class (\$94,400).
This information and more is available on the
MB USA
\href{http://www.mbusa.com/}{website}.


\section{Cyclical sensitivity of consumption by income group}

Just as we saw that some products are more cyclical than others,
so the consumption of some consumers is more cyclical than that of others.
Curiously enough, the consumption expenditures of the rich are more
cyclically sensitive than the consumption expenditures of the poor.
Their consumption expenditures are high --- they're the rich, after all ---
but respond more
in both directions to fluctuations in the economy.

The numbers below summarize this sensitivity.
If the sensitivity is 1.0, then the income group has the same sensitivity
as the average consumer:
if average consumption rises by 1\%, so does the consumption of the group.
A number greater than 1.0 indicates greater sensitivity.
Thus we see, for example, that when average consumption
rises by 1\%,
the consumption of the richest 5\%  rises by 2.52\%.

These numbers are reported by Jonathan Parker and Annette Vissing-Jorgensen
({\it American Economic Review\/}, 2009).
Their rationale seems to be that the rich have significantly more
sensitive incomes, which translates into more sensitive consumption.


\begin{center}
\tabcolsep=0.1in
\begin{tabular}{lcccc}
\multicolumn{5}{l}{Table 2.  Cyclical sensitivity of consumption by income group} \\
\toprule
    &  \multicolumn{4}{c}{Income Group} \\
        \cmidrule(r){2-5}
    &  Bottom 80\%  &  Top 20\% & Top 10\%  &  Top 5\%  \\
\midrule
Consumption sensitivity &  0.79 & 1.83 & 2.15 & 2.52 \\
\bottomrule
\end{tabular}
\end{center}

\section{Questions}

\begin{itemize}
\item How cyclical are Mercedes' product lines?

\item How would you suggest they deal with that?

\item More concretely:  How does it affect decisions on production,
pricing, and product positioning?
\end{itemize}

{\vfill
{\bigskip \centerline{\it \copyright \ \number\year \
David Backus $|$ NYU Stern School of Business}%
}}


\end{document}

OLD STUFF


%
\begin{enumerate}
\topsep=0.05\bigskipamount
\itemsep=0.05\bigskipamount

\item Economic growth:
the growth rate of real GDP through the fourth quarter.

%\item Employment:  the change in the nonfarm payroll employment number
%through next December.

\item Inflation:
the year-on-year inflation rate through December
(consumer price index).

\item Interest rates:  the ``Fed Funds target rate'' reported
after the March 27/28 meeting of the FOMC
(Federal Open Market Committee, the group that sets
US monetary policy).

\item Foreign exchange rates:
the rates reported by the
\href{http://www.federalreserve.gov/releases/h10/update/}
{Federal Reserve} at year-end for
\begin{itemize}
\item the dollar price of one euro and
\item the yen price of one dollar.
\end{itemize}

\end{enumerate}
Each forecast should include a brief rationale
and an equally brief assessment of the impact each item is likely to have on the production, pricing, and positioning of Mercedes in the US.
Do you think they would affect the company's Chrysler brands
the same way?


You barely remember your Global Economy course,
but turn immediately to classmates working at banks,
who provide you with the relevant analysis.



\end{document}

the course's
\href{http://www.stern.nyu.edu/eco/b012303/Backus/ge_resources_db.htm}
{resource page} for help,
starting with the Economist Intelligence Unit's country data and country
intelligence,
the Bloomberg economic calendar,
the Cleveland Fed's {\it Economics Trends\/},
and the Philadelphia Fed's
{\it Survey of Professional Forecasters\/}.


\end{document}

