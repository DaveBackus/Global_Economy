\documentclass[letterpaper,12pt]{article}

\RequirePackage{GE05}
% this inputs graphicx, too
\RequirePackage{comment}
\RequirePackage[hypertex]{hyperref}
%\hypersetup{dvips}
%\hypersetup{backref,colorlinks=false,urlcolor=cyan,linkcolor=blue}
%\hypersetup{pdfpagemode=None,pdfstartview=FitH}

\def\ClassName{The Global Economy}
\def\Category{Mini-Case}
\def\HeadName{Making Sense of FOMC Statements}

\begin{document}
\parindent = 0.0in
\parskip = 0.6\bigskipamount
\thispagestyle{empty}%
\Head

\centerline{\large \bf \HeadName}%
\centerline{Revised:  \today}

\medskip 
When the following statement was released, bond prices rose sharply.  
Why?  

\medskip
{\bf Statement of March 21:} 

The Federal Open Market Committee decided today to keep its target for the federal funds rate at 5-1/4 percent.

Recent indicators have been mixed and the adjustment in the housing sector is ongoing. Nevertheless, the economy seems likely to continue to expand at a moderate pace over coming quarters.

Recent readings on core inflation have been somewhat elevated. Although inflation pressures seem likely to moderate over time, the high level of resource utilization has the potential to sustain those pressures.

In these circumstances, the Committee's predominant policy concern remains the risk that inflation will fail to moderate as expected. Future policy adjustments will depend on the evolution of the outlook for both inflation and economic growth, as implied by incoming information.


\medskip
{\bf Statement of January 31:} 

The Federal Open Market Committee decided today to keep its target for the federal funds rate at 5-1/4 percent.

Recent indicators have suggested somewhat firmer economic growth, and some tentative signs of stabilization have appeared in the housing market. Overall, the economy seems likely to expand at a moderate pace over coming quarters.

Readings on core inflation have improved modestly in recent months, and inflation pressures seem likely to moderate over time. However, the high level of resource utilization has the potential to sustain inflation pressures.

The Committee judges that some inflation risks remain. The extent and timing of any additional firming that may be needed to address these risks will depend on the evolution of the outlook for both inflation and economic growth, as implied by incoming information. 


\vfill \centerline{\it \copyright \ \number\year \ NYU Stern
School of Business}

\end{document} 

