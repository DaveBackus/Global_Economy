\documentclass[12pt]{article}

\usepackage{GE13}
\usepackage{comment}
\usepackage{booktabs}
\usepackage{hyperref}
\urlstyle{rm}   % change fonts for url's (from Chad Jones)
\hypersetup{
    colorlinks=true,        % kills boxes
    allcolors=blue,
    pdfsubject={NYU Stern course GB 2303, Global Economy},
    pdfauthor={Dave Backus @ NYU},
    pdfstartview={FitH},
    pdfpagemode={UseNone},
%    pdfnewwindow=true,      % links in new window
%    linkcolor=blue,         % color of internal links
%    citecolor=blue,         % color of links to bibliography
%    filecolor=blue,         % color of file links
%    urlcolor=blue           % color of external links
% see:  http://www.tug.org/applications/hyperref/manual.html
}

% for ge05.sty
\def\ClassName{The Global Economy}
\def\Category{Mini-Case}
\def\HeadName{Opportunity in Ghana?}

\begin{document}
\parindent = 0.0in
\parskip = \bigskipamount
\thispagestyle{empty}%
\Head

\centerline{\large \bf \HeadName}%
\centerline{Revised:  \today}

\bigskip
As a junior associate at Booz \& Company, you have been asked to
prepare a short report on the possibility of
opening business centers in Ghana.
Your report would serve as input for a sales pitch to Genpact.

Genpact is a global leader in business process management,
a term that now covers a broad range of services:
data entry, customer service, accounting, human resources,
and many more,
together with their overall management.
Genpact began as a unit of GE, where it delivered services to
the rest of the company, many of them performed in India.
It became an independent company in 2007,
trading on the NYSE under ticker symbol G.
It now has operations in India, China, Poland, Morocco, Brazil,
South Africa, and other locations around the globe.

You see Ghana as a potential location for call centers,
data entry, and similar basic business processing,
most of it aimed at the US market,
with more to come if things work out.
Ghana is a former British colony that has been growing rapidly
in recent years after a period of unusually stable politics.
The Economist Intelligence Unit refers to it as a ``robust democracy.''
Its official language is English.
The World Economic Forum ranked Ghana 114th (of 133)
in their Global Competitiveness Report.
They continue:
``The country continues to display strong public institutions and
governance indicators,
particularly in regional comparison. Some
aspects of its infrastructure are also good
by regional standards, particularly the state of its ports.
On the other hand, education levels continue to lag
behind international standards [and]
labor markets continue to be characterized by inefficiencies.''

Using the information on the next page ---
and your own experience and insight ---
jot down your thoughts about these questions:
%
\begin{itemize}
%\item What do we know about this business?
\item What features of a country would make it a good location for this business?
\item Based on the evidence here, how do India and Ghana compare?
What are their strengths and weaknesses along the important dimensions?
\item What other information would you like to have?
\item Suppose Ghana set up a Special Economic Zone,
free from some of the laws and practices operating elsewhere in the country.
What different laws and practices would be most attractive to Genpact?
\item Overall, would you favor expansion into Ghana?
Why or why not?
\end{itemize}


\pagebreak
%
{\bf Exhibit 1 \\
Indicators for Ghana and India}
\begin{center}
\tabcolsep = 0.14in
\begin{tabular}{lrrr}
\toprule
Indicator   &  Ghana    &  India   &  Source  \\

\midrule
\multicolumn{3}{l}{\it Governance indicators\/} \\
Political stability (percentile) & 40    &    10  & WB, WGI \\
Govt effectiveness (percentile)  & 57  & 54  & WB, WGI \\
Regulatory quality (percentile)  & 53  & 42  & WB, WGI \\
Rule of law (percentile)         & 53  & 56  & WB, WGI \\
Control of corruption (percentile)& 60 & 47  & WB, WGI \\

\midrule
\multicolumn{4}{l}{\it Economic and business indicators\/} \\
GDP per capita (USD)        &  3,081 & 3,703 &  IMF \\
Employment rigidity (percentile)   & 27 & 30 & WB, Doing Business \\
Severance costs (weeks of pay) & 178  & 56 & WB, Doing Business \\
Paying taxes:  hours spent     & 224  & 254 & WB, Doing Business \\
Paying taxes:  gross rate (\%) &  34  &  62 & WB, Doing Business \\
Enforcing contracts (cost, \%) &  23  &  40 & WB, Doing Business \\
Investor protection (percentile)    &  60  &  60 & WB, Doing Business \\

\midrule
\multicolumn{4}{l}{\it Education indicators\/} \\
Literacy of adults  (\%)       &    67     &  63   &  WB, WDI \\
Literacy of young (15-24)  (\%) &    67     &  63  & WB, WDI  \\
Primary school completion (\%)  & 87    & 96    & WB, WDI  \\
Average education of adults (years)  & 7.0 & 4.4 & Barro-Lee \\

\midrule
\multicolumn{4}{l}{\it Infrastructure indicators\/} \\
Infrastructure quality (percentile) & 43 & 33 &  WEF \\
Quality of electricity (percentile)  &  20 & 24  & WEF \\
Internet users (\%)   & 8.6 & 7.8 & WB, WDI \\
Broadband penetration (\%) &  0.2 & 0.9 & WB, WDI \\

\bottomrule
\end{tabular}
\end{center}
%
\vspace{-0.15in}
Sources include:  the World Bank (WB),
including their World Development Indicators (WDI),
World Governance Indicators (WGI), and Doing Business;
the International Monetary Fund (IMF);
and the World Economic Forum (WEF).
Percentiles range from 0 (bad) to 100 (good).

\vfill
{
More of this kind of thing at \url{https://sites.google.com/site/nyusternglobal/}
%\url{http://goo.gl/EDybU}

\vspace*{0.25in}
\centerline{\it \copyright \ \number\year \
NYU Stern School of Business}
}

\end{document}

From Peter:  \href{http://www.technologyreview.com/computing/12908/}{Ghana}
(Good story, paywall...)


\begin{solution}
The challenge here is to take a random collection
of data and use it to form a coherent view of
the quality of the local business environment
{\it for this particular business\/}.
Good answers tied these measures to the needs of call centers,
less good answers typically listed the measures and
simply noted which country looked better.
%
\begin{parts}
\part Here's a guess of the main issues
in (roughly) declining order of importance:
wages, English-speaking, literate and somewhat educated,
flexible labor market, good telecommunications infrastructure,
and general business-friendly environment.
Most of the indicators can be mapped into these categories.

Grading:  10 points for something close to this or
something logically coherent
on its own terms, partial credit otherwise.
It's essential that the answer address this specific business.

\part Overall they're pretty similar.
Wages are likely lower in Ghana, since GDP per capita is.
English isn't the official language India, but it's a common
second language and many Indians speak English.
In Ghana, English is the official language.
Literacy and education are a little lower in Ghana, but there's not much difference.
It's not clear how good the infrastructure is, but related measures
(electricity, general infrastructure) are similar or somewhat better
in Ghana.
The countries are similar on employment rigidity
(Ghana is slightly less rigid), but Ghana has higher severance costs.
(It's not reported in the table, but
severance payments here are for workers with 20+ years of employment,
so the number may not be typical for this business.)
As for the general environment:
rule of law, control of corruption, and effectiveness of government
are all similar in the two countries.

Grading:  10 points for similar discussion.

\part I'd say the two countries are similar in most respects.
If you can run this business in India, you can probably run it in Ghana,
where wages are lower.
The one source of concern is severance costs --- we might
want to look more closely at this.

Things I'd want to collect more information about:
wages for the appropriate skilled people,
political situation, how severance works,
whether this quality of education measure is relevant to me.

Grading:  10 points for this or other logical argument.

\end{parts}
\end{solution}



