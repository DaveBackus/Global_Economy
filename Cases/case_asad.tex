\documentclass[12pt]{article}

\usepackage{GE13}
\usepackage{comment}
\usepackage{booktabs}
\usepackage{hyperref}
\urlstyle{rm}   % change fonts for url's (from Chad Jones)
\hypersetup{
    colorlinks=true,        % kills boxes
    allcolors=blue,
    pdfsubject={NYU Stern course GB 2303, Global Economy},
    pdfauthor={Dave Backus @ NYU},
    pdfstartview={FitH},
    pdfpagemode={UseNone},
%    pdfnewwindow=true,      % links in new window
%    linkcolor=blue,         % color of internal links
%    citecolor=blue,         % color of links to bibliography
%    filecolor=blue,         % color of file links
%    urlcolor=blue           % color of external links
% see:  http://www.tug.org/applications/hyperref/manual.html
}

\usepackage{enumitem}
\setitemize{leftmargin=*, topsep=0pt}
\setenumerate{leftmargin=*, topsep=0pt}

% for ge05.sty
\def\ClassName{The Global Economy}
\def\Category{Mini-Case}
\def\HeadName{A Crisis of Confidence?}

\begin{document}
\parindent = 0.0in
\parskip = 0.5\bigskipamount
\thispagestyle{empty}%
\Head

\centerline{\large \bf \HeadName}%
\centerline{Revised:  \today}

\bigskip
You have rotated into the economic analysis group at JP Morgan Chase,
reporting to US Chief Economist Michael Feroli.
As your first assignment, he asks you to work through the possible impact
on the US economy of further deterioration in Europe.
He suggests that this might affect the US economy through three channels.
First, Europeans may buy fewer US products.
Second, financial conditions in Europe might spill over to the US,
making it more difficult, for example, for US firms to finance expansion.
And third, the loss of confidence there might lead to a fall
in confidence of US business leaders.


You go back to your Global Economy notes, in particular the
chapters on the aggregate supply and demand model,
where you find this figure:

\begin{center}
\setlength{\unitlength}{0.075em}
%\setlength{\unitlength}{0.1em}
\begin{picture}(280,200)(0,-10)
%\footnotesize
\thicklines

% horizontal axis
\put(-30,0){\vector(1,0){300}}
\put(255,-16){$Y$}

% vertical axis
\put(0,-20){\vector(0,1){200}}
\put(-15,155){$P$}

% demand
\put(25,165){\line(4,-3){200}}\put(230,10){AD}
%\put(65,165){\line(4,-3){200}}\put(270,10){AD}

% supply
\put(25,13){\line(4,3){200}} \put(230,160){AS}
%\put(65,13){\line(4,3){200}} \put(270,160){AS$'$}
\put(126.4,0){\line(0,1){170}} \put(118,175){AS$^*$}\put(122,-16){$Y^*$}

% equilibrium labels
%\put(105,85){\footnotesize B}
%\put(150,115){\footnotesize A}
%\put(138,64){\footnotesize C}
% dotted lines
%\qbezier[31]{(133,0)(133,46)(133,92)}
%\qbezier[45]{(0,92)(67,92)(133,92)}
%\qbezier[45]{(0,72)(67,72)(133,72)}

\end{picture}
\end{center}



You remind yourself what it means and write down some questions:
%
\begin{itemize}
\item Where is the current short-run equilibrium in this diagram?
Does that seem right to you?
\item Do Feroli's channels correspond to changes in supply or demand?
\item If you decide the primary effect is on demand, how would you expect
GDP growth and inflation to respond in the short run?  In the long run?
\item What, if anything, do you think is missing from this analysis?
\end{itemize}

\vfill
{
%More of this kind of thing at \url{https://sites.google.com/site/nyusternglobal/}
\vspace*{0.1in}
\centerline{\it \copyright \ \number\year \ NYU Stern School of Business}
}



\end{document}


\pagebreak
%
{\bf Exhibit 1 \\
Indicators for Ghana and India}
\begin{center}
\tabcolsep = 0.14in
\begin{tabular}{lrrr}
\toprule
Indicator   &  Ghana    &  India   &  Source  \\

\midrule
\multicolumn{3}{l}{\it Governance indicators\/} \\
Political stability (percentile) & 40    &    10  & WB, WGI \\
Govt effectiveness (percentile)  & 57  & 54  & WB, WGI \\
Regulatory quality (percentile)  & 53  & 42  & WB, WGI \\
Rule of law (percentile)         & 53  & 56  & WB, WGI \\
Control of corruption (percentile)& 60 & 47  & WB, WGI \\

\midrule
\multicolumn{4}{l}{\it Economic and business indicators\/} \\
GDP per capita (USD)        &  3,081 & 3,703 &  IMF \\
Employment rigidity (percentile)   & 27 & 30 & WB, Doing Business \\
Severance costs (weeks of pay) & 178  & 56 & WB, Doing Business \\
Paying taxes:  hours spent     & 224  & 254 & WB, Doing Business \\
Paying taxes:  gross rate (\%) &  34  &  62 & WB, Doing Business \\
Enforcing contracts (cost, \%) &  23  &  40 & WB, Doing Business \\
Investor protection (percentile)    &  60  &  60 & WB, Doing Business \\

\midrule
\multicolumn{4}{l}{\it Education indicators\/} \\
Literacy of adults  (\%)       &    67     &  63   &  WB, WDI \\
Literacy of young (15-24)  (\%) &    67     &  63  & WB, WDI  \\
Primary school completion (\%)  & 87    & 96    & WB, WDI  \\
Average education of adults (years)  & 7.0 & 4.4 & Barro-Lee \\

\midrule
\multicolumn{4}{l}{\it Infrastructure indicators\/} \\
Infrastructure quality (percentile) & 43 & 33 &  WEF \\
Quality of electricity (percentile)  &  20 & 24  & WEF \\
Internet users (\%)   & 8.6 & 7.8 & WB, WDI \\
Broadband penetration (\%) &  0.2 & 0.9 & WB, WDI \\

\bottomrule
\end{tabular}
\end{center}
%
\vspace{-0.15in}
Sources include:  the World Bank (WB),
including their World Development Indicators (WDI),
World Governance Indicators (WGI), and Doing Business;
the International Monetary Fund (IMF);
and the World Economic Forum (WEF).
Percentiles range from 0 (bad) to 100 (good).

\vfill
{
More of this kind of thing at \url{https://sites.google.com/site/nyusternglobal/}
%\url{http://goo.gl/EDybU}

\vspace*{0.25in}
\centerline{\it \copyright \ \number\year \
NYU Stern School of Business}
}

\end{document}

