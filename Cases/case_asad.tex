\documentclass[12pt]{article}

\usepackage{../LaTeX/ge14}
\input{../LaTeX/preamble.tex}

\def\ClassName{The Global Economy}
\def\Category{Mini-Case}
\def\HeadName{Abe's Three Arrows}

\begin{document}
\parindent = 0.0in
\parskip = 0.5\bigskipamount
\thispagestyle{empty}%
\Head

\centerline{\large \bf \HeadName%
\footnote{This was written with Kim Schoenholtz, Japan expert extraordinaire.}
}%
\centerline{Revised:  \today}


% ?? check low unemployment rate, comments:  
% http://marginalrevolution.com/marginalrevolution/2014/11/the-bad-economic-news-out-of-japan-what-does-it-mean.html

\bigskip
Shinzo Abe was elected Prime Minister of Japan in December 2012
after two decades of slow growth and falling prices.
He pledged dramatic policy changes to revive the Japanese economy,
dubbed the ``three arrows'' of ``Abenomics.''
We consult the Economist Intelligence Unit and press reports for specifics:
%
\begin{itemize}
\item Arrow \#1:  Fiscal stimulus.  A sizeable economic stimulus package was passed by parliament in
February of 2013, and a smaller one in October.
The IMF reports that this resulted in a 2013 fiscal deficit of 7.6\% of GDP
and net government debt over 130\% of GDP.

\item Arrow \#2: Monetary stimulus.
After nearly two decades of deflation, the Bank of Japan (BOJ) announced in April 2013
an aggressive plan to expand the money supply
to achieve an inflation target of 2\%.
Falling short of their target, the BOJ announced on October 31, 2014,
a further massive increase in its balance sheet.
Total assets of the BOJ are now projected to exceed 70\% of GDP in 2015,
more than double comparable figures from the Fed,
the European Central Bank, and the Bank of England.

\item Arrow \#3: Structural reform.
The government has proposed a broad array of micro-based reforms,
including looser product-market regulations,
a more flexible labor market,
and smaller subsidies to an inefficient agricultural sector.
However, the government has been reluctant to reduce Japan's high level of agricultural protectionism,
which keeps out a range of agricultural products, including rice, from other countries.
Many observers remain skeptical that the government will promote these structural reforms aggressively.
\end{itemize}
%
Your mission is to explore the impact of the three arrows using
the aggregate supply and demand framework summarized by Figure \ref{fig:asad}.
\begin{itemize}
\item Explain, for each ``arrow,'' whether it affects supply or demand.
Which way does each one shift the appropriate curve(s)?
\item Compare the short- and long-term impact on output of the three policies.
Which are likely to have the greatest impact in the short term?
In the long term?

\item The fiscal stimulus was enacted despite fears that Japan's public debt was reaching
dangerous levels.
To allay these fears,
parliament raised the consumption tax from 5\% to 8\% in April 2014.
At the government's discretion, a further hike to 10\% is scheduled for October 2015.
If the second tax hike is implemented,
the IMF projects Japan's fiscal deficit to decline to 6.7\% of GDP in 2014,
5.5\% in 2015, and 4.7\% in 2016.

What are the likely effects of these tax increases?

\item Overall, which arrow do you think is most important to Japan's future?
\end{itemize}


\vspace*{0.5in}
\begin{figure}[h!]
\caption{Aggregate supply and demand diagram 1}
\label{fig:asad}
\begin{center}
\setlength{\unitlength}{0.075em}
%\setlength{\unitlength}{0.1em}
\begin{picture}(280,200)(-20,-10)
%\footnotesize
\thicklines

% horizontal axis
\put(-30,0){\vector(1,0){300}}
\put(255,-16){$Y$}

% vertical axis
\put(0,-20){\vector(0,1){200}}
\put(-15,155){$P$}

% demand
\put(25,165){\line(4,-3){200}}\put(230,10){AD}
%\put(65,165){\line(4,-3){200}}\put(270,10){AD}

% supply
\put(25,13){\line(4,3){200}} \put(230,160){AS}
%\put(156.4,0){\line(0,1){170}}
%\put(148,175){AS$^*$}\put(152,-16){$Y^*$}
% SUBTRACT 30 FROM X AXIS 
\put(127,0){\line(0,1){170}}
\put(116,175){AS$^*$}\put(122,-16){$Y^*$}
\end{picture}
\end{center}
\end{figure}

%\vspace*{0.5in}
\begin{figure}[h!]
\caption{Aggregate supply and demand diagram 2}
\label{fig:asad}
\begin{center}
\setlength{\unitlength}{0.075em}
%\setlength{\unitlength}{0.1em}
\begin{picture}(280,200)(-20,-10)
%\footnotesize
\thicklines

% horizontal axis
\put(-30,0){\vector(1,0){300}}
\put(255,-16){$Y$}

% vertical axis
\put(0,-20){\vector(0,1){200}}
\put(-15,155){$P$}

% demand
\put(25,165){\line(4,-3){200}}\put(230,10){AD}
%\put(65,165){\line(4,-3){200}}\put(270,10){AD}

% supply
\put(25,13){\line(4,3){200}} \put(230,160){AS}
\put(156.4,0){\line(0,1){170}}
\put(148,175){AS$^*$}\put(152,-16){$Y^*$}
\end{picture}
\end{center}
\end{figure}


\input{../LaTeX/footer.tex}

\end{document}

\begin{comment}
This continues to be topical.
Here's a
\href{http://www.bloomberg.com/news/2013-12-12/abe-pushes-biggest-farm-revamp-since-macarthur-broke-landlords.html}
{recent comment} about agricultural policy.

\begin{itemize}
\item We have:
\begin{itemize}
\item Fiscal stimulus. This shifts aggregate demand to the right.
\item Monetary stimulus. Same.
\item Structural reform. This shifts both aggregate supply curves to the right.
\end{itemize}
Grading:  5 points for each bullet done correctly.
The graphs need not be included, but if they are they should be correct.

\item Fiscal and monetary stimulus will raise output in the short run.
They have no long-run impact on output.

Structural reform, on the other hand, raises output both short-term and long-term.
In this respect, it's likely the most important of the arrows.
(Also, unfortunately, the one that's been executed least effectively.)

Grading: 5 points for each one.
Also:  the original question asks for the short and long-term impact, and doesn't
specify output, so answers might include comments about the impact on prices.
\end{itemize}
\end{comment}

