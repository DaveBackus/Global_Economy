\documentclass[letterpaper,12pt]{article}

\RequirePackage{GE13}
% this inputs graphicx, too
\RequirePackage{comment}
\RequirePackage{hyperref}
%\hypersetup{dvips}
%\hypersetup{backref,colorlinks=false,urlcolor=cyan,linkcolor=blue}
%\hypersetup{pdfpagemode=None,pdfstartview=FitH}

\def\ClassName{The Global Economy}
\def\Category{Mini-Case}
\def\HeadName{Mercedes-Benz USA}

\begin{document}
\parindent = 0.0in
\parskip = 0.6\bigskipamount
\thispagestyle{empty}%
\Head

\centerline{\large \bf \HeadName}%
\centerline{Revised:  \today}

\medskip
It's the first day of your summer internship at Mercedes-Benz USA,
a subsidiary of DaimlerChrysler AG.
You are thrilled by the opportunity to market Mercedes to American consumers,
but somewhat nervous about your new job.
Your boss enters your office with your first assignment:
summarize the macroeconomic conditions most likely to affect
the US market for Mercedes through the end of this year.
She explains that growth, inflation, and interest rates
have a direct impact on consumer spending,
and that exchange rates affect your cost of production
as well as that of your competitors.
She asks, specifically, for your year-end forecasts of:
%
\begin{enumerate}
\topsep=0.05\bigskipamount
\itemsep=0.05\bigskipamount

\item Economic growth:
the growth rate of real GDP.

\item Inflation:
the year-on-year inflation rate.

\item Interest rates:  short-term money market rates.

\item Foreign exchange rates:
\begin{itemize}
\item the dollar price of one euro and
\item the yen price of one dollar.
\end{itemize}

\end{enumerate}
Each forecast should include a brief rationale
and an equally brief assessment of the impact each item is likely to have on the production and pricing of Mercedes in the US.


\vfill \centerline{\it \copyright \ \number\year \ NYU Stern
School of Business}

\end{document}

