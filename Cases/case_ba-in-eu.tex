\documentclass[12pt]{article}

\usepackage{../LaTeX/ge14}
% spacing on page
\oddsidemargin=0.25truein \evensidemargin=0.25truein
\topmargin=-0.5truein \textwidth=6.0truein \textheight=8.75truein

\usepackage{comment}
\usepackage{graphicx}
\usepackage{amssymb}
\usepackage{amsmath}

% layout of figures and tables
\usepackage[margin=0pt, labelsep=period, labelfont=bf]{caption}
%\usepackage{float}

\usepackage{hyperref}
\urlstyle{rm}   % change fonts for url's (from Chad Jones)
\hypersetup{
    colorlinks=true,        % kills boxes
%    allcolors=blue,
    pdfsubject={ECON-UB233, Macroeconomic foundations for asset pricing},
    pdfauthor={Dave Backus @ NYU},
    pdfstartview={FitH},
    pdfpagemode={UseNone},
%    pdfnewwindow=true,      % links in new window
    linkcolor=blue,         % color of internal links
%    citecolor=blue,         % color of links to bibliography
    filecolor=blue,         % color of file links
    urlcolor=blue           % color of external links
% see:  http://www.tug.org/applications/hyperref/manual.html
}

% for listing code in tt font
\usepackage{verbatim}

% for table spacing
\usepackage{booktabs}

% section headers and spacing
\usepackage[tiny, compact]{titlesec}

% list spacing
\usepackage{enumitem}
\setitemize{leftmargin=*, topsep=0pt}
\setenumerate{leftmargin=*, topsep=0pt}

% attach files to the pdf
\usepackage{attachfile}
    \attachfilesetup{color=0.75 0 0.75}

\usepackage{needspace}
% \needspace{4\baselineskip} makes sure we have four lines available before a pagebreak

%\newcommand{\phm}{\phantom{--}}
\newcommand{\NX}{\mbox{\it NX\/}}
\newcommand{\POP}{\mbox{\it POP\/}}
\renewcommand{\log}{\ln}

\renewcommand{\thefootnote}{\fnsymbol{footnote}}


\def\ClassName{The Global Economy}
\def\Category{Mini-Case}
\def\HeadName{Business Analytics in Europe}

\begin{document}
\parindent = 0.0in
\parskip = 0.5\bigskipamount
\thispagestyle{empty}%
\Head

\centerline{\large \bf \HeadName}%
\centerline{Revised:  \today}

\bigskip
As a graduating MBA student at the prestigious ecole des Hautes Etudes Commerciales (HEC) de Paris,
you face a daunting job market as Europe struggles through its 
worst economic climate since the 1930s.  
In a bad job market, you figure the solution is to hire yourself. 
Together with two classmates, you start developing plans for a business analytics startup.
The idea is to provide data insights to a broad range of businesses located
throughout the European Union.
The beauty of the plan, you think, is that you can do it anywhere.


The three of you begin to compare the pros and cons of Paris, Barcelona, and Stockholm,
your respective home bases.
You collect the data in Table \ref{tab:cities} and begin to sketch out a plan.
As you think this through, you run through the issues in your mind:
\begin{itemize} \itemsep=0.0in
\item What features do you need in a city to make it attractive to you and your business?
\item What are the pros and cons of each city along these dimensions?
\item Which city do you think best fits your plans?
\end{itemize}

\bigskip
\begin{table}[h]
%    \tabcolsep = 0.2in
\centering
\begin{tabular}{lrrr}
\toprule
Country Indicators & France  & Spain  & Sweden  \\
\midrule
Ease of doing business (rank) &  38  &  52  &  14   \\
Ease of starting a business (rank) &  41  &  142  &  61   \\
Protecting investors (rank)     &  80  &  98    & 34  \\
Getting electricity (rank)      &  42 &  62  &  9 \\
Resolving insolvency (rank)     &  46 &  22 &  20 \\
%\midrule
Minimum wage (USD/month)        &  778 & 1009   &  none \\
Mandatory severance (weeks of pay)        &  4   &  14    & none \\
Unemployment rate               &  10.5  &  24.4   & 7.4  \\
Employment rate                 & 64.3  &  55.8  &  74.7    \\
Difficulty of dismissals (index, 1-6)   &  2.6 & 2.0 & 2.5  \\
%\midrule
Education of workers aged 25-34 (years) &  12.6  &  11.5  &  12.5  \\
Internet quality (Ookla, index)    &  81.3  &  84.0  &  86.9 \\
\midrule
City Indicators & Paris  & Barcelona  & Stockholm  \\
\midrule
Quality of life (rank, Mercer)  &  34  &  44  &  20  \\
Cost of living (index)          & 226  & 223  &  157 \\
\bottomrule
\end{tabular}
\caption{Business indicators for three cities and countries.}
\label{tab:cities}
\end{table}


{\vfill
{\vfill
{\bigskip \centerline{\it \copyright \ \number\year \
David Backus $|$ NYU Stern School of Business}%
}}

}

\end{document}

From Peter:  \href{http://www.technologyreview.com/computing/12908/}{Ghana}
(Good story, paywall...)


\begin{solution}
The challenge here is to take a random collection
of data and use it to form a coherent view of
the quality of the local business environment
{\it for this particular business\/}.
Good answers tied these measures to the needs of call centers,
less good answers typically listed the measures and
simply noted which country looked better.
%
\begin{parts}
\part Here's a guess of the main issues
in (roughly) declining order of importance:
wages, English-speaking, literate and somewhat educated,
flexible labor market, good telecommunications infrastructure,
and general business-friendly environment.
Most of the indicators can be mapped into these categories.

Grading:  10 points for something close to this or
something logically coherent
on its own terms, partial credit otherwise.
It's essential that the answer address this specific business.

\part Overall they're pretty similar.
Wages are likely lower in Ghana, since GDP per capita is.
English isn't the official language India, but it's a common
second language and many Indians speak English.
In Ghana, English is the official language.
Literacy and education are a little lower in Ghana, but there's not much difference.
It's not clear how good the infrastructure is, but related measures
(electricity, general infrastructure) are similar or somewhat better
in Ghana.
The countries are similar on employment rigidity
(Ghana is slightly less rigid), but Ghana has higher severance costs.
(It's not reported in the table, but
severance payments here are for workers with 20+ years of employment,
so the number may not be typical for this business.)
As for the general environment:
rule of law, control of corruption, and effectiveness of government
are all similar in the two countries.

Grading:  10 points for similar discussion.

\part I'd say the two countries are similar in most respects.
If you can run this business in India, you can probably run it in Ghana,
where wages are lower.
The one source of concern is severance costs --- we might
want to look more closely at this.

Things I'd want to collect more information about:
wages for the appropriate skilled people,
political situation, how severance works,
whether this quality of education measure is relevant to me.

Grading:  10 points for this or other logical argument.

\end{parts}
\end{solution}



