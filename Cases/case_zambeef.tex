\documentclass[12pt]{article}

\usepackage{../LaTeX/ge14}
% spacing on page
\oddsidemargin=0.25truein \evensidemargin=0.25truein
\topmargin=-0.5truein \textwidth=6.0truein \textheight=8.75truein

\usepackage{comment}
\usepackage{graphicx}
\usepackage{amssymb}
\usepackage{amsmath}

% layout of figures and tables
\usepackage[margin=0pt, labelsep=period, labelfont=bf]{caption}
%\usepackage{float}

\usepackage{hyperref}
\urlstyle{rm}   % change fonts for url's (from Chad Jones)
\hypersetup{
    colorlinks=true,        % kills boxes
%    allcolors=blue,
    pdfsubject={ECON-UB233, Macroeconomic foundations for asset pricing},
    pdfauthor={Dave Backus @ NYU},
    pdfstartview={FitH},
    pdfpagemode={UseNone},
%    pdfnewwindow=true,      % links in new window
    linkcolor=blue,         % color of internal links
%    citecolor=blue,         % color of links to bibliography
    filecolor=blue,         % color of file links
    urlcolor=blue           % color of external links
% see:  http://www.tug.org/applications/hyperref/manual.html
}

% for listing code in tt font
\usepackage{verbatim}

% for table spacing
\usepackage{booktabs}

% section headers and spacing
\usepackage[tiny, compact]{titlesec}

% list spacing
\usepackage{enumitem}
\setitemize{leftmargin=*, topsep=0pt}
\setenumerate{leftmargin=*, topsep=0pt}

% attach files to the pdf
\usepackage{attachfile}
    \attachfilesetup{color=0.75 0 0.75}

\usepackage{needspace}
% \needspace{4\baselineskip} makes sure we have four lines available before a pagebreak

%\newcommand{\phm}{\phantom{--}}
\newcommand{\NX}{\mbox{\it NX\/}}
\newcommand{\POP}{\mbox{\it POP\/}}
\renewcommand{\log}{\ln}

\renewcommand{\thefootnote}{\fnsymbol{footnote}}


\def\ClassName{The Global Economy}
\def\Category{Mini-Case}
\def\HeadName{Opportunities for Zambeef}

\begin{document}
\parindent = 0.0in
\parskip = \bigskipamount
\thispagestyle{empty}%
\Head

\centerline{\large \bf \HeadName}%
\centerline{Revised:  \today}

\bigskip
Zambeef, the Zambia-based meat distributor, is looking for new opportunities.
The Economist reports:
``Zambeef operates meat counters at all 20 Shoprite stores across Zambia
as well as its newer outlets in Ghana and Nigeria.
Zambeef also has around 100 shops of its own.
The CEO notes that with markets targeting both low and high-income consumers,
they are able to sell `all of the animal.'
The firm is also vertically integrated;
its `farm-to-fork' model includes farms, retail outlets,
and `cold chain logistics' with its fleet of 78 refrigerated trucks.
The downside of recent expansion, they say, is the demands on its managers.''

\begin{table}[h]
\centering
\tabcolsep = 0.1in
\begin{tabular}{lrrr}
\toprule
Indicator & Zambia &  Botswana & Tanzania \\
\midrule
\multicolumn{2}{l}{\it General} \\
GDP per capita  (2005 USD) &  1690  & 11,300 & 1250  \\
Population (millions)      &   13.5 & 2.0 &  44.9   \\
Doing Business overall (percentile) & 49  & 68 & 27 \\
World Economic Forum overall (percentile) & 37 & 50 & 16\\
\midrule
\multicolumn{2}{l}{\it Governance} \\
Political stability (percentile)  &  66 & 88 & 48 \\
Govt effectiveness (percentile)   &  38 & 68 & 28 \\
Regulatory quality (percentile)   &  37 & 74 & 37 \\
Rule of law (percentile)          &  42 & 75 & 34 \\
Control of corruption (percentile)&  46 & 79 & 22 \\
\midrule
\multicolumn{2}{l}{\it Labor} \\
Minimum wage (USD per month) &   76 & 92 & 52 \\
Severance after 10 years (weeks of pay) & 87 & 36 & 10 \\
Labor market efficiency (percentile) & 37 & 68 & 67  \\
Literacy (percent of adults)        & 71 & 84 & 73 \\
Years of school (adults)        & 6.7 & 9.6 & 5.8  \\
\midrule
\multicolumn{2}{l}{\it Infrastructure and trade} \\
Infrastructure quality (percentile)  & 20 & 36 & 10 \\
%\midrule
%\multicolumn{2}{l}{\it International trade} \\
Import documents required (number) & 8 & 5 & 6\\
Import delay (days) &  56 & 37 & 31  \\
Import cost (USD per container) &  3600 & 3500 & 1600 \\
\bottomrule
\end{tabular}
\caption{Economic and institutional indicators.
Percentiles range from 0 (worst) to 100 (best).
Sources:  Penn World Table, World Economic Forum, World Bank, Doing Business.}
\label{tab:zambia}
\end{table}


Zambeef is now looking to expand further, either in Zambia or in nearby countries.
As a consultant based in Johannesburg,
you have been asked to advise them on the strengths and weaknesses
of neighboring Botswana, Tanzania, and Zambia.
You quickly summarize various measures of institutional quality in the three countries;
see Table \ref{tab:zambia}.
You also turn to the World Economic Forum's {\it Global Competitiveness Report\/},
which includes a survey of business leaders and tabulates the most commonly
reported problems.
For these three countries, the most common complaints were
\begin{itemize} \itemsep=0.0in
\item Zambia:  access to financing, corruption, and inadequate infrastructure.
\item Botswana:  poor work ethic of labor force, inefficient government bureaucracy,
and access to financing.
\item Tanzania:  access to financing, corruption, and inadequate infrastructure.
\end{itemize}


As you work your way through the data, you ask yourself:  
\begin{itemize} \itemsep=0.0in
\item What features of an economy are important to your business?  
\item How do Botswana and Tanzania compare to Zambia on these features?
\item What issues raise the most concern in each country?
How might you deal with them?
\item What location(s) would you recommend?
\end{itemize}

\end{document} 


