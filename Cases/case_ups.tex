\documentclass[letterpaper,12pt]{article}

\RequirePackage{GE05}
% this inputs graphicx, too

\def\ClassName{The Global Economy}
\def\Category{Mini-Case}
\def\HeadName{UPS in Mexico}

\begin{document}
\parindent = 0.0in
\parskip = \bigskipamount
\thispagestyle{empty}%
\Head

\centerline{\large \bf \HeadName}%
\centerline{Revised:  \today}

\bigskip
June 1995.  Ricardo Dadoo, director general of UPS de Mexico, sat
back wearily in the chair of his Mexico City office.  UPS had spent
four years and over \$100m setting up Mexican operations to take
advantage of the cross-border business generated by the 
North American Free Trade Agreement (NAFTA).  The air
service was working well, but Mexican regulations continued to
obstruct his efforts to build an inexpensive ground-based
small-package business in Mexico. 
%The restriction on large trucks, 
%in particular, raised his costs relative to locally-owned competitors.  
He wondered what to do next.

As you think through Dadoo's strategy, consider these questions:
%
\begin{itemize}
\item What was the business opportunity UPS identified in Mexico?  

\item What were Dadoo's problems with Mexican regulations? 
How did they affect his costs?  

\item What was the source of his issues?  

\item What should he do?
\end{itemize}

The attached exhibits come from the American business press.


%\vfill \centerline{\it \copyright \ \number\year \ 
%NYU Stern School of Business}

% --------------------------------------------------------------------
\pagebreak%
{\bf Exhibit 1.} {\bf The Seattle Times, April 15, 1993.}

MILES TO GO BEFORE THEY SLEEP --- AIR CARGO EXPANDING IN MEXICO

BYLINE: William Armbruster, Polly Lane

U.S. air cargo carriers, anticipating a boom in trade after
implementation of the North American Free Trade Agreement, are
dramatically expanding their Mexican operations. Leading the most
recent expansion effort are United Parcel Service and Burlington Air
Express. Both launched new operations two weeks ago, with UPS
opening a \$1.8 million office and terminal at Benito Juarez Airport
in Mexico City and Burlington beginning next-day door-to-door
service within Mexico via a dedicated freighter network run by its
new local agent. ``All the carriers are positioning themselves to
handle the boom when NAFTA is ratified," added George Trevino,
Mexico manager for Seattle-based Airborne Express.


% --------------------------------------------------------------------
\newpage
{\bf Exhibit 2.} {\bf Journal of Commerce, April 27, 1995.}

US BEGINS TRADE ACTION ON UPS' MEXICO COMPLAINT

BYLINE: JOHN MAGGS

DATELINE: WASHINGTON

The Clinton administration began a trade-dispute proceeding against
Mexico this week, after complaints from United Parcel Service that
it had been unfairly discriminated against in its efforts to compete
in Mexico's small-package delivery market.

The case, the first against Mexico under the 1993 North American
Free Trade Agreement, could yield a decision by next fall calling
for Mexico to change its rules.

UPS officials claim that Mexico has dragged its heels for a year in
responding to charges that its regulations violate the Nafta by
favoring Mexican delivery companies.

Mexican officials repeatedly asked for more time to draft
regulations, promising they would address the UPS complaints. But
when that draft was finally furnished this week, UPS Vice President
Bob Elizondo got a rude surprise.

Not only did the draft fail to correct UPS's main complaint, said
Mr. Elizondo, who is responsible for the company's Latin American
business, but ``there were new restrictions that would have made
things worse."

On Tuesday, U.S. Trade Representative Mickey Kantor sent a letter to
Mexican Commerce Secretary Herminio Blanco asking for
``consultations" on the dispute. The request is the first step in
forming a five-member dispute panel under Chapter 20 of the Nafta.
Consultations must be held by May 25, and any time after that, the
United States can insist on a panel.

The five members would be drawn from a roster of trade experts from
Canada, Mexico and the United States. Once formed, the panel would
take up to four months to decide whether Mexico has violated its
obligations under the Nafta. It does not have the power to enforce
its recommendations, but if its members decide that their ruling has
not been carried out, the aggrieved country would have the right to
impose trade sanctions.

UPS claims that the Nafta provision violated by Mexico involves
``national treatment" --- the idea that a U.S. company would be
regulated by the Mexican government in exactly the same way that a
Mexican company is regulated.

Although the two countries have argued about whether Mexico was
exempted from this requirement for small-package deliveries, the
crux of the dispute is that Mexico allows Mexican-owned companies to
use full-size trucks in the small-package business, while preventing
UPS and other foreign carriers from using anything larger than small
delivery vans.

The draft regulations, out this week, were supposed to correct this
problem but did not, according to U.S. officials.

Beyond that shortcoming, they said, the regulations established new
restrictions on small-package deliveries, such as new size and width
limits and a limit on the number of packages that can be delivered
at one time to a customer. It appeared from the draft that Mexican
companies would be exempted from these requirements.


% --------------------------------------------------------------------
\newpage
{\bf Exhibit 3.} {\bf Journal of Commerce, July 13, 1995.}

EDITORIAL:  ROADBLOCK IN MEXICO

MEXICO was supposed to be a land of opportunity after the North
American Free Trade Agreement took effect. Instead, meddlesome
Mexican officials have made life miserable for some businesses,
including America's largest freight transportation company, United
Parcel Service.

UPS this week said it will end its trucking service between the
United States and Mexico because of barriers created --- or
tolerated --- by the Mexican government. Some of the company's
problems are the result of blatant protectionism by the Mexican
government, which is under pressure from its domestic trucking
industry to curb UPS' activities.

Until recently, for example, Mexico barred foreign-owned companies
from using trucks that weighed more than four tons. Mexico recently
agreed to change that rule under pressure from the U.S. government.
But officials added new restrictions on the number of packages UPS
can carry, and on their size and weight. This will force the company
to give up some business it already had.

To make matters worse, UPS faces a nightmare at Mexican customs
locations, where packages must be opened and searched. A frustrated
UPS official told us, ``Our ground operation was designed to be a
low-cost alternative to air service. But (Mexico's regulations)
forced us to dump tons of money into it, so it was no longer a
low-cost alternative."

Mexico has its share of problems today, most of them more pressing
than an angry U.S. transportation company. But in the wake of its
financial meltdown, Mexico should be working hard to attract good
companies and willing investors. Instead, it is driving them away.



% --------------------------------------------------------------------
\newpage
{\bf Exhibit 4.} {\bf Journal of Commerce, October 3, 1995.}

UPS PRESSES ADMINISTRATION TO SEEK MORE FREEDOM IN MEXICAN
OPERATIONS

BYLINE: RIP WATSON

Three months ago, UPS halted all ground operations between the
United States and Mexico, claiming the latter's restrictive rules
made the service inefficient and too costly to operate.

Although UPS won't acknowledge it, the Mexican service apparently
has been a money loser.

UPS operations in Mexico now consist of international air service
with morning delivery and a limited domestic ground operation that
is concentrated in a triangle between Monterrey, Guadalajara and
Mexico City.

``The question is not new services, but the ability for us to be
more efficient," Mr. Elizondo said. ``We're playing a wait-and-see
game. I'm waiting to see what they (the Mexican government) are
going to come up with to level the playing field. We are attempting
to provide the full level of service to all Mexican customers. We
are suffering in Mexico because we cannot operate as efficiently as
we'd like to."

UPS has invested between \$120 million and \$140 million in Mexico,
and has plans to double that within five years.

UPS has laid off 17 percent of its Mexican work force of 1,800
persons because of restrictions on its operations, Mr. Elizondo
said.



% --------------------------------------------------------------------
\newpage
{\bf Exhibit 5.} {\bf Journal of Commerce, March 19, 1996.}

UPS: NAFTA'S BENEFITS OVERESTIMATED IN LAUNCHING GROUND SERVICE TO
MEXICO

BYLINE: KEVIN G. HALL

DATELINE: PUERTO VALLARTA, Mexico

United Parcel Service overestimated benefits of the North American
Free Trade Agreement when it launched its now-defunct ground service
into Mexico, and does not expect resolution this year of its
permitting problems south of the border, a top executive said here.

Speaking frankly to transport executives at this Mexican resort last
week, Robert L. Elizondo said the package-express giant is now more
realistic about hurdles it faces in its Mexico operations.

``I think we overestimated Nafta,'' Mr. Elizondo, a Miami-based vice
president of operations for the Americas, said when asked about UPS'
decision last year to abandon its ground service into Mexico. The
company continues operating its air operations in Mexico supported
by local delivery fleets.

In a conversation with reporters afterwards, Mr. Elizondo said the
Clinton administration's decision to not respect the Nafta's Dec. 18
opening of border states to foreign competition dooms prompt
resolution of UPS' dispute with Mexican regulators over truck
licensing.

UPS has been locked in a two-year dispute with Mexican transport
officials, accusing the government of not providing it national
treatment as required under the Nafta. UPS is unhappy that it is
restricted to 5-ton vehicles on federal highways --- the same limit
on domestic package companies. But UPS insists that limit is not
realistic nor consistent with global treatment of express companies.

``There were numerous discussions on it when President (Ernesto)
Zedillo went to Washington in October last year. From the Congress
to the president himself, (Trade Representative) Mickey Kantor, and
I guess we anticipated in 1996 maybe there would possibly be
resolution,'' Mr. Elizondo said. ``But after the Dec. 18 issue, I
think we're not looking on any resolution on that until'' after the
Nov. 5 presidential elections.

Few predict resolution of the Nafta border-opening dispute until
after the elections, since it is widely viewed that the Clinton
administration bowed to Teamsters demands that the border opening be
delayed for safety concerns.

Union support is crucial to the president's re-election efforts.
But, ironically, in exchange for the border delay, Teamsters --- of
whom UPS is the world's largest employer --- are essentially
blocking UPS' efforts to expand in Mexico.

Mexico is viewed as having little incentive to resolve the UPS
dispute until the border-opening issue is resolved with the United
States.

Meanwhile, the 2-year-old dispute over trucking permits and
licensing for Mexican domestic operations is having a direct effect
on UPS' costs because --- absent their own larger trucks --- they
must contract with Mexican truckers to move goods between
operational centers in Monterrey, Guadalajara and Mexico City.

``The disadvantage obviously is that where we could get away with
one large vehicle type of trailer to make the service, we have to
use two or three vehicles,'' Mr. Elizondo told reporters. ``That's
not exactly the economy of scale we're looking for.''

Canacar, Mexico's trucking association, has vigorously opposed
changes that would let UPS operate its own large trucks between
metropolitan areas. The association says UPS would then be a
long-haul carrier and take domestic business away from its members.


% --------------------------------------------------------------------
\newpage
{\bf Exhibit 6.} {\bf Journal of Commerce, December 18, 1997.}

Mexico tired of waiting for border to open to truck traffic; After 2
years of delays, next stop could be a dispute panel, says commerce
chief

BYLINE: BY KEVIN G. HALL

DATELINE: MEXICO CITY

Truckers wanting to cross the U.S.-Mexican border have been stuck in
neutral for two years.

Today is the second anniversary of the day the border was supposed
to have opened to international trucking under the North American
Free Trade Agreement.

It hasn't, and Mexico has grown weary of the gridlock.

Mexican Commerce Secretary Herminio Blanco says two years of
discussions have yielded few results, and he expects to seek a
formal resolution next month before a Nafta dispute panel.

``We need to make a decision, and I think that should be done quite
soon,'' he said in an interview Tuesday on the 10th floor of the
towering building housing the commerce agency Secofi.

At issue is the unilateral decision by the United States to accept
applications from Mexican truckers to operate in border states but
not act on them, thus keeping the frontier closed to cross-border
truck traffic.

The Clinton administration, in what then was viewed as a bow to
organized labor before a coming election year, cited safety concerns
in delaying the agreement --- even as the Department of
Transportation had inspectors providing tours at the U.S. border
professing readiness.

Additionally, the border was to open completely to cross-border bus
operations on Jan. 1, 1997, but that too was delayed by the Clinton
administration.

``Unless the U.S. makes a move, we're going to have to initiate
formal consultations on passengers (buses) and we're going to have
to go ahead and ask for a meeting of the (Nafta) Commission on
trucking,'' Mr. Blanco said.

Under the dispute settlement rules of Nafta, after two governments
consult on a dispute, the plaintiff may convene a meeting of the
trade ministers of Mexico, Canada and the United States. If the
dispute is not settled by the commission, a panel of outside trade
experts is convened to hear it.

The trade pact, said Mr. Blanco --- who was Mexico's chief Nafta
negotiator --- does not allow the United States to impose
market-access restrictions on Mexican trucks because of what laws
and regulatory oversight Mexico does or does not have.

Nafta, he said, requires only that Mexican carriers operating in the
United States meet all U.S. rules and requirements.

``What Nafta says is, You will open these border states, but anybody
that drives in the other country has to fulfill 100 percent the
rules and regulations of the other country,'' said Mr. Blanco.
``It's very simple.''

In Washington, U.S. Secretary of Transportation Rodney Slater said
that although progress had been made, a breakthrough was not in
sight.

``We're close on a lot of these things,'' he said, referring to U.S.
safety concerns. But he added, ``Quite frankly, there are some
things that we want from them, and we don't want to really sign off
on that until we get them.''

... Mexico has shown little interest in resolving disputes such as
new package-express rules and harmonizing size and weight standards
while the border issue remains an issue.

% --------------------------------------------------------------------
\newpage
{\bf Exhibit 7.} {\bf Wall Street Journal, November 26, 2007.}

Mexican Truck Stop \\
By MARY ANASTASIA O'GRADY \\
November 26, 2007; Page A20

It's hard to say who came out on top in the Nov. 15 debate among 
Democratic presidential candidates held in Nevada. But we do know 
that free trade took a beating. A majority of the candidates 
disapproved of some or all of the U.S. bilateral and regional trade 
agreements -- including the North American Free Trade Agreement -- 
and pledged to reverse the trend toward market opening if given the 
chance. Hillary Clinton stopped short of promising to undo Nafta but 
she called for a ``trade timeout."

%It is troubling to hear the protectionist drumbeat growing louder in 
%the Democratic Party, particularly as it concerns Latin America. The 
%last time Washington adopted an anti-trade bias by signing into law 
%the Smoot-Hawley tariffs in 1930, it set off a world-wide depression 
%-- and a period of isolationism in Latin America that took some 60 
%years to begin to reverse. Now Democrats seem to be saying that if 
%they can only capture the White House, they are committed to 
%reliving this painful history.

%The Democrats' anti-trade agenda is already playing out in Congress, 
%with both houses continuing to block the full opening of the 
%southern border to Mexican long-haul trucks under Nafta. Congress's 
%actions could damage the U.S. economy because Mexico has the legal 
%right to retaliate. What's worse is what this flouting of U.S. 
%commitments to Mexico suggests to the Mexican people about Yankee 
%integrity.

[...]


The problem dates back to 1995, when Bill Clinton issued an 
executive order -- in violation of Nafta, which he had signed into 
law -- to stop Mexican long-haul trucks from crossing the border. 
Mr. Clinton was responding to pressure from Teamsters, who didn't 
want any new competition. He cited safety concerns -- things like 
substandard drivers and vehicles -- which to this day have never 
been supported by evidence.

In fact, Mexican trucking companies have a long history of operating 
in the U.S. and with no notably inferior safety record. Yet their 
numbers have been limited since 1982, when the Reagan administration 
announced that until Mexico opened its markets to U.S. competitors, 
no new licenses would be granted to Mexican carriers. Existing 
Mexican long-haul trucking businesses had their permits 
grandfathered, and from 1992-2002 some 1,300 Mexican-domiciled 
companies -- all of which were majority U.S.-owned -- received 
``certificates of registration" to deliver ``exempt commodities" 
from Mexico to the U.S.

In other words, there have been plenty of Mexican trucks on U.S. 
roads all these years -- although not as many as there might be 
under Nafta. Nevertheless, in a 2002 appropriations bill Congress 
demanded that they be subject to a new set of safety regulations, 
some of which are more stringent than U.S. standards. Since that 
year, the Department of Transportation's Inspector General has 
audited the safety process at the border annually and has been able 
to certify that it is working. Mexican carriers are also more 
heavily insured than their U.S. competitors. Every Mexican truck is 
required to carry U.S. insurance on top of the insurance it carries 
in Mexico.

Earlier this year, the DOT analyzed the safety record of Mexican 
carriers in the U.S. from 2003-2006. It looked at the rate in which 
trucks received an ``out-of-service" designation by DOT inspectors 
targeting companies with the worst records. The out-of-service rate 
for U.S. trucks was 23.5\%, compared to a rate for trucks from 
Mexico of 21.29\%. Mexican short-haul trucks operating in the border 
zone also had a better record than the U.S. trucks, with an 
out-of-service rate of 22.5\%.

These statistics ought to be enough to end the debate. But with 
Teamster pull still strong in Congress, the Bush administration this 
year offered to introduce a pilot program to allow a limited number 
of new trucking companies to begin doing business in the U.S. under 
close DOT scrutiny. The program kicked off on Sept. 6, and there are 
now seven Mexican companies operating 44 vehicles in the U.S. and 
four U.S. companies operating 41 vehicles in Mexico. You'd think 
that those with safety worries would be glad to see such a vigilant 
approach to the problem. But just after the program started, both 
the House and the Senate voted to strip its funding in the 2008 
budget.

It's not clear whether this budget cut will be sustained. But the 
effort makes it obvious that Congress is no honest broker. As John 
Hill, administrator of the Federal Motor Carrier Safety 
Administration, told me last week: ``Every time we move closer to 
implementing the provisions of Nafta, Congress adds a new provision. 
It's hard to hit a moving target."

Mr. Hill also challenges the charge that Mexican trucks are not 
safe. "We've applied strong enforcement guidelines and Mexico has 
met them. The opponents of Nafta are looking for any way they can 
find to drum up fear among Americans, even though Mexican trucks 
have been operating safely in this country for years."

Mexico doesn't have to sit still for this. In February 2001, a Nafta 
arbitration panel issued a unanimous decision against the U.S. block 
on Mexican long-haul trucks. Mexico could retaliate with import 
tariffs on U.S. goods to the tune of \$2 billion. In the Nov. 20 
issue of the Latin American business magazine Poder y Negocios, 
Mexican Secretary of Communications and Transportation Luis Tellez 
said that his country has not ruled out that possibility.

He also expressed frustration with Congress: ``The problem is that 
the Congress is no longer in the frame of mind in which it sees 
Nafta as something important or something that the U.S. government 
has to comply with." There are those in Congress, he said, who don't 
have a clear idea of what Nafta is and others who don't want to 
``lose points and Teamster support."


During the free trade bashing in Nevada, Mrs. Clinton said that 
instead of signing new agreements, we ``need to get back to 
enforcing the ones we have, which the Bush administration has not 
done." When it comes to Nafta and the Mexican trucks we can all 
agree on the first part of that statement, but the fault lies with 
Congress, not the president.


\end{document}

Notes:
* UPS enters Mexico to take advantage of NAFTA boom
* Mexicans hang all kinds of regulations on them to protect Mexican truckers
* UPS goes to US govt asking for help
* US got promised to help, but gives up when Teamsters complain about Mexican truckers competing with them in the US.
