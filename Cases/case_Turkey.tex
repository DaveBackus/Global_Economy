\documentclass[letterpaper,12pt]{article}

\RequirePackage{comment}
\RequirePackage[hypertex]{hyperref}
\RequirePackage{GE05}
% this inputs graphicx, too
\usepackage{booktabs}

\newcommand{\NX}{\mbox{\em NX\/}}
\newcommand{\POP}{\mbox{\em POP\/}}

\def\ClassName{The Global Economy}
\def\Category{Mini-Case}
\def\HeadName{Turkey}

\begin{document}
\parindent = 0.0in
\parskip = \bigskipamount
\thispagestyle{empty}%
\Head

\centerline{\large \bf \HeadName}%
\centerline{Revised:  \today}

\bigskip

\begin{center}
\begin{tabular}{lrrrrrrr}
\toprule 
         &  2004  &  2005  &  2006   & 2007  & 2008 &  2009  &  2010 \\%
\midrule 
Real GDP growth  & 9.6 & 8.4 & 6.9 & 4.7 & 0.7 & --4.7 & 4.0 \\
Inflation   & 8.6 & 8.2 & 9.6 & 8.8 & 10.4 & 6.3 & 10.0 \\
TFP growth  & 6.1 & 4.7 & 4.3 & 5.7 & --2.4 & --5.7 & 2.2 \\ 
Investment rate & 20.2 & 21.0 & 22.3 & 21.4 & 19.9 & 16.8 & 15.9 \\
Saving rate     & 19.4 & 20.0 & 22.0 & 21.1 & 21.8 & 14.8 & 16.0 \\
Current account & --3.7 & --4.6 & --6.0 & --5.8 & --5.6 & --2.3 & --4.0 \\
%Trade balance   &             
Budget balance  & --5.4 & --1.3 & --0.6 & --1.6 & --1.8 & --5.5 & --5.3 \\ 
Primary balance & 4.7 & 5.8 & 5.5 & 4.2 & 3.5 & 0.1 & --0.1 \\
Public debt      & 56.6 & 51.1 & 45.5 & 39.6 & 40.0 & 46.3 \\
Net foreign assets   & --41 & --35 & --39 & --39 & --38 & --44 & --40 \\%
Interest rate:  short & 21.4 & 14.7 & 15.6 & 17.2 & 16.0 & 9.2 & 10.0\\
%Interest rate:  long  &  \\
Real exchange rate & 131 & 146 & 145 & 159 & 162 & 150 & 165 \\
Reserves           &  37 & 52 & 63 & 77 & 74 & 75 \\
\bottomrule 
\end{tabular}
\end{center}
{\it 
Economic indicators for Turkey.  
(i)~Investment, saving, current account, budget balances, 
public debt, and net foreign assets are expressed as 
percentages of GDP (ratio to GDP multiplied by 100).  
(ii)~Government budget numbers (budget balance, primary balance): 
negative numbers indicate deficits, 
positive numbers indicate surpluses.  
(iii)~The real exchange rate is a weighted average across trading partners, 
with weights tied to the amount of trade;
high numbers indicate that prices of local goods are high relative to 
prices in other countries.
(iv)~Foreign exchange reserves are expressed in billions of USD.  
(v)~2010 numbers are estimates.
All numbers courtesy of the Economist Intelligence Unit.} 

You have been asked to write a short report summarizing 
economic prospects in Turkey over the next 2-3 years.  
Is Turkey more likely to grow like China or collapse like Greece?  
Its level of development is between the two, 
with GDP per capita double China but less than half of Greece. 

Having some experience with such situations, 
you study the Economist Intelligence Unit's 
various reports and summarize the relevant sections:  
%
\begin{itemize}
\item Modern Turkish politics has evolved from the secular 
single-party republic established by Mustafa Kemal (Ataturk) in 1923 into 
a multiparty democracy.  

\item The ruling Justice Party (AKP) gained power in 2002 and is likely to retain power until elections in 2011.  
    The AKP's ``pro-Islamist roots'' are an ongoing source of tension
    with the ``secularist elite'' and a source of 
    political uncertainty.  

\item Turkey has a free-trade agreement (``customs union'') 
with the EU and is exploring closer ties, including membership.  

\item Large government deficits in the 1990s led to inflation rates between 50 and 100 percent. 
    An accord with the IMF, in which loans were tied to fiscal stringency,
    was in effect from 1999 to 2008.  
    The government has ``promised to reverse the deterioration [in 
    the budget during the global financial crisis].''
    
\item Turkey has a flexible exchange rate.  

\item Turkey has ``full capital account convertibility:''
unlike China, there are few restrictions on international capital flows.  
At present, about two-thirds of foreign liabilities are private 
and one-third public.  

\item The banking system is ``stable.'' 

\item After a strong second half of 2009 and first quarter of 2010, 
the IMF raised its forecast of GDP growth for 2010 from 3.7 to 5.2\%.  
\end{itemize}

With this information in hand, you start to sketch out your report:  
\begin{enumerate}
\item Describe Turkey's debt-to-GDP ratio for the period in the table.
What seem to be the major factors in its behavior? 
(5~points) 

\item 
What is your estimate of the debt-to-GDP ratio 
at year-end 2010?  
What factors might change your estimate?  

\item Do you see any ``red flags'' that make you concerned 
about the near-term future of the Turkish economy?  
Explain why or why not using whatever list of issues you think is appropriate. 

\item How do you see Turkey's near-term prospects?
Is Chinese success or Greek tragedy more likely?
Why?  
\end{enumerate}


George Friedman:  

Those who argue that the Turkish government is radically Islamist are simply wrong, for two reasons. First, Turkey is deeply divided, with the powerful heirs of the secular traditions of Kemal Ataturk on one side. They are too strong to have radical Islam imposed on them. Second, the Islamism of the Turkish government cannot possibly be compared to that of Saudi Arabia, for example. Islam comes in many hues, as does Christianity, and the Turkish version derives from Ottoman history. It is subtle, flexible and above all pragmatic. It derives from a history in which Turkish Islam was allied with Catholic Venice to dominate the Mediterranean. So Turkish Islam is not strong enough to impose itself on the secularists and too urbane to succumb to simplistic radicalism. It will do what it has to do, but helping al Qaeda is not on its agenda. Still, it will be good to talk to the secularists, who regard the current government with fear and distrust, and see whether they remain as brittle as ever.

Read more: Geopolitical Journey, Part 2: Borderlands | STRATFOR 

\url{http://www.stratfor.com/weekly/20101108_geopolitical_journey_part_2_borderlands}


\vfill \centerline{\it \copyright \ \number\year \ NYU Stern
School of Business}

\end{document} 