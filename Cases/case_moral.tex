\documentclass[letterpaper,12pt]{article}

\usepackage{ge13}
\usepackage{comment}
\usepackage{graphicx}
\usepackage{hyperref}
\urlstyle{rm}   % change fonts for url's (from Chad Jones)
\hypersetup{
    colorlinks=true,        % kills boxes
    allcolors=blue,
    pdfsubject={NYU Stern course GB 2303, Global Economy},
    pdfauthor={Dave Backus @ NYU},
    pdfstartview={FitH},
    pdfpagemode={UseNone},
%    pdfnewwindow=true,      % links in new window
%    linkcolor=blue,         % color of internal links
%    citecolor=blue,         % color of links to bibliography
%    filecolor=blue,         % color of file links
%    urlcolor=blue           % color of external links
% see:  http://www.tug.org/applications/hyperref/manual.html
}
\usepackage[small,compact]{titlesec}

\def\ClassName{The Global Economy}
\def\Category{Mini-Case}
\def\HeadName{Are Markets Moral?}

\begin{document}
\parindent = 0.0in
\parskip = \bigskipamount
\thispagestyle{empty}%
\Head

\centerline{\large \bf \HeadName}%
\centerline{Revised:  \today}

\bigskip
{\it Remarks made by David Backus at a debate organized by Professor
Bruce Buchanan, Director of the Markets, Ethics, and Law
Program at NYU's Stern School of Business.
At the debate, roughly 75\% of the audience (mostly business students)
disagreed with the proposition:  markets are moral.\/}

*** New things \\
Marijuana:  http://www.nytimes.com/2013/10/27/us/few-problems-with-cannabis-for-california.html \\
Slaves:  http://qz.com/136664/there-could-be-slaves-in-the-supply-chain-of-your-chocolate-smartphone-and-sushi/

Kidneys:  http://www.volokh.com/2013/11/01/poor-rationale-banning-organ-markets/
http://www.becker-posner-blog.com/2006/01/organ-sales--posners-comment.html

http://www.bloombergview.com/articles/2014-05-29/would-you-pay-84-000-for-a-new-liver

http://www.imf.org/external/pubs/ft/fandd/2014/06/taylor.htm

Markets are free of coercion and therefore the best and most moral means of transaction available.
�The argument for liberty is not an argument against organization�. But an argument against all exclusive, monopolistic organization, against the use of coercion to prevent others from trying to do better
� Coercion occurs when one man�s actions are made to serve another man�s will, not for his own but for the other�s purpose� coercion implies both the threat of inflicting harm and the intention thereby to bring about certain conduct.

�Coercion thus is bad because it prevents a person from using his mental powers to the full and consequently from making the greatest contribution that he is capable of to the community

-          F A Hayek, �The Constitution of Liberty�

Prostitution:  http://www.economist.com/news/europe/21589922-has-liberalisation-oldest-profession-gone-too-far-giant-teutonic-brothel

Also:
\url{http://timharford.com/2014/01/what-price-supply-and-demand/}

\url{http://www.npr.org/blogs/money/2014/01/24/265396928/when-a-65-cab-ride-costs-192}

\url{http://mungowitzend.blogspot.com/2014/01/morality-and-markets.html}

*****

I'm an economist, so the idea of morality is wasted on me, not to
mention good taste.  Maybe I should have mentioned that to Bruce
earlier.  Markets, though, I think about a lot:  markets for food,
beer, drugs, baseball tickets, election futures, %husbands and wives,
%sperm,
body parts, you name it.  If you can imagine it, I'm happy to
talk about it and eBay is probably happy to sell it.

So are markets moral?  I asked a bunch of my friends, and none of
them thought the question made any sense.  Maybe that's telling me I
need new friends.  But I think their point is that markets are
a tool, and tools are neither moral nor immoral.  We could say the
same for hammers, saws, and even guns.  On the whole,
markets are a lot less dangerous than guns.  As my colleague John
Leahy put it:  ``Markets don't kill people, guns do."

If you look around the world, it's hard not to conclude that
markets are an extremely useful tool for generating prosperity.
Even Marx and Engels admitted that.  A couple decades ago, we found
lots of people trying to go from East to West Germany, but very few
the other way.  I think we can assume that they thought this was a
good idea.  Today we see people trying to move from Mexico to the
US, or from North Korea to China.  Not all of this is about markets,
but a lot of it is.  The US has more and better markets than Mexico,
and God knows what passes for markets in North Korea.  Right now,
we're seeing the greatest reduction in poverty the world has ever
seen as China and India undergo relatively modest increases in their
use of markets.

It's true, of course, that markets have a dark side.  Reality TV is
one.  And in my business there's the shock of finding out that your
dumbest classmate in graduate school is now worth a billion dollars.

That's the usual advertising for so-called free markets, but I'd
like to make a different point with an example.  A few years ago, I
was at a conference in Budapest.  I took the train from Prague with
a friend and got to Budapest about midnight.  We weren't ready for bed,
so we went for a walk along the Danube.
As we walked along, women would go
out of their way to say hello.  I thought to myself, what a friendly
town!  The next night we were drinking beer with friends, which is
what we do at conferences, and asked them whether that market was
legal in Hungary.  We didn't get an answer for Hungary, but
were told that in Turkey it's legal to sell, but not to buy.  That
sounds strange, but it turns out the Turks have devised a brilliant
system.  They not only protect the sellers, they have an ongoing
opportunity to embarrass politicians, who seem to be among the most
active buyers.  Imagine what Jon Stewart could do if the US adopted
this system.

The next year I was at the same conference, this time in Vancouver.
I got up one morning and was reading Canada's leading business newspaper,
the National Post, and ran across an article about prostitution in Germany.
Apparently Germany did one better than Turkey and legalized both buying and selling.
What they noticed (according to the article) was that legalization led to a
better market experience for both buyers and sellers.
Sellers found that they could now call the police when they were robbed or beaten,
and buyers appreciated the regular health checkups imposed on the sellers.

The point is not that we should legalize prostitution --- I'll let
someone else take that one on --- but that markets work best when
there's a legal infrastructure, what we would call institutions.
That's why I referred earlier to better markets:  it's not just the
existence of  markets, but the institutions that support fair and
honest exchange.  In this country, we're the beneficiaries of
centuries of institutional development that have turned out to work
pretty well for most people.  In many other countries, not so much.
China has been doing very well, but it's still possible to be
poisoned by cough syrup or milk.  In Russia, an
entrepreneur stands a fair chance of having his business stolen ---
taken from him lock, stock, and barrel, possibly with the connivance
of the government.  In Mexico, more than in the US, government is
often used to protect the businesses of the rich and powerful.
Carlos Slim is one of the richest men in the world,
and a large part of his wealth is said to have been accumulated
with the help of government policies that
discouraged competitors.

In short, markets are great tools, but they work best when they have
strong legal and political support.  Bruce's second question is
whether markets encourage immoral behavior.  Certainly it's easy to
think of people who have succeeded by lying, cheating, and stealing.
Some got caught, some didn't.  Did markets make them that way?  Let
me quote Warren Buffett:  ``Of the billionaires I have known, money
just brings out the basic traits in them.  If they were jerks before
they had money, they are simply jerks with a billion dollars."

%So to summarize:  markets are tools, and people are people.


\subsection*{Related quotations}

Adam Smith ({\it Wealth of Nations\/}):  ``As every individual,
therefore, endeavours as much as he can both to employ his capital
in the support of domestic industry, and so to direct that industry
that its produce may be of the greatest value; every individual
necessarily labours to render the annual revenue of the society as
great as he can. He generally, indeed, neither intends to promote
the public interest, nor knows how much he is promoting it.
By preferring the support of domestic to that of foreign industry, he
intends only his own security; and by directing that industry in
such a manner as its produce may be of the greatest value, he
intends only his own gain, and he is in this, as in many other
cases, led by an invisible hand to promote an end which was no part
of his intention. Nor is it always the worse for the society that it
was no part of it. By pursuing his own interest he frequently
promotes that of the society more effectually than when he really
intends to promote it. I have never known much good done by those
who affected to trade for the public good. It is an affectation,
indeed, not very common among merchants, and very few words need be
employed in dissuading them from it.''

\begin{comment}
James Madison ({\it Federalist Papers No 51\/}):  ``But what is
government itself, but the greatest of all reflections on human
nature.  If men were angels, no government would be necessary.  If
angels were to govern men, neither external nor internal controls on
government would be necessary.  In forming a government which is to
be administered by men over men, the great difficulty lies in this:
you must first enable government to control the governed; and in the
next place oblige it to control itself."
\end{comment}

Karl Marx and Frederich Engels ({\it Communist Manifesto\/}):  ``The
bourgeoisie has been the first to show what man's activity can bring
about.  It has accomplished wonders far surpassing Egyptian
pyramids, Roman aqueducts, and Gothic cathedrals; it has conducted
expeditions that put in the shade all former exoduses of nations and
crusades.''

Theodore Roosevelt (``The New Nationalism," 1910):  ``Now, this
means that our government, national and State, must be freed from
the sinister influence or control of special interests. ...  The
Constitution guarantees protections to property, and we must make
that promise good.  But it does not give the right of suffrage to
any corporation.  ... The citizens of the United States must
effectively control the mighty commercial forces which they have
themselves called into being.''

Lou Dobbs (online blurb for his book, {\it Exporting America\/}):
``The shipment of American jobs to cheap foreign labor markets threatens not only millions of workers and their families, but also the American way of life.  Corporate raiders are breaking down our borders in search of the lowest-price labor available anywhere in the world.  For the first time in history, corporations are laying off Americans from well-paying jobs and replacing them with low-paid foreign workers. ... Corporate America isn�t doing all this alone:  Big business and Washington are in cahoots, trading our nation's livelihood for short-term gain.''

%Global Trade Watch (web site):
%``The data are in and they clearly show the damage NAFTA has wrought for millions of people in the US, Mexico, and Canada.  As we predicted:  a race-to-the-bottom in wages, destruction of hundreds of thousands of good US jobs, undermining of democratic control of domestic policy-making, and threats to  health, environmental and food safety standards. �  Millions of campesinos throughout Mexico have lost a significant source of income and left their small corn farms.''

Ernesto Zedillo, former president of Mexico:
``Half of the world's population lives on less than \$2 a day.  I say to you, with all conviction, that a vital part of the solution consists of promoting more globalization.  More international trade, more investment flowing across countries, more knowledge diffused internationally among communities and individuals ... will do much to defeat the evil of poverty during this new century.''



Richard Posner (Becker-Posner blog, October 23, 2006, on price
gouging):  ``But here is an interesting wrinkle. Admiralty law and
common law (both are systems of judge-made law, but they are
classified separately by lawyers because they used to be
administered by separate courts) alike forbid certain practices that
might be described as `price gouging.' Suppose a ship is sinking,
and another ship comes along in time to save the cargo and
passengers of the first. The second ship demands, as its price for
saving the cargo and passengers of the first ship, that the owner of
the ship give it the ship and two-thirds of the rescued cargo, and
the captain of the first ship, on behalf of the owner, being
desperate agrees. The contract would not be legally enforceable;
under the admiralty doctrine of `salvage,' the second ship would be
entitled to a `fair' price for rescuing the first, but to no more.
In a parallel case, also maritime but governed by common law
rather than admiralty law (the Alaska Packers case, well known to
law students), seamen on board a ship that was fishing for salmon in
Alaska waters went on strike, demanding higher wages. The captain of
the ship agreed because, the fishing season in these waters being
very short, he could not have hired a replacement crew in time to
make his quota. Again, however, the court refused to enforce the
contract, in essence because it had been obtained under duress.''

Laura Meckler ({\it Wall Street Journal\/}, November 13, 2007):
``Amid a severe kidney-donor shortage, an idea long considered
anathema in the medical community is gaining new currency: payments
for people willing to give up a kidney.  One of the most outspoken
voices on the topic isn't a free-market libertarian, but a prominent
transplant surgeon named Arthur Matas. Dr. Matas, 59 years old, is a
Canadian-born physician best known for his research at the
University of Minnesota. Lately, he's been traveling the country
trying to make the case that barring kidney sales is tantamount to
sentencing some patients to death. `There's one clear argument for
sales,' Dr. Matas told a gathering of surgeons earlier this year.
The practice, currently illegal in the U.S., `would increase the
supply of kidneys, save lives and improve the quality of life for
those with end-stage renal disease.' ...  Among his opponents on the
issue is a friend and colleague, Francis Delmonico. A Harvard
University professor who has played a central role in shaping
national transplant policy, the 62-year-old physician has several
objections to organ sales. He fears such a system would attract the
poor, vulnerable and unhealthy, and that altruistic donations might
wither away.  `Payments eventually result in the exploitation of the
individual,' says Dr. Delmonico, who also worries about encouraging
black-market sales both here and in developing countries. `It's the
poor person who sells.'"

Stanley Zin (NYU Stern professor):  ``Money can't buy happiness, but it's an
experiment most people want to run for themselves.''

%\newpage
\subsection*{Discussion questions}

\setlength{\leftmargini}{.5\leftmargini}
\begin{enumerate}

\item Give examples in which you think markets work well,
and other examples in which they work poorly.
How did you decide whether or not they worked well or not?
What factors led them to work well or poorly?

%\item Do markets encourage lying, cheating, and stealing?
%Would you expect to find more of this in a market-based economy --- or less?

\item Do you think we should have more or less market activity
in the following areas:
%
\begin{itemize}
\item international trade in goods and services?
\item international trade in assets?
\item shortages following natural disasters?
\item addictive drugs?
\item human organs?
\end{itemize}

\item Do markets contribute to prosperity?

\item Are markets moral?  Is that the same thing?

\end{enumerate}


\vfill \centerline{\it \copyright \ \number\year \
NYU Stern School of Business}

\end{document}

\url{http://www.enlightenmenteconomics.com/blog/index.php/2014/03/enoughness/}

I left with another book the two Skidelskys have edited, Are Markets Moral? Glancing at the contents, I think its answer is �no�. But even to pose the question is to make markets overly-abstract. Markets are institutions in which people have social relations. The markets for tea bags, accountancy services, and radio spectrum have entirely different structures and characteristics. �Moral� isn�t a description that can apply to abstract nouns at all. One can sensibly ask if bankers are moral, at risk of generalising, but not markets.
