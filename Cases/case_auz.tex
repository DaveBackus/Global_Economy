\documentclass[letterpaper,12pt]{exam}

%\usepackage[hypertex]{hyperref}
\usepackage{amsmath}
\usepackage{natbib}
%\usepackage[dvips]{graphicx}
\usepackage{comment}
\RequirePackage{GE05}
% this inputs graphicx, too

\newcommand{\GDP}{\mbox{\em GDP\/}}
\newcommand{\NDP}{\mbox{\em NDP\/}}
\newcommand{\GNP}{\mbox{\em GNP\/}}
\newcommand{\NX}{\mbox{\em NX\/}}
\newcommand{\NY}{\mbox{\em NY\/}}
\newcommand{\CA}{\mbox{\em CA\/}}
\newcommand{\NFA}{\mbox{\em NFA\/}}
\newcommand{\Def}{\mbox{\em Def\/}}
\newcommand{\CPI}{\mbox{\em CPI\/}}
\newcommand{\phm}{\phantom{--}}

\def\ClassName{The Global Economy}
\def\Category{Micro-Case}
\def\HeadName{Risk Analysis:  Australia, April 2009}

\begin{document}
\parindent = 0.0in
\parskip = \bigskipamount
\thispagestyle{empty}%
\Head

\centerline{\large \bf \HeadName}%
\centerline{Revised:  \today}

\bigskip
As a European investor in short-term Australian securities, 
you have made a fair amount of money over the last decade
betting that Australia's high interest rates would not 
be offset by declines in the value of its currency.  
You wonder, however, whether it's time to take your money 
and run.  

Having some experience with such situations, 
you check the Economist Intelligence Unit's Country Data, 
summarized in Table \ref{tab:auz}, 
and Country Risk Service, 
which reports:  
%
\begin{itemize}
\item The exchange rate is flexible, and could move either way against
the euro.  
\item Australia's large net foreign liability position reflects 
a combination of direct investment in Australian businesses, 
notably mining, 
and the carry trade, 
in which investors purchase AUD-denominated assets 
in order to benefit from relatively high local interest rates.  
\item The banking system faces higher financing costs due to global 
credit conditions, but is stable.  Government guarantee of deposits
has reduced the risk of failures. 
\end{itemize}

With this information in hand, you how would you assess your risk 
of continuing to invest in short-term Aussie debt?  
Would you continue to invest in Australia or look for 
opportunities elsewhere?  

%
\begin{table}[h]
\vspace{1em}%
\centering%
\hspace{-6cm}%
\begin{minipage}
{0.52\textwidth}%
\begin{center}{\small
\begin{tabular}{lrrrrrrrr}%
\vspace{-0.6cm}\\
\hline%
\vspace{-.3cm}\\
    & 2002 & 2003 & 2004 & 2005 & 2006 &  2007 &  2008 & 2009 \\%
\vspace{-.3cm}\\
\hline%
\vspace{-.2cm}\\
GDP growth (real) & 4.3 & 3.0 & 3.8 & 2.8 & 2.8 & 4.0 & 2.1 
        & --1.6 \\
Inflation 
        & 3.0 & 2.8 & 2.3 & 2.7 & 3.5 & 2.3 & 4.4 & 1.2 \\
Interest rate:  short
        & 4.6 & 4.8 & 5.3 & 5.5 & 5.8 & 6.4 & 6.7 & 2.8 \\
Interest rate:  long 
        & 5.8 & 5.4 & 5.6 & 5.3 & 5.6 & 6.0 & 5.8 & 3.3 \\            
Investment rate    
        & 24.1 & 25.2 & 25.5 & 26.5 & 26.8 & 27.8 & 28.7 &  \\
Saving rate          
        & 20.1 & 20.4 & 20.0 & 21.3 & 21.2 & 21.9 & 24.3 &  \\
Current account 
        & --3.8 & --5.5 & --6.1 & --5.8 & --5.5 
        & --6.4 & --4.2 & --3.5 \\
Govt budget:  total
        & 1.3 & 1.8 & 1.1 & 1.5 & 1.5 & 1.6 & 1.8 & --3.3 \\
Govt budget:  primary 
        & 2.9 & 3.2 & 2.4 & 2.7 & 2.6 & 2.6 & 2.7 & --2.6 \\
Govt debt
        & 20.1 & 18.5 & 17.5 & 17.0 & 16.4 & 15.4 & 13.9  \\
Exchange rate 
        & 1.84 & 1.54 & 1.36 & 1.31 & 1.33 & 1.20 & 1.19 \\        
Real exchange rate 
        & 100 &  113 & 121 & 125 & 125 & 133 & 132 \\
FX reserves (USD) 
        & 21 & 32 & 36 & 42 & 53 & 25 & 31 & \\
FX reserves (months) 
        &  2.9 & 3.7 & 3.3 & 3.4 & 4.0 & 1.6 & 1.7 \\        
Net foreign assets  
        & --68 & --68 & --65 & --71 & --77 & --89 & --103 \\ 
\vspace{-3mm} \\
\hline 
\end{tabular}
}
\end{center}
\end{minipage}
\caption{Economic indicators for Australia.  
Notes:  
(i)~Investment, saving, current account, government budget, 
government debt, and net foreign assets
are expressed as percentages of GDP (ratios multiplied by 100).  
(ii)~Government budget numbers are ``balances'':  positive 
numbers are surpluses, negative numbers are deficits.  
(iii)~The exchange rate is the Aussie dollar (AUD) price 
of one US dollar (USD);
high numbers indicate that foreign currency is expensive.  
The real exchange rate is a weighted average across trading partners.  
The convention is the inverse of the exchange rate:  
high numbers indicate that local goods are expensive relative to foreign 
goods.  
(iv)~Foreign exchange reserves are expressed, first, 
in billions of USD, second, 
as a ratio to monthly imports. 
Thus the number 2.9 means that reserves are 2.9 times one
month's imports.
(v)~2009 numbers are estimates.}
\label{tab:auz} 
\end{table}



\vfill \centerline{\it \copyright \ \number\year \ 
NYU Stern School of Business}


\end{document}

