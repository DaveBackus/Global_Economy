\documentclass[11pt]{article}

\usepackage{FM}

\title{Course Syllabus}

\begin{document}

\makehead

\paragraph*{Professor}
Timothy Van Zandt\kern 1em $\langle$tvanzand@stern.nyu.edu$\rangle$


\paragraph*{Objectives}
This course analyzes economic forces relevant to managerial decision making. Our focus is on the economic fundamentals of businesses and industries: production and costs; demand for the product; pricing strategies; competition between industry participants; and trade with private information. The microeconomic concepts and tools you will learn have many applications beyond those treated explicitly in the  course; you will see such applications, for example, in subsequent marketing, strategy, and finance courses.

\paragraph*{Readings}
The main readings are the \emph{Prices and Markets} ``class guides'', contained in the course pack. These focus on core analytic material and exercises.

The course pack also contains clippings from the press and a some short cases. These readings illustrate and extend ideas discussed in class.

For supplementary reading, students can consult optional textbooks. Those who want to reenforce the material in this class or explore additional topics can consult \emph{Microeconomics, 5th edition} (Prentice Hall, 2001), by  Robert Pindyck and Daniel Rubinfeld. This book is available from the bookstore, along with a useful \emph{Study Guide}.

Students who have little background in microeconomics and would like a more elementary introduction to the topics can consult \emph{Principles of Microeconomics, 2nd edition} (Harcourt, 2000), by Gregory Mankiew.

\paragraph*{Preparing for class}

You should read the session's class guide(s) and attempt the exercises  beforehand. We can then use class time to review key ideas, go more deeply into those areas you find the most difficult, and use the exercises as a basis for class discussion.

You are not asked to hand in written solutions to the exercises in the class guides. I hand out complete solutions to the exercises for self-grading. Their central nature to the course---both as a vehicle for learning the material and as a basis for class discussion---provide sufficient incentive for you to take them seriously.

A few graded homework assignments will be given out during the course.

\paragraph*{Course grading}

The course grade is based on graded homework (10\%), a 1-hour midterm exam (25\%), and a 2.5-hour final exam (65\%). Each exam is closed book, but you can bring to the exam a single letter-size sheet with anything you want written on both sides. The material covered on the exams consists of the class guides, solutions to exercises, slides, and other in-class material and discussions.

While a textbook may help you understand this material, only a few textbook readings are required for the exam. Copies of these readings will be distributed to those who have opted not to purchase the textbook.

You are expected to understand the message of the articles and cases but not memorize them.

Though class participation is not graded (to keep it natural and spontaneous), it is an important component of this class. I frequently cold call in order to give everyone a chance to participate. Attendance is required and excessive unexcused absences will result in a failing grade.


\end{document}
