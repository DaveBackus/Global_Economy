\chapter{Government Debt and Deficits}\label{chp:dbdf}
\hypertarget{deficits}{}

\textbf{Tools:} Government budget constraint; debt dynamics.

\textbf{Key Words:} Government budget; government debt and deficits; primary deficit/surplus;
the debt-to-GDP ratio; hidden liabilities.

\textbf{Big Ideas:}
\vspace{-0.1in}
\begin{itemize}
\item Government spending must be paid for, either now through taxes, or in the
future.
\item Current debt must be balanced by future primary surpluses.
\item Changes in the ratio of debt to GDP have three sources:
interest, growth, and primary deficits.
\end{itemize}
\rule{\textwidth}{1pt}

The fundamental principle of government finance is that
governments must finance spending with taxes.
Issuing debt postpones this obligation but does not eliminate it.
If a government doesn't collect enough tax revenue now,
it must collect it later --- or face default.
Since investors like to be repaid, they pay close attention
to government debt and deficits.
If they're too high, investors may demand higher yields
or even stop buying government securities altogether.
The consequences of this sequence of events are never pretty.

Debt finance poses a \hyperref[sec:time_cons]{\textit{time-consistency} problem} because
(i) governments have an incentive to issue debt and tax future
generations who do not vote today;
and (ii) future governments may have an incentive to inflate away
or repudiate past debt.
Countries with good governance solve this problem either through
sound budget policy or institutional design.
Otherwise, not so much.


\section{Government revenues, expenses, and debt}

We start with a quick overview of
government spending and revenue decisions ---
what is conventionally called {\it fiscal policy\/}.

Countries differ in the size of government relative to the economy,
in the sources of tax revenues,
and in their expenditures.
Governments everywhere
purchase goods and services (schools, police, courts, roads, military),
transfer money to individuals (social insurance, health care),
and collect revenue (largely through taxes).
The distinction between purchases and transfers is important.
Only purchases show up in the expenditure identity.
Transfers are, nevertheless, a large part of total expenses in many economies,
particularly developed economies.
Governments also pay interest on outstanding government debt,
an expense we track separately.

We'll look at data for each of these in class.
As a rule, government spending and revenue are a larger fraction of GDP
in rich countries than in poor ones.
Rich countries also spend more on transfers.
There is, however, a lot of variation at all levels of development.

We put these elements together in a relation we'll call
the {\it government budget constraint\/}.
On the expense side, we label government purchases of goods and services $G$,
transfers $V$, and interest payments $iB$
(the product of the government debt $B$ and whatever interest rate $i$
the government pays on it).
On the revenue side, we label tax revenue $T$.
(Note: $T$ is tax revenue, not the tax rate.)
By convention, all of these things are nominal:
they're measured in local currency units.
The government budget constraint is, then,
\begin{eqnarray}
    G_t + V_t + i_t B_{t-1} - T_t  &=& B_{t} - B_{t-1} .
    \label{eq:gbc}
\end{eqnarray}
Here, $B_{t-1}$ is the amount of debt outstanding at the end of
period $t-1$.
The left-hand side of (\ref{eq:gbc})
is the government deficit,
the right the change in the quantity of debt.
The equation says, in essence, that any surplus or deficit is matched
by a change in the quantity of debt.
A government deficit, for example, is financed by issuing more debt.


The elements of equation (\ref{eq:gbc})
are often used to generate summary measures
of fiscal policy.
The most common are ratios  of
the government deficit and government debt to GDP.
We'll look at both, as well as the connection between them.


%\end{document}
\section{Debt and (primary) deficits}


Governments need to finance their spending with taxes.
It's not quite true --- governments have other sources of revenue --- but it's close enough to be worth remembering.
Issuing debt allows a government to postpone taxes,
just as a credit card allows an individual to postpone paying
for purchases,
but does not eliminate the obligation.
Delay, in fact, comes with a cost:  We need to pay the original
obligation, plus interest.
In the rest of this section, we make the same point more formally.

We're going to take our budget constraint, equation (\ref{eq:gbc}),
and use it to relate debt to past and future deficits.
To make things a little simpler,
define the {\it primary deficit\/} $D$ as the deficit
net of interest payments (sort of an ``EBITDA'' number):
\begin{eqnarray*}
    D_t &=&  G_t + V_t - T_t .
\end{eqnarray*}
(This is sometimes reported with the opposite
sign and called the primary budget balance or surplus.)
With this simplification, (\ref{eq:gbc}) can be expressed as
\begin{eqnarray}
    B_{t}   &=& (1+i) B_{t-1} + D_t
   \label{eq:gbc-bdynamics1}
\end{eqnarray}
or
\begin{eqnarray}
    B_{t-1}  &=&  B_{t}/(1+i) - D_t/(1+i) .
    \label{eq:gbc-bdynamics2}
\end{eqnarray}
They're the same equation, but the first one looks backward from $t$ to $t-1$,
and the second looks forward from $t-1$ to $t$.
We'll put both to work.
The $i$ in these equations is the \textit{nominal interest rate} that the government pays on its debt.
We'll assume it is constant for now --- it makes the math simpler ---
but allow it to change in the next section.

Equation (\ref{eq:gbc-bdynamics1}) tells us where the debt came from.
If we substitute over and over again, back to some period $t-n$,
we have
\begin{eqnarray*}
    B_{t}  &=& D_t + (1+i) B_{t-1} \\
            &=& D_t + (1+i) [(1+i) B_{t-2} + D_{t-1}] \\
            &=& D_t + (1+i) D_{t-1} + (1+i)^2 D_{t-2} + \cdots + (1+i)^n B_{t-n}.
\end{eqnarray*}
In words:  The current debt is the debt we started with $n$ periods ago
plus the current value of past deficits plus accumulated interest.
It's like your credit card bill: Your current balance consists of past shortfalls plus accumulated interest.


Equation (\ref{eq:gbc-bdynamics2}) tells us what we need to do in the future
to service the current debt.
If we substitute repeatedly, we find
\begin{eqnarray*}
    B_{t-1}   &=&  B_{t+1}/(1+i)^2 - [D_t/(1+i)+ D_{t+1}/(1+i)^2] \\
            &=&  B_{t+2}/(1+i)^3 -
                [D_t/(1+i)+ D_{t+1}/(1+i)^2 + D_{t+2}/(1+i)^3] \\
            &=&  B_{t+n-1}/(1+i)^n -
                [D_t/(1+i)+ \cdots + D_{t+n-1}/(1+i)^n] .
%    \label{eq:gbc-bdynamics-rec}
\end{eqnarray*}
If we assume that debt can't grow faster than the interest rate forever,
then, as we continue to substitute, the first term goes to zero.
[The technical condition is $B_{t+n}/(1+i)^n$ goes to zero as $n$
approaches infinity.
It amounts to not allowing the government to run a Ponzi scheme,
paying off old debt by issuing new debt, forever.]
The relation then becomes
\begin{eqnarray}
    B_{t-1}   &=&  -[D_t/(1+i)+ D_{t+1}/(1+i)^2 +
                   D_{t+2}/(1+i)^3] +   \cdots ]  \nonumber \\
            &=& - \mbox{Present Discounted Value of Primary Deficits}
                \phantom{\sum} \nonumber \\
            &=& \mbox{Present Discounted Value of Primary Surpluses} .
            \label{eq:debt-future-surpluses}
\end{eqnarray}
In words:   The current government debt must be matched
by the present discounted value of future primary surpluses.
As we said at the start, all spending must be financed by tax revenue --- eventually.
It's not enough to shrink the deficit.
Eventually we have to run surpluses, measured net of interest payments.

Analysis of this sort often uses the term {\it sustainable\/}.
We say the debt is sustainable if current debt is balanced
by plans for future surpluses, as in equation (\ref{eq:debt-future-surpluses}).
If not, we say the government's budget is {\it unsustainable\/}.
In this case, we can paraphrase the economist Herbert Stein:
``Something must change, so it will.''


\begin{comment}
Although (primary) government deficits must eventually be reversed,
they can affect the economy while they last.
One route is distribution.
If taxes are delayed long enough, the tax burden
will be shifted from current to future generations.
Thus the US contribution to World War II was financed largely with debt,
shifting some of the burden from those alive at the time to those born later.

Is deficit financing a good or bad thing?
One approach to this question is based on tax smoothing.
As we mentioned in ``Notes on taxes,''
the economic disincentives built into taxes are minimized
by having relatively constant tax rates even if tax revenues
vary over time.
On average this is likely to lead to governments running (modest)
surpluses in booms and deficits in recessions,
as tax revenues go up and down with the economy.
\end{comment}


%\end{document}
\section{Debt dynamics}

Investors watch government debt and deficits
for signs that a government may not honor its debts.
Even a hint of this can change the rate at which the government
borrows or even its ability to access capital markets.
In practice, it's common to look at them as ratios to GDP.
In such ratios, we measure both in local currency units,
so we have (for example) the ratio of the nominal debt to nominal GDP.

So how does the debt-to-GDP ratio change from one period to the next?
There's a useful decomposition of changes
into components due to the real interest rate, GDP growth, and the
primary deficit.
It's a little complicated, so we'll work our way up to it.
Recall that debt evolves according to
\begin{eqnarray}
    B_{t} &=& D_t + i_t B_{t-1} + B_{t-1}  ,
    \label{eq:gbc-bdynamics1a}
\end{eqnarray}
where $D$ is the primary deficit
and $D_t + i_t B_{t-1}$ is the total deficit.
You should recognize this as equation (\ref{eq:gbc-bdynamics1})
in slightly different form.
Here's how it looks with real numbers.

{\bf Example (US, 2012).}
Consider the numbers for US government debt and deficits,
expressed in trillions of US dollars:

\begin{center}
\begin{tabular}{lcr}
\toprule
%
%\midrule
Government debt, year end 2011  &  $B_{t-1}$    & 12.103 \\
Total deficit, 2012             &  $ D_t + i_t B_{t-1}$ & 1.010 \\
Primary deficit, 2012           &  $D_t$        &  0.674 \\
\bottomrule
\end{tabular}
\end{center}

The data are October 2012 estimates from the IMF's
\href{http://www.imf.org/external/ns/cs.aspx?id=28}{World Economic Outlook database}.

Use what we know about debt dynamics to compute debt $B_t$ at the end of 2012.
What is the interest rate $i_t$ paid on the debt?

Answer.
Debt at the end of 2012 follows from equation (\ref{eq:gbc-bdynamics1a}):
$B_t = 1.010 + 12.103 = 13.113$.
Interest payments are the difference between the two deficit numbers:
$ i_t B_{t-1} = D_t + i_t B_{t-1} - D_t = 1.010 - 0.674 = 0.336 $.
The implied interest rate on the debt is
the ratio of interest payments to debt:
$ i_t B_{t-1}/B_{t-1} = 0.336/12.103 = 0.0278 = 2.78\%$.


Now back to the dynamics of the debt-to-GDP ratio.
The bottom line is the relation
\begin{eqnarray}
    \frac{B_{t}}{Y_{t}}
            &\approx&
                \frac{B_{t-1}}{Y_{t-1}} + (i_t-\pi_t) \frac{B_{t-1}}{Y_{t-1}}
                - g_t \frac{B_{t-1}}{Y_{t-1}}
             +    \frac{D_{t}}{Y_{t}}  .
    \label{eq:debtdynamics}
\end{eqnarray}
Circle this equation, it's important.
It gives us three sources
of change in the ratio of debt to GDP.
The first is the real interest on the debt, which accumulates
as long as the debt and real interest rate are positive.
The second is the growth of the economy,
which reduces the ratio by increasing the denominator.
The third is the primary deficit.
Each makes a contribution to changes in the debt-to-GDP ratio.


It's not required, but if you're interested in where this comes from,
here are the details.
We divide equation (\ref{eq:gbc-bdynamics1a}) by nominal GDP $Y_t$ to get
\begin{eqnarray*}
    \frac{B_{t}}{Y_{t}} &=& %\frac{D_{t}}{Y_{t}} + \left( 1+i_t \right)  \frac{B_{t-1}}{Y_{t}}
%        \;\;=\;\;
            \frac{D_{t}}{Y_{t}}
            +  (1+i_t) \frac{B_{t-1}}{Y_{t}}  .
%           \label{eq:debtdynamics1}
\end{eqnarray*}
In the last term, note that the denominator is $Y_t$, not $Y_{t-1}$.
If the growth rate of real GDP is $g_t$ and the inflation rate is $\pi_t$,
then the growth rate of nominal GDP is approximately $g_t + \pi_t$.
Therefore,
\[
    Y_{t} \;\;\approx\;\; (1+g_t+\pi_t) Y_{t-1}.
\]
The ratio of debt to GDP then follows
\begin{eqnarray*}
    \frac{B_{t}}{Y_{t}}
            &\approx&
                \left( \frac{1+i_t}{1+g_t+\pi_t} \right)  \frac{B_{t-1}}{Y_{t-1}}
             +    \frac{D_{t}}{Y_{t}}  \nonumber \\
            &\approx&
                \left[ 1 + i_t - (g_t+\pi_t) \right]  \frac{B_{t-1}}{Y_{t-1}}
             +    \frac{D_{t}}{Y_{t}}   \nonumber \\
            &\approx&
                \frac{B_{t-1}}{Y_{t-1}} + (i_t-\pi_t) \frac{B_{t-1}}{Y_{t-1}}
                - g_t \frac{B_{t-1}}{Y_{t-1}}
             +    \frac{D_{t}}{Y_{t}}  .
%    \label{eq:debtdynamics}
\end{eqnarray*}
The second equation is based on the approximation
\begin{eqnarray*}
    \frac{1+i_t}{1+g_t+\pi_t} &\approx& 1 + i_t - g_t - \pi_t ,
\end{eqnarray*}
good for small values of $i_t$, $g_t$, and $\pi_t$.
All you need to know is that the right side is simpler than the left.
The third rearranges terms, combining $i_t-\pi_t$ into a real
interest-rate term.
We're left with (\ref{eq:debtdynamics}), as promised.


This is a mechanical analysis, but a useful one.
By looking at the components of equation (\ref{eq:debtdynamics}),
we can get a sense of the origins of past changes in the debt-to-GDP ratio
and the potential sources of future changes.


{\bf Example (US, 2012, continued).}
The numbers we saw earlier look this way expressed as ratios to GDP:

\begin{center}
\begin{tabular}{lcr}
\toprule
%
%\midrule
Government debt, year end 2011  &  $B_{t-1}/Y_{t-1}$ & 0.8028 \\
Primary deficit, 2012           &  $D_t/Y_t$         & 0.0431 \\
Interest rate                   &  $i_t$             & 0.0278 \\
Real GDP growth rate            &  $g_t$             & 0.0217 \\
Inflation rate                  &  $\pi_t$           & 0.0213 \\
\bottomrule
\end{tabular}
\end{center}

What is the ratio of debt to GDP at the end of 2012?

Answer.  We apply the formula, equation (\ref{eq:debtdynamics}):
\begin{eqnarray*}
    {B_{t}}/{Y_{t}} &=&  0.8028  + (0.0278-0.0213)*0.8028 - 0.0217*0.8028 + 0.0431 \\
            &=&  0.8028 + 0.0052 - 0.0174 + 0.0431 \;\;=\;\; 0.8337.
\end{eqnarray*}
In words:  the ratio of debt to GDP rose from 80.3 percent to 83.4 percent,
an increase of 3.1 percent of GDP.
The primary deficit contributes 4.3 percent of the change (more than the total),
and growth drives it the other way by 1.7 percent.

\textbf{Example (Peru, 2003-2007).}
Between 2003 and 2007,
Peru's debt fell from 47 percent of GDP to 25 percent.
Using the numbers in the table below, what happened?
What was the primary source of this decline?

\begin{center}
%\begin{table}
\begin{tabular*}{\textwidth}{lcccc}
\toprule
        & Debt   & Interest &  Growth
            & \phantom{x}Deficit\phantom{x} \\
        &  $B_{t}/Y_{t}$ &  $(i_t-\pi_t)B_{t-1}/Y_{t-1}$  &  $-g_t B_{t-1}/Y_{t-1}$
                & $D_t/Y_t$ \\
\midrule
2003\phantom{xxxx}
            & 47.1 &  \\
2004        & 44.3 & 0.2 & $-2.4$ & $-0.6$  \\
2005        & 37.7 & 1.1 & $-3.0$ & $-4.6$  \\
2006        & 33.1 & 1.0 & $-2.9$ & $-2.7$  \\
2007        & 30.9 & 1.1 & $-2.9$ & $-0.4$  \\
2008        & 25.0 & $-0.3\phantom{-}$ & $-3.0$ & $-2.5$ \\
\midrule
Sum         &      & 3.1 & $-14.3$ & $-10.9$ \\
\bottomrule
\end{tabular*}
\end{center}
% Data:  WEO, Sep 2011.  Numbers don't add right, so I made the primary deficit
% the residual -- ie, it's dummied up
%\end{comment}

Answer.  You can see from the final row that
growth and a primary deficit both played large roles.
Growth probably gets some credit even for the latter term, too,
since it tends to increase government revenue.



\section{What's missing?}

Our summary of debt dynamics,
captured in equation (\ref{eq:debtdynamics}), buries some issues beneath the mathematics.

One issue is the link between the interest rate
and the fiscal situation (the debt and deficit).
If investors start to worry about a government's
willingness to honor its debt,
they may demand higher interest rates,
which, in turn, raises future debt --- and so on.
%It's not a cycle you want to get caught in.
Such credit spreads can rise sharply
if the government has not shown sufficient
fiscal discipline.
(Here, ``sufficient'' means whatever is needed to reassure
investors.)
It can also rise because global financial markets
place a higher premium on risk,
as they did during the 2008-09 financial crisis.
Over this period, spreads on emerging market debt of all kinds
widened, even in countries with fundamentally sound
fiscal positions.
%For that reason, it's often useful to consider scenarios
%in which the interest rate on government debt rises sharply.

Another issue is the link between growth and deficits.
If growth rises, as in Peru (example, above),
that reduces the debt-to-GDP ratio through
the impact on $g$ in equation (\ref{eq:debtdynamics}).
But it also generates higher tax revenues and, hence,
a lower primary deficit, even if tax {\it rates\/}
do not change.
This, in turn, reduces future debt further.
That's one reason that Peru's fiscal situation improved:
The economy boomed.
Growth, then, is the cure for many problems.

A third issue is hidden government liabilities.
The idea behind our analysis
is that the primary deficit determines how the debt evolves.
In fact, current decisions often involve commitments
for future expenditures that don't show up in the
current government budget but are nevertheless important.
In principle these {\it hidden liabilities\/}
should show up in the budget when they're incurred,
but in practice they don't.
Here are some common examples:
%
\begin{itemize}
\item Social security and pensions.
Many countries have implicit commitments to pay
money to retired people in the future
that are not accounted for properly.
In the US, for example, we simply look at the current cash flow of
social security receipts and payments.
In principle, there should be an entry for unfunded pension
liabilities,
as there is for firms.
In many countries, aging populations have made these looming payments
a serious concern.
Healthcare payments are similar.

\item Financial bailouts.
We tend to treat these as one-offs, but, in fact, they happen all the time and they're invariably expensive.
A country that bails out its banks may find its debt rise sharply.

\item Regional governments.
Relations between central governments and local authorities
differ widely around the world.
In the US, the precedent was set in the 1840s for state and local
governments to finance their own activities without help from
the central government.
In other countries, debt problems of regional governments are often
passed to the central government.
These implicit liabilities of the central government were a concern in
Argentina and Brazil in the 1990s and in Spain right now.
\end{itemize}

A serious analysis of fiscal policy should, therefore,
start with equation (\ref{eq:debtdynamics})
but go on to consider all possible sources of change in its components.
It's no different from financial accounting for firms:
you need to know what lies behind the numbers.


\section{How much debt is too much?}

How much debt is too much?
At what point should investors be concerned?
There is, unfortunately, no clear answer.
Or rather there is an answer, which is that
the quality of governance is more important than the debt numbers.
Argentina defaulted with debt of about 40 percent of GDP,
but the UK had debt well over 100 percent of GDP after World War II
and didn't generate undue concern.
In many cases you're stuck trying to guess how the local politics will play out.
%There's no shortcut for that.

With that warning, here are some rules of thumb:
\begin{itemize}
\item Worry if the government debt is above 50 percent of GDP.
This is very rough, but it's a start.

\item Worry if the deficit is above 5 percent of GDP ---
and is expected to stay that way.
The issue is not so much any particular deficit,
but the long-term posture.

\item Use higher numbers for sovereigns with strong institutions,
lower numbers for those with weak institutions. The institutions
that matter most are those that help contain the \hyperref[sec:time_cons]{time-consistency problem}:
namely, the risk that a future policymaker will repudiate or
inflate away the debt. Relevant institutions would include an
independent central bank (to limit inflation), a robust financial system
(that promotes market discipline and limits bailouts), and fiscal arrangements
that limit debt accumulation or prevent its repudiation (e.g., transparent
and comprehensive long-term budgeting, pay-go or balanced budget
rules, and legal requirements for prioritizing debt payments).

\item Worry if the debt is primarily short-term or denominated in
foreign currency.
Even if the debt is stable, countries
may find themselves in difficulty if they have to refinance
a large fraction of their debt over a short period of time.
(Companies are no different:  think Lehman Brothers, or Drexel before that.)
In late 1994, for example, Mexico had much of its debt in short-term
securities.
When investors refused to buy new issues, it triggered a crisis.
This despite relatively modest debt and deficits.

Foreign debt has similar risks.
Many developing countries issue debt denominated in hard currency
(dollars, say, or euros), but it's a mixed blessing.
Investors use it to avoid the risk of a currency collapse,
but if the currency collapses,
a country's debt can rise sharply,
perhaps increasing the odds of default.
Depending on how it plays out, the Euro Area could be an example,
if the debt of an exiting country remains denominated in euros.
%For example, if a country exits the Euro Area, its existing debt will become foreign currency debt.
\end{itemize}



\section*{Executive summary}

\setlength{\leftmargini}{.5\oldleftmargini}
\begin{enumerate}

\item Countries differ enormously in the magnitude and composition
of government spending, taxes, and debt.

\item Government spending must be paid for, either now through
taxes or in the future by running primary surpluses.

\item The following factors govern changes in the debt-to-GDP ratio:
(a)~interest on the debt;
(b)~GDP growth;
and (c)~the primary deficit.

\item Institutions that limit the incentive of future governments to inflate
away or repudiate the debt can help to promote fiscal discipline.

\end{enumerate}
\setlength{\leftmargini}{\oldleftmargini}

%\begin{comment}
\section*{Review questions}

\setlength{\leftmargini}{.5\oldleftmargini}
\begin{enumerate}
\item Deficits down under.
Consider these data for Australia:
%
\begin{center}
\begin{tabular}{lrr}
\toprule
        & 2010 & 2011 \\
\midrule
Real GDP growth (annual percent) & 2.6 & 2.9 \\
Inflation  (annual percent)      & 5.2 & 4.8 \\
Interest rate  (annual percent)  & 5.3 & 5.9 \\
Government deficit (primary, percent of GDP)    & 2.9 & 0.7 \\
Government deficit (total, percent of GDP)      & 4.1 & 2.2 \\
Government debt (end of period, percent of GDP) & 25.3 \\
\bottomrule
\end{tabular}
\end{center}
\begin{enumerate}
\item Why are the primary and total government deficits different?
%\item What is the implied interest rate on government debt in 2011?
\item What is the government debt ratio at the end of 2011?
\end{enumerate}

Answer.
\begin{enumerate}
\item The difference is interest payments on government debt.
Apparently in 2011 they amounted to 1.5 percent of GDP.
%\item Evidently 5.9 percent = 1.5/25.3.
\item We use the debt dynamics equation:
\begin{eqnarray*}
    \frac{B_{t}}{Y_t}
            &=&
                \frac{B_{t-1}}{Y_{t-1}} + (i_t-\pi_t) \frac{B_{t-1}}{Y_{t-1}}
                - g_t \frac{B_{t-1}}{Y_{t-1}}
                +  \frac{D_{t}}{Y_{t}}  \\
             &=& 25.3 + (0.059-0.048)*25.3 - 0.029*25.3 + 0.7 \\
%             &=& 25.3 + 0.3 [=(0.059-0.048)*25.3] - 0.7 [=0.029*25.3] + 0.7 \\
             &=&  25.6 .
\end{eqnarray*}
We used a shortcut here on the interest rate, taking the number
from the table (a typical market rate) rather than computing
the interest rate paid on government debt.
\end{enumerate}


\item How the US financed World War II.
The short answer is that they issued debt,
but how did they pay off the debt?
Between 1945 and 1974, the ratio of debt to GDP fell from 66 percent to 11 percent.
What led to the change?
George Hall and Thomas Sargent (source below) computed the following:

\begin{center}
%\begin{table}
\begin{tabular*}{0.9\textwidth}{l@{\extracolsep{\fill}}ccc}
\toprule
       & Interest &  Growth & Primary Deficit \\
       &  $(i_t-\pi_t)B_{t-1}/Y_{t-1}$  &  $-g_t B_{t-1}/Y_{t-1}$
                & $D_t/Y_t$ \\
\midrule
1945-1974  & --12.5  & --21.6  &  --20.8 \\
\bottomrule
\end{tabular*}
\end{center}
All numbers are percentages.

Answer.  If you look at the numbers, they tell you that
growth (the same debt looked smaller when GDP grew)
and primary surpluses account for most of this.
You also see a negative contribution from real interest
payments.
What does this tell us?
With hindsight,
we would say that investors lost money in real terms
because inflation was higher than they expected when they
purchased government debt.
If the government had paid (say) a one percent real return,
this contribution would have been positive
and the 1974 debt level would have been higher.
In that sense, the US used inflation to reduce the debt burden.


\end{enumerate}
\setlength{\leftmargini}{\oldleftmargini}

\section*{If you're looking for more}

It's too technical for this course,
but some of the material on debt dynamics was adapted from
Craig Burnside, ed., {\it Fiscal Sustainability in Theory and Practice\/},
World Bank, 2005.

More user-friendly (and very good)
is George Hall's summary of his work with Thomas Sargent,
``\href{http://www.brandeis.edu/global/rosenberg/briefs/hall_brief.html}
{How will we pay down the debt?}''
They describe sources of changes in the US debt position
from World War II to 2008.

*** [Fine point in case you try this yourself:
most debt numbers are inconsistent with reported deficits.
Here we constructed deficits from changes in debt.
The differences are generally small, but it's embarrassing
that even semi-official numbers are not reported on a consistent basis.]


If you're looking for international data on government debt over 200 years or more,
you might want to look at the IMF's
\href{http://www.imf.org/external/pubs/cat/longres.cfm?sk=24332.0}{historical database}.
Search:  ``imf debt database.''

\needspace{20\baselineskip}
\section*{Symbols and data used in this chapter}

\needspace{10\baselineskip}
\begin{table}[H]
\centering
\caption{Symbol table.}
\begin{tabular*}{0.9\textwidth}{l@{\extracolsep{\fill}}l}
\toprule
Symbol & Definition\\
\midrule
$G$    &Government purchases of goods and services\\
$V$    &Transfers\\
$i$    &Nominal interest rate\\
$B$    &Stock of government debt\\
$T$    &Tax revenues (not a tax {\it rate\/})\\
$D$    &Primary deficit ($=G+V-T$)\\
%$DS$    &Debt service $(i B/Y)$\\
$g$    &Discretely-compounded growth rate of real GDP\\
$\pi$     &Inflation rate \\
$g+\pi$ &Discretely-compounded growth rate of nominal GDP\\
$Y$    &Nominal GDP\\
\bottomrule
\addlinespace
\end{tabular*}
\begin{minipage}{0.9\textwidth}
\footnotesize{Note:  In this chapter we have dealt only with \textit{nominal} variables.}
\end{minipage}
\end{table}

\needspace{8\baselineskip}
\begin{table}[htb]
\centering
\caption{Data table.}
\begin{tabular*}{0.9\textwidth}{l@{\extracolsep{\fill}}l}
\toprule
Variable & Source\\
\midrule
Federal government debt        &GFDEBTN\\
Federal government debt  held by the public    &FYGFDPUN\\
Federal government net surplus    &FGDEF\\
Federal interest outlays    &FYOINT\\
Nominal GDP    &GDP\\
Real GDP    &GDPC1\\
GDP deflator    &GDPDEF\\
\bottomrule
\addlinespace
\end{tabular*}
\begin{minipage}{0.9\textwidth}
\footnotesize{To retrieve the data online, add the identifier from the source column to \url{http://research.stlouisfed.org/fred2/series/}.  For example, to retrieve real GDP, point your browser to \url{http://research.stlouisfed.org/fred2/series/GDPC1}}
\end{minipage}
\end{table}




