\part{Short-Term Economic Performance}


\chapter*{Short-Term Overview}
\hypertarget{srp}{}

\rule{\textwidth}{1pt}

This outline covers key concepts from the second part of the course:
short-term economic performance.
It is not exhaustive, but is meant to help you
(i)~anticipate what is coming and
(ii)~organize your thoughts later on.


\medskip
\textbf{\hyperref[chp:bcpr]{\underline{Business Cycle Properties}}} \textbf{and} \textbf{\hyperref[chp:bcin]{\underline{Business-Cycle Indicators}}}

\textbf{Tools:} Basic statistics (standard deviation, correlation); cross-correlation function.

\textbf{Key Words:} Volatility; procyclical and countercyclical;  leading, lagging, and coincident.

\textbf{Big Ideas:}
\vspace{-0.1in}
\begin{itemize}
\item Economies do not grow smoothly or regularly. 
We refer to fluctuations in economic activity as business cycles.
\item Components of GDP vary in systematic ways over the business cycle: spending on investment and consumer durables is more volatile than output; spending on services and nondurable goods is less volatile than output. Labor and capital markets move with the cycle as well.
\item Business cycle indicators are characterized by several properties: leads, lags, procyclical, countercyclical. Cross-correlation functions identify these properties.
\end{itemize}


\hyperref[chp:mpin]{\textbf{\underline{Money and Inflation}}}

\textbf{Tools:} The quantity theory.

\textbf{Key Words:} Quantity theory of money; velocity; hyperinflation.

\needspace{4\baselineskip}
\textbf{Big Ideas:}
\vspace{-0.1in}
\begin{itemize}
\item The quantity theory links the money supply with the price level and output. In the long run, an increase in the money supply results in an increase in the price level (inflation).
\item Hyperinflation refers to an inflation rate of 100\% per year or more.
Hyperinflations follow a standard pattern: government deficits are financed with money,
which produces inflation.
\end{itemize}


\hyperref[chp:asad]{\textbf{\underline{Aggregate Supply and Demand}}} \textbf{and} \hyperref[chp:pasad]{\textbf{\underline{Policy in the AS/AD model}}}

\textbf{Tools:} Graphical analysis of aggregate supply and demand (AS/AD) model.

\textbf{Key Words:} Short- and long-run aggregate supply; sticky wages; supply/demand shocks; policy objectives and tools; long-run equilibrium or potential output; output gap.

\textbf{Big Ideas:}
\vspace{-0.1in}
\begin{itemize}
\item The AS/AD model relates output and prices in the short and long runs.
The model is composed of (i) an upward-sloping short-run aggregate supply curve,
which inherits its shape from sticky wages;
(ii) a vertical long-run aggregate supply curve;
and (iii) a downward-sloping aggregate demand curve.
\item Monetary policy should respond
differently to demand and supply shocks.
As a general rule, policy should resist/offset changes in output triggered by shifts in demand
and accommodate/reinforce changes triggered by shifts in supply.
\end{itemize}


\hyperref[chp:mpir]{\textbf{\underline{Money and Interest Rates}}}

\textbf{Tools:} Open market operations; central bank balance sheet; Taylor rule.

\textbf{Key Words:} Real interest rate; nominal interest rate; expected inflation; real money balances; inflation targeting; interest-rate rules; rules vs. discretion; zero lower bound; quantitative easing; credit easing; policy duration commitment.

\textbf{Big Ideas:}
\vspace{-0.1in}
\begin{itemize}
\item In conventional practice, central banks use open market operations
to manage interest rates.
These policy actions are equivalent to managing the money supply directly.
\item The Taylor rule provides a guide to how central banks manage their target interest rates
in response to data on inflation and (real) GDP growth.
\item When interest rates are at or near zero, a central bank can resort to unconventional monetary policy,
 including quantitative easing, credit easing, and policy-duration commitments.
 These policies have been in widespread use since 2008.
\end{itemize}
%\needspace{4\baselineskip}
