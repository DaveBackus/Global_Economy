\documentclass[12pt]{article}

\usepackage{../../LaTeX/ge14}
% spacing on page
\oddsidemargin=0.25truein \evensidemargin=0.25truein
\topmargin=-0.5truein \textwidth=6.0truein \textheight=8.75truein

\usepackage{comment}
\usepackage{graphicx}
\usepackage{amssymb}
\usepackage{amsmath}

% layout of figures and tables
\usepackage[margin=0pt, labelsep=period, labelfont=bf]{caption}
%\usepackage{float}

\usepackage{hyperref}
\urlstyle{rm}   % change fonts for url's (from Chad Jones)
\hypersetup{
    colorlinks=true,        % kills boxes
%    allcolors=blue,
    pdfsubject={ECON-UB233, Macroeconomic foundations for asset pricing},
    pdfauthor={Dave Backus @ NYU},
    pdfstartview={FitH},
    pdfpagemode={UseNone},
%    pdfnewwindow=true,      % links in new window
    linkcolor=blue,         % color of internal links
%    citecolor=blue,         % color of links to bibliography
    filecolor=blue,         % color of file links
    urlcolor=blue           % color of external links
% see:  http://www.tug.org/applications/hyperref/manual.html
}

% for listing code in tt font
\usepackage{verbatim}

% for table spacing
\usepackage{booktabs}

% section headers and spacing
\usepackage[tiny, compact]{titlesec}

% list spacing
\usepackage{enumitem}
\setitemize{leftmargin=*, topsep=0pt}
\setenumerate{leftmargin=*, topsep=0pt}

% attach files to the pdf
\usepackage{attachfile}
    \attachfilesetup{color=0.75 0 0.75}

\usepackage{needspace}
% \needspace{4\baselineskip} makes sure we have four lines available before a pagebreak

%\newcommand{\phm}{\phantom{--}}
\newcommand{\NX}{\mbox{\it NX\/}}
\newcommand{\POP}{\mbox{\it POP\/}}
\renewcommand{\log}{\ln}

\renewcommand{\thefootnote}{\fnsymbol{footnote}}

% for gexx.sty
\def\ClassName{The Global Economy}
\def\Category{Professor David Backus}
\def\HeadName{The Global Economy:  Syllabus}

\begin{document}
\thispagestyle{empty}
\parskip=0.65\bigskipamount
\Head
\centerline{\large\bf The Global Economy:  Syllabus}
\vspace{1mm}
\centerline{\large\bf COR1-GB 2303.00/12 $|$ Fall 2014}
\vspace{1mm}
\centerline{Revised:  \today}

\section{About the course}

This course is about the performance of countries and
the businesses that operate in them.
We will use the tools of macroeconomics, and macroeconomic data,
to assess countries' economic and business conditions, both now and in the future.
Topics include:  long-term economic performance (why is Germany more prosperous than Greece? what are the challenges of doing business in China and India?), short-term fluctuations (where is the US economy headed in the next twelve months?), and macroeconomic crises (what's going on in Europe?).
By the end of the course, you will be able to:
%
\begin{itemize}
\item Explain how differences in local conditions
and institutions
affect the nature and cost of running a business.

\item Describe the sources of good long-term performance: why per
capita income is higher in the US and France than in China and India,
and why China and India are among the fastest growing countries in the
world.

\item Evaluate indicators of good short-term performance
and their impact on product and financial markets.

%\item Identify countries with promising business opportunities.

\item Find and use data relevant to all of these activities.

\end{itemize}
%
We (meaning the team that has put this course together)
think these skills will serve you well, whether you are
advising clients on their international operations,
managing an emerging market hedge fund,
marketing consumer products,
or even working for the New York Mets.


\section{Course website}

Almost everything you need for the course will be posted on the course website:

\vspace*{\parskip}
\centerline{\url{https://sites.google.com/site/nyusternglobal/home}.}

This includes readings, assignments, online discussion,
slides, practice exams, and so on.
I will hand out hardcopies of most of these things in class,
but the online versions include color graphics, links, and attachments.
{\bf Do not look for course materials on NYU Classes},
there's nothing there.

\section{Important dates}

Please note the assignment and exam dates posted on the
\href{https://sites.google.com/site/nyusternglobal/home/outline}{course outline}.


\section{Prerequisites}

I expect you to be able to apply the basic tools of economics,
mathematics, statistics, and spreadsheets.  We will use
logarithms and spreadsheets (extensively) and calculus (somewhat).
To make sure you're ready,
please work through the
\href{http://www.stern.nyu.edu/~dbackus/2303/notes_math.pdf}{Math Review}
before the term starts.
You can also get started on
\href{http://www.stern.nyu.edu/~dbackus/2303/ps0_f14.pdf}{Problem Set \#0}, % ?? update
a check on how well you understand the review
that is due at the start of the second class.

\section{Data}

We will use online data sources extensively.
%There's a collection of links on the macro resources page,
%
%\vspace*{\parskip}
%\centerline{\url{http://pages.stern.nyu.edu/~dbackus/macro_resources.htm},}
The best one is the St Louis Fed's
\href{http://research.stlouisfed.org/fred2/}{FRED}.
You will have an opportunity to use FRED, and perhaps the FRED apps or Excel add-in,
in Problem Set \#0.



\section{Help}

If you need help, please send me an email or post a question
on the
\href{https://sites.google.com/site/nyusternglobal/home/announcements}{Announcements \& Discussion}
page (which will go to me, too).
You can also post a question by sending an email to
\href{mailto:global_economy_discussion@googlegroups.com}{global\_economy\_discussion@googlegroups.com}
%{global_economy_discussion@googlegroups.com}
--- but your email must from from the email address you used to register for the group.
%More on the last one shortly.

If you'd like to speak in person, my office is KMC 7-68:
take the elevator to seven,
go right out of the elevator, through the doors, left at the wall, fourth office on the right.


For most questions, I recommend the
\href{https://sites.google.com/site/nyusternglobal/home/announcements}{Announcements \& Discussion}
page.
If you think you know the answer to someone else's question, please post that, too.
The idea is to create an environment in which we teach ourselves,
which is both effective and fun.
I'll weigh in if I think greater clarity is called for.
{\bf Plan ahead:  you may not be able to get a useful response to anything posted
less than 24 hours before an assignment is due.}


\section{Course materials}

The materials include:
%
\begin{itemize}
\item {\it The Book.\/}
We developed a book specifically for this course:
just what you need, no more, no less.
Please read the relevant sections before class.
It will be distributed in the first class,
posted on the course website,
and sold through
\href{http://www.amazon.com/dp/1500837105/}{Amazon}
(Version 2.1, dated August 2014).
Use whatever format you prefer.
The online version comes with links and color graphs.

%\item {\it Optional textbook.\/}
%If you would like to use a traditional textbook, we recommend
%Cowen and Tabarrock's {\it Modern Principles:  Macroeconomics\/}
%and Mankiw's {\it Macroeconomics\/}.
%I'll post readings from the 2009 edition of
%Cowen-Tabarrok
%(\href{http://www.amazon.com/Modern-Principles-Macroeconomics-Tyler-Cowen/dp/1429202491}{link})
%and the sixth (2006) edition of Mankiw
%(\href{http://www.amazon.com/Macroeconomics-N-Gregory-Mankiw/dp/0716767112/}{link}),
%which are (much) less expensive than current editions
%and no less useful for our purposes.
%(If you use one of these books and find it helpful, please let me know.
%Otherwise, I may eliminate this option from future versions of the course.)
%%The cost of a used copy should be no more than \$20.


\item {\it The Economist\/}.
My advice is to get a subscription.
It gives you a weekly summary of what's going on in the world,
some of which we'll discuss in class.
There's a link on the course website.

\item {\it Slides\/}.
I will distribute copies at the start of class and post pdf's on the course website.
If you want the full Powerpoint experience, the originals are on
\href{https://github.com/DaveBackus/Global_Economy}{GitHub} in the Slides directory.

If you find the slides difficult to read on their own,
remind yourself that's a feature not a bug.
They're designed to facilitate discussion
and have intentional gaps that we will fill in during class.
Let me repeat: {\bf The slides are not intended to be read on their own.}
They are an input to class discussion, not a summary of where that discussion
leads.
Reading them without attending class is likely to be a frustrating experience.

\item {\it Videos.\/}
Classes will be videotaped, but keep in mind:
(i) I would prefer to have you in class and
(ii)~the taping system has failed periodically in the recent past.

\end{itemize}


\section{Grades}

Your grade will be computed from:
%
\begin{center}
\begin{tabular}{lcc}
    Problem sets   &&  20\% \\
    Midterm exam   &&  35\% \\
    Final exam     &&  45\%
\end{tabular}
\end{center}
%
In addition, attendance and participation can influence
the grades of students who are outliers in either direction.
Final grades will conform with the school's guideline for core courses:
no more than 35\% of the class will receive an A or A--.
That means, for example, that you can be above average yet get a B.
%Remind yourself:  A B is a good grade.

\needspace{2\baselineskip}
The fine print:
\begin{itemize}

\item \textit{Class attendance and participation}.
We all learn more and have more fun when everyone attends class and participates.
Participation includes making thoughtful comments in class,
asking thoughtful questions,
and posting comments on the Announcements \& Discussion page.
As a favor to me, and to make sure your participation is noted,
please bring your nameplate to class and add your name to the
sign-in sheet.

\item \textit{Problem sets}.
There are five problem sets,
due in hardcopy at the start of class.
{\bf Late problem sets will not be accepted:
anything submitted after the start of class on the due date will be given a grade of zero.}
You should do Problem Set \#0 on your own,
but for the others you may work in groups.
I recommend groups of two to four,
but a group of one is doable, and five might be permitted in unusual circumstances.
The work you hand in should be the work of you or your group
and include the names of everyone involved on the first page.
Outside sources should be noted; ditto quotations.

%Form counts as well as substance.
{\bf Everything you hand in should be a  professional product.
Anything less will be downgraded accordingly.}
You should state your answer clearly and prominently,
and not bury it in a pile of calculations.
You should not include printouts of data.
Any spreadsheet files should be printed on letter paper
with a portrait orientation.
If through some emergency you submit your work by email,
it should be printable in one step,
pdf format preferred.

One last thing.  Some groups like to divide up the work:
one person does the first one, another does the second, and so on.
The downside is that the others don't learn what the assignment is designed
to teach.
My best guess is that this shows up in exam grades.


\item \textit{Practice problems}.
Similar to problem sets, but they are not collected or graded.
You will nevertheless find them helpful in reviewing the
material covered in class and preparing for exams.


\item \textit{Exams}.
The midterm and final exams will be held in class.
The midterm will last 90 minutes, the final no more than 120 minutes.
They will cover material covered in class, reviewed in assignments,
or assigned as reading.
You can use one sheet of notes: letter paper, both sides, any size type you like.
You may also use a calculator, but may not use any device
capable of wireless transmission.  Proximity to any such
device during the exam will be treated as a violation of the honor
code (see below).

\item \textit{Grading}.
{\bf Questions about grading must be made in writing no more than two weeks
after the graded material is returned.}
Keep in mind:  Our prime directive is to treat everyone the same way,
whether they appeal or not.
For this reason, we will correct clear mistakes but will not
reconsider judgement calls.
%If we made a clear mistake, we'll correct it,
%but we will not reconsider judgement calls.

%\item \textit{Appeals}.
%Our prime directive is to treat everyone the same way.
%For that reason, we feel it's important ???
%If we made a clear mistake, we'll correct it,
%but we will not reconsider judgement calls.
%[??]
\end{itemize}

\section{Extras}

You may notice that the book and the course outline contain references to
extra sources of information.
All of them are optional.
The last section of each chapter of the book,
labeled ``if you're looking for more,''
includes references and links to other sources.
We envision this as a resource if you happen to find yourself in
a situation that goes beyond what you find in the book.
You can also send me an email:  the course comes with what we like
to call lifetime technical support.

The course outline, posted on the course website,
contains lighter references related to each class
under the heading ``something extra.''
They're generally short, and sometime humorous,
if I can use that word in such close proximity to economics.
They are intended
to give you a broader perspective of the issues at hand.
If you find others that fit, please pass them on ---
to me and, via Announcements \& Discussion, to your classmates.

\section{Other sections of the course}

Several of us teach sections of this course,
but there is generally not much difference in content or materials.


\section{Honor Code}

The Stern Honor Code was instituted by students
and requires every student to act with integrity in all
academic activities and to hold his or her peers to the same
standard.

In this course, you may discuss assignments with anyone
(in fact, I encourage it), but {\bf any work submitted for a
grade should be your own} (for individual work) {\bf or your group's}
(for group work).
This means it should be in your own words and based on your own calculations.
Submissions may be scanned electronically to identify
content similar to other submissions or to material from other sources
used without attribution.
On exams, you may bring in and consult one piece
of paper with anything on it you like (letter size, both sides),
but your answers should be entirely your own work.




\section{Professional behavior}

%As a general guideline, please do not do anything
%that will disrupt me or your classmates during class.
In the interest of having a high-quality experience for all,
please
%
\begin{itemize}
\item Arrive a few minutes early,
both at the start of class and after the break.

\item Put away your 
\href{http://marginalrevolution.com/marginalrevolution/2015/04/why-you-should-take-notes-by-hand-not-on-a-laptop.html}
{laptop},
iPhone, Android, etc.

\item Bring your nameplate.

%\item Join your classmates.
%Avoid the last row.
%
\end{itemize}
Thank you in advance.


\section{Students with disabilities}

If you have a qualified disability that requires academic accommodation,
please contact the Moses Center for Students with Disabilities (CSD, 212-998-4980) and ask them to
send me a letter verifying your registration and outlining the accommodation they recommend.
{\bf If you need to take an exam at the CSD,
you must submit a completed Exam Accommodations Form to them
at least one week prior to the scheduled exam time.}

{\vfill
{\bigskip \centerline{\it \copyright \ \number\year \
David Backus $|$ NYU Stern School of Business}%
}}

\end{document}
