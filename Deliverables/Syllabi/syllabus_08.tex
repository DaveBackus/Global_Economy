\documentclass[letterpaper,12pt]{article}

\RequirePackage{ge05}
% this inputs graphicx, too
\RequirePackage{comment}
\RequirePackage[hypertex]{hyperref}
    \hypersetup{colorlinks=true,urlcolor=blue,linkcolor=red}
%    \hypersetup{letterpaper=true}
%    \hypersetup{colorlinks=true,citecolor=black,linkcolor=black,urlcolor=black}
%    \hypersetup{bookmarks=true,pdfpagemode=UseThumbs}
    \hypersetup{pdfauthor={David Backus}} 
%    \hypersetup{pdfsubject={Global Economy}} 

% for ge05.sty
\def\ClassName{The Global Economy}
\def\Category{Professor David Backus}
\def\HeadName{The Global Economy:  Syllabus}


\begin{document}
\thispagestyle{empty}%
\Head
\centerline{\large\bf Syllabus:  The Global Economy} 
\vspace{1mm}
\centerline{\large\bf B01.2303.21/22 $|$ Spring 2008}
\vspace{1mm}
\centerline{Revised:  \today}


\subsubsection*{About the course}

We will use economics as a tool for exploring the enormous 
differences in the economic and business environments of 
countries. By the end of the course, you will be able to:
%
\begin{itemize}
\item Explain how differences in local conditions 
and institutions 
affect the nature and cost of doing business.   

\item Describe the causes of good long-term performance: why per
capita income is higher in the US and France than in China and India, 
and why China and India are among the fastest growing countries in the 
world.  

\item Evaluate indicators of good short-term performance
and their impact on product and financial markets.  

\item Identify countries with promising business opportunities.   

\end{itemize}
%
I think these skills will serve you well, whether you are selling
corporate bonds, 
marketing consumer products, 
advising clients on their international operations, 
managing an emerging market hedge fund,
or even working for the Metropolitan Opera.


\subsubsection*{Prerequisites}

I expect you to be able to apply basic tools of economics,
statistics, and mathematics.  We will use 
logarithms (extensively) and calculus (somewhat). 
If you're not sure what this means, 
please work through the 
``\href{http://www.stern.nyu.edu/eco/b012303/Backus/notes_math.pdf}
{Mathematics Review}''
before the term starts.   


\subsubsection*{Help} 

There are times when everyone needs a little help. 
If that happens to you, please come see me.  
My office hours are Monday 12:00-2:00, 
but I'm around most of the time and would be happy to talk 
if not otherwise engaged.   
I'm in KMC 7-68 (turn right
out of the elevator, through the doors to the economics department, 
left when you hit the wall, third door on your right, behind the tree).  
I generally answer email (dbackus@stern.nyu.edu) quickly 
and will answer the phone (212.998.0873) if I'm in my office.  

We also have two first-rate Teaching Fellows.
Their contact information is on Blackboard.  


\subsubsection*{Course material}

We will use a variety of materials, including:  
%
\begin{itemize}

\item {\it Notes\/}.  
We will use custom-designed (``bespoke'') notes that will be
distributed the first day of class and posted on Blackboard.
They are much more concise than a textbook. 
I expect you to read -- or at least skim -- them before class.  
Some of the material is demanding, so it's helpful to get a preview
before class.  

\item {\it Optional textbook.\/} 
Most students do not use a textbook for this course.
But if economics is new to you, or you prefer a more leisurely
presentation, you might consider buying 
Mankiw's {\it Macroeconomics\/}.
If you do, I suggest you buy it used on Amazon or eBay.  
I'll post readings to the most recent edition on Blackboard, 
but if you get a good deal on an earlier edition, 
let me know and I'll post the appropriate chapter and page numbers
to that one, too.   

NB:  Do NOT buy Mankiw's {\it Principles of Macroeconomics\/}.
That's the macro half of his principles book, not 
his macroeconomics books.  
Here's a 
\href{http://www.amazon.com/Macroeconomics-N-Gregory-Mankiw/dp/0716762137/ref=pd_bbs_sr_3?ie=UTF8&s=books&qid=1196705962&sr=8-3}
{link} 
to the right one.  

\item {\it The Economist\/}.    
Consider it required reading.  
We will refer to articles regularly in and out of class.  

\item {\it Slides\/}.  
I will post them on Blackboard approximately one day before class.
I reserve the right to change them at any time up to the start of class, 
but will post the latest version after class.    

\end{itemize}


\subsubsection*{Grades}

Your grade will be based on:
%
\begin{center}
\begin{tabular}{lcc}
    Class participation &\hspace*{0.50in}&   10\%  \\
    Group projects (best 5 of 7) &&  40\% \\
    Midterm exam &&  25\% \\
    Final exam   &&  25\%
\end{tabular}
\end{center}
%
Final grades will conform with the school's guideline for core courses:
no more than 35\% of the class will receive an A or A--. 


The fine print:
\begin{itemize}

\item \textit{Class attendance and participation}. Your thoughtful
participation makes the course more interesting and productive for everyone, 
including yourself.  
You can participate by asking interesting questions, 
raising issues from articles in {\it The Economist\/} and 
other publications, 
offering your own insights in class, 
or posting articles and comments on the class's 
discussion board. 
The participation grade is based on attendance, 
participation in class, and activity on the discussion board.  

\item \textit{Group projects}. 
Group projects review and extend the material covered in class.   
There are six group projects; your course grade will be based 
on your five best project grades.  
Answers should be submitted through Blackboard 
by 9am on the due date.
Late assignments will be penalized 10 points for every hour or part
thereof past 9am.  


To submit your project, go to ``Submit Projects" on Blackboard and 
follow the instructions.   
Be sure you submit your project:  saving is not the same as submitting.  
Please label the submission with its name (eg, Group Project \#2) 
and include the names of all the members of your group on that assignment. If you submit a spreadsheet, it must be printable on letter paper simply by clicking on print.  Better yet, convert it to a pdf file and submit that. 

If you would like to submit a revision, 
please label it as such and have another member of your group submit it --- Blackboard will not allow 
multiple submissions by the same person.  

The Final Group Project is an opportunity to 
apply the lessons of the course to a realistic 
business situation of your choice.  

\item \textit{Exams}. The two exams will take 75 minutes each. 
You can use one sheet of notes: letter paper, both sides, any size type you like. You may also use a calculator, but may not use any electronic device
that is capable of wireless transmission.  Proximity to any such
device during the exam will be treated as a violation of the honor
code (see below).  

\end{itemize}



\subsubsection*{Groups}

Group projects should be done in groups of no more than five.
Members can change between projects, and you can form groups
with members from both sections.  


\subsubsection*{Blackboard}

Virtually everything you need for this course will be posted on Blackboard: 
notes, assignments, slides, and links to electronic information sources.  
Some online documents, including this one, 
have links to outside sources that
may not be apparent in the printed version.


\subsubsection*{Office 2007}
 
From our IT group:  If you submit Word 2007 or other Office 2007 
documents through Blackboard, make sure you save the files as Word 97-2003 documents 
(or in Excel, Excel 97-2003 workbooks). 
This ensures not only that Blackboard recognizes the file extension, 
but that others (ie, me) will be able to open the document.


\subsubsection*{Honor Code}

At the Stern School, 
we believe that honesty and integrity are necessary for a
rewarding educational experience. These qualities form the basis
for the trust among members of the community (students,
faculty, and administrators) that is essential for educational
excellence. The Honor Code was instituted several years ago by
students, and requires each student to act with integrity in all
academic activities and to hold his or her peers to the same
standard.

In this course, you may discuss assignments with anyone  -- in
fact, I encourage it -- but any work submitted for a
grade should be your own (for individual work) or your group's
(for group work). On exams, you may bring in and consult one piece
of paper with anything on it you like (letter size, both sides),
but your answers should be entirely your own work.


\subsubsection*{Professional behavior}

In the interest of having a high-quality experience for all, 
your classmates and I request that you 
%
\begin{itemize}
\item Come on time.  I recommend you aim for 5 minutes early, 
both at the start of class and after the break.  

%\item Avoid the right section (from my perspective) 
%and the last two rows.  

\item Restrict laptop use to class activities -- 
and then only in the last (official) row.  
%
\end{itemize}
We thank you in advance.  


\subsubsection*{Students with disabilities}

If you have a qualified disability that requires academic accommodation during this course, 
please contact the Moses Center for Students with Disabilities (CSD, 212-998-4980) and ask them to 
send me a letter verifying your registration and outlining the accommodation they recommend.  
If you need to take an exam at the CSD, 
you must submit a completed Exam Accommodations Form to them 
at least one week prior to the scheduled exam time to be guaranteed accommodation.

\vfill \centerline{\it \copyright \ \number\year \ 
NYU Stern School of Business}


\end{document}
