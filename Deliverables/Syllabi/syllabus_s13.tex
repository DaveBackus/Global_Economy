\documentclass[12pt]{article}

\usepackage{ge13}
\usepackage{comment}
\usepackage{hyperref}
\urlstyle{rm}   % change fonts for url's (from Chad Jones)
\hypersetup{
    colorlinks=true,        % kills boxes
    allcolors=blue,
    pdfsubject={NYU Stern course GB 2303, Global Economy},
    pdfauthor={Dave Backus @ NYU},
    pdfstartview={FitH},
    pdfpagemode={UseNone},
%    bookmarks=false,
%    pdfnewwindow=true,      % links in new window
%    linkcolor=blue,         % color of internal links
%    citecolor=blue,         % color of links to bibliography
%    filecolor=blue,         % color of file links
%    urlcolor=blue           % color of external links
% see:  http://www.tug.org/applications/hyperref/manual.html
}

% for ge13.sty
\def\ClassName{The Global Economy}
\def\Category{Professor David Backus}
\def\HeadName{The Global Economy:  Syllabus}


\begin{document}
\thispagestyle{empty}
\parskip=0.55\bigskipamount
\Head
\centerline{\large\bf The Global Economy:  Syllabus}
\vspace{1mm}
\centerline{\large\bf COR1-GB 2303.00/30 $|$ Spring 2013}
%\vspace{2mm}
%\centerline{\large\bf **** ROUGH DRAFT ****}
\vspace{1mm}
\centerline{Revised:  \today}

\subsubsection*{About the course}

The performance of businesses is tightly linked to the
economic environments of the countries in which they operate.
The idea behind the course is to use the tools of macroeconomics
--- and macroeconomic data ---
to assess the economic performance of countries
and the challenges facing businesses operating in them.
By the end of the course, you will be able to:
%
\begin{itemize}
\item Explain how differences in local conditions
and institutions
affect the nature and cost of running a business.

\item Describe the causes of good long-term performance: why per
capita income is higher in the US and France than in China and India,
and why China and India are among the fastest growing countries in the
world.

\item Evaluate indicators of good short-term performance
and their impact on product and financial markets.

%\item Identify countries with promising business opportunities.

\item Find and use data relevant to all of these activities.

\end{itemize}
%
These skills should serve you well, whether you are
advising clients on their international operations,
managing an emerging market hedge fund,
marketing consumer products,
or even working for the New York Mets.


\subsubsection*{Important dates}

The midterm and final exams will be held in class on the following dates:
%
\begin{itemize}
\item Saturday section:  March 30 (midterm) and May 11 (final).
\item Monday section:  April 1 (midterm) and May 13 (final).
\end{itemize}
These and other dates are posted on the
\href{https://sites.google.com/site/nyusternglobal/home/outline}{course website}.

\subsubsection*{Course website}

Almost everything you need for the course will be posted on the course website:

\vspace*{\parskip}
\centerline{\url{https://sites.google.com/site/nyusternglobal/}.}

This includes readings, assignments, online discussion,
slides, practice exams, and so on.
I will hand out hardcopies of most of these things in class,
but the online versions include color graphics and links.
{\bf Do not look for course materials on NYU Classes},
there won't be anything there.

\subsubsection*{Data}

We will use online data sources extensively.
There's a complete collection of links on the resources page,

\vspace*{\parskip}
\centerline{\url{http://pages.stern.nyu.edu/~dbackus/macro_resources.htm},}

but the best one is the St Louis Fed's
\href{http://research.stlouisfed.org/fred2/}{FRED}.
You'll have an opportunity to use FRED, and perhaps the FRED apps or Excel add-in,
in Problem Set \#0.


\subsubsection*{Prerequisites}

I expect you to be able to apply the basic tools of economics,
mathematics, statistics, and spreadsheets.  We will use
logarithms and spreadsheets (extensively) and calculus (somewhat).
To make sure you're ready,
please work through the
\href{http://www.stern.nyu.edu/~dbackus/2303/notes_math.pdf}{Math Review}
before the term starts.
You can also get started on
\href{http://www.stern.nyu.edu/~dbackus/2303/ps0_s13.pdf}{Problem Set \#0},
a check on how well you understand the review
that is due at the start of the second class.

\subsubsection*{Help}

If you need help, please contact me or the teaching fellow, or post a question
on the
\href{https://sites.google.com/site/nyusternglobal/home/announcements}{Announcements \& Discussion}
page.
More on the last one shortly.
My office is KMC 7-68:
right out of the elevator, through the doors, left at the wall, fourth office on the right.
%I usually reply quickly to email
%(\href{mailto:dbackus@stern.nyu.edu}{dbackus@stern.nyu.edu}), but
%see below.
The teaching fellow's contact information is listed
on the course website.
%are Olenna Tysiak (Saturday class,
%\href{mailto:ogt202@stern.nyu.edu}{ogt202@stern.nyu.edu})
%and Varun Bahl (Monday class,
%\href{mailto:vb680@stern.nyu.edu}{vb680@stern.nyu.edu}).

For most questions, I recommend the
\href{https://sites.google.com/site/nyusternglobal/home/announcements}{Announcements \& Discussion}
page.
If you think you know the answer to someone else's question, please post that, too.
The idea is to create an environment in which we teach ourselves,
which lots of research shows is the best way to learn.
I'll weigh in if I think greater clarity is called for.
{\bf Plan ahead:  you may not be able to get a response to anything posted
less than 24 hours before an assignment is due.}

%\subsubsection*{Online resources}


\subsubsection*{Course materials}

The materials include:
%
\begin{itemize}
\item {\it Book.\/}
We developed a book specifically for this course.
Please read the relevant sections before class.
It will be distributed at the start of the term,
posted on the course website,
and sold through
\href{http://www.amazon.com/Global-Economy-Stern-Department-Economics/dp/0615728006/}{Amazon}.
Use whatever format you prefer.
The online version comes with links and color graphs.

\item {\it Optional textbook.\/}
If you would like to use a traditional textbook, we recommend
Cowen and Tabarrock's {\it Modern Principles:  Macroeconomics\/}
and Mankiw's {\it Macroeconomics\/}.
I'll post readings from the 2009 edition of
Cowen-Tabarrok
(\href{http://www.amazon.com/Modern-Principles-Macroeconomics-Tyler-Cowen/dp/1429202491}{link})
and the sixth (2006) edition of Mankiw
(\href{http://www.amazon.com/Macroeconomics-N-Gregory-Mankiw/dp/0716767112/}{link}),
which are (much) less expensive than the current editions
and no less useful for our purposes.
(If you use one of these and find it helpful, please let me know at the end of the term.)
%The cost of a used copy should be no more than \$20.


\item {\it The Economist\/}.
I won't check, but my advice is to get a subscription.
It gives you a weekly summary of what's going on in the world.
There's a link on the course website.

\item {\it Slides\/}.
I will distribute copies at the start of class and post them on the course website.
If you find them difficult to read on their own,
remind yourself it's a feature not a bug.
They're designed to facilitate discussion
and have gaps designed for us to fill in during class.
\end{itemize}


\subsubsection*{Grades}

Your grade will be computed from:
%
\begin{center}
\begin{tabular}{lcc}
    Problem sets   &&  20\% \\
    Midterm exam   &&  35\% \\
    Final exam     &&  45\%
\end{tabular}
\end{center}
%
In addition, attendance and participation can influence
the grades of students who are outliers in either direction.
Final grades will conform with the school's guideline for core courses:
no more than 35\% of the class will receive an A or A--.
That means, for example, that you can be above average yet get a B.

The fine print:
\begin{itemize}

\item \textit{Class attendance and participation}.
Everyone learns more and has more fun when everyone participates in class ---
and on the Announcements \& Discussion page.
Participation includes making thoughtful comments in class,
asking thoughtful questions,
and posting comments on the discussion page.
As a favor to me, and to make sure your participation is noted,
please bring your nameplate to class and add your name to the
sign-in sheet.

\item \textit{Problem sets}.
There are five problem sets,
due in hardcopy at the start of class.
{\bf Late problem sets will not be accepted:
anything submitted after the start of class on the due date will be given a grade of zero.}
You should do Problem Set \#0 on your own,
but for the others you may work in groups.
I recommend groups of three or four,
but a group of one is doable, and five might be permitted in unusual circumstances.
The work you hand in should be the work of you or your group
and include the names of everyone involved on the first page.
Outside sources should be noted.

Form counts as well as substance.
What you hand in should look professional.
You should be clear about what your proposed answer is.
The grader will not be expected to search through (say) a spreadsheet to find the appropriate number.
You should not include printouts of data or other irrelevant aspects of your calculations.
Any spreadsheet files should be printed on letter paper
with a portrait orientation.
Anything submitted by email should be printable in one step.

\item \textit{Practice problems}.
Similar to problem sets, but they are not collected or graded.
You will nevertheless find them helpful in reviewing the
material covered in class.


\item \textit{Exams}.
The midterm and final exams will be in class.
The midterm will last 90 minutes, the final no more than 120 minutes.
They will cover material reviewed in class or assigned as reading.
You can use one sheet of notes: letter paper, both sides, any size type you like.
You may also use a calculator, but may not use any device
capable of wireless transmission.  Proximity to any such
device during the exam will be treated as a violation of the honor
code (see below).

\item \textit{Grading}.
{\bf Questions about grading must be made in writing no more than one week
after the graded material is returned to you.}


%\item \textit{Appeals}.
%Our prime directive is to treat everyone the same way.
%For that reason, we feel it's important ???
%If we made a clear mistake, we'll correct it,
%but we will not reconsider judgment calls.
%[??]
\end{itemize}


\subsubsection*{Honor Code}

The Stern Honor Code was instituted by students
and requires every student to act with integrity in all
academic activities and to hold his or her peers to the same
standard.

In this course, you may discuss assignments with anyone
(in fact, I encourage it), but {\bf any work submitted for a
grade should be your own} (for individual work) {\bf or your group's}
(for group work).
This means it should be in your own words and based on your own calculations.
Submissions may be scanned electronically to identify
content similar to other submissions or to material from other sources
used without attribution.
On exams, you may bring in and consult one piece
of paper with anything on it you like (letter size, both sides),
but your answers should be entirely your own work.


\subsubsection*{Professional behavior}

%As a general guideline, please do not do anything
%that will disrupt me or your classmates during class.
In the interest of having a high-quality experience for all,
please
%
\begin{itemize}
\item Arrive a few minutes early,
both at the start of class and after the break.

\item Put away your laptop, iPhone, Android, etc.

%\item Join your classmates.
%Avoid the last row.
%
\end{itemize}
Thank you in advance.


\subsubsection*{Students with disabilities}

If you have a qualified disability that requires academic accommodation,
please contact the Moses Center for Students with Disabilities (CSD, 212-998-4980) and ask them to
send me a letter verifying your registration and outlining the accommodation they recommend.
{\bf If you need to take an exam at the CSD,
you must submit a completed Exam Accommodations Form to them
at least one week prior to the scheduled exam time.}

\vfill
\centerline{\it \copyright \ \number\year \ NYU Stern School of Business}

\end{document}
