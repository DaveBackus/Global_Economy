\documentclass[12pt]{article}

\usepackage{ge05}
\usepackage{comment}
\usepackage[dvipdfm]{hyperref}
\urlstyle{rm}   % change fonts for url's (from Chad Jones)
\hypersetup{
    colorlinks=true,        % kills boxes
    allcolors=blue,
    pdfsubject={NYU Stern course GB 2303, Global Economy},
    pdfauthor={Dave Backus @ NYU},
    pdfstartview={FitH},
    pdfpagemode={UseNone},
%    bookmarks=false,
%    pdfnewwindow=true,      % links in new window
%    linkcolor=blue,         % color of internal links
%    citecolor=blue,         % color of links to bibliography
%    filecolor=blue,         % color of file links
%    urlcolor=blue           % color of external links
% see:  http://www.tug.org/applications/hyperref/manual.html
}

% for ge05.sty
\def\ClassName{The Global Economy}
\def\Category{Professor David Backus}
\def\HeadName{Global Economy:  Syllabus}


\begin{document}
\thispagestyle{empty}
\parskip=0.75\bigskipamount
\Head
\centerline{\large\bf The Global Economy:  Syllabus}
\vspace{1mm}
\centerline{\large\bf COR1-GB 2303.00/30 $|$ Spring 2012}
%\vspace{2mm}
%\centerline{\large\bf **** ROUGH DRAFT ****}
\vspace{1mm}
\centerline{Revised:  \today}

\subsubsection*{About the course}

We will use economics as a tool for exploring the enormous
differences in the economic and business environments of
countries. By the end of the course, you will be able to:
%
\begin{itemize}
\item Explain how differences in local conditions
and institutions
affect the nature and cost of doing business.

\item Describe the causes of good long-term performance: why per
capita income is higher in the US and France than in China and India,
and why China and India are among the fastest growing countries in the
world.

\item Evaluate indicators of good short-term performance
and their impact on product and financial markets.

\item Identify countries with promising business opportunities.

\end{itemize}
%
These skills will serve you well, whether you are
marketing consumer products,
advising clients on their international operations,
managing an emerging market hedge fund,
or even working for movie studio.


\subsubsection*{Important dates}

Please check the midterm and final exam dates on the course website:  \\

\centerline{\url{https://sites.google.com/site/nyusternglobal/}}

The
\href{https://sites.google.com/site/nyusternglobal/home/outline}{complete schedule}
is posted on the same site.
%Dates for ``deliverables'' (assignments and exams) are marked in red.


\subsubsection*{Prerequisites}

I expect you to be able to apply basic tools of economics,
mathematics, statistics, and spreadsheets.  We will use
logarithms and spreadsheets (extensively) and calculus (somewhat).
To make sure you're ready,
please work through the
\href{http://www.stern.nyu.edu/~dbackus/2303/notes_math.pdf}{Math Review}
before the term starts.
You can also get started on
\href{http://www.stern.nyu.edu/~dbackus/2303/ps0_s12.pdf}{Problem Set \#0}
(a check on how well you understand the review),
which is due at the start of the second class.

\subsubsection*{Help}

If you need help, please contact me or the teaching fellow.
I am available after class, by appointment, and by email.
My office is KMC 7-68:
right out of the elevator, through the doors, left at the wall, fourth office on the right.
I usually reply quickly to email
(\href{mailto:dbackus@stern.nyu.edu}{dbackus@stern.nyu.edu}).
If you don't get a response in
24 hours, please send a friendly reminder.
The teaching fellows
are Olenna Tysiak (Saturday class,
\href{mailto:ogt202@stern.nyu.edu}{ogt202@stern.nyu.edu})
and Varun Bahl (Monday class,
\href{mailto:vb680@stern.nyu.edu}{vb680@stern.nyu.edu}).

Questions about assignments will be posted --- with answers ---
under Announcements on the course website so that everyone can see them.
You should check there first if you have a question.
You should also plan ahead:  I don't guarantee I'll be
able to answer emails sent
less than 24 hours before an assignment is due.

\subsubsection*{Blackboard}

I'm trying to move away from Blackboard for lots of reasons,
but will use it for grades, class emails, and maybe discussions.
%Discussions are an essential part of the course:  more on them later.
Materials --- notes, assignments, and slides ---
will be posted on the course website.


\subsubsection*{Course materials}

We will use a variety of materials, including:
%
\begin{itemize}

\item {\it Optional textbook.\/}
There is no required textbook.
If you would like to use one, we recommend either
Cowen and Tabarrock's {\it Modern Principles:  Macroeconomics\/}
or Mankiw's {\it Macroeconomics\/}.
I'll post readings from the 2009 edition of
Cowen-Tabarrock
(\href{http://www.amazon.com/Modern-Principles-Macroeconomics-Tyler-Cowen/dp/1429202491}{link})
and the sixth (2006) edition of Mankiw
(\href{http://www.amazon.com/Macroeconomics-N-Gregory-Mankiw/dp/0716767112/}{link}),
which are much less expensive than the current editions
and no less useful for our purposes.
The cost of a used copy should be no more than \$5-20.

\item {\it Notes\/}.
Instead of a one-size-fits-all textbook,
we will use notes written specifically for this course.
They will be distributed at the start of the term and
posted on the course website.
I expect you to read them before class.

\item {\it The Economist\/}.
Consider it recommended reading.
It's a useful weekly summary of what's going on in the world.
There's a link on the website.

\item {\it Slides\/}.
I will post a preliminary version before class
and, if necessary, an updated version after class.
If you find them difficult to read on their own,
that's a feature not a bug:
they're designed to facilitate discussion
and have gaps for you to fill in.
\end{itemize}


\subsubsection*{Grades}

Your grade will be computed from:
%
\begin{center}
\begin{tabular}{lcc}
    Problem sets   &&  20\% \\
    Midterm exam   &&  35\% \\
    Final exam     &&  45\%
\end{tabular}
\end{center}
%
In addition, attendance and participation can influence
the grades of students who are outliers in either direction.
Final grades will conform with the school's guideline for core courses:
no more than 35\% of the class will receive an A or A--.

The fine print:
\begin{itemize}

\item \textit{Class attendance and participation}.
Everyone learns more and has more fun when people participate
actively in class.
Class participation includes making thoughtful comments
and asking thoughtful questions,
including connections to current events.
%and ---  posting comments on the discussion board.
To make sure your participation is noted, please
bring your nameplate to class.
Attendance matters, so be sure to add your name to the
sign-in sheet for each class.
%Once during the semester, every student is expected to
%post a comment to lead the discussion on a specific topic.
%Other contributions to the discussion board are voluntary,
%but will be acknowledged as participation.

\item \textit{Problem sets}.
There will be one individual (the math review)
and four group problem sets,
due in hardcopy at the start of class.
{\bf Late problem sets will not be accepted:  
anything submitted after the due date will be given a grade of zero.}  
For all of them but the math review,
which you should do on your own,
you may work on problem sets in groups of up to five students.
The work you hand in should be the work of your group
and include the names of everyone involved on the first page.
Use of outside sources should be noted where appropriate.


\item \textit{Exams}.
The midterm and final exams will be in class.
The midterm will last 90 minutes, the final 120 minutes.
They will cover material reviewed in class or assigned as reading.
You can use one sheet of notes: letter paper, both sides, any size type you like.
You may also use a calculator, but may not use any device
capable of wireless transmission.  Proximity to any such
device during the exam will be treated as a violation of the honor
code (see below).

%\item \textit{Appeals}.
%Our prime directive is to treat everyone the same way.
%For that reason, we feel it's important ???
%If we made a clear mistake, we'll correct it,
%but we will not reconsider judgment calls.
%[??]

\end{itemize}

We will also distribute practice problems
and practice exams.
They're what the names suggest:  good practice
to help you review what you've learned.


\subsubsection*{Honor Code}

The Stern Honor Code was instituted by students
and requires every student to act with integrity in all
academic activities and to hold his or her peers to the same
standard.

In this course, you may discuss assignments with anyone
(in fact, we encourage it), but any work submitted for a
grade should be your own (for individual work) or your group's
(for group work).
This means it should be in your own words and based on your own calculation.
Submissions may be scanned electronically to identify
content similar to other submissions or to material from other sources
used without attribution.
On exams, you may bring in and consult one piece
of paper with anything on it you like (letter size, both sides),
but your answers should be entirely your own work.

\subsubsection*{Professional behavior}

%As a general guideline, please do not do anything
%that will disrupt me or your classmates during class.
In the interest of having a high-quality experience for all,
please
%
\begin{itemize}
\item Arrive a few minutes early,
both at the start of class and after the break.

\item Put away your laptop, iPhone, and Blackberry.

\item Join your classmates.
Avoid the last row.
%
\end{itemize}
Thank you in advance.


\subsubsection*{Students with disabilities}

If you have a qualified disability that requires academic accommodation,
please contact the Moses Center for Students with Disabilities (CSD, 212-998-4980) and ask them to
send me a letter verifying your registration and outlining the accommodation they recommend.
If you need to take an exam at the CSD,
you must submit a completed Exam Accommodations Form to them
at least one week prior to the scheduled exam time to be guaranteed accommodation.

\vfill
\centerline{\it \copyright \ \number\year \ NYU Stern School of Business}

\end{document}
