\documentclass[letterpaper,12pt]{article}

\usepackage{comment}
%\usepackage{amsmath}


\usepackage{amssymb}
\renewcommand{\labelitemi}{$\Box$}
\renewcommand{\labelitemii}{$\Box$}

\usepackage[compact]{titlesec}

\RequirePackage{GE13}
% this inputs graphicx, too

\def\ClassName{The Global Economy}
\def\Category{Checklist}
\def\HeadName{Global Economy Short-Term Checklist}

\begin{document}
\parindent = 0.0in
\parskip = \bigskipamount
\thispagestyle{empty}%
\Head

\centerline{\large \bf \HeadName}%
\centerline{Revised:  \today}

\bigskip
A checklist of indicators and tools for assessing the economic conditions of a country before entering a short-term business relationship.


\subsubsection*{1. Measures of short-term country performance}

\renewcommand{\labelitemi}{$\Box$}
\begin{itemize}

\item Macroeconomic conditions:  GDP growth; consumption, investment, and saving; employment and unemployment; inflation; short and long-term interest rates; government deficit and debt; real and nominal exchange rates.
    %; net exports, current account balance, and net foreign asset position.
%    [Remind yourself what each is, and what it tells you about the economy.]

\item Business cycle indicators:  housing starts, retail sales,
    new claims for unemployment insurance, stock prices, many others;
    details vary by country.

\item Comment:  All countries experience fluctuations in growth rates.
    In developed countries, near-term growth is somewhat predictable
    from a variety of economic indicators.
    In developing countries, volatility is greater and there is a greater
    chance of a complete change in long-term growth prospects, up or down.
\end{itemize}


\subsubsection*{2. Analytical tools}

\begin{itemize}
\item Graphs:  plot indicator (or its growth rate);
    draw lines for mean and mean +/-- one standard deviation
    for comparison.

\item Business-cycle scorecard:  a table that summarizes
the strengths of a number of business-cycle indicators.

\item Cross-correlation function:  graphical tool for identifying leads and lags in economic indicators.
\item Aggregate supply and demand:  the benchmark theoretical framework for thinking about short-term movements in GDP growth and inflation.
\item Taylor rule:  bond traders' guide to monetary policy; indicates how
    short-term
    interest rate is likely to respond to changes in economic growth and inflation; helps to identify ``neutral'' policy so that you can say whether current policy is unusual.
\item Government debt dynamics:  shows sources of year-to-year changes in the ratio of government debt to GDP.  Critical inputs:  interest on debt, GDP growth, primary deficit.
%\item Net foreign asset dynamics:  shows sources of year-to-year changes in the ratio %of net foreign assets to GDP.  Critical inputs:  GDP growth, interest on NFA, net %exports.
\item Exchange rates:  purchasing power parity, interest rate parity.
    Comment:  short-term fluctuations in currency prices are
    large and mostly inexplicable.
\end{itemize}


\subsubsection*{3. Crisis indicators}

%Here's the crisis checklist:
%
\begin{itemize}
\item Government debt and deficits.
Common rules of thumb:  worry if government
deficit is more than 5\% of GDP or debt is more than 50\% of GDP.
Adjust upward for developed countries, downward for developing countries.
But note:  quantitative benchmarks are less important than the political environment.
Default is more closely associated with weak institutions than high levels of debt.  

Fine points:  worry further if debt is short-term and/or denominated in foreign currency.
Short-term debt subjects the government to refinancing risk:
 markets may demand better terms or refuse to refinance.  Foreign-denominated debt subjects government to risk
if currency falls in value, increasing the debt in local terms.
Watch out for hidden liabilities:  pensions, health care, bank bailouts, etc.

\item Banking/financial system.
A financial collapse hurts the economy
and may leave the government with a large expense.
Not a topic for this course, but analysts track leverage
and nonperforming loans.

\item Currency and capital controls.
Does the central bank limit purchases and sales of local currency?
Can you move money (``capital'') in and out of the country?
Under what conditions?

\item Exchange rate and reserves.
Fixed exchange rate regimes sometimes blow up.
Rule of thumb:  worry if the exchange rate is fixed, or close to it,
and the currency is overvalued in PPP terms
(Big Macs cost 30\% more than in other currencies,
real exchange rate has risen more than 30\% in last 2-5 years).
Worry more if foreign exchange reserves
are low or have fallen significantly.

The trilemma.  You can only have two of
(i)~fixed exchange rate, (ii)~free international movement of capital,
and (iii)~independent monetary policy.  If you try to have all three,
something will give, probably the exchange rate.


%\item Current account and net foreign assets.
%If net foreign assets is negative we say a country is a net borrower.
%Rule of thumb:  worry if net foreign borrowing is greater than 50\% of GDP. Worry more %if foreign borrowing finances government deficits.
%Worry further is the current account deficit is more than 5\% of GDP.

\item Politics and institutions.
Especially in emerging markets, it's often more important to follow the politics than the economic numbers.
Worry if the political situation is unstable, especially in a country
with otherwise weak institutions.
\end{itemize}


\subsubsection*{4. Integration}

The challenge is to put all these pieces together:
to use our tools and economic data to come up with a coherent picture of
current and near-term future economic conditions,
and how these conditions are likely to affect any business opportunities
you are considering.
Remember, too, that bad current conditions can be wonderful opportunities if things improve.

\bigskip
\vfill \centerline{\it \copyright \ \number\year \ NYU Stern School of Business}

\end{document}
