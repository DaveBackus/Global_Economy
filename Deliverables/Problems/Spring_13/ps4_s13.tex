\documentclass[12pt]{exam}

\usepackage{ge13}
\usepackage{comment}
\usepackage{booktabs}
\usepackage{hyperref}
\urlstyle{rm}   % change fonts for url's (from Chad Jones)
\hypersetup{
    colorlinks=true,        % kills boxes
    allcolors=blue,
    pdfsubject={NYU Stern course GB 2303, Global Economy},
    pdfauthor={Dave Backus @ NYU},
    pdfstartview={FitH},
    pdfpagemode={UseNone},
%    pdfnewwindow=true,      % links in new window
%    linkcolor=blue,         % color of internal links
%    citecolor=blue,         % color of links to bibliography
%    filecolor=blue,         % color of file links
%    urlcolor=blue           % color of external links
% see:  http://www.tug.org/applications/hyperref/manual.html
}

% for ge05.sty
\def\ClassName{The Global Economy}
%\def\Category{Professor David Backus}
\def\Category{Backus \& Cooley}
\def\HeadName{Problem Set \#4}
\newcommand{\phm}{\phantom{--}}
\newcommand{\NX}{\mbox{\it NX\/}}
\newcommand{\POP}{\mbox{\it POP\/}}

\printanswers

\begin{document}
\parindent = 0.0in
\parskip = \bigskipamount
\thispagestyle{empty}%
\Head

\centerline{\large \bf \HeadName: Monetary \& Fiscal Policy}
\centerline{Revised:  \today}

\medskip
{\it You may do this assignment in a group.
Whatever you hand in should be the work of your group
and include the names of all of the contributors.}

\begin{questions}
% --------------------------------------------------------------------
\question The Taylor rule (50 points).
You have been asked by your boss to explain the Taylor rule
and use it as a guide to current US monetary policy.
You review your Global Economy class notes and do the following:
\begin{parts}
\part Using data from FRED (see data guide below),
you plot the Taylor rule and the fed funds rate for the period 1960 to present.
(15~points)
\part In which periods was Fed policy significantly different from the Taylor rule?
On the whole, do you see the periods of difference
as mistakes by the Fed or reasonable departures from the rule?
(15~points)
\part What does the Taylor rule suggest now?
How does it compare to current Fed policy,
as represented by the current fed funds rate and the Fed's most
recent
\href{http://federalreserve.gov/monetarypolicy/fomccalendars.htm}
{policy statement}?
(20~points)
\end{parts}
Data guide.  To implement the Taylor rule,
you'll need quarterly data on
\begin{itemize}
\item Real GDP (FRED variable GDPC1):  use year-on-year growth rate.
\item Consumer prices (FRED variable PCEPI)):  use year-on-year growth rate.
\item Fed funds rate (FRED variable FEDFUNDS):  use as is.
\end{itemize}
You can download all of them from FRED.
It's not required, but you can also generate the graph(s) directly in FRED.

\begin{solution}
\begin{parts}
\part The Taylor rule calculation is
\begin{eqnarray*}
    i_t  \;=\;  r^* + \pi_t + (1/2) (\pi_t - \pi^*)
            + (1/2) (g_t - g^* )   ,
\end{eqnarray*}
You'll see different versions of this equation,
discussed in the notes,
but this is the one we will use.

Here's a FRED figure of the rule and the actual fed funds rate:

\url{http://research.stlouisfed.org/fred2/graph/?g=6D5}.

The link includes a description of how it was computed.
In the figure, the blue line is the actual fed funds rate and
the red line is the fed funds rate implied by the Taylor rule.
By convention, we've set $r^* = 2\%$, $\pi^* = 2\%$, and $g^* = 3\%$.

Some people replace the last term
in the Taylor rule with the percentage difference
between real GDP and its long-run equilibrium level,
often referred to as ``potential output.''
The challenge here, of course, is deciding what the long-run equilibrium
is.
One of the standard numbers is available in FRED (FRED variable GDPPOT).
It's neither recommended nor required, but
if you took this route, your Taylor rule would look like this one,

\url{http://research.stlouisfed.org/fred2/graph/?g=6Pg},

which isn't a lot different from the previous version.


\part The list would include:
(i)~the 1970s, when the Fed set interest rates below the Taylor rule;
(ii)~the early 2000s, when they did the same out of fear of deflation;
(iii)~2009, when the Taylor rule suggests a negative interest rate;
and (iv)~now, when the rate is again below the Taylor rule.
Most people regard (i) as a mistake.
(iii) was unavoidable.  The other two we can debate.

\part Right now, the Taylor rule is pointing to a higher interest rate ---
between 2.7 and 3\% by my calculation.
The Fed, however,
has announced that it expects to keep interest rates near zero for some time.
The FOMC statement includes:
``To support continued progress toward maximum employment and price stability,
the Committee expects that a highly accommodative stance of monetary policy will remain
 appropriate for a considerable time ...
In particular, the Committee decided to keep the target range for the federal funds rate at 0 to 1/4 percent and currently anticipates that this exceptionally low range for the federal funds rate will be appropriate at least as long as the unemployment rate remains above 6-1/2 percent, inflation ... is projected to be no more than a half percentage point above the Committee's 2 percent longer-run goal,
and longer-term inflation expectations continue to be well anchored.''
What does this mean?
Most interpret it as saying they'll keep the rate close to zero for another year or more.

Is that a good policy?  We'll have to wait and see.
The question is whether the Fed is making the same mistake it made in the 1970s,
keeping the interest rate too low.
Some see deviating from the Taylor rule as a mistake that will
lead to sharply higher inflation in the future.
An alternative is that the Fed thinks the Taylor rule is misleading under current conditions.
We can reconvene in three years and see how this worked out.




\end{parts}
\end{solution}


% --------------------------------------------------------------------
\question Fiscal policy in Turkey (50 points).
The Turkish economy has had its ups and downs over the last century,
but some think it could be poised for Chinese-like growth.
One concern, however, is fiscal policy.
Large government deficits in the 1990s led to inflation rates
consistently above 50 percent a year.
An accord with the IMF, in which loans were tied to fiscal stringency,
was in effect from 1999 to 2008.
Despite a jump up in 2008-09, government deficits remain modest.
%They have also benefitted from strong growth in the economy.

Your mission is to assess the risks of fiscal policy to the economy.
Having some experience with such situations,
you collect some data:

\begin{center}
%{\small
\begin{tabular}{lrrrrr}
\toprule
         & 2008 &  2009  &  2010 & 2011 & 2012 \\%
\midrule
Real GDP growth  & 0.7 & --4.8 & 9.2 & 8.5 & 3.0 \\
Inflation        & 12.0& 5.3 & 5.7 & 8.9 & 6.2 \\
Govt expenditures& 34.1 & 38.0 & 35.9 & 34.8 & 35.3 \\
Budget balance   &--4.3 &--4.3 &--2.2 &--1.9 &--0.8 \\
Primary balance  &  0.0 & 0.0 & 0.1 & 0.1 & 0.2 \\
Public debt (net)& 33.4 \\ %& 51.1 & 45.5 & 39.6 & 40.0 & 46.3 \\
Interest rate (short) \hspace*{0.2in}
                 & 16.0 & 9.2 & 5.8 & 3.0 & 5.9 \\
\bottomrule
\end{tabular}
%}
\end{center}
The data comes from the IMF's
\href{http://www.imf.org/external/pubs/ft/weo/2012/02/weodata/index.aspx}{WEO Database}.
(That's information for another time, but not for now.)
(i)~Government expenditures, (government) budget balances,
and public debt are expressed as
percentages of GDP (ratios to GDP multiplied by 100).
(ii)~Government budget numbers are balances:
negative numbers indicate deficits,
positive numbers indicate surpluses.
(iii)~Entries for 2012 are estimates.
(iv)~The interest rate is nominal.
(v)~Growth, inflation, and interest are percentages.

After giving the numbers a quick look,
you study the Economist Intelligence Unit's
various reports for background information and summarize the relevant sections:
%
\begin{itemize}
\item Modern Turkish politics has evolved from the secular
single-party republic established by Mustafa Kemal (Ataturk) in 1923 into
a multiparty democracy.

\item The ruling Justice Party (AKP) gained power in 2002
and is likely to retain power until elections in 2015 and beyond.
    The AKP's ``pro-Islamist roots'' are an ongoing source of tension
    with the ``secularist elite'' and a source of
    political uncertainty.

\item Turkey has a free-trade agreement (``customs union'')
with the EU and is exploring closer ties, including membership.
Some think this process imposes discipline on economic policies.
\end{itemize}

With this information in hand, you start to sketch out your report:
\begin{parts}
\part Why is the budget balance in deficit while the primary balance
shows a modest surplus?
(10~points)

\part Compute Turkey's debt-to-GDP ratio for the period in the table.
Use our analysis of debt dynamics to determine what factors
contributed to the change in debt-to-GDP between 2008 and 2012?
(20~points)

\part
What factors might alter your estimate of how the debt ratio evolves
during 2013 and after?
(10~points)

\part After skimming the EIU's Country Risk Report ---
and using your own good judgement ---
do you see any specific concerns looming about
fiscal policy over the next 3-4 years?
Other concerns?
(10~points)
\end{parts}

Accessing the EIU's Country Risk Reports.
Go to NYU's
\href{http://library.nyu.edu/vbl/}{Virtual Business Library},
then click on Country Information,
EIU Country Risk Service (login as requested if off-campus),
Country Risk Service (again), and (in this case) Turkey.
Choose the latest report and give it a quick read.


\begin{solution}
\begin{parts}
\part The difference between the primary and total budget balances
reflects interest payments on the debt.
In 2012,
the budget surplus was $-0.8$ percent (that is, a deficit)
and the primary surplus 0.2 percent.
The difference of 1.0 percent is interest payments on the debt.

\part This is a debt dynamics calculation.
The key equation is
\begin{eqnarray}
    \Delta ({B_{t}}/{Y_{t}})
            &=&
                (i_t-\pi_t) ({B_{t-1}}/{Y_{t-1}})
                - g_t ({B_{t-1}}/{Y_{t-1}})
             +    ({D_{t}}/{Y_{t}})  .
             \label{eq:debt-dynamics}
\end{eqnarray}
Make sure you know what all the pieces are.
We derived it as an approximation, but ignore that.

Most of the things you need are given, you just put them to work in the equation.
The only one about which there might be some question is
the interest rate $i_t$.
You could either take the given interest rate
or compute the implied interest rate on the debt from interest payments.
Either is fine as far as grading is concerned, but the former is far easier.

We'll use the reported rate,
but if you choose to calculate the implied interest rate
here's what's involved.
From (a), we know interest payments on the debt $i_t B_{t-1}/Y_t$.
Why the different timing?  Because interest is paid on last year's debt ($B_{t-1}$)
but is reported as a ratio to current GDP ($Y_t$).
Thus in 2003 we have $i_t B_{t-1}/Y_t = 13.0\%$.
To extract $i_t$, we calculate
\begin{eqnarray*}
    i_t &=&  \left[ \frac{(i_t B_{t-1}/Y_t)}
            {(B_{t-1}/Y_{t-1})} \right]
            (Y_t/Y_{t-1}) .
\end{eqnarray*}
(Do the cancelations to show that you really do get $i$ here.)
If you do this for 2009, you get 12.3\%, a but higher than the reported rate of 9.2\%.
Why is this?  Because some of the debt is long-term, and
was issued in the past when rates were higher.
Another issue you'll run across:  you need to do this together with calculating
the debt, because that's an input into the rate calculation.

If you try this for 2004, you get 19.5\%,
which is a bit lower than the 21.4\%
interest rate reported in the table.

Back to the calculation.
We compute each of the terms in equation (\ref{eq:debt-dynamics})
and see how they add up.
If we add the relevant columns to our table, we have

\begin{center}
%{\small
\begin{tabular}{lrrrrr}
\toprule
         & 2008 &  2009  &  2010 & 2011 & 2012 \\%
\midrule
Real GDP growth  & 0.7 & --4.8 & 9.2 & 8.5 & 3.0 \\
Inflation        & 12.0& 5.3 & 5.7 & 8.9 & 6.2 \\
Govt expenditures& 34.1 & 38.0 & 35.9 & 34.8 & 35.3 \\
Budget balance   &--4.3 &--4.3 &--2.2 &--1.9 &--0.8 \\
Primary balance  &  0.0 & 0.0 & 0.1 & 0.1 & 0.2 \\
Public debt (net)& 33.4 \\ % & 36.2 & 32.8 & 28.1 & 27.0  \\
Interest rate (short) \hspace*{0.2in}
                 & 16.0 & 9.2 & 5.8 & 3.0 & 5.9 \\
\midrule
Public debt (net)& 33.4 & 36.2 & 32.8 & 28.1 & 27.0  \\
Real interest $(i_t-\pi_t)(B_{t-1}/Y_{t-1})$ & & 1.3 & 0.0 & --1.9 & --0.1 \\
Growth  $- g_t(B_{t-1}/Y_{t-1})$ & & 1.6 & --3.3 & --2.8 & --0.8 \\
Primary deficit $D_t/Y_t$ & & 0.0 & --0.1 & --0.1 & --0.2 \\
\bottomrule
\end{tabular}
%}
\end{center}
\medskip

Overall, we see that debt fell from 33.4 percent of GDP to 27.0 percent of GDP,
a decline of 6.4 percent.
Where did that come from?
The 2009-2012 totals for the three terms in (\ref{eq:debt-dynamics})\
are $-0.7$\% from real interest (note that inflation was above the interest rate),
$-5.3$\% from growth,
and $-0.4$\% from primary surpluses.
Evidently growth was the major contributor.


Bottom line:  after having serious debt and deficit problems
between 1990 and 2004,
Turkey has shown great fiscal discipline and now finds its debt to GDP
ratio falling.

For the calculations, see

\url{http://pages.stern.nyu.edu/~dbackus/2303/ps4_s13_answerkey.xls}



\part What could go wrong?
Answers may differ, but a standard list might include:
a sharp rise in interest rates, which have been low;
slower growth;
and unforseen budget problems.
That is, the terms in the equation.
Growth is the most important, because it also affects tax receipts
and hence the deficit.

A related concern is political, because
political problems often lead to budget problems.
In this sense the ``ongoing tension'' with the elites could be an issue.

\part The EIU Country Risk Report includes these notable comments:
\begin{itemize}
\item Fiscal issues. Modest increase in fiscal deficit from weaker growth and higher interest rates.
Turkey faces a series of elections in 2014 and 2015, and public expenditure
may overshoot government targets during this period.
The government’s fiscal deficit reduction plans include a substantial
increase in privatisation receipts.
A weaker Turkish lira will make it more costly to service foreign-denominated debt.
At the end of 2012 non-lira-denominated debt accounted for about 27\% of total central government debt.
\item Political.  The Kurdish issue and related terrorist attacks will continue to pose a major risk
to social and political stability. The domestic political scene will remain polarised.
The risk of military conflict with war-torn Syria has risen sharply.
\end{itemize}
\phantom{xx}

\end{parts}
\end{solution}

\end{questions}

\vfill \centerline{\it \copyright \ \number\year \
NYU Stern School of Business}

\end{document}

