\documentclass[12pt]{exam}

\usepackage{ge13}
%\usepackage{comment}
\usepackage{booktabs}
\usepackage{hyperref}
\urlstyle{rm}   % change fonts for url's (from Chad Jones)
\hypersetup{
    colorlinks=true,        % kills boxes
    allcolors=blue,
    pdfsubject={NYU Stern course GB 2303, Global Economy},
    pdfauthor={Dave Backus @ NYU},
    pdfstartview={FitH},
    pdfpagemode={UseNone},
%    pdfnewwindow=true,      % links in new window
%    linkcolor=blue,         % color of internal links
%    citecolor=blue,         % color of links to bibliography
%    filecolor=blue,         % color of file links
%    urlcolor=blue           % color of external links
% see:  http://www.tug.org/applications/hyperref/manual.html
}

% for ge05.sty
\def\ClassName{The Global Economy}
\def\Category{Backus \& Cooley}
\def\HeadName{Problem Set \#2}
\newcommand{\phm}{\phantom{--}}
\newcommand{\NX}{\mbox{\it NX\/}}
\newcommand{\POP}{\mbox{\it POP\/}}

\printanswers

\begin{document}
\parindent = 0.0in
\parskip = \bigskipamount
\thispagestyle{empty}%
\Head

\centerline{\large \bf \HeadName: Long-Term Economic Performance}
%\centerline{\large \bf Problem Set \#2:  Long-Term Economic Performance}
\centerline{Revised:  \today}

%\medskip
%{\it I had software problems, had to kill off some of the formatting
%to get this to work. }

\medskip
{\it You may do this assignment in a group.
Whatever you hand in should be the work of your group
and include the names of all of the contributors.}

\begin{questions}

\begin{solution}
Brief answers follow,
but see also the spreadsheet posted on the course website.
\end{solution}

% --------------------------------------------------------------------
\question Sources of Korean success  (35~points).
The Republic of Korea (``South Korea'')
has been one of the great economic success stories of world history.
Since the end of the Korean War in 1953,
GDP per capita has risen by a factor of about 18.
Over the same period, US income rose by a factor of 3.
As a result, the gap between the two countries has shrunk dramatically.
In 1953, average income in Korea was about 10\% of US income,
but by 2010 (the most recent comparable number) it was about 65\%.


Was Korea a classic productivity story,
or did capital formation and hours worked play more important
roles than in other countries?
We know, for example, that the saving rate and hours worked
are both unusually high.
Let's check the numbers and see where the differences
in GDP per person come from.

Use the Penn World Table spreadsheet, posted at

\vspace*{\parskip}
\centerline{\url{http://pages.stern.nyu.edu/~dbackus/2303/pwt71.xlsx},}

to fill in this table for 2010:
%
\begin{center}
%\tabcolsep = 0.2in
\begin{tabular}{lcc}
\toprule
        &  South Korea & United States \\
\midrule
GDP per person ($Y/\POP$) \phantom{xxxx}&  \\
GDP per worker ($Y/L$)    &  \\
Capital-output ratio ($K/Y$)  & \\
Capital per worker ($K/L$)    & \\
Employment rate ($L/\POP$)      & \\
\bottomrule
\end{tabular}
\end{center}
\smallskip
Note that you may have to compute some of the entries
yourself from the numbers in the spreadsheet.

\begin{parts}

\part What is the ratio of GDP per person in the two countries
(Korea over US)?
(5~points)

\part Use the production function to derive total factor productivity (TFP)
 in each country from the numbers in the table.
 What is the ratio of the two countries?  How does it compare to the ratio
 you computed in (a)?
 (10~points)

\part Overall, what factors contribute to the difference in GDP per person?
How important is capital?
(10~points)

\part You have heard that Koreans work exceptionally long hours.
The OECD Employment Outlook reports that the average employee in Korea worked 2111 hours in 2010,
while the average American employee worked only 1787 hours.
How would this information change your calculation of TFP?
How does it change your assessment of the relative productivity
of Korea and the US?
(10~points)
\end{parts}

\begin{solution}
Brief answers follow.
See the spreadsheet for the calculations.
The table becomes
\begin{center}
%\tabcolsep = 0.2in
\begin{tabular}{lrrr}
\toprule
        &  Korea & USA & Ratio\\
\midrule
GDP per person ($Y/\POP$) \phantom{xxxx}& 26,609 & 41,365 & 0.643  \\
GDP per worker ($Y/L$)                  & 54,315 & 82,359 & 0.659 \\
Capital-output ratio ($K/Y$)  & 3.942 & 2.682 & 1.302 \\
Capital per worker ($K/L$)    & 189,642 & 220,887 & 0.859 \\
Employment rate ($L/\POP$)      & 0.490 & 0.502 & 0.976 \\
\bottomrule
\end{tabular}
\end{center}

%
\begin{parts}
\part The ratio is $0.643 = 26,609/41,365$:
Korea has, by this measure, a living standard equal to 64.3\% of the US's.
The rest of the question is devoted to explaining the sources
of this difference.

\part We compute productivity the usual way from measures in output and inputs.
If the production function is $ Y/L = A (K/L)^\alpha$, then $ A = (Y/L) / (K/L)^\alpha$
where (as usual) $\alpha = 1/3$.
Thus $A_{K} = 54315/189642^{1/3} = 945$.  You can also look this number up in the table,
but explaining the calculation is essential to a complete answer.
The ratio of productivities is 0.694,
which is a bit higher than the ratio of GDP per capita.

\part What we have in mind is a level comparison:
\begin{eqnarray*}
    \frac{(Y/\POP)_K}{(Y/\POP)_{US}} &=&
            \left( \frac{(L/\POP)_K}{(L/\POP)_{US}} \right)
            \left( \frac{A_K}{A_{US}} \right)
            \left( \frac{(K/L)_K}{(K/L)_{US}} \right)^{1/3} \\
            &=&  0.976 \times 0.694 \times 0.950
                \;\;=\;\; 0.643 .\phantom{sum^K}
\end{eqnarray*}
You see here that most of the difference comes from productivity.

\part This question is intentionally more demanding.
We modify the production function to include hours of work.
There's more than one route to this answer,
among them $ Y = A K^\alpha (hL)^{1-\alpha} $.
Either way, productivity (``corrected'' for hours worked)
is now 5.745 in Korea and 9.252 in the US
(the use of hours data changes the units).
The ratio is 0.621, which is below our earlier calculation of 0.694.
In words:  part of what we attributed to productivity before
was really long hours.
Put another way:  Korea looks better (relative to the US)
in a comparison of output per worker than in a comparison
of TFP, because output per worker includes a contribution
from long hours.

\end{parts}
\end{solution}

% --------------------------------------------------------------------
\question Argentina and Chile (35 points).
Argentina and Chile have each experienced
dramatic growth and painful reversals over the last century.
Argentina was one of the richest countries in the world in 1900,
but mixed economic performance since then has
dropped it back into the world of emerging markets.
Chile suffered a traumatic change in government in the 1970s,
but emerged a decade later as a fast-growing democracy.
Chile now has the highest per capita GDP in Latin America.
Your mission is to tell us how --- and maybe why
--- their experiences have differed.

Fill in the following table using the same
source as the previous question:

\begin{center}
%\tabcolsep = 0.2in
\begin{tabular}{lcccc}
\toprule
        &  \multicolumn{2}{c}{Argentina}
        &  \multicolumn{2}{c}{Chile} \\
        \cmidrule(r){2-3}  \cmidrule(r){4-5}
        &  1960 & 2010 & 1960 & 2010 \\
\midrule
GDP per capita ($Y/\POP$) \phantom{xxxx}&  \\
GDP per worker ($Y/L$)    &  \\
Capital-output ratio ($K/Y$)  & \\
Capital per worker ($K/L$)    & \\
Employment rate ($L/\POP$)      & \\
\bottomrule
\end{tabular}
\end{center}
\medskip
%

\begin{parts}
\part Compute the (average annual continuously compounded)
growth rates of GDP per capita and GDP per worker in the two countries
over the period 1960 to 2010.  (10~points)

\part Use our growth accounting methodology to allocate growth in
GDP per worker to growth in productivity and capital per worker.
Which factor has been most important over the last 50 years? (15~points)

\part Use the World Bank's
\href{http://info.worldbank.org/governance/wgi/sc_country.asp}{Governance Indicators}
and \href{http://www.enterprisesurveys.org/Reports}{Enterprise Surveys}
to comment on the differences in the political and business environments of the two countries.
%(Click on the chart for each country for a nice visual summary.)
%Introduce additional information as needed.
What role do you think these institutional differences played in the countries'
economic performance?
(10~points)
\end{parts}


\begin{solution}
GDP per capita in Chile has gone
from well below Argentina in 1960 to significantly above
[correction:  about the same].
How did this happen?
Most of this problem is concerned with aggregate data:
how much comes from capital, TFP, etc?
The last part is an opportunity to look further
into features of the two economies that might account
for what we find in the production function numbers.

To make the numbers easier to manage, I've divided
$Y/L$ and $K/L$ by 1000, which changes the units of TFP, too.
The growth rates, of course, are the same either way.
See the spreadsheet for details.
%
\begin{parts}
\part The (continuously compounded annual)
growth rate of GDP per capita for Argentina is
\[
    \gamma \;=\; \log (12.340/6.043) / (2010 - 1960) \;=\; 1.428\%.
\]
(We give 3 digits here to allow you to compare your own numbers,
but it's excessive accuracy, not justified by the quality of the data.)
If you're not sure why this works,
refer to the discussion of growth rates in the math review.
A similar calculation gives us 2.446\% for Chile, so about one percent per year higher.
Analogous growth rates for GDP per worker are 1.231\% and 1.741\%, respectively.
The difference between the two numbers comes from growth in
$L/\POP$.

Bottom line:  faster growth in Chile, in part because of an increase in
the fraction of the population working.

\part We use the production function
\[
    Y/L \;=\; A (K/L)^{1/3} ,
\]
so TFP is $ A = (Y/L) / (K/L)^{1/3}  $.
Again, we expect you to explain the calculation
even if you look the number up from the table.
In growth rates, with the adjustment for employment,
this translates into
\begin{eqnarray*}
    \gamma_{Y/L} &=&   \gamma_A  + \alpha \gamma_{K/L}  ,
\end{eqnarray*}
where $\alpha = 1/3$ as usual.
A few calculations give us

\medskip
\begin{center}
%\tabcolsep = 0.12in
\begin{tabular}{lccc}
\toprule
             &  $Y/L$    &   $A$   &  $K/L$     \\
\midrule
\multicolumn{3}{l}{\it Argentina} \\
Growth rate (\%) &  1.231   &   0.835 &  1.187   \\
Contribution to growth (\%)
                 &  1.231   &   0.835 &  0.396  \\
\midrule
\multicolumn{3}{l}{\it Chile} \\
Growth rate (\%) &  1.741  &  1.376   &   1.096    \\
Contribution to growth (\%)
                &   1.741  &  1.376   &   0.365   \\
\bottomrule
\end{tabular}
\end{center}

\medskip
Here ``growth rate'' is computed as in (a) and multiplied
by 100 to give us a percentage.
``Contribution'' modifies these growth
rates as they occur in the formula:  the growth rate of $K/L$ gets
multiplied by $\alpha = 1/3$ (its exponent in the production function).
The numbers tell us that the difference in growth
reflects differences in TFP growth.

One last thought on the numbers:  the PWT used to show
significantly higher GDP per capita in Chile than Argentina.
It's not clear why that's no longer true.
There could be something funny in the numbers.

\part The World Bank's Governance Indicators include these
percentile ranks:

\medskip
\begin{center}
\begin{tabular}{lcc}
\toprule
                        & Argentina & Chile \\
\midrule
Voice and accountability	& 57.7 & 81.2  \\
Political stability	        & 53.8 & 65.1 \\
Govt effectiveness	        & 48.8 & 83.9 \\
Regulatory quality	        & 25.1 & 93.4 \\
Rule of law		            & 33.3 & 88.3 \\
Control of corruption	    & 42.2 & 91.9 \\
\bottomrule
\end{tabular}
\end{center}

\bigskip
The Enterprise Surveys are less useful in comparing countries,
but they identify some of the obvious problems.
In Argentina, businesses cite
taxes, access to finance, labor regulations, political instability,
and the informal economy as the five largest obstacles to performance.
In Chile, the same list is labor regulations, education of the workforce,
access to finance, the informal sector, and transportation.

Almost all of the institutional indicators are stronger for Chile
than Argentina, so perhaps it's not surprising that economic performance
has been better.
Much of this has come in the last 25 years, as Chile has reestablished a
stable democracy with sensible economic policies.
Argentina continues to fluctuate between policies that favor long-term
performance and those that focus on the short term to the detriment of
the long term.

Our belief is that the strong economic and political
foundations Chile has laid will continue to generate higher growth
than we'll see in Argentina.
For one thing, we've had different parties in power over the last 20 years,
yet the same basic economic policies have been followed.
One we find intriguing is a privatized social security system,
which helped them develop a domestic capital market.

A temporary factor is copper prices,
which play an unusually important role in Chile.
It's a tricky measurement issue, since
high prices  tend to show up as high GDP
and therefore high TFP.
History tells us that probably won't last.
With that in mind, the government saved revenue from copper
for several years,
despite intense political pressure to spend it.
When the crisis hit, they had a pot of money they could use to mitigate
its effects.

If you're looking for more anecdotes about Argentina,
try
\href{http://en.search.wordpress.com/?q=argentina&site=nyusterneconomics.wordpress.com}{here}.
\end{parts}
\end{solution}

% --------------------------------------------------------------------
\question Labor market conditions (30 points).
Your first day on the job at General Electric,
you are given 4 hours to collect information for a 5-minute
presentation to your group summarizing the labor market
conditions a manufacturer would face in Brazil, Poland, and Singapore.
Once you get over your initial panic, you contact your Global Economy
professor, who suggests you look at the
\href{http://pages.stern.nyu.edu/~dbackus/macro_resources.htm}{resource page},
especially
%
\begin{itemize}
\item The Bureau of Labor Statistics'
\href{http://www.bls.gov/fls/}{International Labor Comparisons}
page, which includes wage rates in a number of different countries,
collected on a comparable basis.
Click on a country name to get a summary table.

\item The World Bank's
\href{http://data.worldbank.org/topic}{Global Development Indicators}
and the
\href{http://www.barrolee.com/}{Barro-Lee dataset},
which include
information about the education and literacy of the population.

\item The World Bank's Doing Business website, which includes
institutional information about the labor market,
labeled
\href{http://www.doingbusiness.org/data/exploretopics/employing-workers}
{Employing Workers}.

\end{itemize}
%
Use this information to put together a short report
summarizing labor market conditions in these three countries.


\begin{solution}
The basic tradeoff here is between cost (the wage data)
and quality, including the quality of the institutions
you'd be dealing with.
Here's a table of some of the numbers you might find,
but yours may differ depending on the source:

\begin{center}
\begin{tabular}{lrrr}
\toprule
		& Brazil & Poland  & Singapore  \\
\midrule
Hourly compensation (USD)   & 11.65  & 8.83 & 22.60 \\
\midrule
Literacy of adults  (\%)    &  90 & 100 & 96 \\
Primary school enrollment (\%) & 127 & 97 & --- \\
Secondary school enrollment (\%) & 101 & 97 & --- \\
\midrule
Minimum wage (monthly, USD) & 300 & 386 & --- \\
Severance (weeks, 20 years experience)
                    & 33.3 & 0  & 0 \\
\midrule
Strictness of employment protection (index)
                & 2.75 & 1.9 & --- \\
\midrule
Average years of school (age $\geq$ 25) & 7.2 & 10.0 & 8.9 \\
\bottomrule
\end{tabular}
\end{center}

The lines represent different sources;
in the order they appear,
BLS, World Bank GDI, Doing Business, OECD,
and Barro-Lee via our PWT spreadsheet.

A quick summary:
\begin{itemize}
\item Singapore:  educated workers, flexible labor market, expensive.
\item Brazil:  the least educated of the three, also the least flexible labor market,
but relatively cheap.
\item Poland:  even cheaper than Brazil, better educated workers,
somewhat more flexible labor market, part of the EU.
\end{itemize}

\end{solution}


\end{questions}

\vfill \centerline{\it \copyright \ \number\year \
NYU Stern School of Business}

\end{document}

