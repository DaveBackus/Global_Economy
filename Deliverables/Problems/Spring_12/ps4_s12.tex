\documentclass[12pt]{exam}

\usepackage{ge05}
\usepackage{comment}
\usepackage{booktabs}
\usepackage[dvipdfm]{hyperref}
\urlstyle{rm}   % change fonts for url's (from Chad Jones)
\hypersetup{
    colorlinks=true,        % kills boxes
    allcolors=blue,
    pdfsubject={NYU Stern course GB 2303, Global Economy},
    pdfauthor={Dave Backus @ NYU},
    pdfstartview={FitH},
    pdfpagemode={UseNone},
%    pdfnewwindow=true,      % links in new window
%    linkcolor=blue,         % color of internal links
%    citecolor=blue,         % color of links to bibliography
%    filecolor=blue,         % color of file links
%    urlcolor=blue           % color of external links
% see:  http://www.tug.org/applications/hyperref/manual.html
}

% for ge05.sty
\def\ClassName{The Global Economy}
%\def\Category{Professor David Backus}
\def\Category{Backus \& Cooley}
\def\HeadName{Problem Set \#4}
\newcommand{\phm}{\phantom{--}}
\newcommand{\NX}{\mbox{\it NX\/}}
\newcommand{\POP}{\mbox{\it POP\/}}

\printanswers

\begin{document}
\parindent = 0.0in
\parskip = \bigskipamount
\thispagestyle{empty}%
\Head

\centerline{\large \bf \HeadName: Monetary \& Fiscal Policy}
\centerline{Revised:  \today}

\medskip
{\it You may do this assignment in a group of up to five people.
Whatever you hand in should be the work of your group.}

\begin{questions}
% --------------------------------------------------------------------
\question The Tayor Rule (50 points).
You have been asked by your boss to explain the Taylor Rule
and use it as a guide to current US monetary policy.
You review your Global Economy class notes and do the following:
\begin{parts}
\part Using data from FRED (see data guide below),
you plot the Taylor Rule and the fed funds rate for the period 1960 to present.
(15~points)
\part What do you think was the impact on output and inflation
of keeping the fed funds rate below the Taylor
Rule during the 1970s?
(10~points)
\part How did policy compare to the Taylor Rule from 1985 to 2000?
What do you think was the impact on output and inflation?
(10~points)
\part What does the Taylor Rule suggest now?
What options does the Fed have for dealing with the current situation?
How does that compare to the Fed's stated policy in their
most recent
\href{http://federalreserve.gov/monetarypolicy/fomccalendars.htm}
{policy statement}?
(15~points)
\end{parts}
Data guide.  To implement the Taylor Rule,
you'll need quarterly data on
\begin{itemize}
\item GDP (FRED variable GDP):  use year-on-year growth rate
\item Consumer prices (FRED variable PCEPI)):  use year-on-year growth rate
\item Fed Funds rate (FRED variable FEDFUNDS):  use as is.
\end{itemize}
You can download all of them from FRED.
It's not required, but you can also generate the graph(s) directly in FRED.

\begin{solution}
\begin{parts}
\part The Taylor rule calculation is
\begin{eqnarray*}
    i_t  \;=\;  r^* + \pi_t + (1/2) (\pi_t - \pi^*)
            + (1/2) (g_t - g^* )   ,
\end{eqnarray*}
You'll see different versions of this equation,
discussed in the notes,
but this is the one we'll use.

Here's a FRED figure of the rule and the actual fed funds rate:

\url{http://research.stlouisfed.org/fred2/graph/?g=6D5}

The link includes a description of how it was computed.
In the figure, the blue line is the actual fed funds rate and
the red line is the fed funds rate implied by the Taylor rule.
By convention, we've set $r^* = 2\%$, $\pi^* = 2\%$, and $g^* = 3\%$.

Some people replace the last term 
in the Taylor rule with the percentage difference 
between real GDP and its long-run equilibrium level, 
often referred to as ``potential output.''
The challenge here, of course, is deciding what the long-run equilibrium
is.  
One of the standard numbers is available in FRED.
It's neither recommended nor required, but 
if you took this route, your Taylor rule might look like 

\url{http://research.stlouisfed.org/fred2/graph/?g=6Pg}

which isn't a lot different from the previous version.  


\part You can see in the figure that the Fed set interest rates in the 1970s below
the Taylor rule.
The standard interpretation is that the Fed was overly ``expansionary,''
setting the interest rate too low and (and implicitly allowing the money supply
to grow too fast).
This led to the inflation we saw way back then.

\part From 1985 to 2000, there's not much difference between the Taylor rule
and Fed policy, except that the Fed smoothed out the wiggles in the rule.
As for outcomes:  we experienced, on the whole, low and stable inflation rates.
We also experienced above-average GDP growth and relatively stable GDP growth,
with only the 1991 recession to detract from overall performance.
See this FRED figure documenting inflation and GDP growth over the same period:

\url{http://research.stlouisfed.org/fred2/graph/?g=6D7}

It's not clear how much credit to give to monetary policy for (say) the dot-com boom,
but even a skeptic would have to say that the policy
doesn't seem to have caused any damage and probably did some good.

\part Right now, the Taylor rule is pointing to higher interest rates
but the Fed has announced that it expects to keep interest rates near zero
through most of 2014.
The question is whether the Fed is making the same mistake it made in the 1970s,
keeping the interest rate too low.
An alternative is that the Fed thinks the Taylor rule is misleading under current conditions.
We can reconvene in three years and see how this works out.

\end{parts}
\end{solution}


% --------------------------------------------------------------------
\question Fiscal policy in Turkey (50 points).
The Turkish economy has had its ups and downs over the last century,
but some think it could be poised for Chinese-like growth.
One concern, however, is fiscal policy.
Large government deficits in the 1990s led to inflation rates
consistently above 50 percent.
 An accord with the IMF, in which loans were tied to fiscal stringency,
  was in effect from 1999 to 2008.
  Despite a ``spike'' during 2008-09, the government debt and ongoing deficits
    remain modest.
    Both have benefitted from strong growth in the economy.
but some think it may be poised for Chinese-like growth.
Your mission is to assess the risks to the economy,
starting with fiscal policy,
and the likelihood of sustained rapid growth.

Having some experience with such situations,
you collect some data:

\begin{center}
{\small
\begin{tabular}{lrrrrrrrrr}
\toprule
         & 2003 &  2004  &  2005  &  2006   & 2007  & 2008 &  2009  &  2010 & 2011 \\%
\midrule
Real GDP growth  & 5.2 & 9.3 & 8.4 & 6.9 & 4.7 & 0.7 & --4.8 & 9.1 & 7.8 \\
Inflation        &23.4 &10.9 & 8.4 & 9.8 & 6.6 & 10.8& 5.0 & 8.4 & 6.1 \\
Govt expenditures&30.9 & 25.3 & 22.6 & 23.5 & 24.2 & 23.9 & 28.2 & 26.6 & 25.9 \\
Budget balance   &--8.9&--5.4 &--1.3 &--0.6 &--1.6 &--1.8 &--5.5 &--3.6 &--1.7 \\
Primary balance  & 4.1 & 4.7 & 5.8 & 5.5 & 4.2 & 3.5 & 0.1 & 0.8 & 1.6 \\
Public debt      & 62.3 \\ %& 51.1 & 45.5 & 39.6 & 40.0 & 46.3 \\
Interest rate (short) & 36.2 & 21.4 & 14.7 & 15.6 & 17.2 & 16.0 & 9.2 & 5.8 & 3.0 \\
\bottomrule
\end{tabular}
}
\end{center}
The data are from the Economist Intelligence Unit's CountryData.
(i)~Government expenditures, (government) budget balances,
and public debt are expressed as
percentages of GDP (ratio to GDP multiplied by 100).
(ii)~Government budget numbers are balances:
negative numbers indicate deficits,
positive numbers indicate surpluses.
(iii)~Entries for 2011 are estimates.
(iv)~The interest rate is nominal.  

After giving the numbers a quick look,
you study the Economist Intelligence Unit's
various reports for background information and summarize the relevant sections:
%
\begin{itemize}
\item Modern Turkish politics has evolved from the secular
single-party republic established by Mustafa Kemal (Ataturk) in 1923 into
a multiparty democracy.

\item The ruling Justice Party (AKP) gained power in 2002
and is likely to retain power until elections in 2015 and beyond.
    The AKP's ``pro-Islamist roots'' are an ongoing source of tension
    with the ``secularist elite'' and a source of
    political uncertainty.

\item Turkey has a free-trade agreement (``customs union'')
with the EU and is exploring closer ties, including membership.
Some think this process imposes discipline on economic policies.
\end{itemize}

With this information in hand, you start to sketch out your report:
\begin{parts}
\part Why are the differences between the primary and total budget
balances so large?
(10~points)

\part Compute Turkey's debt-to-GDP ratio for the period in the table.
What seem to be the major factors in its behavior?
(20~points)

\part
What factors might change your estimate of how the debt ratio could evolve
during 2012 and after?
(10~points)

\part After skimming the EIU's Country Risk Report ---
and using your own good judgement ---
do you see any specific concerns looming about
fiscal policy over the next 3-5 years?
Other concerns?
(10~points)
\end{parts}

Accessing the EIU's Country Risk Reports.
Go to NYU's
\href{http://library.nyu.edu/vbl/}{Virtual Business Library},
then click on Country Information,
EIU Country Risk Service (login as requested if off-campus),
Country Risk Service (again), and (in this case) Turkey.
Choose the latest report and give it a quick read.

\begin{solution}
\begin{parts}
\part The difference between the primary and total budget balances
reflects interest payments on the debt.
Like Brazil, Turkey pays relatively high interest rates on its
debt, which increases its cost.
For example, in 2003, the total deficit was 8.9\% of GDP,
listed in the table as a budget balance of $-8.9\%$.
Nevertheless, there was a primary balance of 4.1\% (a surplus),
which is a primary deficit with the opposite sign:
$-4.1\%$.
The difference of 13.0\% of GDP (yikes!)
was interest payments on the debt.

\part This is an extensive calculation.
For the details, see the spreadsheet

\url{http://pages.stern.nyu.edu/~dbackus/2303/ps4_s12_answerkey.xls}

The key equation is
\begin{eqnarray*}
    \Delta ({B_{t}}/{Y_{t}})
            &=&
                (i-\pi) ({B_{t-1}}/{Y_{t-1}})
                - g ({B_{t-1}}/{Y_{t-1}})
             +    ({D_{t}}/{Y_{t}})  .
\end{eqnarray*}
Make sure you know what all the pieces are.
We derived it as an approximation, but ignore that.

Most of the things you need are given, you just put them to work in the equation.
The only one about which there might be some question is
the interest rate $i_t$.
You could either take the given interest rate
or compute the implied interest rate on the debt from interest payments.
Either is fine as far as grading is concerned,
but here's what involved in using the implied interest rate,
which is what is done in the spreadsheet.
From (a), we know interest payments on the debt $i_t B_{t-1}/Y_t$.
Why the different timing?  Because interest is paid on last year's debt ($B_{t-1}$)
but is reported as a ratio to current GDP ($Y_t$).
Thus in 2003 we have $i_t B_{t-1}/Y_t = 13.0\%$.
To extract $i_t$, we calculate
\begin{eqnarray*}
    i_t &=&  \left[ \frac{(i_t B_{t-1}/Y_t)}
            {(B_{t-1}/Y_{t-1})} \right]
            (Y_t/Y_{t-1}) .
\end{eqnarray*}
If you try this for 2004, you get 19.5\%,
which is a bit lower than the 21.4\%
interest rate reported in the table.

Back to the calculation.
We compute each of the terms in the equation and see how they add up.
The results include a ratio of debt to GDP 35.1\% in 2011,
a decline of 27.2\% of GDP since 2003.
How did that happen?
The components are

\begin{center}
%\begin{table}
\begin{tabular}{lccc}
\toprule
       & Interest &  Growth & Primary Deficit \\
       &  $(i-\pi)B_{t-1}/Y_{t-1}$  &  $-g B_{t-1}/Y_{t-1}$
                & $D_t/Y_t$ \\
\midrule
2003-2011 \hspace*{0.4in} & 20.3  & --21.3  &  --26.2 \\
\bottomrule
\end{tabular}
\end{center}

We see that interest is pushing us the other way, raising the debt-to-GDP ratio.
But we overcome that with strong growth and primary surpluses.

Bottom line:  after having serious debt and deficit problems,
Turkey has shown great fiscal discipline over the last decade.


\part What could go wrong?
Answers may differ, but a standard list might include:
a sharp rise in interest rates, which have some down since 2003;
slower growth;
some unseen budget problems;
and so on.

\part The economic fundamentals look fine.
The EIU Country Risk Report rates Turkey as stable along most dimensions.
The biggest concern is political.
There's a long history to this, alluded to in the question.
The EIU's update includes:
``The Kurdish issue and related terrorist attacks pose a major risk to social and
political stability. Tension between the religiously conservative ruling Justice
and Development Party and the secularist-nationalist elites will remain high.''

\end{parts}
\end{solution}

\end{questions}

\vfill \centerline{\it \copyright \ \number\year \
NYU Stern School of Business}

\end{document}

