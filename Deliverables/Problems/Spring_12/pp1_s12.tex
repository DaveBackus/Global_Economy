\documentclass[12pt]{exam}

\usepackage{ge05}
\usepackage{comment}
\usepackage{booktabs}
\usepackage[dvipdfm]{hyperref}
\urlstyle{rm}   % change fonts for url's (from Chad Jones)
\hypersetup{
    colorlinks=true,        % kills boxes
    allcolors=blue,
    pdfsubject={NYU Stern course GB 2303, Global Economy},
    pdfauthor={Dave Backus @ NYU},
    pdfstartview={FitH},
    pdfpagemode={UseNone},
%    pdfnewwindow=true,      % links in new window
%    linkcolor=blue,         % color of internal links
%    citecolor=blue,         % color of links to bibliography
%    filecolor=blue,         % color of file links
%    urlcolor=blue           % color of external links
% see:  http://www.tug.org/applications/hyperref/manual.html
}

% for ge05.sty
\def\ClassName{The Global Economy}
%\def\Category{Professor David Backus}
\def\Category{Backus \& Cooley}
\def\HeadName{Practice Problems A}
\newcommand{\phm}{\phantom{--}}
\newcommand{\NX}{\mbox{\it NX\/}}

\noprintanswers
%\printanswers

\begin{document}
\parindent = 0.0in
\parskip = \bigskipamount
\thispagestyle{empty}%
\Head

\centerline{\large \bf \HeadName: Production \& Capital Formation}
\centerline{Revised:  \today}

\medskip
{\it This will not be collected or graded, but it's a good way to make sure you're up to speed.
We recommend you do it before the next class.}

\begin{questions}
% --------------------------------------------------------------------
\question Capital update...  growth and K/Y ?? 

% --------------------------------------------------------------------
\question Solow -- war??  
The impact of Solow's analysis is that productivity growth 
is more important than capital formation in the long-term 
performance of countries, 
but it's a mistake to disregard capital formation altogether.  
The standard example is war.
Wars often destroy a lot of the physical capital;
subsequent recovery is exactly what the Solow model does.  

We'll consider an economy that starts at its long-run 
``stationary'' level of capital, then go on
to study its recovery following a 20\% drop.  
%
\begin{parts}
\part Consider an economy whose parameters include
$A = 1$, $L=100$, $\alpha = 1/3$, $\delta = 0.1$, and $s = 0.2$.
Explain what each of these does.  

\part 
Suppose we start out with $K = $?? 
Using the spreadsheet, what does capital tend to?  

\part Now kill of 20\% of $K$.  How fast is recovery?  
What does this depend on?  

\end{parts}

\end{questions}

\vfill \centerline{\it \copyright \ \number\year \
NYU Stern School of Business}

\end{document}

