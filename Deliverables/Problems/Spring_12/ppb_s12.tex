\documentclass[12pt]{exam}

\usepackage{ge05}
\usepackage{comment}
\usepackage{booktabs}
\usepackage[dvipdfm]{hyperref}
\urlstyle{rm}   % change fonts for url's (from Chad Jones)
\hypersetup{
    colorlinks=true,        % kills boxes
    allcolors=blue,
    pdfsubject={NYU Stern course GB 2303, Global Economy},
    pdfauthor={Dave Backus @ NYU},
    pdfstartview={FitH},
    pdfpagemode={UseNone},
%    pdfnewwindow=true,      % links in new window
%    linkcolor=blue,         % color of internal links
%    citecolor=blue,         % color of links to bibliography
%    filecolor=blue,         % color of file links
%    urlcolor=blue           % color of external links
% see:  http://www.tug.org/applications/hyperref/manual.html
}

% for ge05.sty
\def\ClassName{The Global Economy}
%\def\Category{Professor David Backus}
\def\Category{Backus \& Cooley}
\def\HeadName{Practice Problems B}
\newcommand{\phm}{\phantom{--}}
\newcommand{\NX}{\mbox{\it NX\/}}

\printanswers

\begin{document}
\parindent = 0.0in
\parskip = \bigskipamount
\thispagestyle{empty}%
\Head

\centerline{\large \bf \HeadName:  Labor Markets \& Trade}
\centerline{Revised:  \today}

\medskip
{\it This will not be collected or graded,
but it's a good way to make sure you're up to speed.
We recommend you do it before the next class.}

\begin{questions}

% --------------------------------------------------------------------
\question Labor market analysis.
As the TF in a Global Economy section,
you have been asked to work through an example to illustrate
the impact of a minimum wage and payroll taxes on employment and unemployment.
you start with these labor demand and supply functions:
\begin{eqnarray*}
    \mbox{Demand:} && L^d(w) \;\;=\;\; 1/(1.5 w)^3 \\
    \mbox{Supply:} && L^s(w) \;\;=\;\; w^2 .
\end{eqnarray*}
%
\begin{parts}
\part Describe first how a frictionless labor market might work.
What wage rate clears the market (that is, equates supply and demand)?
How much labor is employed?
What is the unemployment rate?
Sketch the result in a supply/demand diagram.

\part Now consider introducing a minimum wage.  If the minimum wage is
$w_m = 1$, how much labor is employed?
How much is supplied?
What is the unemployment rate?
Show how this works in your diagram.

\part Suppose the minimum wage is $w_m = 1/2$.
How much labor is employed?  Supplied?
What is the unemployment rate?

\part Suppose there is no minimum wage, but the government
imposes 5\% payroll tax:
a tax on labor paid by the employer.
That is:  if $w$ is the wage received by the employee,
$ 1.05 \cdot w$ is what the employer pays.
What is the equilibrium wage?  Employment?
How do they compare to the equilibrium without the tax?
\end{parts}

\begin{solution}
Same idea as class, with numbers added.
See the spreadsheet for calculations.
\begin{parts}
\part In a frictionless market,
prices and quantities adjust until supply equals demand.
If we do that, it implies (sorry)
\begin{eqnarray*}
    w^5 &=& (1/1.5)^3
\end{eqnarray*}
or $w = 0.784$.
If we substitute into supply or demand, we find $L = 0.615$.
At these values, everyone works who wants to and unemployment is zero.

\part If the minimum wage is $w_{min} = 1$, note that this is above
the wage in the frictionless market.
At this wage, more people want to work than firms want to hire.
We have $L^d = 0.296$, $L^s = 1$, employment = 0.296,
unemployment $= 1 - 0.296 = 0.704$, and the unemployment rate is
$ 0.704 / (0.296 + 0.704) = 70.4\%$.

\part If the minimum wage is 1/2, then we're back to the solution in (a):
0.615 units of labor are employed and unemployment is zero.
Why?  The market wage is above this minimum, so it's irrelevant.

\part The labor demand function becomes
\begin{eqnarray*}
    L^s(w) &=& 1/( 1.5 \times 1.05 w)^3,
\end{eqnarray*}
where $w$ is the pretax wage.
Setting supply equal demand gives us a wage of 0.761.
Employment is 0.580.
There is no unemployment, but both employment and the wage
are lower than they were without the tax.
Why?  Because the cost of hiring has gone up for firms.
Even with the lower (pretax) wage,
the cost including the tax has gone up to 0.800.

It's probably easier to follow the logic with generic supply and demand curves.  
The tax shifts the demand curve for labor down by the amount of the tax.
If demand slopes down and supply slopes up,
we'll see a decline in employment.
This isn't any different from other markets.  
We tax cigarettes, for example, because we want to reduce the quantity.
It's the same logic.

\end{parts}
\end{solution}

% --------------------------------------------------------------------
\question Protecting sugar.
The US has a long-running tradition of protecting sugar producers,
going back to 1789 when Treasury Secretary Alexander Hamilton helped
pass a tariff on sugar imports.
(Tariffs at that time were the major source of federal government revenue
and were needed to service the country's debt.)
Currently the US restricts the quantity of sugar imports (quotas) and guarantees
prices to US producers that are well above world prices.

\begin{parts}
\part Who benefits from these policies?
\part Who loses?
\part Economists believe that the cost of sugar protection outweigh
the benefits, yet these policies have been in place for 200 years.
Why?
\end{parts}

\begin{solution}
\begin{parts}
\part Producers of sugar or sugar substitutes benefit:
Sugar cane and sugar beet farmers, as well as corn growers,
who produce the ubiquitous ``high-fructose corn syrup,''
a common substitute for sugar in American soft drinks and candy.
The Iowa primary usually illustrates this connection in stark terms.

\part Anyone who consumes or produces a product that contains sugar is hurt.
Also people who use substitutes.

\part Sad to say, economists do not make policy.
Policy is made by people who wanted to get reelected.
Sugar producers are a small group, so you might think their influence
would be small, but they have a strong interest in such policies.
Everyone else, a much larger group, generally loses from these policies,
but their losses, while large in total, are small to them individually,
hence unlikely to change their votes.
\end{parts}
\end{solution}

% --------------------------------------------------------------------
\question Dumping coffee.
You are working for Illy USA, the American subsidiary of
an Italian company that imports and sells coffee products in the US.
You boss tells you that domestic coffee roasters have filed a dumping complaint
against the company,
but does not know what that means.
He asks:
%
\begin{parts}
\part What must they show to sustain a judgement of dumping against Illy?

\part What damage could this do to Illy USA?

\part Why has dumping become more common in the recent past?
\end{parts}

\begin{solution}
\begin{parts}
\part They must show two things.  First, that Illy has priced
coffee ``unfairly'' in the US.
For example, that Illy sells coffee in the US below its estimated cost of production
or below what it charges in Italy.
Second, that they have suffered ``injury'' as a result.

\part It could result in duties (tariffs) imposed on Illy's imports,
not to mention the endless litigation associated with
many dumping cases.

\part Antidumping cases have risen in popularity has
World Trade Organization (WTO) rules have eliminated other
trade barriers.
\end{parts}
%
Wikipedia has a nice summary:  \\
\centerline{\url{http://en.wikipedia.org/wiki/Dumping_(pricing_policy)}}
\end{solution}


\end{questions}

\vfill \centerline{\it \copyright \ \number\year \
NYU Stern School of Business}

\end{document}

