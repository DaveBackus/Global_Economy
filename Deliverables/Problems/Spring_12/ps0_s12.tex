\documentclass[12pt]{exam}

\usepackage{ge05}
\usepackage{comment}
\usepackage{booktabs}
\usepackage[dvipdfm]{hyperref}
\urlstyle{rm}   % change fonts for url's (from Chad Jones)
\hypersetup{
    colorlinks=true,        % kills boxes
    allcolors=blue,
    pdfsubject={NYU Stern course GB 2303, Global Economy},
    pdfauthor={Dave Backus @ NYU},
    pdfstartview={FitH},
    pdfpagemode={UseNone},
%    pdfnewwindow=true,      % links in new window
%    linkcolor=blue,         % color of internal links
%    citecolor=blue,         % color of links to bibliography
%    filecolor=blue,         % color of file links
%    urlcolor=blue           % color of external links
% see:  http://www.tug.org/applications/hyperref/manual.html
}

% for ge05.sty
\def\ClassName{The Global Economy}
%\def\Category{Professor David Backus}
\def\Category{Backus \& Cooley}
\def\HeadName{Problem Set \#0}
\newcommand{\phm}{\phantom{--}}
\newcommand{\NX}{\mbox{\it NX\/}}

%\printanswers

\begin{document}
\parindent = 0.0in
\parskip = \bigskipamount
\thispagestyle{empty}%
\Head

\centerline{\large \bf \HeadName: Math \& Spreadsheet Review}
\centerline{Revised:  \today}

\medskip
{\it This is an individual assignment:  you may speak to others about it,
but what you hand in should be your own work.}

{\bf Hint:  This will be easier if you read the Math Review before you start.}

\begin{questions}

\begin{solution}
Brief answers follow,
but see also the spreadsheet posted on the course website.
\end{solution}

% --------------------------------------------------------------------
\question Exponents and logarithms (30 points).
A {\it production function\/} (next class) is a mathematical
model of the relation between a firm or a country's output
and its inputs, which here will be capital and labor.
We'll use the function
\begin{eqnarray}
    Y &=& K^\alpha L^{1-\alpha} ,
    \label{eq:production}
\end{eqnarray}
where $Y$ is the quantity of output,
$K$ is the quantity of capital (plant and equipment),
$L$ is the quantity of labor (number of employees),
and $\alpha = 1/3$ is a parameter (take this as given).
We'll set $K=100$ and $L = 50$.

\begin{parts}
\part
Compute output $Y$ by entering the following formula in
a spreadsheet:
\begin{verbatim}
= 100^(1/3)*50^(2/3)
\end{verbatim}
Make sure you understand what this does.
What value do you get?
(10~points)

\part
What is the natural logarithm of $Y$?
(Remember:  use the spreadsheet function {\tt LN}.)
(5~points)

\part Show that the production function can be written
\begin{eqnarray*}
    \log Y &=&  \alpha \log K + (1-\alpha) \log L ,
\end{eqnarray*}
where $\log x$ means the natural logarithm of the variable $x$.
What properties of logarithms do we need to derive this from
equation (\ref{eq:production})?
(10~points)

\part Compute $\log Y$ using the expression you verified in (c),
which translates into the spreadsheet formula
\begin{verbatim}
= (1/3)*LN(100) + (2/3)*LN(50)
\end{verbatim}
Verify that your answer is the same as the one you computed in (b).
%(If it isn't, at least one of them is wrong.)
(5~points)
\end{parts}

\begin{solution}
Answers include more digits than required to make sure they agree with yours.
Fewer digits is fine, even recommended.
\begin{parts}
\part Answer:  62.9961.
\part LN(62.9961) = 4.1431.
\part You need two properties of logarithms:
(i)~$\log(xy) = \log x + \log y$ and (ii)~$\log (x^a) = a \log x$.
\part Same answer as (b).
\end{parts}
\end{solution}

% --------------------------------------------------------------------
\question How much capital (40 points)?
Our mission is to use the production function
and market prices
to decide how much capital a firm should ``rent.''
Let us say that a firm sells output at price $p$ per unit,
rents/hires capital and labor at prices $r$ and $w$,
and produces according to the production function (\ref{eq:production}).
Its profit is therefore revenue minus cost:
\begin{eqnarray*}
    \mbox{Profit} &=& p Y - (rK + wL)
            \;\;=\;\; p K^\alpha L^{1-\alpha} - (rK + wL) .
\end{eqnarray*}
How much capital does the firm want?
To make this concrete, let us say that
$ p = 1$, $w = 1/2$, $r = 3/16 = 0.1875 $ (18.75\%), $L=27$,
and (always) $\alpha = 1/3$.
We'll solve the problem two ways:
using a spreadsheet, and using calculus.

\begin{parts}
\part In column A of a spreadsheet,
create a column of values for capital $K$ running from 1 to 100 in increments of 1.
Then, in column B, compute revenue for each value of $K$.
Does revenue increase or decrease as you increase capital?
(10~points)

\part In column C, compute cost for each value of $K$.
Does cost increase or decrease as you increase capital?
(10~points)

\part In column D, compute profit for each value of $K$.
At which value of capital is profit highest?
(It's helpful here to graph profit against capital,
but since graphics depend on the software you use,
I'll leave it to you to decide whether to try that.
My recommendation is to do it;
as they say, a picture is worth a thousand words.)
(10~points)

\part Now we do the same thing using calculus.
Think of Profit as a function of capital $K$.
Compute its derivative and set it equal to zero.
For what value of $K$ is the derivative equal to zero?
Why does this give us the profit-maximizing
value of $K$?
(10~points)
\end{parts}

\begin{solution}
\begin{parts}
\part As you increase $K$, revenue increases.
If you look carefully, you'll see it increases at a decreasing rate:
changes in revenue get smaller.
\part Cost increases, but at a constant rate:
changes are constant.
\part Profit hits its highest value of 10.5 at $K=64$.
\part We can solve the same problem using calculus.
We differentiate Profit with respect to $K$, set the result
equal to zero, and solve for $K$:
\begin{eqnarray*}
    \partial \mbox{Profit} / \partial K &=& \alpha p K^{\alpha-1} L^{1-\alpha} - r \;\;=\;\; 0 \\
    \Rightarrow \;\;\;  K^{1-\alpha} &=& L^{1-\alpha} (\alpha p /r) \\
    \Rightarrow \;\;\;\;\;\;\;\;  K &=& L (\alpha p /r)^{1/(1-\alpha)}
                \;\;=\;\; 64.
\end{eqnarray*}
\end{parts}
\end{solution}


% --------------------------------------------------------------------
\question Macroeconomic volatility (30 points).
The term ``business cycle'' refers to the periodic ups and downs of the economy,
evident in GDP (a measure of the total output of the economy)
and in many other things as well (employment, the stock market, retail sales, ...).
Years of experience tells us that
many things go up and down together, but some sectors go up and down
more.
We say they're more volatile, in the sense that the standard deviations
of their growth rates are larger.

Our mission is to verify both facts using quarterly data on GDP
and two of its expenditure components, consumption and investment.
Use the spreadsheet linked on the course outline page and
%
\href{http://pages.stern.nyu.edu/~dbackus/2303/ps0_data.xls}{here}.
%
It has four columns:
Column A is the date (the first month of the quarter),
column B is real GDP (labeled GDPC96),
column C is real private investment (labeled GPDIC96),
and column D is real personal consumption expenditures
(labeled PCECC96).
(The data are from FRED, the St Louis Fed's data portal.
It gives you access to a broad range of macroeconomic data
and has some graphics options, too.
It's a good site, worth a look when you have a few minutes.)

For each of the three data series,
compute simple growth rates over the period
1950Q1 to present using the formula
\begin{eqnarray*}
    g_t &=& 4 (x_t - x_{t-1})/x_{t-1} .
\end{eqnarray*}
(The ``4'' converts a quarterly growth rate to annual units.
There are other ways to do this, but there's a lot to be said for simple.)
Put these growth rates in separate columns, one for each series.

\begin{parts}
\part For each of the three series, compute the mean and standard
deviation.
How do the standard deviations compare?
Do they make sense to you?
(10~points)

\part What are the correlations of consumption and investment growth
with GDP?
Do they make sense to you?
Do you agree that these variables ``move up and down together
but have different volatilities''?
(A graph would be useful here, too, but is not required.)
(10~points)

\part Repeat (a,b) using continuously-compounded growth rates
\begin{eqnarray*}
    \gamma_t &=& 4 (\log x_t - \log x_{t-1}) .
\end{eqnarray*}
How do the means compare?  The standard deviations?
(10~points)
\end{parts}

\begin{solution}
For (a,b), the numbers are
%
\begin{center}
\begin{tabular}{lccc}
\toprule
            &   GDP   &  INV  &   CON  \\
\midrule
Mean        &  0.0324 &  0.0432 & 0.0334 \\
Std dev     &  0.0400 &  0.2048 & 0.0344 \\
\midrule
\multicolumn{2}{l}{Correlations} \\
GDP         &  1.0000 \\
INV         &  0.7943 & 1.0000 \\
CON         &  0.6289 & 0.2647 & 1.0000 \\
\bottomrule
\end{tabular}
\end{center}
%
What do we see?  The standard deviation is much larger for investment
than GDP,
and a little smaller for consumption.
The former is a classic feature of economic fluctuations:
most of the action is in what accountants call ``capex'':
production of new plant and equipment, which here includes housing.

For (c) the numbers are similar.
For some purposes, continuous compounding is cleaner.
You'll see it again later in the course.
%
\begin{center}
\begin{tabular}{lccc}
\toprule
            &   GDP   &  INV  &   CON  \\
\midrule
Mean        &  0.0321 &  0.0378 & 0.0331 \\
Std dev     &  0.0397 &  0.2036 & 0.0342 \\
\midrule
\multicolumn{2}{l}{Correlations} \\
GDP         &  1.0000 \\
INV         &  0.7924 & 1.0000 \\
CON         &  0.6268 & 0.2631 & 1.0000 \\
\bottomrule
\end{tabular}
\end{center}

There's some big picture content here worth noting.
We see in business cycles that lots of things move up and down together.
We see that here in the correlations with GDP.
They're even higher if we use year-on-year growth rates.
We also see that some things move more than others.
The example here is investment,
which is far more variable than GDP or consumption.


\end{solution}

\end{questions}


\vfill \centerline{\it \copyright \ \number\year \
NYU Stern School of Business}

\end{document}

