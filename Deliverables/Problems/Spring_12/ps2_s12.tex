\documentclass[12pt]{exam}

%\usepackage{ge05}
%\usepackage{comment}
\usepackage{booktabs}
\usepackage[dvipdfm]{hyperref}
\urlstyle{rm}   % change fonts for url's (from Chad Jones)
\hypersetup{
    colorlinks=true,        % kills boxes
    allcolors=blue,
    pdfsubject={NYU Stern course GB 2303, Global Economy},
    pdfauthor={Dave Backus @ NYU},
    pdfstartview={FitH},
    pdfpagemode={UseNone},
%    pdfnewwindow=true,      % links in new window
%    linkcolor=blue,         % color of internal links
%    citecolor=blue,         % color of links to bibliography
%    filecolor=blue,         % color of file links
%    urlcolor=blue           % color of external links
% see:  http://www.tug.org/applications/hyperref/manual.html
}

% for ge05.sty
%\def\ClassName{The Global Economy}
%\def\Category{Backus \& Cooley}
%\def\HeadName{Problem Set \#2}
\newcommand{\phm}{\phantom{--}}
\newcommand{\NX}{\mbox{\it NX\/}}
\newcommand{\POP}{\mbox{\it POP\/}}

\printanswers

\begin{document}
\parindent = 0.0in
\parskip = \bigskipamount
\thispagestyle{empty}%
%\Head

%\centerline{\large \bf \HeadName: Long-Term Economic Performance}
\centerline{\large \bf Problem Set \#2:  Long-Term Economic Performance}
\centerline{Revised:  \today}

\medskip
{\it I had software problems, had to kill off some of the formatting 
to get this to work. } 

\medskip
{\it You may do this assignment in a group of up to five people.
Whatever you hand in should be the work of your group.}

\begin{questions}

\begin{solution}
Brief answers follow,
but see also the spreadsheet posted on the course website.
\end{solution}

% --------------------------------------------------------------------
\question Sources of Korean success  (35~points).
The Republic of Korea (``South Korea'')
has been one of the great economic success stories of world history.
Since the end of the Korean War in 1953,
GDP per capita has risen by a factor of 15.
Over the same period, US income rose by a factor of 3.
As a result, the gap between the two countries has shrunk.
In 1953, average income in Korea was 11\% of US income,
but by 2007 (the most recent comparable number) it was 56\%.


Was Korea a classic productivity story,
or did capital formation and hours worked play more important
roles than in other countries?
We know, for example, that the saving rate and hours worked
are both unusually high.
Let's check the numbers and see where the differences
in GDP per person come from.

The numbers:
%
\begin{center}
\tabcolsep = 0.2in
\begin{tabular}{lcc}
\toprule
        &  Korea & United States \\
\midrule
GDP per person (Y/POP) &  \$23,850  & \$42,887 \\
GDP per worker (Y/L)    &  \$47,723  & \$84,342 \\
Capital-output ratio (K/Y)    &  4.41 &  3.25 \\
Capital per worker (K/L)      &  \$210,432  &  \$274,080 \\
Employment rate (L/POP)      & 0.500 &  0.508 \\
%Hour worked per year           &  2266 & 1799 \\
\bottomrule
\end{tabular}
\end{center}
%
The numbers come from the Penn World Tables, Version 7.0.
%The last one (hours worked) comes from the OECD.
All of them are for 2007.
%
\begin{parts}

\part What is the ratio of GDP per person in the two countries
(Korea over US)?
(5~points)

\part Use the production function to derive total factor productivity (TFP)
 in each country from the numbers in the table.
 What is the ratio of the two countries?  How does it compare to the ratio
 you computed in (a)?
 (10~points)

\part Overall, what factors contribute to the difference in GDP per person?
How important is capital?
What would the GDP ratio be if the capital-output ratio in Korea equalled that
of the US?
(10~points)

\part You have heard that Koreans work exceptionally long hours.
The OECD reports that the average employee in Korea worked 2266 hours in 2007,
while the average American worked only 1799 hours.
How would this information change your calculation of TFP?
How does it change your assessment of the relative productivity
of Korea and the US?
(10~points)
\end{parts}

\begin{solution}
Brief answers follow.
See the spreadsheet for the calculations.
%
\begin{parts}
\part The ratio is $0.556 = 23.850/42.887$:
Korea has, by this measure, a living standard equal to 55.6\% of the US's.
The rest of the question is devoted to explaining the sources
of this difference.

\part We compute productivity the usual way:  if the production
function is $ Y/L = A (K/L)^\alpha$ then $ A = (Y/L) / (K/L)^\alpha$
where (as usual) $\alpha = 1/3$.
The ratio of productivities is
$0.618 = 8.023/12.984$,
which is a bit higher than the ratio of GDP
per capita.

\part What we had in mind was two level comparisons,
one with the given capital data, the other when we
give Korea the same capital-output ratio as the US.
The traditional level comparison gives us
\begin{eqnarray*}
    \frac{(Y/\POP)_K}{(Y/\POP)_{US}} &=&
            \left( \frac{(L/\POP)_K}{(L/\POP)_{US}} \right)
            \left( \frac{A_K}{A_{US}} \right)
            \left( \frac{(K/L)_K}{(K/L)_{US}} \right)^{1/3} \\
            &=&  0.984 \times 0.618 \times 0.916
                \;\;=\;\; 0.556 .\phantom{sum^K}
\end{eqnarray*}

The second comparison is optional.
The idea is to redo the calculation when Korea's capital-output ratio is
the same as the US.
Since the ratio is $1.357 = 4.41/3.25 $, we divide Korea's $K/L$
by 1.357.
The new calculation (a ``counterfactual'') gives us
\begin{eqnarray*}
    \frac{(Y/\POP)_K}{(Y/\POP)_{US}}
            &=&  0.984 \times 0.618 \times 0.827
                \;\;=\;\; 0.503 .\phantom{sum^K}
\end{eqnarray*}
Note that only the last term changes.
This tells us that the lower capital-output ratio
drops GDP per person to 0.503.

\part This question is intentionally more demanding.
We modify the production function to include hours of work.
There's more than one route to this answer,
among them $ Y/L = A (K/L)^\alpha h^{1-\alpha} $
and $ Y/hL = A (K/hL)^\alpha  $.
Either way, productivity (``corrected'' for hours worked)
is now 0.0465 in Korea and 0.0878 in the US
(the use of hours data changes the units).
The ratio is 0.530, well below our earlier calculation of 0.618.
In words:  part of what we attributed to productivity before
was really long hours.
Put another way:  Korea looks better (relative to the US)
in a comparison of output per worker than in a comparison
of TFP, because output per worker includes a contribution
from long hours.

\end{parts}
\end{solution}

% --------------------------------------------------------------------
\question Argentina and Chile (35 points).
Argentina and Chile have each experienced
dramatic growth and painful reversals over the last century.
Argentina was one of the richest countries in the world in 1900,
but mixed economic performance since then has
dropped it back into the world of emerging markets.
Chile suffered a traumatic change in government in the 1970s,
but emerged a decade later as a fast-growing democracy.
Chile now has the highest per capita GDP in Latin America.
Your mission is to tell us how --- and maybe why
--- their experiences have differed.

\begin{center}
%\tabcolsep = 0.12in
\begin{tabular}{lccccc}
\toprule
Country   &  Year   &  POP   &  Employment &  Y/POP &  K/POP  \\
\midrule
Argentina \hspace*{0.1in} &  1960   &  20.6m &  \phantom{1}8.1m
                &  \phantom{1}7,838  &  12,713  \\
Argentina &  2004   &  39.1m &  16.2m             &  10,939 &  24,343  \\
Chile   &  1960  & 7.59m &  2.55m
                &  \phantom{1}5,086   &   16,666  \\
Chile     &  2004   & 15.67m\phantom{1} &  6.57m   &  12,678  &  29,437  \\
\bottomrule
\end{tabular}
\end{center}
%
In the table,
POP is population (millions), Employment is the number of people working
(millions), Y/POP is per capita GDP (2000 US dollars),
and K/POP  is per capita capital (2000 US dollars).
Data are from the Penn World Tables, version 6.3.

\begin{parts}
\part Compute the (average annual continuously compounded)
growth rates of GDP per capita and GDP per worker in the two countries
over the period reported in the table.  (10~points)

\part Use our growth accounting methodology to allocate growth in
output per worker to growth in TFP and capital per worker.
Which factor has been most important over the last 40+ years? (15~points)

\part Use the World Bank's
\href{http://info.worldbank.org/governance/wgi/sc_country.asp}{Governance Indicators}
to comment on the differences in the political environment between the two countries.
(Click on the chart for each country for a nice visual summary.)
What role do you think these institutional differences played in the countries'
economic performance?
(10~points)
\end{parts}


\begin{solution}
GDP per capita in Chile has gone
from well below Argentina in 1960 to significantly above in 2004.
How did this happen?
Most of this problem is concerned with aggregate data:
how much comes from capital, TFP, etc?
The last part is an opportunity to look further
into features of the two economies that might account
for what we find in the production function numbers.
%
\begin{parts}
\part The (continuously compounded annual)
growth rate of GDP per capita for Argentina is
\[
    \gamma \;=\; \log (10.939/7.838) / (2004 - 1960) \;=\; 0.758\%. \; (!)
\]
(We give 3 digits here to allow you to compare your own numbers,
but it's excessive accuracy, not justified by the quality of the data.)
Refer to the discussion of growth rates in ``Sources of Growth''
if you're not sure why this works.
A similar calculation gives us 2.076\% for Chile.
GDP per worker is GDP per capita times the population then
divided by the number of workers.
Its growth rates are 0.639\% (Argentina) and 1.572\% (Chile).
Bottom line:  faster growth in Chile, in part because a larger
fraction of the population is working.

\part We use the production function
\[
    Y/L \;=\; A (K/L)^{1/3} ,
\]
so TFP is $ A = (Y/L) / (K/L)^{1/3}  $.
In growth rates, with the adjustment for employment,
this translates into
\begin{eqnarray*}
    \gamma_{Y/POP} &=&  \gamma_{L/POP}  + \gamma_A
                + \alpha \gamma_{K/L}  ,
\end{eqnarray*}
where $\alpha = 1/3$ as usual.
This allows us to decompose growth in per capita GDP into
components due to the employment rate, productivity,
and capital per worker.
A few calculations give us

\medskip
\begin{center}
%\tabcolsep = 0.12in
\begin{tabular}{lcccc}
\toprule
            &  $Y/\POP$      &  $L/\POP$    &   $A$   &  $K/L$     \\
\midrule
\multicolumn{5}{l}{\it Argentina} \\
Growth rate (\%) &  0.758  &  0.119   &   0.186 &  1.358   \\
Contribution to growth (\%)
                &   0.758  &  0.119   &   0.186 &  0.453  \\
\midrule
\multicolumn{5}{l}{\it Chile} \\
Growth rate (\%) &  2.076  &  0.503   &   1.309 &   0.789   \\
Contribution to growth (\%)
                &   2.076  &  0.503   &  1.309  &   0.263  \\
\bottomrule
\end{tabular}
\end{center}

\medskip
Here ``growth rate'' is computed as in (a) and multiplied
by 100 to give us a percentage.
``Contribution'' modifies these growth
rates as they occur in the formula:  the growth rate of $K/L$ gets
multiplied by $\alpha = 1/3$ (its exponent in the production function).
The numbers tell us that most of the difference in growth
reflects differences in TFP growth.


The question calls for a slightly simpler calculation:
to compute the components in growth of GDP per worker.
It's the same, except we skip the growth rate of $L/\POP$.
In Argentina, for example, the growth rate of GDP per worker (0.639\%)
consists of growth in capital per worker (0.453\%)
and growth in TFP (0.186\%).

\part The World Bank's Governance Indicators include these
percentile ranks:

\begin{center}
\begin{tabular}{lcc}
\toprule
                        & Argentina & Chile \\
\midrule
Voice and accountability	& 57.3	& 82.0  \\
Political stability	& 45.3 & 	67.5 \\
Govt effectiveness	& 46.9 & 	83.7 \\
Regulatory quality	& 26.8 & 	91.4 \\
Rule of law		   & 32.7 & 	87.7 \\
Control of corruption	&	39.7 & 90.9 \\
\bottomrule
\end{tabular}
\end{center}

\bigskip
Almost all of the institutional indicators are stronger for Chile
than Argentina, so perhaps it's not surprising that economic performance
has been better.
Much of this has come in the last 25 years, as Chile has reestablished a
stable democracy with sensible economic policies.
Argentina continues to fluctuate between policies that favor long-term
performance and those that focus on the short term to the detriment of
the long term.

Our belief is that the strong economic and political
foundations Chile has laid will continue to generate higher growth
than we'll see in Argentina.
For one thing, we've had different parties in power over the last 20 years,
yet the same basic economic policies have been followed.
One we find intriguing is a privatized social security system,
which helped them develop a domestic capital market.

A temporary factor is copper prices,
which play an unusually important role in Chile.
It's a tricky measurement issue, since
high prices  tend to show up as high GDP
and therefore high TFP.
History tells us that probably won't last.
With that in mind, the government saved revenue from copper
for several years,
despite intense political pressure to spend it.
When the crisis hit, they had a pot of money they could use to mitigate
its effects.
\end{parts}
\end{solution}

% --------------------------------------------------------------------
\question Labor market conditions (30 points).
Your first day on the job at General Electric,
you are given 4 hours to collection information for a 5-minute
presentation to your group summarizing the labor market
conditions a manufacturer would face in Brazil, Poland, and Singapore.
Once you get over your initial panic, you contact your Global Economy
professor, who suggests you look at the
\href{http://pages.stern.nyu.edu/~dbackus/macro_resources.htm}{resource page},
especially
%
\begin{itemize}
\item The Bureau of Labor Statistics'
\href{http://www.bls.gov/fls/}{International Labor Comparisons}
page includes wage rates in a number of different countries,
collected on a comparable basis.
Click on a country name to get a summary table.

\item The World Bank's
\href{http://data.worldbank.org/topic}{Global Development Indicators}
includes
information about the education and literacy of the population.

\item The World Bank's Doing Business website includes
institutional information about the labor market,
labeled
\href{http://www.doingbusiness.org/data/exploretopics/employing-workers}
{Employing Workers}.

\end{itemize}
%
Use this information to put together a short report
summarizing labor market conditions in these three countries.


\begin{solution}
The basic tradeoff here is between cost (the wage data)
and quality, including the quality of the institutions
in which you'd be operating.  Here's a table:

\begin{center}
\begin{tabular}{lrrr}
\toprule
		& Brazil & Poland  & Singapore  \\
\midrule
Hourly compensation (USD)   & 10.08  & 8.01 & 19.10 \\
\midrule
Literacy of adults  (\%)    &  90 & 100 & 95 \\
Primary school enrollment (\%) & 127 & 97 & --- \\
Secondary school enrollment (\%) & 101 & 97 & --- \\
\midrule
Minimum wage (monthly, USD) & 300 & 386 & --- \\
Severance (weeks, 20 years experience)
                    & 33.3 & 0  & 0 \\
\midrule
Strictness of employment protection (index)
                & 2.75 & 1.9 & --- \\
\midrule
Average years of school  & 4.9 & 9.7 & 7.0 \\
\bottomrule
\end{tabular}
\end{center}

The lines represent different sources;
in the order they appear,
BLS, World Bank, Doing Business, OECD,
and UN via Nationmaster.

A quick summary:
\begin{itemize}
\item Singapore:  educated workers, flexible labor market, most expensive.
\item Brazil:  the least educated of the three, also the least flexible labor market,
but less expensive.
\item Poland:  similar cost to Brazil, better educated workers,
somewhat more flexible labor market.
\end{itemize}

\end{solution}


\end{questions}

\vfill \centerline{\it \copyright \ \number\year \
NYU Stern School of Business}

\end{document}

