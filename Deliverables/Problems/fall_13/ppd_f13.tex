\documentclass[12pt]{exam}

\usepackage{ge13}
\usepackage{comment}
\usepackage{booktabs}
\usepackage{hyperref}
\urlstyle{rm}   % change fonts for url's (from Chad Jones)
\hypersetup{
    colorlinks=true,        % kills boxes
    allcolors=blue,
    pdfsubject={NYU Stern course GB 2303, Global Economy},
    pdfauthor={Dave Backus @ NYU},
    pdfstartview={FitH},
    pdfpagemode={UseNone},
%    pdfnewwindow=true,      % links in new window
%    linkcolor=blue,         % color of internal links
%    citecolor=blue,         % color of links to bibliography
%    filecolor=blue,         % color of file links
%    urlcolor=blue           % color of external links
% see:  http://www.tug.org/applications/hyperref/manual.html
}

\usepackage{enumitem}
\setitemize{leftmargin=*, topsep=0pt}
\setenumerate{leftmargin=*, topsep=0pt}
\usepackage{attachfile}
    \attachfilesetup{color=0.5 0 0.5}
\usepackage{needspace}
% example:  \needspace{4\baselineskip} makes sure we have four lines available before pagebreak

% for ge05.sty
\def\ClassName{The Global Economy}
\def\Category{Professor David Backus}
%\def\Category{Backus \& Cooley}
\def\HeadName{Practice Problems D}
\newcommand{\phm}{\phantom{--}}
\newcommand{\NX}{\mbox{\it NX\/}}

%\printanswers

\begin{document}
\parindent = 0.0in
\parskip = \bigskipamount
\thispagestyle{empty}%
\Head

\centerline{\large \bf \HeadName:  Exchange Rates \& Crises}
\centerline{Revised:  \today}

\medskip
{\it This will not be collected or graded,
but it's a good way to make sure you're up to speed.
We recommend you do it before the next class.}


\begin{questions}
% --------------------------------------------------------------------
\question {\it Purchasing power parity for Big Macs.\/}
The Economist reports the following data for
local prices of Big Macs and US dollars in July 2012:

\begin{center}
\begin{tabular}{lccc}
\toprule
        & Big Mac Price  &  Exchange Rate    \\
        &(Local Currency)&(LCUs per Dollar)   \\
\midrule
Argentina       &  19.0  &  4.16  \\
Brazil          &  10.1  &  0.97 \\
India           &  89.0  &  56.2   \\
United States   &  4.33  &  1.00   \\
\bottomrule
\end{tabular}
\end{center}
\medskip

\begin{parts}
\part What is the dollar price of a Big Mac in each of these locations?
\part What exchange rates for the first three currencies would
equate the dollar prices of Big Macs to the US price?
\part Based on Big Mac prices,
how much are the first three currencies over- or under-valued
relative to the US dollar?
\end{parts}

\begin{solution}
The calculations are summarized in

{\small
\begin{center}
\begin{tabular}{lccccc}
\toprule
        & Big Mac  &  Exch Rate & Big Mac   &  Big Mac  & Overvaluation\\
        &(LCU)     &(LCU/USD)  &  (USD)&   Parity &  (percent)\\
        &(A)&(B)& (C) & (D) & (E)  \\
\midrule
Argentina       &  19.0  &  4.16  &  4.57 &  4.39  & $5.5$\\
Brazil          &  10.1  &  0.97  &  10.41&  2.33  & 140\\
India           &  89.0  &  56.2  &  1.58 &  20.55 & $-63$\\
United States   &  4.33  &  1.00  &  4.33 &  1.00  & 0.0\\
\bottomrule
\end{tabular}
\end{center}
}
See the spreadsheet for calculations
(download this pdf file, open it with the Adobe Reader or the equivalent,
and click on the pushpin):
\attachfile{ppd_f13_answerkey.xlsx}

\begin{parts}
\part Dollar prices of Big Macs are reported above as column (C),
computed as (A)/(B).

\part The calculation is reported in column (D).
The idea is to pick the exchange rate that makes the dollar price
of a Big Mac the same in both countries.

A more elaborate rationale.  The relative price of Big Macs is like a real exchange rate.
The real exchange rate is the ratio of prices converted to a common currency:
\begin{eqnarray*}
    \mbox{RER}  &=&  eP^*/P .
\end{eqnarray*}
Usually we use prices indexes for $P$ and $P^*$, here we use prices of Big Macs.

Mathematically, we set RER equal to one and solve for $e = P/P^*$.
In the table, we computed this as the ratio of entries on column (A) to the US
entry in the same column.
The results are reported in (D).
They give us a ``Big Mac Parity'' benchmark for what the exchange rate should be.

\part If we compare our calculation of the PPP exchange rate in (b) to the actual,
we can see how far off we are.
In the table, we compute ``overvaluation'' as the percentage difference between
column (D) and true exchange rates (B):  {\tt 100*[(D)/(B)-1]}.
We see that the Brazilian real is overvalued (Big Macs are expensive there)
and the Indian rupee is undervalued (Big Macs are cheap there).
Argentina is about even, at least at the official exchange rate.
Most people think overstates the value of their currency and that Argentine
Big Macs are cheaper than is reported here.
\end{parts}
\end{solution}

% --------------------------------------------------------------------
\question {\it Foreign exchange reserves.\/}
Countries often buy and sell foreign currency as part of their
day-to-day central bank operations.

\begin{parts}
\part Describe the central bank's balance sheet.
Where do foreign exchange reserves appear?
\part If private citizens choose to buy foreign currency
from the central bank, what happens to the central bank's balance sheet?
What limits how much of this the central bank can do?
\end{parts}

\begin{solution}
\begin{parts}
\part Foreign exchange reserves are an asset of the central bank.
\part The balance sheets look something like this:

\pagebreak
\begin{center}
\begin{tabular}{lr|lr}
\multicolumn{4}{l}{} \\
\multicolumn{4}{l}{Central Bank} \\
\midrule
Assets  &&  Liabilities \\
\midrule
Bonds &  10    & Money & 20 \\
FX reserves & 10 \\

\multicolumn{4}{l}{} \\
\multicolumn{4}{l}{Households \& firms (everyone else)} \\
\midrule
Assets  &&  Liabilities \\
\midrule
Money & 10 \\
FX    &  50 \\
Bonds & 180 \\
\multicolumn{4}{l}{}
\end{tabular}
\end{center}

\part Suppose private citizens buy 5 worth of ``FX'' (foreign currency)
from the central banks, paying in local currency (money).
Balance sheets change like this:

\begin{center}
\begin{tabular}{lr|lr}
\multicolumn{4}{l}{} \\
\multicolumn{4}{l}{Central Bank} \\
\midrule
Assets  &&  Liabilities \\
\midrule
Bonds &  10    & Money & 15 \\
FX reserves &  5 \\

\multicolumn{4}{l}{} \\
\multicolumn{4}{l}{Households \& firms (everyone else)} \\
\midrule
Assets  &&  Liabilities \\
\midrule
Money & 5 \\
FX    &  55 \\
Bonds & 180 \\
\multicolumn{4}{l}{}
\end{tabular}
\end{center}

The central bank is limited by the quantity of reserves it holds.
Once it runs out, it either finds a way to get more
or stops trading currencies.

\end{parts}
\end{solution}

% --------------------------------------------------------------------
\question {\it The trilemma in action.\/}
In 1992, the Bank of England found
that its commitment to maintain a quasi-fixed parity of the pound against
the Deutschemark forced it to raise short-term interest rates during
a recession.
What were its choices?
How does this illustrate the trilemma?

\begin{solution}
The trilemma says you can have at most two of these three things:
(i) independent monetary policy, (ii) fixed exchange rate,
and (iii) free capital mobility.

In the UK, they took (iii) as given, but (i) and (ii) were in conflict.
The economy was in a recession, so the central bank wanted to lower
interest rates.  But that conflicted with their agreement to maintain
a quasi-fixed exchange rate with European currencies, including the
Deutschemark.
In the end, they kept (i) and abandoned (ii).
\end{solution}


% --------------------------------------------------------------------
\question {\it Crisis triggers.\/}
Economic crises have been with us throughout recorded history.
Although they often come as a surprise, their forms are familiar
from past experience.

\begin{parts}
\part What are the classic crisis triggers?
\part What indicators would you use to assess
each source of crisis risk?
\part What are the standard responses to each one?
\part What would you do now if you were the benevolent dictator
of the European Union?
\end{parts}

\begin{solution}
\begin{parts}
\part Classic triggers:  government debt, finance/banking,
and fixed exchange rates.

\part Common indicators:
\begin{itemize}
\item Government debt:  lots of debt, continuing deficits,
and (most important) the political situation.
\item Banking:  not part of this course, but people would look
at capital ratios, nonperforming loans, and asset problems in
the economy (real estate is a standard source of problems
dating back to Roman times).
\item Fixed exchange rate:  overvaluation either from PPP or compared to
the recent past, reserves.
\end{itemize}

\part Common responses:
\begin{itemize}
\item Government debt:  fiscal discipline,
perhaps an IMF program to indicate commitment and finance
a transition period.
\item Banking:  if the issue is illiquidity,
the central bank supplies liquidity to the financial system;
if the issue is insolvency, it's important to recapitalize the
system and get it working again.
\item Fixed exchange rate:  give it up, let the currency float.
\end{itemize}

\part Where do we start?
\end{parts}
\end{solution}

\end{questions}

\vfill \centerline{\it \copyright \ \number\year \
NYU Stern School of Business}

\end{document}

