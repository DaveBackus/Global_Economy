\documentclass[12pt]{exam}

\usepackage{ge13}
\usepackage{comment}
\usepackage{booktabs}
\usepackage{hyperref}
\hypersetup{
    colorlinks=true,        % kills boxes
    allcolors=blue,
    pdfsubject={NYU Stern course GB 2303, Global Economy},
    pdfauthor={Dave Backus @ NYU},
    pdfstartview={FitH},
    pdfpagemode={UseNone},
%    pdfnewwindow=true,      % links in new window
%    linkcolor=blue,         % color of internal links
%    citecolor=blue,         % color of links to bibliography
%    filecolor=blue,         % color of file links
%    urlcolor=blue           % color of external links
% see:  http://www.tug.org/applications/hyperref/manual.html
}
\urlstyle{rm}   % change fonts for url's (from Chad Jones)

\usepackage{enumitem}
    \setitemize{leftmargin=*, topsep=0pt}
    \setenumerate{leftmargin=*, topsep=0pt}

\usepackage{attachfile}
    \attachfilesetup{color=0.5 0 0.5}
\usepackage{embedfile}
    \embedfile{ps0_f13_answerkey.xlsx}
% http://bay.uchicago.edu/tex-archive/macros/latex/contrib/oberdiek/embedfile.pdf

\usepackage{needspace}
% example:  \needspace{4\baselineskip} makes sure we have four lines available before pagebreak

% for ge05.sty
\def\ClassName{The Global Economy}
\def\Category{Professor David Backus}
%\def\Category{Backus \& Cooley}
\def\HeadName{Problem Set \#0}
\newcommand{\phm}{\phantom{--}}
\newcommand{\NX}{\mbox{\it NX\/}}
\renewcommand{\log}{\ln}

\printanswers

\begin{document}
\parindent = 0.0in
\parskip = 0.7\bigskipamount
\thispagestyle{empty}%
\Head

\centerline{\large \bf \HeadName: Math \& Spreadsheet Review}
\centerline{Revised:  \today}

\medskip
{\it This is an individual assignment:  you may speak to others about it,
but what you hand in should be your own work.

Suggestions:
\begin{itemize} \itemsep=0pt
\item Read the Math Review before you start.
\item Start Question 3 first to make sure you're able to download the data from FRED.
\item Read the syllabus about handing in a professional product.
\end{itemize}
}


\begin{questions}

\begin{solution}
Brief answers follow,
but see also the attached spreadsheet:
download this pdf file, open it with the Adobe Reader or the equivalent, 
and click on the pushpin:
\attachfile{ps0_f13_answerkey.xlsx} \\
{\it Warning:  If you don't see a pushpin above, my guess is you have a Mac.
The pushpin doesn't appear in Preview,
but you can use the Adobe Reader or the equivalent.}
\end{solution}



% --------------------------------------------------------------------
\question {\it Exponents and logarithms (30 points).\/}
A {\it production function\/} (next class) is a mathematical
model of the relation between a firm or country's output
and its inputs, which here will be capital and labor.
We'll use the function
\begin{eqnarray}
    Y &=& K^\alpha L^{1-\alpha} ,
    \label{eq:production}
\end{eqnarray}
where $Y$ is the quantity of output,
$K$ is the quantity of capital input (plant and equipment),
$L$ is the quantity of labor input (number of employees),
and $\alpha = 1/3$ is a parameter (take this as given, always).
We'll set $K=150$ and $L = 50$.

\begin{parts}
\part
Compute output $Y$ by entering the following formula in
a spreadsheet:
\begin{verbatim}
= 150^(1/3)*50^(2/3)
\end{verbatim}
Make sure you understand what this does.
What value do you get?
(10~points)

\part
What is the natural logarithm of $Y$?
(Remember:  use the spreadsheet function {\tt LN}.)
(5~points)

\part Show that the production function can be written
\begin{eqnarray*}
    \log Y &=&  \alpha \log K + (1-\alpha) \log L ,
\end{eqnarray*}
where $\log x$ means the natural logarithm of the variable $x$.
What properties of logarithms do we need to derive this from
equation (\ref{eq:production})?
(10~points)

\part Compute $\log Y$ using the expression you verified in (c),
which translates into the spreadsheet formula
\begin{verbatim}
= (1/3)*LN(150) + (2/3)*LN(50)
\end{verbatim}
Verify that your answer is the same as the one you computed in (b).
%(If it isn't, at least one of them is wrong.)
(5~points)
\end{parts}

\begin{solution}
Answers include more digits than required to make sure they agree with yours.
Fewer digits is fine, even recommended.
\begin{parts}
\part Answer:  72.1125.
\part LN(72.1125) = 4.2782.
\part You need two properties of logarithms:
(i)~$\log(xy) = \log x + \log y$ and (ii)~$\log (x^a) = a \log x$.
\part Same answer as (b).
\end{parts}
\end{solution}

% --------------------------------------------------------------------
\question {\it How much capital (40 points)?\/}
Our mission here is to use the production function
and market prices
to decide how much capital a firm should ``rent.''
Let us say that a firm sells output at price $p$ per unit,
rents/hires capital $K$ and labor $L$ at prices $r$ and $w$,
and produces output $Y$ according to the production function (\ref{eq:production}).
Its profit is therefore revenue minus cost:
\begin{eqnarray}
    \mbox{Profit} &=& p Y - (rK + wL)
            \;\;=\;\; p K^\alpha L^{1-\alpha} - (rK + wL) .
            \label{eq:profit}
\end{eqnarray}
How much capital does the firm want?
To make this concrete, let us say that
$ p = 1$, $w = 1/2$, $r = 3/16 = 0.1875 $ (18.75\%), $L=27$,
and (always) $\alpha = 1/3$.
We'll solve the problem two ways:
using a spreadsheet, and using calculus.

\begin{parts}
\part In column A of a spreadsheet,
create a column of values for capital $K$ running from 1 to 100 in increments of 1.
Then, in column B, compute revenue $pY = pK^\alpha L^{1-\alpha}$ for each value of $K$.
Does revenue increase or decrease as you increase capital?
(10~points)

\part In column C, compute cost ($rK+wL$) for each value of $K$.
Does cost increase or decrease as you increase capital?
(10~points)

\part In column D, compute profit for each value of $K$.
Turn in a graph with profit on the y-axis and capital on the x-axis,
with the axes clearly labeled.
At which value of capital is profit highest?
(10~points)

\part Now we do the same thing using calculus.
Think of profit --- equation (\ref{eq:profit}) --- as a function of capital $K$.
Compute its derivative and set it equal to zero.
For what value of $K$ is the derivative equal to zero?
Why does this give us the profit-maximizing
value of $K$?
(10~points)
\end{parts}

\begin{solution}
\begin{parts}
\part As you increase $K$, revenue increases.
If you look carefully, you'll see it increases at a decreasing rate:
changes in revenue get smaller.
\part Cost increases, but at a constant rate:
changes are constant.
\part Profit hits its highest value of 10.5 at $K=64$.
\part We can solve the same problem using calculus.
We differentiate Profit with respect to $K$, set the result
equal to zero, and solve for $K$:
\begin{eqnarray*}
    \partial \mbox{Profit} / \partial K &=& \alpha p K^{\alpha-1} L^{1-\alpha} - r \;\;=\;\; 0 \\
    \Rightarrow \;\;\;  K^{1-\alpha} &=& L^{1-\alpha} (\alpha p /r) \\
    \Rightarrow \;\;\;\;\;\;\;\;  K &=& L (\alpha p /r)^{1/(1-\alpha)}
                \;\;=\;\; 64.
\end{eqnarray*}
\end{parts}
\end{solution}


% --------------------------------------------------------------------
\question {\it Macroeconomic volatility (30 points).\/}
The term ``business cycle'' refers to the periodic ups and downs of the economy,
evident in GDP (a measure of the total output of the economy)
and many other things (employment, retail sales, stock prices, and so on).
Years of experience tells us that
many things go up and down together, but some components go up and down
more.
We say they're more volatile, in the sense that the standard deviations
of their growth rates are larger.

Our mission is to verify both facts using quarterly data on
GDP and two of its expenditure components, consumption and investment.
The first step is to download the data from FRED:
{\url{http://research.stlouisfed.org/fred2/}.}

\begin{table}[h]
\centering
\caption{FRED Data Codes.}
\label{tab:fred}
\vspace{1mm}
\begin{tabular*}{0.5\textwidth}{l@{\extracolsep{\fill}}l}
\toprule
Data        &  FRED Code\\
\midrule
GDP        &    GDPC96 \\
Consumption &    PCECC96\\
Investment  &    GPDIC96\\
\bottomrule
\end{tabular*}
\end{table}

The FRED series codes are given in Table \ref{tab:fred}.
Most people use FRED's Excel add-in to download the
data straight from FRED to a spreadsheet on their computer.
If you have difficulty with this, the following somewhat more cumbersome
method is close to foolproof:
\begin{itemize}
\item Go to FRED.
\item Type GDPC96 in the search box in the upper right corner.
That will generate a graph of real GDP.
\item Click on Edit Graph, a link at the bottom of the graph on the left.
\item Click on Add Data Series, a link at the bottom.
\item Enter the code for consumption and press enter.
You should now have two lines, one for GDP, the other for consumption.
\item Repeat the last two steps with investment.
\item Once you have all three lines on the graph, click on
Download Data in Graph, the link above the graph's upper left corner.
\end{itemize}
See also our FRED videos at \url{http://www.youtube.com/user/NYUSternGE}.
Whatever method you use,
you should end up with a spreadsheet that has dates in the first column
and the three series in the next three columns.

Once you have the data,
compute, for each series,
discretely-compounded annual rates of growth over the period 1950Q1 to 2013Q2.
The formula is
\begin{eqnarray*}
    g_t &=& \left[ \left({x_t}-x_{t-1}\right) /{x_{t-1}}\right] \times 100.
        \;\;=\;\; \left[ \left({x_t}/{x_{t-1}}\right) -1 \right] \times 100.
\end{eqnarray*}
The 100 at the end converts the (quarterly) growth rate to a percentage.
Place each growth rate in its own column.
Note that the growth rate for 1950Q1 requires data for 1949Q4.

Now you're ready to do some analysis:
\begin{parts}
\part For each of the three growth rates, compute the mean and standard
deviation.
How do the standard deviations compare?
%Do they make sense to you?
(10~points)
\part What are the correlations among the three variables?
%Do they make sense to you?
Do you agree that these variables ``move up and down together
but have different volatilities''?
(A graph would be useful here, too, but is not required.)
(10~points)

\part Compute the continuously-compounded growth rate of GDP using the formula
\begin{eqnarray*}
    \gamma_t &=& (\log x_t - \log x_{t-1}) \times 100 .
\end{eqnarray*}
How do the mean and standard deviation compare to those of the discretely-compounded growth rate?
Why are the two growth rates so similar?
(10 points)
\end{parts}

\begin{solution}
For (a,b), the numbers are
%
\begin{center}
\begin{tabular}{lrrr}
\toprule
            &   GDP   &  CON  & INV    \\
\midrule
Mean        &  0.82 &  0.84 & 1.10  \\
Std dev     &  0.97 &  0.85 & 4.65  \\
\midrule
\multicolumn{2}{l}{Correlations} \\
GDP         &  1.000 \\
CON         &  0.617 & 1.000 \\
INV         &  0.794 & 0.264 & 1.000 \\
\bottomrule
\end{tabular}
\end{center}
%
What do we see?  The standard deviation for investment
is much larger than for GDP,
and a little smaller than GDP for consumption.
The former is a classic feature of economic fluctuations:
most of the action is in what accountants call ``capex'':
production of new plant and equipment, which here includes housing.

For (c) the numbers are almost the same.
The mean is 0.81, the standard deviation is 0.96,
and the correlation between the two growth rates is 0.99996.
For some purposes, continuous compounding is cleaner.
We'll see it again for precisely that reason.

There's some big picture content here worth noting.
We see in business cycles that lots of things move up and down together.
We see that here in the correlations with GDP.
They're even higher if we use year-on-year growth rates.
We also see that some things move more than others.
The example here is investment,
which is far more variable than GDP or consumption.
You can see this clearly in the FRED graph, done
for year-on-year growth rates:

\vspace*{\parskip}
\centerline{\url{http://research.stlouisfed.org/fred2/graph/?g=fHo}}
\end{solution}

\end{questions}
\embedfilefinish

\vfill \centerline{\it \copyright \ \number\year \
NYU Stern School of Business}

\end{document}
