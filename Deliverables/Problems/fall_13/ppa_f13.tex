\documentclass[12pt]{exam}

\usepackage{ge13}
\usepackage{comment}
\usepackage{booktabs}
\usepackage{hyperref}
\urlstyle{rm}   % change fonts for url's (from Chad Jones)
\hypersetup{
    colorlinks=true,        % kills boxes
    allcolors=blue,
    pdfsubject={NYU Stern course GB 2303, Global Economy},
    pdfauthor={Dave Backus @ NYU},
    pdfstartview={FitH},
    pdfpagemode={UseNone},
%    pdfnewwindow=true,      % links in new window
%    linkcolor=blue,         % color of internal links
%    citecolor=blue,         % color of links to bibliography
%    filecolor=blue,         % color of file links
%    urlcolor=blue           % color of external links
% see:  http://www.tug.org/applications/hyperref/manual.html
}

\usepackage{enumitem}
\setitemize{leftmargin=*, topsep=0pt}
\setenumerate{leftmargin=*, topsep=0pt}
\usepackage{attachfile}
    \attachfilesetup{color=0.5 0 0.5}
\usepackage{needspace}
% example:  \needspace{4\baselineskip} makes sure we have four lines available before pagebreak


% for ge05.sty
\def\ClassName{The Global Economy}
\def\Category{Professor David Backus}
%\def\Category{Backus \& Cooley}
\def\HeadName{Practice Problems A}
\newcommand{\NX}{\mbox{\it NX\/}}

\printanswers

\begin{document}
\parindent = 0.0in
\parskip = \bigskipamount
\thispagestyle{empty}%
\Head

\centerline{\large \bf \HeadName: Production \& Capital Formation}
\centerline{Revised:  \today}

\medskip
{\it This will not be collected or graded,
but it's a good way to make sure you're up to speed.
We recommend you do it before the next class.}

\begin{questions}

\begin{solution}
Brief answers follow,
but see also the attached spreadsheet:
download this pdf file, open it with the Adobe Reader or the equivalent,
and click on the pushpin:
\attachfile{ppa_f13_answerkey.xlsx} \\
{\it If you don't see a pushpin above, my guess is you have a Mac.
The pushpin doesn't appear in Preview,
but you can use the Adobe Reader or the equivalent.}
\end{solution}

% --------------------------------------------------------------------
\question {\it Labor productivity.\/}
Consider our production function,
\begin{eqnarray*}
    Y &=& A K^\alpha L^{1-\alpha} .
\end{eqnarray*}
When needed, we'll use these numerical values:
$\alpha = 1/3$, $A=1$, $L=100$, and $K=250$.

\begin{parts}
\part What are the components of the production function?  What does it tell us?

\part A common summary number is the average product of labor,
 $Y/L$, often referred to simply as labor productivity.
Using the numbers given, what are output $Y$ and labor productivity $Y/L$?

\part Suppose $A$ rises from 1 to 1.25.
What happens to output?  Labor productivity?

%\part With $A=1$ again, suppose $K$ rises from 250 to 300.
%What happens to output?  Labor productivity?

\part Consider a similar country in which $Y=150$, $K=150$, and $L = 100$.
Which country has higher total factor productivity $A$?
Which has higher labor productivity $Y/L$?
\end{parts}

\begin{solution}
\begin{parts}
\part In order:  $Y$ is output, total production in a country, usually measured by real GDP;
$A$ is total factor productivity (TFP), a measure of how effectively inputs are used to produce output;
$K$ the quantity of physical capital, plant and equipment;
$L$ the number of employees working to produce output;
$\alpha$ is a parameter, which we set equal to 1/3
(one-third of output is paid to capital, two-thirds to labor).
\part We get $Y = 135.721$ and $Y/L = 1.357$.
\part If $A$ rises 25\%, output and output-per-worker rise proportionately:
$Y = 169.651$ and $Y/L = 1.697$.

\part Total factor productivity $A$ in the new country is 1.310, 
which is higher than the previous country.
Labor productivity $Y/L$ is 1.500, which is also higher.  
\end{parts}
\end{solution}


% --------------------------------------------------------------------
\question {\it Dynamics of the capital-output ratio.\/}
We often express the capital stock as a ratio to GDP ---
the so-called capital-output ratio.
Both are real variables;
the idea is that the ratio summarizes the capital intensity
of the economy.
Most countries --- rich, poor, and in between ---
have ratios between 2 and 3.

Over time, variation in the capital-output ratio reflects changes
in both its numerator and denominator.
The idea here is to run through some simple calculations
to remind ourselves how this works.
We'll use these numbers:

\begin{center}
%\tabcolsep = 0.2in
\begin{tabular}{lcc}
\toprule
Year \phantom{xxxxx}   & Real Investment ($I$)  &  Real GDP ($Y$) \\
\midrule
2005    &  23  &  107 \\
2006    &  25  &  110 \\
2007    &  25  &  114 \\
2008    &  27  &  116 \\
2009    &  15  &  110 \\
2010    &  18  &  113 \\
\bottomrule
\end{tabular}
\end{center}
%

\begin{parts}
\part Suppose the capital stock at the start of 2005 is $K = 210$
and depreciation is $\delta = 0.10$.
How does the capital stock evolve from 2006 to 2010?
The capital-output ratio?

\part Take as given the following:  If the growth rate $g$ and investment rate $I/Y$
are constant, then the capital-output ratio tends toward
\begin{eqnarray*}
    K/Y &=& [1/(\delta+g)] \ I/Y .
\end{eqnarray*}
We refer to this as its ``steady state'' value.
Using mean values from the table you constructed in (a),
what is the steady state capital-output ratio.
How close is it to the 2010 value?
\end{parts}


\begin{solution}
This problem is partly drill, partly an illustration of a broader
point:  that when we look at changes in $K/Y$ (or similar object),
a lot of the action comes from the denominator.
As a concrete illustration, China has very high saving and investment rates,
yet nevertheless has a value of $K/Y$ not much different from the US.
How can this be?  The answer is the growth rate of output $g$.
When you grow that fast, you need a lot of investment just to keep
$K/Y$ constant.
We'll see the same thing when we look at the ratio of
government debt to GDP.

\begin{parts}

\part The table below is based on the formula,
$K_{t+1} = (1-\delta)K_t + I_t$.
For details, see the spreadsheet.

\pagebreak
\begin{center}
\tabcolsep = 0.1in
\begin{tabular}{lcccc}
\toprule
Year \phantom{xx}   & $I$  &  $Y$ & $K$  & $K/Y$ \\
\midrule
2005    &  23  &  107 &  210.00  &  1.963 \\
2006    &  25  &  110 &  212.00  &  1.927 \\
2007    &  25  &  114 &  215.80  &  1.893 \\
2008    &  27  &  116 &  219.22  &  1.890 \\
2009    &  15  &  110 &  224.30  &  2.039 \\
2010    &  18  &  113 &  216.87  &  1.919 \\
\bottomrule
\end{tabular}
\end{center}
%

\part We calculate means of $g=1.150\%$ and $I/Y = 0.198$.
That gives us $K/Y = [1/(0.10 + 0.0150)] \times 0.198 = 1.778$,
which is (a little?) below what we see above.

This wasn't part of the question, but you might ask where the
steady state relation comes from.
Recall that the capital stock follows
\begin{eqnarray*}
    K_{t+1} &=& (1-\delta) K_t + I_t ,
\end{eqnarray*}
and the growth rate of output is
\begin{eqnarray*}
    Y_{t+1} &=& (1+g_{t+1}) Y_t .
\end{eqnarray*}
Taking the ratio of the first to the second, we have
\begin{eqnarray*}
    (1+g_{t+1})\frac{K_{t+1}}{Y_{t+1}}  &=& (1-\delta)\frac{ K_t}{Y_t}
                    + \frac{I_t}{Y_t} .
\end{eqnarray*}
In the steady state, $K_{t+1}/Y_{t+1} = K_t/Y_t = K/Y$, $I_t/Y_t = I/Y$, and $g_{t+1} = g$
are all constant.
That gives us the steady state relation used above.

The idea is that the capital-output ratio is affected by three things:
depreciation $\delta$, which reduces future capital; growth $g$,
which increases the denominator and decreases the capital-output ratio;
and investment, which increases future capital.
We haven't given much attention to $g$ so far,
but it will reappear when we study debt dynamics.

\end{parts}
\end{solution}

% --------------------------------------------------------------------
\question {\it Recovering from war in the Solow model.\/}
Solow's analysis tells us that productivity growth
is more important than capital formation for the long-term
performance of countries,
but it's a mistake to disregard capital formation altogether.
The standard example is war,
which can destroy a lot of the physical capital.
The subsequent recoveries typically follow the Solow model pretty well,
with growth rates high at first, then declining as capital catches up
to its prewar level.

We'll consider an economy that starts at its long-run
``stationary'' level of capital, then go on
to study its recovery following a 20\% drop.
%
\begin{parts}
\part Consider an economy whose parameters include
$A = 1$, $L=100$, $\alpha = 1/3$, $\delta = 0.1$, and $s = 0.2$.
Explain what each of these does.

\part
Suppose we start out with $K = 250$.
If you adapt the spreadsheet and compute $K$ for many periods,
what value does it seem to be heading toward?

\part Suppose you start at the steady state computed in (b),
but  the capital stock falls suddenly by 20\% (think:  war).
Enter this in the spreadsheet and see what happens.
What happens to the growth rate of GDP in subsequent years?
How many periods does it take for the capital stock to recover to
90\% of its steady state value?
\end{parts}

\begin{solution}
\begin{parts}
\part Most of these were covered in Question 1.
The new elements are the depreciation rate $\delta$
(depreciation equals $\delta$ times the current stock of capital)
and the saving rate (the fraction of income that's saved, which equals
the fraction of output devoted to investment in this model).
\part After 100 years, $K=282.81$.
[Not required, but the steady state calculation gives us
$ K = (sA/\delta)^{1/(1-\alpha)} L = 282.84$,
so we're pretty close.]

\part If we start at 226 (about 20\% lower than the steady state),
we get to 254 (halfway to the steady state) in 10-11 years.
The GDP growth rate declines steadily, much as we saw in Japan and Germany
following World War II.

\end{parts}
\end{solution}

\end{questions}

\vfill \centerline{\it \copyright \ \number\year \
NYU Stern School of Business}

\end{document}

