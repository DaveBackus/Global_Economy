\documentclass[12pt]{exam}

\usepackage{ge13}
%\usepackage{comment}
\usepackage{booktabs}
\usepackage{hyperref}
\urlstyle{rm}   % change fonts for url's (from Chad Jones)
\hypersetup{
    colorlinks=true,        % kills boxes
    allcolors=blue,
    pdfsubject={NYU Stern course GB 2303, Global Economy},
    pdfauthor={Dave Backus @ NYU},
    pdfstartview={FitH},
    pdfpagemode={UseNone},
%    pdfnewwindow=true,      % links in new window
%    linkcolor=blue,         % color of internal links
%    citecolor=blue,         % color of links to bibliography
%    filecolor=blue,         % color of file links
%    urlcolor=blue           % color of external links
% see:  http://www.tug.org/applications/hyperref/manual.html
}

\usepackage{enumitem}
    \setitemize{leftmargin=*, topsep=0pt}
    \setenumerate{leftmargin=*, topsep=0pt}
\usepackage{attachfile}
    \attachfilesetup{color=0.5 0 0.5}
\usepackage{needspace}
% example:  \needspace{4\baselineskip} makes sure we have four lines available before pagebreak

% for ge05.sty
\def\ClassName{The Global Economy}
\def\Category{Professor David Backus}
\def\HeadName{Problem Set \#2}
\newcommand{\phm}{\phantom{--}}
\newcommand{\NX}{\mbox{\it NX\/}}
\newcommand{\POP}{\mbox{\it POP\/}}
\renewcommand{\log}{\ln}

\printanswers

\begin{document}
\parindent = 0.0in
\parskip = \bigskipamount
\thispagestyle{empty}%
\Head

\centerline{\large \bf \HeadName: Long-Term Economic Performance}
%\centerline{\large \bf Problem Set \#2:  Long-Term Economic Performance}
\centerline{Revised:  \today}

\medskip
{\it You may do this assignment in a group.
Whatever you hand in should be the work of your group
and include the names of all of the contributors.}

\begin{questions}

\begin{solution}
Brief answers follow,
but see also the attached spreadsheet:
download this pdf file, open it with the Adobe Reader or the equivalent,
and click on the pushpin:
\attachfile{ps2_f13_answerkey.xlsx} \\
{\it If you don't see a pushpin above, my guess is you have a Mac.
The pushpin doesn't appear in Preview,
but you can use the Adobe Reader or the equivalent.}
\end{solution}

% --------------------------------------------------------------------
\question {\it Sources of Korean success  (35~points).\/}
The Republic of Korea (``South Korea'')
has been one of the great economic success stories of world history.
Since the end of the Korean War in 1953,
GDP per capita has risen by a factor of almost 20.
Over the same period, US income rose by a factor of 3.
As a result, the gap between the two countries has shrunk dramatically.
In 1953, average income in Korea was about 10\% of US income,
but by 2010 (the most recent comparable number) it was about 65\%.

Was Korea a classic productivity story,
or did capital formation and hours worked play more important
roles than in other countries?
We know, for example, that the saving rate and hours worked
are both unusually high.
Let's check the numbers and see where the remaining difference
in GDP per person comes from.

%Go to the Penn World Table
%\href{http://www.rug.nl/research/ggdc/data/penn-world-table|}{website},
Use the Penn World Table summary spreadsheet,

\vspace*{\parskip}
\centerline{\url{http://pages.stern.nyu.edu/~dbackus/2303/pwt80_GlobalEconomy.xlsx},}

to fill in this table for 2011:
%
\begin{center}
%\tabcolsep = 0.2in
\begin{tabular}{lcc}
\toprule
        &  South Korea & United States \\
\midrule
GDP per person ($Y/\POP$) \phantom{xxxx}&  \\
GDP per worker ($Y/L$)    &  \\
Capital-output ratio ($K/Y$)  & \\
Capital per worker ($K/L$)    & \\
Employment rate ($L/\POP$)      & \\
\bottomrule
\end{tabular}
\end{center}
\smallskip
%Note that you may have to compute some of the entries
%yourself from the numbers in the spreadsheet.

\begin{parts}

\part What is the ratio of GDP per person in the two countries
(Korea over US)?
(5~points)

\part Use the production function to derive total factor productivity (TFP)
 in each country from the numbers in the table.
 What is the ratio of the two countries?  How does it compare to the ratio
 you computed in (a)?
 (10~points)

\part Overall, what factors contribute to the difference in GDP per person?
How important is capital?
(10~points)

\part You have heard that Koreans work exceptionally long hours.
The OECD Employment Outlook reports that the average employee in Korea worked 2090 hours in 2011,
while the average American employee worked only 1787 hours.
How would this information change your calculation of TFP?
How does it change your assessment of the relative productivity
of Korea and the US?
(10~points)
\end{parts}

\begin{solution}
Brief answers follow.
See the spreadsheet for the calculations.
The table becomes
\begin{center}
%\tabcolsep = 0.2in
\begin{tabular}{lrrr}
\toprule
        &  Korea & USA & \phantom{x}Ratio\\
\midrule
GDP per person ($Y/\POP$) \phantom{xxxx}& 29,272 & 42,140 & 0.695  \\
GDP per worker ($Y/L$)                  & 58,864 & 93,038 & 0.633 \\
Capital-output ratio ($K/Y$)            & 3.981 & 3.146 & 1.265 \\
Capital per worker ($K/L$)              & 234,314 & 292,659 & 0.801 \\
Employment rate ($L/\POP$)              & 0.497 & 0.453 & 1.097 \\
\bottomrule
\end{tabular}
\end{center}

%
\begin{parts}
\part The ratio is $0.695 = 29,272/42,140$:
Korea has, by this measure, a living standard about 70\% of the US's.
The rest of the question is devoted to explaining the sources
of this difference.

\part We compute productivity the usual way from measures in output and inputs.
If the production function is $ Y/L = A (K/L)^\alpha$, then \\ $ A = (Y/L) / (K/L)^\alpha$
with (as usual) $\alpha = 1/3$.
Thus for Korea we have $A_{K} = 58,864/234,314^{1/3} = 955$.
The ratio of productivities is 0.681,
which is a little bit less than the ratio of GDP per capita.

\part What we have in mind is a level comparison:
\begin{eqnarray*}
    \frac{(Y/\POP)_K}{(Y/\POP)_{US}} &=&
            \left( \frac{(L/\POP)_K}{(L/\POP)_{US}} \right)
            \left( \frac{A_K}{A_{US}} \right)
            \left( \frac{(K/L)_K}{(K/L)_{US}} \right)^{1/3} \\
            &=&  1.097 \times 0.681 \times 0.929
                \;\;=\;\; 0.694 .\phantom{sum^K}
\end{eqnarray*}
You see here that most of the difference comes from productivity.

\part This question is intentionally more demanding.
We modify the production function to include hours of work.
There's more than one route to this answer,
among them $ Y = A K^\alpha (hL)^{1-\alpha} $.
Productivity (``corrected'' for hours worked)
is now 5.841 in Korea and 9.516 in the US.
(The use of hours data changes the units, so they're not comparable to the previous numbers.)
The ratio is 0.615, which is well below our earlier calculation of 0.681.
In words:  part of what we attributed to productivity before
was really a difference in hours worked.
\end{parts}
\end{solution}

% --------------------------------------------------------------------
\question {\it La Dolce Vita (35 points).\/}
Like most of Western Europe, Italy grew rapidly after World War II.
It differs from most other European countries in slowing
almost to a halt over the last decade.
The question is why.

We'll start by looking at the numbers.
Complete the following table using the same source as the previous question:
%
\begin{center}
%\tabcolsep = 0.2in
\begin{tabular}{lrrr}
\toprule
        &  1950 & 2000 & 2011 \\
\midrule
GDP per capita ($Y/\POP$) \phantom{xxxx}&  \\
GDP per worker ($Y/L$)    &  \\
Capital-output ratio ($K/Y$)  & \\
Capital per worker ($K/L$)    & \\
Employment rate ($L/\POP$)      & \\
\bottomrule
\end{tabular}
\end{center}
\medskip
%

\begin{parts}
\part Compute the (average annual continuously compounded)
growth rates of GDP per capita and GDP per worker
over the periods 1950-2000 and 2000-2011.
10~points)

\part Use our growth accounting methodology to allocate growth in
GDP per worker to growth in productivity and capital per worker.
Which factor changed most between the two periods? (15~points)

\part Use the World Bank's
\href{http://www.doingbusiness.org/rankings}{Doing Business}
rankings, and any other sources you deem relevant,
to assess Italy's business environment.
What are its strengths?  Its weaknesses?
What factors would you blame for Italy's current malaise?
(10~points)
\end{parts}


\begin{solution}
The idea is to take a quick look at Italy's economic performance.
To make the numbers easier to manage, I've divided
$Y/L$ and $K/L$ by 1000, which changes the units of TFP, too.
The growth rates, of course, are the same either way.
See the spreadsheet for details.
%
\begin{parts}
\part The numbers are
%
{\small
\begin{center}
%\tabcolsep = 0.2in
\begin{tabular}{lrrrrr}
\toprule
&&&& \multicolumn{2}{c}{Growth Rates} \\
\cmidrule(r){5-6}
        &  1950 & 2000 & 2011 & 1950-2000 & 2000-11 \\
\midrule
$Y/\POP$ \phantom{xxxx}&
          3.596 & 28.734 & 29.051 & 4.157 & 0.100 \\
$Y/L$  & 10.105 & 72.449 & 71.678 & 3.94 & (0.097)  \\
$K/Y$  & 2.718 & 3.446 & 4.769 & 0.475 & 2.954\\
$K/L$  & 27.465 & 249.637 & 341.831 & 4.414 & 2.857 \\
$L/\POP$  & 0.356 & 0.397 & 0.405 & 0.218 & 0.181 \\
\bottomrule
\end{tabular}
\end{center}
}
\medskip
%
Growth rates are percentages.  Numbers in parentheses are negative.


To take one number:  the (average annual continuously compounded)
growth rate of GDP per capita for the period 1950-2000 is
\[
    \gamma \;=\; \log (28.734/3.596) / (2000 - 1950) \;=\; 4.157\%.
\]
(We give 3 digits here to allow you to compare your own numbers,
but it's excessive accuracy, not justified by the quality of the data.)
The other growth rates are computed the same way.
If you're not sure why this works,
refer to the discussion of growth rates in the math review.

For the later period, the growth rate is 0.100\% (zero, really),
so there's been a sharp drop in economic performance.
If you look more closely, it's not simply a reflection
of the financial crisis;
growth stopped before that.

\part We use the production function
$    Y/L \;=\; A (K/L)^{1/3} $,
so TFP is $ A = (Y/L) / (K/L)^{1/3}  $.
In growth rates, this becomes
\begin{eqnarray*}
    && \phantom{xx} \gamma_{Y/L} \;\;=\;\;   \gamma_A  + \alpha \gamma_{K/L}  \\
    \mbox{1950-2000}: && \phantom{-}3.940 \;\;=\;\; 2.468 + 1.471 \\
    \mbox{2000-11}:   && -0.097 \;\;=\;\; -1.050 + 0.952 .
\end{eqnarray*}
We see here that almost all of the drop growth comes from productivity.
For whatever reason, productivity stopped growing.
In fact, it's going down. But why?

\part Italy remains one of the more prosperous countries in the world,
with a long and distinguished history and culture.
The World Bank ranks it 24th in GDP per capita.
Many of its institutions, however, are well below that.
Doing Business ranks Italy 73rd in overall ease of doing business,
84th in ease of starting a business,
131st in difficulty and expense of paying taxes,
and 160th in efficiency of enforcing contracts.
This is a picture, in short, of a country with some
fundamental economic problems.


The Economist Intelligence Unit comments in its {\it Country Commerce\/} report:
\begin{itemize}
\item ``The main attraction for foreign companies to set up operations in Italy is the
size of the domestic market. Deterrents include the complexity of Italy's legal
system; inefficiencies in public administration; inadequate infrastructure,
particularly in the south of the country; and often high regulatory barriers to
entry.''
\item ``Italy's rigid labour market consists mainly of two tiers of highly protected
workers with lifetime jobs and those newly hired with short-term contracts that
include little or no security. The European Commission has long called for
changes to labour market legislation but the contentious nature of such reform
in Italy resulted in very little progress.''
\end{itemize}

The World Economic Forum's {\it Global Competitiveness Report\/} says:
``Italy was 49th this year, with a lack of clear political direction over the past year increasing business
uncertainty. ...
Its labor market remains extremely rigid, hindering employment creation.
Italy's  financial markets are not sufficiently developed to provide
needed finance for business development.
Other institutional weaknesses include high levels of
corruption and organized crime and a perceived lack of
independence within the judicial system, which increase
business costs and undermine investor confidence.''

It's not hard to imagine that such things cut into productivity,
with the results we noted earlier.
\end{parts}
\end{solution}

% --------------------------------------------------------------------
\question {\it Labor market conditions (30 points).\/}
Your first day on the job at General Electric,
you are given 4 hours to collect information for a 5-minute
presentation to your group summarizing the labor market
conditions a manufacturer would face in Brazil, the Czech Republic, and Singapore.
Once you get over your initial panic, you contact your Global Economy
professor, who suggests you look at the
\href{http://pages.stern.nyu.edu/~dbackus/macro_resources.htm}{resource page},
starting with:
%
\begin{itemize}
\item The Bureau of Labor Statistics'
\href{http://www.bls.gov/fls/}{International Labor Comparisons},
especially the section titled
\href{http://www.bls.gov/fls/ichcc.htm}{Hourly Compensation Costs},
which includes wage costs in a number of countries,
collected on a comparable basis.

\item The World Bank's
\href{http://data.worldbank.org/topic}{Global Development Indicators}
and the
\href{http://www.barrolee.com/}{Barro-Lee dataset},
which include
information about the education and literacy of the population.

\item The World Bank's Doing Business website, which includes
institutional information about the labor market,
labeled
\href{http://www.doingbusiness.org/data/exploretopics/employing-workers}
{Employing Workers}.

\item The Economist Intelligence Unit's {Country Commerce Reports},
particularly the section on human resources,
which describes the legal and business environment governing employment.
To access the reports go to NYU's
\href{http://library.nyu.edu/vbl/}{Virtual Business Library},
click on Country Information, then EIU Country Commerce,
login as directed using your NYU id and pw,
click on Country Commerce again, and choose the country of interest.
\end{itemize}
%
Use this information to put together a short report
summarizing labor market conditions in these three countries.
A summary table or chart would be ideal.

\begin{solution}
The basic tradeoff here is between cost (the wage data)
and quality, including the skill of the labor force and
the quality of the institutions you'll be dealing with.
Here's a table of some of the numbers you might find,
but yours may differ depending on the source:

\begin{center}
\begin{tabular}{lrrr}
\toprule
		& Brazil & Czecho  & Singapore  \\
\midrule
Hourly direct pay (USD)           & 5.95   & 7.23& 15.71 \\
Hourly total compensation (USD)   & 11.20  & 11.95 & 24.16 \\
\midrule
Literacy of adults  (\%)    &  90 & 100 & 96 \\
%Primary school enrollment (\%) & 127 & 97 & --- \\
%Secondary school enrollment (\%) & 101 & 97 & --- \\
\midrule
Minimum wage (monthly, USD) & 350 & 440 & 0 \\
Severance (weeks, 10 years experience)
                    & 36 & 13 & 0 \\
\midrule
Strictness of employment protection (index)
                & 1.75 & 2.66 & --- \\
\midrule
Average years of school (age $\geq$ 15) & 7.6 & 12.1 & 9.1 \\
\bottomrule
\end{tabular}
\end{center}

The lines represent different sources;
in the order they appear,
BLS, World Bank GDI, Doing Business, OECD,
and Barro-Lee.

Some of the highlights from the human resources sections
of the EIU's {\it Country Commerce\/} reports:
\begin{itemize}
\item Brazil:
Brazil has a highly regulated and costly labour system for the formal sector, and
there continues to be a sizeable informal sector that employs many workers.

The 1988 federal constitution legalises unions, collective-bargaining negotiations
and the right to strike in both the public and private
sectors. The constitution also sets overtime rates, provides a monthly minimum
wage and regulates working hours. It lists a variety of labour entitlements,
including maternity leave, annual leave, worker's compensation,
social services, medical assistance and unemployment benefits.

Labour costs in Brazil are high. This is mainly because of the many mandatory
charges and taxes attached to employment. Although wages remain quite
moderate, they account for at most two-thirds of the total costs of hiring labour.

\item Czecho:
The Czech Republic has a highly skilled workforce, particularly in technology
and engineering. Educational and literacy levels are high. Companies report
few difficulties in recruiting skilled and unskilled workers, particularly in
industrial areas.

Czech labour law derives from the Communist-era Labour Code of
1965, though it continues to evolve. ...
Changes introduced in January 2011 exempt workers with tenure of
less than two years from mandatory three-month severance pay in the event of
involuntary termination. The revision, made with the goal of boosting
job-creation, cut
severance for newer employees to one month's wages for those with less than
one year of service and two months' wages for those with 1–2 years on the job.

\item Singapore:
Singapore's workforce is among the most stable and productive in Asia.
...
The workforce is greying rapidly, and there are fewer economically
active people in the younger age brackets. ...
...
Technically qualified local staff is available from local universities.

Absent any agreed notice period in a collective agreement,
an employee must receive one day of notice for fewer than 26 weeks of service,
one week of notice for 26 weeks to two years of service, two weeks' notice for
2–5 years of service, and four weeks' notice for five or more years of service.
...  Severance is usually covered by labor contracts.

\end{itemize}


My summary:
\begin{itemize}
\item Brazil:  lowest education level of the three, somewhat rigid labor market,
cost comparable to Czecho, much cheaper than Singapore.
\item Czech Republic:  the most educated labor force of the three, labor available,
comparable in price
 to Brazil, more rigid labor market (but evolving), part of the EU.
\item Singapore:  educated workers, limited supply of young workers,
flexible labor market, expensive.
\end{itemize}

\end{solution}


\end{questions}

\vfill \centerline{\it \copyright \ \number\year \
NYU Stern School of Business}

\end{document}

