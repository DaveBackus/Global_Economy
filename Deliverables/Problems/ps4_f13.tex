\documentclass[12pt]{exam}

\usepackage{ge13}
\usepackage{comment}
\usepackage{booktabs}
\usepackage{hyperref}
\urlstyle{rm}   % change fonts for url's (from Chad Jones)
\hypersetup{
    colorlinks=true,        % kills boxes
    allcolors=blue,
    pdfsubject={NYU Stern course GB 2303, Global Economy},
    pdfauthor={Dave Backus @ NYU},
    pdfstartview={FitH},
    pdfpagemode={UseNone},
%    pdfnewwindow=true,      % links in new window
%    linkcolor=blue,         % color of internal links
%    citecolor=blue,         % color of links to bibliography
%    filecolor=blue,         % color of file links
%    urlcolor=blue           % color of external links
% see:  http://www.tug.org/applications/hyperref/manual.html
}

\usepackage{enumitem}
    \setitemize{leftmargin=*, topsep=0pt}
    \setenumerate{leftmargin=*, topsep=0pt}
\usepackage{attachfile}
    \attachfilesetup{color=0.5 0 0.5}
\usepackage{needspace}
% example:  \needspace{4\baselineskip} makes sure we have four lines available before pagebreak

% for ge05.sty
\def\ClassName{The Global Economy}
\def\Category{David Backus}
%\def\Category{Backus \& Cooley}
\def\HeadName{Problem Set \#4}
\newcommand{\phm}{\phantom{--}}
\newcommand{\NX}{\mbox{\it NX\/}}
\newcommand{\POP}{\mbox{\it POP\/}}

\printanswers

\begin{document}
\parindent = 0.0in
\parskip = \bigskipamount
\thispagestyle{empty}%
\Head

\centerline{\large \bf \HeadName: Monetary \& Fiscal Policy}
\centerline{Revised:  \today}

\medskip
{\it You may do this assignment in a group.
Whatever you hand in should be the work of your group
and include the names of all of the contributors.}

\begin{questions}
% --------------------------------------------------------------------
\question {\it The Taylor rule in the Euro Area (50 points).\/}
You're a trader on Deutsche Bank's fixed income desk
and have just been transferred from New York to London.
You realize, among other things, that you must come
to terms quickly with differences between American and European monetary policy.
You wonder, given the chaos right now in the Euro Area, whether
the Taylor rule is a reasonable guide.
You review your Global Economy class notes and do the following:
\begin{parts}
\item Using data from FRED (see data guide below),
you plot inflation and GDP growth for the Euro Area.
(10~points)
\item You also plot the Euro Area interbank rate (a representative short-term interest rate)
and the Taylor rule for the same rate for the period 1999-present.
Since it's not clear what ``potential output'' is right now,
you use the growth rate version of the rule:
\begin{eqnarray*}
    i_t  \;=\;  r^* + \pi_t + 0.5 (\pi_t - \pi^*)
            + 0.5 (g_t - g^*)   ,
\end{eqnarray*}
where $g_t$ is the growth rate of real GDP.
As usual, you use year-on-year inflation and growth rates
and set $r^* = \pi^* = 2$.
You also set $g^* = 2$, but wonder whether another value would be appropriate.
(10~points)
\item How does the policy rate compare to the Taylor rule in 2009?
Do you think the ECB's policy was appropriate?
(10~points)
\item How does the policy rate compare to the rule now?
What justification does President Mario Draghi give in his most recent
\href{http://www.ecb.europa.eu/press/pressconf/2013/html/index.en.html}{press conference}?
Do you think the policy is appropriate?
(20~points)
\end{parts}
Data guide.  To implement the Taylor rule,
you'll need quarterly data for
\begin{itemize}
\item Real GDP (FRED code NAEXKP01EZQ661S):  use year-on-year growth rate.
\item Consumer prices (FRED code CP0000EZ17M086NEST):  use year-on-year growth rate.
\item Euro area interbank rate (FRED code IR3TIB01EZQ156N):  use as is.
\end{itemize}
You can download all of them from FRED.
It's not required, but you can also generate the graph(s) directly in FRED.

\begin{solution}
\begin{parts}
\item The graph is at

\url{http://research.stlouisfed.org/fred2/graph/?g=pHg}.

Right now, inflation (year-on-year) is about 0.8\% and GDP growth is about --0.2\%.
Relative to 2011 and 2012, inflation is significantly lower and growth a bit higher.

\item The interest rate and the Taylor rule are pictured together at

\url{http://research.stlouisfed.org/fred2/graph/?g=pQ4}.

In the figure, the blue line is the interbank rate and
the red line is the Taylor rule.
The link includes a description of how it was computed.\
Note that the two lines are (roughly) similar between 2000 and 2008; in this respect,
the Taylor rule has been a reasonable guide to ECB interest rate policy.
The two lines are even closer over this period if the target 
growth rate is 3\%.  

\item In 2009, the Taylor rule points to a negative interest rate,
which of course isn't feasible.
The ECB reduced the interbank rate to 1.75\%,
which was a less aggressive response than we saw from the Fed.

Would another value be appropriate?
One possibility:
You could suggest they should have reduced the rate further,
as the Fed did.
Or you could argue that the Taylor rule isn't really appropriate for the ECB,
with its mandated emphasis on inflation.

\item Right now, the Taylor rule is about 1\% above the actual interest rate.
In this respect, the ECB is being more expansionary than the rule suggests. 

Draghi said in his press conference:
\begin{quote}
Underlying price pressures in the euro area are expected to remain subdued ...
[and other factors suggest] we may experience a prolonged period of low inflation.  ...
Our monetary policy stance will remain accommodative for as long as necessary,
and will thereby continue to assist the gradual economic recovery.
\end{quote}
In short:  a low interest rate is justified by the inflation numbers and, furthermore,
this will contribute to growth.
Unlike similar statements from the Fed, the emphasis is clearly on inflation.
See
\href{http://www.ecb.europa.eu/press/pressconf/2013/html/is131205.en.html}{here}.

\end{parts}
\end{solution}


% --------------------------------------------------------------------
\question {\it Fiscal policy in India (50 points).\/}
%
After a century or more of poor macroeconomic performance,
India began growing rapidly in the 1980s and briefly approached Chinese rates of growth.
It remains a poor country, but much less so than thirty years ago.
Last year, however, growth dropped dramatically,
and analysts fear there could be worse to come.
Among their concerns:
\begin{itemize}
\item The ruling Congress party has done little to continue
the liberalization that led to high growth over the last thirty years.
\item Some of the most restrictive labor market regulations
anywhere in the world have limited the participation
of the large population of unskilled workers,
as firms invest in automation and hire skilled workers instead.
Think:  call centers and software, not manufacturing.
\item Infrastructure remains poor across the board.
\item The rupee dropped 20\% in August, as investors reacted
to the prospect of higher interest rates in the developed world
and the government's misguided attempt to impose capital controls.
\item Major redistribution programs, designed to help the poor and garner votes,
have exacerbated a continuing problem with government debt and deficits.
The Economist Intelligence Unit puts it this way in their Country Risk Service:
``India's sovereign risk rating
is constrained by poor fiscal management and high levels of public debt.''
\end{itemize}

Your mission is to focus on the last item:
to assess the fiscal policy risks to the economy.
Having some experience with such situations,
you collect some data:
%
\begin{center}
\begin{tabular}{lrrrr}
\toprule
         & 2011 &  2012  &  2013 & 2014  \\%
\midrule
Real GDP growth (percent)	& 7.75 & 3.99 & 5.68 & 6.23 \\
Inflation (percent)			& 9.30 & 8.10 &	7.85 & 7.35 \\
Interest rate on debt (percent) & 6.39 & 6.61 & 6.86 & 7.25 \\
Govt expenditures (percent of GDP) & 27.19 & 27.50 & 27.78 & 28.01 \\
Government deficit (percent of GDP) & 8.44 & 8.31 & 8.31 & 8.45 \\
Government primary deficit (percent of GDP)	& 4.20 & 3.90 & 3.77 & 3.68 \\
Government debt (percent of GDP) & 66.36	\\ % 66.84	66.39	66.69
\bottomrule
\end{tabular}
\end{center}
Entries for 2013 and 2014 are forecasts.

With this information in hand, you start to sketch out your report:
\begin{parts}
\item What is the difference between the government's deficit and primary deficit?
Why is the latter smaller?
(10~points)

\item Compute India's debt-to-GDP ratio for the period in the table.
Over the period 2011-2014, what factors account for the change in the ratio?
(20~points)

\item
How would your estimate change of the debt-to-GDP ratio at year-end 2014
if  (i)~the interest rate paid on debt rose by 2\%
or (ii)~the growth rate fell by 2\%?
(10~points)

\item
After skimming the EIU's Country Risk Report ---
and using your own good judgement ---
how would you rate the risk from government debt and deficits
to the Indian economy?
What specific concerns would you point to?
(10~points)
\end{parts}

Accessing the EIU's Country Risk Reports.
Go to NYU's
\href{http://library.nyu.edu/vbl/}{Virtual Business Library},
then click on Country Information,
EIU Country Risk Service (login as requested if off-campus),
Country Risk Service (again), and (in this case) India.
Choose the latest report and give it a quick read.


\begin{solution}
\begin{parts}
\item The difference between the primary and total budget balances
reflects interest payments on the debt.
In 2012,
the government deficit was $8.31$ percent of GDP
and the primary deficit was 3.90 percent.
The difference of 4.41 percent is interest payments on the debt.

\item This is a call to apply our debt dynamics tool.
The equation is
\begin{eqnarray*}
    \Delta ({B_{t}}/{Y_{t}})
            &=&
                (i_t-\pi_t) ({B_{t-1}}/{Y_{t-1}})
                - g_t ({B_{t-1}}/{Y_{t-1}})
             +    ({D_{t}}/{Y_{t}})  .
%             \label{eq:debt-dynamics}
\end{eqnarray*}
We'll label these terms (A), (B), and (C), as we did in class.

The results of these calculations are listed below.
See also the spreadsheet (save this pdf file, open it, and click on the pushpin):
\attachfile{ps4_f13_answerkey.xlsx}
%
\begin{center}
\begin{tabular}{lrrrrr}
\toprule
         & 2011 &  2012  &  2013 & 2014  & 2012-14\\%
\midrule
Real GDP growth 	& 7.75 & 3.99 & 5.68 & 6.23 \\
Inflation 			& 9.30 & 8.10 &	7.85 & 7.35 \\
Interest rate on debt & 6.39 & 6.61 & 6.86 & 7.25 \\
Govt expenditures  & 27.19 & 27.50 & 27.78 & 28.01 \\
Government deficit  & 8.44 & 8.31 & 8.31 & 8.45 \\
Primary deficit 	& 4.20 & 3.90 & 3.77 & 3.68 \\
Ratio of debt to GDP  & 66.36	\\ % 66.84	66.39	66.69
\midrule
(A) real interest & & --0.98 & --0.66 & --0.07 & --1.71 \\
(B) growth        & & --2.64 & --3.78 & --4.11 & --10.54 \\
(C) primary deficit & & 3.90 & 3.77	& 3.68 & 11.35 \\
\midrule
Total (A) + (B) + (C) & & 0.27 & --0.67 & --0.50 & --0.90 \\
Ratio of debt to GDP & & 66.63 & 65.96 & 65.46 \\
\bottomrule
\end{tabular}
\end{center}
\medskip

Overall, we see that debt falls from 66.36 to 65.46, a change of --0.90.
We see in the far right column that this modest amount reflects two offsetting
effects:  a decline of 10.54 from economic growth, and a rise of 11.25 from
primary deficits.
We also see a fortuitous contribution from negative real interest rates.

[A few brave souls noted that our debt dynamics equation is an approximation.
It's not necessary to go beyond that, but if you did, you get
a point or two for bravery.]

\item You may notice that the effect of these changes is the same:
both add $0.02*B_{t-1}/Y{t-1} $ to the change in $B/Y$.
Over the three years, this adds almost 4\% to the change,
with the effect that the ratio of debt to GDP rises by 3.08\%,
rather then falls by 0.90\%.
The point is that modest changes in conditions can change the debt situation in
a noticeable way.
Details on the calculations in the spreadsheet.

\item The EIU Country Risk Report includes these notable comments:
\begin{itemize}
\item Growth.  The government has undertaken a series of reforms to revive the torpid economy.
However, real GDP growth slowed sharply in the first quarter of the fiscal year.

\item Interest. In October the Reserve Bank of India (the central bank) again raised its main
policy interest rate, the repurchase rate, which now stands at 7.75\%,
in a bid to moderate inflation.
We forecast that the RBI's policy measures will help to moderate inflationary
expectations, although headline consumer price rises will remain high in 2013.

\item Political. The government's budgetary strategy has been repeatedly blown
off course by a series of unfavourable developments since 2008.  In
its budget for 2013/14, the UPA has set itself the target of cutting the fiscal
shortfall to the equivalent of 4.8\% of GDP. However, the sharp depreciation of
the rupee in June-August 2013 is likely to scupper the government's plans to
lower its subsidy bill, which accounts for around 16\% of public expenditure.
With elections looming, there is also a growing risk that the government will
boost spending in the run-up to the polls, in a bid to secure voter approval.

\item Two things we haven't discussed:  \\
(i) Banking.  Following a downgrade to India's rating for banking sector risk in October, in
view of continued stress on the sector, The Economist Intelligence Unit does
not envisage a further downgrade. \\
(ii) Foreign debt.  Although the level of domestic public debt is high, external debt as a
proportion of GDP remains low, and so it is unlikely that India will experience
an external debt crisis in 2014-15.

\end{itemize}

\end{parts}
\end{solution}

\end{questions}

\vfill \centerline{\it \copyright \ \number\year \
NYU Stern School of Business}

\end{document}

