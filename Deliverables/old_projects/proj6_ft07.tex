\documentclass[letterpaper,12pt]{article}

\RequirePackage{GE05}
% this inputs graphicx, too
\RequirePackage{comment}
\RequirePackage[hypertex]{hyperref}

\def\ClassName{The Global Economy}
\def\Category{Professor David Backus}
\def\HeadName{Group Project \#6}

\begin{document}
\parindent = 0.0in
\parskip = \bigskipamount
\thispagestyle{empty}%
\Head

\centerline{\large \bf \HeadName}%
\centerline{Revised:  \today}

\medskip
{\it Submit via Blackboard by 9am March 29.}
\medskip

You are delighted to have a summer internship at Citigroup Capital Markets.  
Your first rotation:  Fixed Income Research.  
On your first day, the Managing Director gives you a small project to get your feet wet. 
Noting that bond markets are driven largely by macroeconomic news, 
he asks you to write a report summarizing the prospects for the US 
economy over the coming year --- specifically, 
the growth rates of industrial production over the next 3, 6, and 12 months.  
Your report should include:  
%
\begin{itemize}
\item A list of what you think are the most useful economic indicators
along with a short description of what they are and why you think they 
are useful.  

\item A statistical analysis of what each of these indicators implies  
for economic activity over the next 3-12 months.  

\item A combination forecast that aggregates  --- in some way --- 
the information in each of your chosen indicators into a 
12-month forecast of the growth rate of industrial production.  

\item A narrative to be shared with clients 
that weaves these ingredients into a coherent story 
about bond markets and the US economy.  
If appropriate, point out aspects of the current situation 
that are uncertain or difficult to explain.  
\end{itemize}
%
He hands you a spreadsheet containing a large number of 
economic indicators and walks away.  


\vfill \centerline{\it \copyright \ \number\year \ 
NYU Stern School of Business}


%\end{document}
% --------------------------------------------------------------------------
\newpage
{The project is a loosely structured real-world problem;
it does not lend itself to a simple mechanical answer.
Here's a sketch of what I'd do: }  
\begin{itemize}
\item List of indicators.  
We've looked at many, take your pick.  
Among the most popular:  
employment, stock market, new claims for unemployment insurance, 
building permits, and housing starts.
The challenge here is that a 12-month forecast is very demanding:
you need a substantial lead in anything you use.  

\item Statistical analysis.  
I think cross-correlation functions are useful ways to get across leads and lags, but you may be able to get across similar 
information in different ways --- regressions with lags, for example.  

\item Combination.  
I'd use a multivariate regression, with the forward-looking growth 
rate of industrial production as the dependent variable
and several indicators as independent variables.
I'd include additional lags of the indicators and the dependent 
variable as additional independent variables.
You'd have to take a course in time series statistics 
to understand this fully, 
but the idea is that they allow more flexibility in the dynamic relations
among the variables.  

\item Narrative.  This would depend on what you find.
If you find that (say) housing variables are weak,
and that this weakness translates (through the regression)
into a forecast of weak growth, 
you could join Professor Roubini and many others and talk about that.

\end{itemize}

\end{document} 
