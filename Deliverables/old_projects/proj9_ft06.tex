\documentclass[letterpaper,12pt]{article}

\RequirePackage{GE05}
% this inputs graphicx, too
\RequirePackage{comment}
\RequirePackage{hyperref}
%\RequirePackage{eurosym}  % for euro symbol -- use \euro

\def\ClassName{The Global Economy}
\def\Category{Professor David Backus}
\def\HeadName{Group Project \#9}

\begin{document}
\parindent = 0.0in
\parskip = \bigskipamount
\thispagestyle{empty}%
\Head

\centerline{\large \bf \HeadName}%
\centerline{Revised:  \today}

\medskip
{\it Due April 27 at the start of class. }
\medskip

As the new marketing director of the Heineken brand for Heineken USA, 
you are beginning to realize that exchange rates compete with 
advertising campaigns for your attention.  
Heineken is the \#2 imported beer in the US  
(after Grupo Modelo's Corona)  
and an important source of revenue and earnings 
for your parent company, Heineken NV.   
You have learned that while your revenues are entirely in dollars, 
a substantial fraction of your costs are euros,   
making the bottom line sensitive to the dollar/euro exchange rate.  


You know very little about exchange rates, 
so you turn to your recently-hired assistant, 
the proud owner of a Stern MBA.  
You ask her to write a memo answering these questions:  
%
\begin{itemize}%\itemsep=0.01in \topsep=0.01in
\item How likely is the dollar/euro exchange rate to go up or down 
by 10\% or more in the coming year?  
%
%\begin{itemize}
%\item  Possible data source:  
%the Economist Intelligence Unit's 
%\href{http://www.countrydata.bvdep.com/cgi/template.dll?product=101&user=ipaddress}
%{Country Data} site.  
%(Click on:  Data Selection $\rightarrow$ Countries, then choose 
%Regional aggregate $\rightarrow$ Euro Area. 
%Then click on:  Data Selection $\rightarrow$ Series  
%$\rightarrow$ Monthly series $\rightarrow$ Exchange rate 
%$\rightarrow$ Exchange rate LCU:US\$ (av).)  [?? check??] 
%\end{itemize}

\begin{comment}
\item Is the euro under- or overvalued on a PPP basis?  
Possible data sources:  
%
\begin{itemize}
\item  The Economist's 
\href{http://www.economist.com/markets/Bigmac/index.cfm}
{Big Mac index}.  

\item The CPI-based real exchange rate between the Euro Zone and the US.  
[The Euro Zone's CPI is called the ``Harmonized Index of Consumer Prices'' (HICP).] 

\end{itemize}
\end{comment}

\item Based on PPP and any other evidence you think relevant, 
how would you expect the exchange rate to change over the next 12 months?  

\item If the euro rises by 10\%, 
how should we adjust the US dollar price per unit?  
You suggest she start with these ballpark figures:  
%
\begin{itemize} 
\item One unit:  one 6-pack (6 355-ml bottles or the equivalent).   
\item Current price per unit:  6.10 dollars.
\item Current exchange rate:  one euro is worth 1.184 dollars.  
\item Monthly volume:  14 million units. 
\item Production costs for US delivery:  1.96 euros.  
\item US distribution and marketing expenses:  1.75 dollars.  
\item Elasticity of demand in US market:  $-3.0$.  
\end{itemize}

\item With the change in price proposed above:  
\begin{itemize} 
\item By what percentage does your price change?  
\item By what percentage does your volume change?  
\item By what percentage does your profit change?  
\end{itemize}

\item How might your pricing strategy change if the value of the peso 
fell substantially?  

\end{itemize}

{\it Acknowledgement.\/}  This project is motivated by the research 
of Rebecca Hellerstein, 
an economist at the Federal Reserve Bank of New York 
who studies pricing strategies of international businesses, 
including brewers.  


\vfill
\centerline{\it \copyright \ \number\year \ NYU Stern School of Business}


% --------------------------------------------------------------------------
\newpage
{\bf Sketch of an answer:}  

\begin{itemize}

\item The idea is to estimate the annual standard deviation (volatility)  of the euro, 
either by taking year-on-year changes or extrapolating changes 
over shorter intervals.  
Or you could look at the implied volatility for currency options.  
Estimates vary, but a ballpark volatility is 0.15 = 15\%.  
If we assume a normal distribution (wrong, but a good place to start) 
then the probability of an annual change greater than 2/3 of a standard deviation in either direction is about 50\%.  
[The standard normal is greater than 2/3 about 25\% of the time;
in Excel, try ``1--NORMSDIST(2/3)''.]

\item The euro is overvalued by the Big Mac index (Big Macs are 
more expensive in the EuroZone) but that is relatively 
unhelpful in forecasting the exchange rate over the next year.  
Interest rates are lower in the EZ, but interest rates (like prices) 
have only modest power to forecast exchange rates.  
You won't go far wrong by assuming that the exchange rate is unpredictable.  

\item A good starting point:  
apply the pricing formula from Firms \& Markets:  
\[
    p  \;=\;  \mbox{MC} \left( \frac{1} {1+1/\varepsilon} \right) .
\]
where $\varepsilon$ is the price elasticity of demand ($-3$ in this case).  
We'll assume the elasticity is constant, which means the demand curve has the form 
\[
    q \;=\; a p^{\displaystyle \varepsilon} .
\]
With the given numbers, $\mbox{MC} = 1.96*1.184 + 1.75 = 4.07$ 
and the optimal price is $ 6.106$ (roughly what we report).  
If the euro rises 10\%, then $\mbox{MC}$ rises to 4.30 and the optimal price 
rises to 6.454.  
In words:  higher cost leads you to raise your price.  

Comment:  Note that we have violated PPP at the micro level.  
The price of beer doesn't change in Euroland (we didn't say that, but let's 
assume that's the case) and rises by 5.7\% in the US --- in dollars. 
In euros, the US price of Heineken has fallen by (approximately) 4.3\%,
so it's gotten less expensive in the US relative to Europe.  
Why the difference?  Because of local content, 
the US costs that go into US sales.  
No doubt there are other reasons, too, but you get the idea:  
there's no reason to expect 
PPP to work as anything but a rough approximation.  

Continuing with the calculations: volume falls by about 17\%  
(the price rise multiplied by the elasticity $-3$) and revenue 
falls by the difference.  
In proportional terms, profit falls by the same amount as revenue.  

Bottom line (in case you got lost):  
In business terms, your costs have gone up.  What do you do?  
You raise price, but this lowers your volume 
(and, presumably, your market share).  

\item What about the peso?  
Corona is a leading competitor in the imported beer market, 
so anything that affects its cost of production will 
indirectly affect Heineken. 
In fact, Heineken was the leading imported beer in the US for decades.
But following the peso crisis in December 1994, 
Corona's cost of production fell sharply (a weak peso), 
which they used to increase their market share 
(lower price, aggressive marketing, brilliant use of limes 
to cover a nondescript product).  
They've been number one ever since.  

\end{itemize} 

\end{document}

2005 version:  

You're in the treasury department of a large European retailer.  The company has a significant
cash position that it invests in ``money market'' securities (maturities under six months). Given
the low returns currently available on Euro-zone paper, your team is exploring the possibility of
investing farther afield.  After spending an hour at the Bloomberg terminal, you suggest to your
boss that she consider Australia, where money market rates are well above US and Euro-zone rates.
She recalls, however, that Australia has a current account deficit of more than 5\% of GDP. Noting
that current account deficits in Mexico, Korea, and Argentina were followed by massive
depreciations of their currencies, she asks you to write a short report summarizing your view of
the Australian dollar and addressing, in particular, these issues:
%
\begin{enumerate}
\item The current account:  how long has the deficit persisted, and why? Does it pose a risk to
the Auzzie dollar (AUD)?

\item The real exchange rate:  is the AUD expensive or cheap in a PPP sense?

\item Other factors that could lead to a significant depreciation over the next six months.
\end{enumerate}
%
Not knowing where to start, you contact your Global Economy professor, who suggests the following:
\begin{itemize}
\item For online sources in general, try NYU's
\href{http://www.nyu.edu/library/bobst/vbl/}{Virtual Business Library} and Nouriel Roubini's
\href{http://www.stern.nyu.edu/globalmacro/}{Global Macro site} (click on ``Countries").

\item For economic and political analysis of Australia, try the Economist Intelligence Unit's
\href{http://db.eiu.com/index.asp?layout=publicationTypes}{Country Intelligence site}, especially
the Country Profile.

\item For macroeconomic data, try the EIU's
\href{http://www.countrydata.bvdep.com/cgi/template.dll?product=101&user=ipaddress}{Country Data}
or the International Monetary Fund's
\href{http://ifs.apdi.net/imf/ifsbrowser.aspx?branch=ROOT}{International Financial Statistics}.

\item For current analysis, try investment bank reports (access varies, but use your contacts).
\end{itemize}


\vfill
\centerline{\it \copyright \ \number\year \ NYU Stern School of Business}

\end{document}
