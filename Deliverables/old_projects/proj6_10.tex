\documentclass[letterpaper,12pt]{article}

\RequirePackage{GE05}
% this inputs graphicx, too
\RequirePackage{comment}
\RequirePackage[hypertex]{hyperref}
    \hypersetup{colorlinks=true,urlcolor=blue,linkcolor=red}

\def\ClassName{The Global Economy}
\def\Category{Professor David Backus}
\def\HeadName{Group Project \#6}

\begin{document}
\parindent = 0.0in
\parskip = \bigskipamount
\thispagestyle{empty}%
\Head

\centerline{\large \bf \HeadName:  European Monetary Policy}%
\centerline{Revised:  \today}

\medskip
{\it Submit via Blackboard by 9am April 13.}
\medskip

As Chief Economist for Euro Zone Markets in Deutsche Bank's London office,
you have been asked to address your New York colleagues 
about monetary policy and interest rates in the Euro Zone:
the countries of the European Union that have adopted 
the euro as their currency.  
Your US counterpart asks specifically that you comment on these issues:   
%
\begin{itemize}
\item The Euro Zone:  What countries does it include, 
and how do they collaborate in setting monetary policy?  

\item The European Central Bank (ECB):  How do its mission 
and political mandate differ from that of the Federal Reserve?  

\item Economic conditions in the Euro Zone:   
What are inflation and output growth?
How are they expected to change during 2010?    
NB:  description only --- no statistical work called for.  
Use forecasts from a reliable source, such as the EIU.   

\item The recent policy stance of the ECB:  
How have short-term interest rates changed in the recent past?  
How do they compare now to what the Taylor rule suggests?    
 
\item Comparison with the US:  
How do you see short-term interest rates evolving over the next 6 months relative to the US?
If you see differences, do they reflect differences in 
circumstances or differences in policy orientation?  
\end{itemize}
%
Your report (5 pages maximum) or slides (10 slides maximum)  
should reflect its professional audience in both content and style. 
   
You start by putting together a collection of sources and 
links that might be helpful to your New York audience: 
%
\begin{itemize}

\item Descriptions and comparisons of central bank policies and procedures:
%
\begin{itemize}
\item From the ECB:  
\href{http://www.ecb.int/mopo/html/index.en.html}{monetary policy}. 
And, of course, their 
\href{http://www.ecb.int/ecb/educational/pricestab/html/index.en.html}
{cartoon} about price stability.  

\item From Vox EU, 
a blog focused on economic issues related to Europe:
entries on 
    \href{http://www.voxeu.org/index.php?q=node/30}{EU institutions} and 
\href{http://www.voxeu.org/index.php?q=node/32}{monetary policy}.  
I thought 
\href{http://www.voxeu.org/index.php?q=node/3010}{this one} 
was particularly good. 
See also the 
\href{http://www.ecb.int/pub/pdf/scpwps/ecbwp742.pdf}{complete paper}. 

\item From the St Louis Fed:  
\href{http://research.stlouisfed.org/publications/review/03/01/Pollard.pdf}
{two central banks}.   

%\item From the BIS:  
%\begin{itemize}
%\item \href{http://www.bis.org/publ/bispap09.htm}{Comparing monetary policy procedures}
%\item \href{http://papers.ssrn.com/sol3/papers.cfm?abstract_id=856944}
%{The Taylor rule in the Euro Zone} 
%\item \href{http://www.bis.org/publ/bppdf/bispap12s.pdf}
%{The ECB and money markets} 

\end{itemize}

\item Data and commentary:  
\begin{itemize}
\item Euro Zone aggregates are available from the ECB website 
(see above) and the Economist Intelligence Unit's 
Country Data (see Group Project \#1).    

\item Professor Roubini's 
\href{http://www.rgemonitor.com/}{Roubini Global Economics}
  is a good source of current commentary 
  (free access from NYU IP address).
  
\item The Financial Times' 
\href{http://blogs.ft.com/money-supply/}
{money supply blog} covers central banks worldwide.  

\end{itemize}

\end{itemize}


\vfill \centerline{\it \copyright \ \number\year \ 
NYU Stern School of Business}


\end{document}

% --------------------------------------------------------------------------
\newpage
Outline of an answer.

\end{document} 
