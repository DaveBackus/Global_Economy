\documentclass[letterpaper,12pt]{article}

\RequirePackage{comment}
\RequirePackage{GE05} % this inputs graphicx, too
\RequirePackage[hypertex]{hyperref}


\def\ClassName{The Global Economy}
\def\Category{Professor David Backus}
\def\HeadName{Group Project \#7}

\begin{document}
\parindent = 0.0in
\parskip = \bigskipamount
\thispagestyle{empty}%
\Head

\centerline{\large \bf \HeadName}%
\centerline{Revised:  \today}

\medskip
{\it Submit via Blackboard by 9am Apr 26.}
\medskip

As a principal in an international equity fund, 
you have been asked to address New York University's Board of Trustees 
about possible new investments.
Although NYU is well known for its conservative investment philosophy, 
they tell you that the apparent opportunities in emerging markets 
and NYU's global ambitions have led them to reconsider.
They ask you to make a recommendation to the Board 
on how they might invest up to \$500 million in a portfolio 
that will be managed on a one-time buy-and-hold basis with a 2-year
investment horizon.  
Your recommendation should focus on 
an emerging market country you think offers unusual promise 
and specific sectors and companies in that country 
that are likely to benefit from that promise.  


Your recommendation is to take the following form:   
%
\begin{itemize}%

\item A one-page executive summary outlining your country, 
sector, and company recommendations, and your rationale for each.  

\item An in-depth analysis of the risks and opportunities 
represented by the country you recommend.  
These risk and opportunities should be supported by 
data and analysis as far as possible.  

\item Further analysis of specific sectors and countries, 
along with supporting evidence of their suitability
as investments for NYU.  

\end{itemize}


Your experience to date has been primarily in developed countries, 
but you understand that emerging markets are largely ``macro'' markets, 
in the sense that stocks do well if the countries do.  
Accordingly, you look at a range of possible target countries 
and examine their economic fundamentals.  
Once you select a country, you use macroeconomic fundamentals 
to guide your search for appropriate sectors and companies.   
In addition to the usual sources on Global Economy 
\href{http://www.stern.nyu.edu/eco/b012303/Backus/ge_resources_db.htm}
{resource page},
you check several of the following:  
the Economist Intelligence Unit's 
\href{http://db.eiu.com/index.asp?layout=AllTitles}{Country Intelligence},
the World Bank's 
\href{http://www.doingbusiness.org/}
{Doing Business},  
\href{http://site.securities.com/}
{ISI's Emerging Markets}, 
Institutional Investor, 
Pensions and Investments Age, 
and the Financial Times.  


\vfill \centerline{\it \copyright \ \number\year \  
NYU Stern School of Business}


\end{document}

