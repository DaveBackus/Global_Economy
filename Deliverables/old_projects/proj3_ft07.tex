\documentclass[letterpaper,12pt]{article}

\RequirePackage{GE05} % this inputs graphicx, too
\RequirePackage[hypertex]{hyperref}

\def\ClassName{The Global Economy}
\def\Category{Professor David Backus}
\def\HeadName{Project \#3}

\begin{document}
\parindent = 0.0in
\parskip = 0.85\bigskipamount
\thispagestyle{empty}%
\Head

\centerline{\large \bf Group Project \#3}%
\centerline{Revised:  \today}

\medskip
{\it Submit via Blackboard by 9am February 8.}
\medskip

You are the head of strategic planning for an international wine producer and distributor
with sales primarily in the US and Europe.
Noting the recent success of Chilean wines in both markets,
your CEO wonders whether an investment in an Argentine vintner would make sense.
The upside:  the potential to produce high-quality wine at a cost
significantly below California and France.
The downside:  the risk of doing business in a new country,
one with a long history of political instability and poor economic performance.


Your assignment:  write a preliminary report (5 pages maximum,
including tables, charts, and executive summary) that outlines the primary opportunities of the proposed venture.  
Your report should include:  
%
\begin{itemize}%

\item A description of the major opportunities and challenges
you would face.  It is advisable to focus on the differences between
Argentina and the developed countries with which the CEO is familiar.

\item A brief comparison of Argentina with other developing
economies also suitable for your investment.

\item A description of the risks the venture would
face, along with suggestions for minimizing them. 
[We suggest you summarize the magnitude and likelihood of the primary
sources of risk in a ``risk map".  A risk map is a consultant's tool:
a two-dimensional ``matrix'' with probability (low to high) on the
vertical axis and magnitude (also low to high) on the horizontal
axis.] 

\end{itemize}
%
Cite evidence where possible.  
You can expect the CEO to ask for both your 
assessments and the reasoning behind them.  


You immediately check the Global Economy 
\href{http://www.stern.nyu.edu/eco/b012303/Backus/ge_resources_db.htm}
{resource page},
starting with its links to the 
\href{http://www.cia.gov/cia/publications/factbook/}{CIA Factbook}, 
the Economist Intelligence Unit's 
\href{http://db.eiu.com/index.asp?layout=AllTitles}{Country Intelligence},
and the World Bank's 
\href{http://www.doingbusiness.org/}
{Doing Business} 
and 
\href{http://devdata.worldbank.org/dataonline/}
{World Development Indicators}.  
Then you call a former classmate at McKinsey,
who sends you a copy of 
``%\href{http://weblinks1.epnet.com/externalframe.asp?tb=1&_ua=bt+ID++%22MCK%22+shn+1+db+buhjnh+bo+B%5F+88FC&_ug=sid+A5AEB81B%2D5A73%2D4D0C%2DA8B4%2DBD67F63C79D6%40sessionmgr2+dbs+buh+cp+1+D4A0&_us=hd+False+fcl+Aut+or+Date+frn+11+sm+ES+sl+%2D1+dstb+ES+ri+KAAACBZD00036509+9874&_uh=btn+N+6C9C&_uso=st%5B0+%2DJN++%22McKinsey++Quarterly%22++and++DT++20020601+tg%5B0+%2D+db%5B0+%2Dbuh+op%5B0+%2D+hd+False+C8DD&fi=buh_6586173_AN&lpdf=true&pdfs=1.5MB&bk=R&tn=22&tp=CAP&es=cs%5Fclient%2Easp%3FT%3DP%26P%3DAN%26K%3D6586173%26rn%3D18%26db%3Dbuh%26is%3D00475394%26sc%3DR%26S%3DR%26D%3Dbuh%26title%3DMcKinsey%2BQuarterly%26year%3D2002%26bk%3D&fn=11&rn=18}
{Micro lessons for Argentina}'' (posted on Blackboard).  
Finally, you search the student directory for classmates from Argentina.

%**** Risk map a la Daniel Oriesek 


\vfill \centerline{\it \copyright \ \number\year \ NYU Stern
School of Business}

%\end{document}
% Answers *************************************************
\newpage

{\bf Answer.  We discussed this in class at some length, 
but here's an outline of what we were looking for:}
\begin{itemize}
\item A thoughtful, 
professional-looking document that an executive would be pleased to receive.  

\item A sensible identification of the major risks and 
a quantitative basis for judging their probability and magnitude.

\item This is likely a long-term investment, esp if it's a startup (new vines) rather than a purchase of an existing winery.  
If so, long-term  political and economic stability are very important. 
That's something you could address:  
the advantages and disadvantages of the two approaches. 

\item Some of the risks:  
\begin{itemize}
\item Corruption.  Might include bribes for the various approvals needed to 
run a business.  
\item Political risk to property rights.  
Could a new government change 
the property rights of foreign investors?  
Has Kirchner's debt settlement increased the risk or reduced it?  
\item Taxes.  Could we be subject to high taxes? 
Will the export tax continue?  
\item Currency risk.  The value of the peso may vary through time, 
which might work in your favor or against.  
Related to inflation, which might also affect your costs.  
\item Inspections.  Are there mandated government inspections
in this business that could increase costs and lead to 
opportunities for corruption?  
\item Security.  Do we need our own?  Is it expensive?  
\item Labor costs.  Labor is relatively cheap, but the institutions
make hiring and firing difficult.  
\item Skill level of the labor force.  
Highly educated society, not clear how useful that is in this business.  
\item Environment.  Could pollution from local mining 
lead to contaminants in the product?  
Is bad weather (hail, for example) an issue?  
Earthquakes?
\end{itemize}

\item Some ways to mitigate them:   
\begin{itemize}
\item Organizational structure.  A direct investment in a startup 
is probably the most risky.  
Other arrangements you could consider:  
buy a local producer, form a joint venture, 
or enter into a marketing agreement.  
As you move through the list, the risk goes down, 
but so does your control over the operation.  
In a business like this, where quality is important, 
that could be an important consideration.  

\item Financial arrangements.  
You have country risk (taxes or even ownership could change) 
and currency risk, among others.  
For the former, local financing may help, as would the 
organizational issues mentioned above.
With currency risk, 
you might want to match assets and liabilities:  
eg, borrow in local currency.  
\end{itemize}

\end{itemize}

\end{document}
