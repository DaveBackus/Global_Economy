\documentclass[letterpaper,12pt]{article}

\RequirePackage{GE05} % this inputs graphicx, too
\RequirePackage[hypertex]{hyperref}
    \hypersetup{colorlinks=true,urlcolor=blue,linkcolor=red}

\def\ClassName{The Global Economy}
\def\Category{Professor David Backus}
\def\HeadName{Group Project \#2}

\begin{document}
\parindent = 0.0in
\parskip = 1.0\bigskipamount
\thispagestyle{empty}%
\Head

\centerline{\large \bf \HeadName:  Argentine Wine}%
\centerline{Revised:  \today}

\medskip
{\it Submit via Blackboard by 9am February 9.}
\medskip

You are the head of strategic planning for an international wine producer and distributor
with sales primarily in the US and Europe.
Noting the success of Chilean wines in both markets,
your CEO wonders whether an investment in an Argentine vintner would make sense.
The upside:  the potential to produce high-quality wine at a cost
significantly below California and France.
The downside:  the risk of doing business in a country
with a long history of political instability and 
poor economic performance.


Your assignment:  write a preliminary report (5 pages maximum,
including tables, charts, and executive summary) that outlines the primary opportunities of the proposed venture.  
Your report should look like a professional business document 
and include:  
%
\begin{itemize}%

\item A description of the major opportunities and challenges
you would face.  It is advisable to focus on the differences between
Argentina and one or two developed countries with which the CEO is familiar.

\item A brief comparison of Argentina with one or two other developing
economies also suitable for your investment.

\item An overall summary description of the risks the venture would
face, along with suggestions for minimizing them. 
[We suggest you summarize the magnitude and likelihood of the primary
sources of risk in a ``risk map".  A risk map is a consultant's tool:
a ``2 by 2 matrix'' with probability (low to high) on the
vertical axis and magnitude (also low to high) on the horizontal
axis.] 

\item A recommendation on whether or not to pursue this opportunity.  

\end{itemize}
%
The report should not simply be your opinion:  you should 
use as much evidence as you can to support your case.  
You can expect the CEO to ask for both your 
assessments and the reasoning behind them.  

You immediately check the Global Economy 
\href{http://www.stern.nyu.edu/eco/b012303/Backus/ge_resources_db.htm}
{resource page},
starting with its links to the 
\href{https://www.cia.gov/library/publications/the-world-factbook/index.html}
{CIA Factbook}, 
the Economist Intelligence Unit's 
\href{http://db.eiu.com/index.asp?layout=AllTitles}{Country Intelligence},
the World Bank's 
\href{http://www.doingbusiness.org/}
{Doing Business},  
various reports from \href{http://site.securities.com/}{ISI Emerging Markets}, and the World Economic Forum's 
\href{http://www.weforum.org/en/index.htm}
{Global Competitiveness Report}.
Then you call a former classmate at McKinsey,
who sends you a copy of 
``%\href{http://weblinks1.epnet.com/externalframe.asp?tb=1&_ua=bt+ID++%22MCK%22+shn+1+db+buhjnh+bo+B%5F+88FC&_ug=sid+A5AEB81B%2D5A73%2D4D0C%2DA8B4%2DBD67F63C79D6%40sessionmgr2+dbs+buh+cp+1+D4A0&_us=hd+False+fcl+Aut+or+Date+frn+11+sm+ES+sl+%2D1+dstb+ES+ri+KAAACBZD00036509+9874&_uh=btn+N+6C9C&_uso=st%5B0+%2DJN++%22McKinsey++Quarterly%22++and++DT++20020601+tg%5B0+%2D+db%5B0+%2Dbuh+op%5B0+%2D+hd+False+C8DD&fi=buh_6586173_AN&lpdf=true&pdfs=1.5MB&bk=R&tn=22&tp=CAP&es=cs%5Fclient%2Easp%3FT%3DP%26P%3DAN%26K%3D6586173%26rn%3D18%26db%3Dbuh%26is%3D00475394%26sc%3DR%26S%3DR%26D%3Dbuh%26title%3DMcKinsey%2BQuarterly%26year%3D2002%26bk%3D&fn=11&rn=18}
{Micro lessons for Argentina}'' (posted on Blackboard).  
Finally, you search the student directory for 
classmates from Argentina.  

%**** Risk map a la Daniel Oriesek 


\subsubsection*{Creating professional business documents} 

It's essential (and good practice for the future) 
that your report be a professional document.  
You'll see a wide range of formats and structures 
for business communications:
memos, reports, slides, flip charts, and so on.
There's no definitive best model,
but my favorite examples have the following elements:  
%
\begin{itemize}
\item Names.  
Make sure your names are on the document.  
Remember:  you are always selling yourself.  

\item Reader-friendly.  
The important information should be easy to find and read.  
If you don't know what the important information is, 
sit down and think before writing. 
Then you should make sure that this information is 
apparent to even a casual reader.
Remember:  everyone's busy these days, make it easy for them.  

\item Executive summary.  Start with a short focused summary.  
It could be 3-5 sentences, bullet points, or something else.  
    
\item Clear headings.  
If done well, the title and section headings should 
give you an outline of the argument.  
You don't want to overdo bold and italics, 
but selective use can help here.  

\item Effective graphics.  Attractive and informative graphics 
can be extremely helpful in getting your ideas across.
Try to make them understandable
on their own without reference to the text.  
Good titles and labels help.  

\item Clear conclusion.  Make sure there's no confusion 
about the point you're trying to make.
It should show up at the start and at the end, 
and be supported by the information you develop in the middle. 

\end{itemize}
%
Last piece of advice:  when you read business (or other) writing, 
pay attention to format and structure.  
If you see good ideas, copy them.  

For further information, see:   \\ 
Guide to business reports:  \url{http://www.businessballs.com/writing.htm} \\
Samples:  \url{http://www.stern.nyu.edu/eco/B012303/Backus/Writing_samples/}


\vfill \centerline{\it \copyright \ \number\year \ NYU Stern
School of Business}

\end{document}

% Answers *************************************************
\newpage

\subsection*{Sketch of an answer} 

We'll focus on the potential problems, but a good report 
would also explore ways to mitigate them.  
We discussed most of this in class, 
but here's an outline:  
%
\begin{itemize}
\item A thoughtful, 
professional-looking document that an executive would be pleased to receive.  

\item A sensible identification of the major risks and 
a quantitative basis for judging their probability and magnitude.

\item This is likely a long-term investment, esp if it's a startup (new vines) rather than a purchase of an existing winery.  
If so, long-term  political and economic stability are very important. 
Conversely, you might decide to enter in a way that leaves 
you some future flexibility.

\item Possible comparisons:  US or France for developed countries, 
Chile or South Africa for developing.  Examples only --- 
many other choices would be interesting.  

\item Some of the risks:  
\begin{itemize}
\item Corruption.  Might include bribes for the various approvals needed to 
run a business.  
\item Political risk to property rights.  
Could a new government change 
the property rights of foreign investors?  
\item Taxes.  Could we be subject to high taxes? 
Will the export tax continue?  
\item Currency risk.  The value of the peso may vary through time, 
which might work in your favor or against.  
Related to inflation, which might also affect your costs.  
\item Inspections.  Are there mandated government inspections
in this business that could increase costs and lead to 
opportunities for corruption?  
\item Security.  Do we need our own?  Is it expensive?  
\item Labor costs.  Labor is relatively cheap, but the institutions
make hiring and firing difficult.  
\item Skill level of the labor force.  
Highly educated society, not clear how useful that is in this business.  
\item Environment.  Could pollution from local mining 
lead to contaminants in the product?  
Is bad weather (hail, for example) an issue?  
Earthquakes?

\item Energy.  Are regular power outages an issue for our business?  

\end{itemize}

\item Some ways to mitigate them:   
\begin{itemize}
\item Organizational structure.  A direct investment in a startup 
is probably the most risky.  
Other arrangements you could consider:  
buy a local producer, form a joint venture, 
or enter into a marketing agreement.  
As you move through the list, the risk goes down, 
but so does your control over the operation.  
In a business like this, where quality is important, 
that could be an important consideration.  

\item Financial arrangements.  
You have country risk (taxes or even ownership could change) 
and currency risk, among others.  
For the former, local financing may help, as would the 
organizational issues mentioned above.
With currency risk, 
you might want to match assets and liabilities:  
eg, borrow in local currency.  
\end{itemize}

General discussion.  From the perspective of someone running a business, 
the point is that price (crudely approximated by GDP per capita) 
is not the only factor.  Other things (hidden costs) can crop up that may 
make a low price look less attractive.  
From the perspective of understanding where TFP comes from, 
the point is that institutions that increase the difficulty and cost
of running a business often translate into lower productivity.
That's the point of the McKinsey piece:
that institutional features of the Argentine economy raise 
costs of producers and lower productivity for the economy.  

\end{itemize}

\end{document}
