\documentclass[letterpaper,12pt]{article}

\RequirePackage{GE05}
% this inputs graphicx, too
\RequirePackage{comment}
\RequirePackage[hypertex]{hyperref}

\def\ClassName{The Global Economy}
\def\Category{Professor David Backus}
\def\HeadName{Group Project \#5}

\begin{document}
\parindent = 0.0in
\parskip = \bigskipamount
\thispagestyle{empty}%
\Head

\centerline{\large \bf \HeadName}%
\centerline{Revised:  \today}

\medskip
{\it Submit via Blackboard by 9am March 1.}

\begin{enumerate}

\item {\it Labor market theory.\/} As background to a
consulting assignment, you have been asked to work through an
example to illustrate the impact of
minimum wage legislation and payroll taxes on employment and
unemployment. You decide to adapt the framework of your 
{Global Economy} class notes, 
working through the demand for labor
of a single firm producing widgets according to the production
function $Y= A K^{1/3}L^{2/3}$. For simplicity, you assume that
$A=K=1$ and that the price of widgets is 1, too.  The supply of
labor rises with the wage $w$: $L^{s}=w^{2}$.
%
\begin{enumerate}

\item Describe, first, how the labor market might work in this
economy if there is no labor market regulation.  
What is the wage?
Employment? Unemployment?  (10~points)

\item Now consider the effect of introducing a minimum wage
$w_{m}$.  What are employment and unemployment rates if $w_{m}=1$?
If $w_{m}=0.5$?  How does this market differ from the one above?
How much would an unemployed person be willing to pay a recruiter
to find a job?  
(15~points)

\item Suppose workers differ in productivity, 
with some contributing effectively 1.5 units of labor 
per unit paid, and others contributing only 0.5 units.  
How would you expect a minimum wage of 0.75 
to affect the two types?  
(10~points)

\end{enumerate}

%\begin{comment}
Answer. The firm's optimal level of labor (its labor demand) is
such that the gain of adding one more worker is equal to its cost.
The marginal gain is the derivative of sales with respect to
labor: $\frac{2}{3}pAK^{1/3}L^{-1/3}$. The marginal cost is simply
the wage $w$. Therefore, since $p=K=A=1$, the demand for labor is
\[
    L^{d} \;=\; \left(\frac{3}{2}w\right)^{-3}.
\] 

\begin{enumerate}
\item The equilibrium is reached when demand equals supply. That
is:
\[ 
    \left(\frac{3}{2}w\right)^{-3} \;=\; w^{2}.
\] 
Solving for $w$, we obtain that the equilibrium wage is
\[ 
    w^{*} \;=\; \left(\frac{3}{2}\right)^{-3/5}=.78.
\] 
The level of employment is
\[ 
    L^{*} \;=\; \left(\frac{3}{2}w^{*}\right)^{-3}={w^{*}}^{2}=.61.
\] 
The unemployment rate is $0$. Since there is no distortion, all
individuals that would like to work at the equilibrium wage rate
get a job.

\item If $w_{m}=1$, the demand for labor is
$L^{d}=\left(\frac{3}{2}\times 1\right)^{-3}=.30$. Instead the
supply is $L^{s}=1^{2}=1$. Therefore the level of employment is
$.30$. The unemployment rate is $\frac{1-.3}{1}=70\%$. Every
unemployed will be willing to pay a recruiter up to the difference
between the ongoing wage ($1$) and his/her reservation wage. The
reservation wage is different across unemployed, and can be read
off the labor supply curve.

%%%%%%%%%%%%%%%%%%%%%%%%%%%%%%%%%%%%%%%%%%%%%%%%%%%%%%%%%%%%%%%%%%%%%%%%%%%%

\begin{figure}[h]
\begin{center}
\begin{picture}(300,200)(-30,-20)%

\footnotesize%
\put(-30,0){\vector(1,0){300}}%
\put (0,-20){\vector(0,1){200}}%
\put(265,-14){$L$}%
\put(-15,175){$w$}%
\put (154,157){$L^{s}(w)$}%
\put(44,157){$L^{d}(w)$}%

% Demand Curve
\qbezier(45,145)(90,41)(165,10)%

% Supply Curve
\qbezier(45,25)(80,41)(175,140)%

\qbezier[60] (0,92)(65,92)(128,92)%
\qbezier[40] (71,0)(71,46)(71,93)%
\qbezier[40] (128,0)(128,46)(128,93)%

\qbezier[40] (0,62)(50,62)(95,62)%
\qbezier[30] (95,0)(95,31)(95,62)%

\put(-38,60){$w^{*}=.76$}%
\put(-33,90){$w_{m}=1$}%

\put(95,-18){$.61$}%

\put(65,-18){$.30$}%
\put(130,-18){$1$}%

\end{picture}
\end{center}
\caption{With and without unemployment.} \label{fig:minimum1}
\end{figure}

%%%%%%%%%%%%%%%%%%%%%%%%%%%%%%%%%%%%%%%%%%%%%%%%%%%%%%%%%%%%%%%%%%%%%%%%%%%%

The wage $w_{m}=.5$ is below the equilibrium rate. This means that
the minimum wage is not binding. The outcome in terms of wage and
employment is going to be the same as in part (a).

\item You might guess that the minimum wage would  affect 
the low-skilled workers more than the high-skilled. 
Why?  Because the high-skilled are worth more than the minimum 
wage, so they will be hired (presumably) something close to their 
productivity.
But if the productivity of the low-skilled is below the minimum wage, 
they won't be hired. 
The result is that the unemployed will consist predominantly 
(perhaps entirely in this setup) 
on the low-skilled. 

\end{enumerate}
%\end{comment}

% ----------------------------------------------------------------------
\item {\it Labor market practice.\/}  
Your first day on the job at General Electric, 
you are given 4 hours to prepare
a 5-minute presentation to your group summarizing the 
labor market issues a manufacturer would face in 
Belarus, France, and Ireland.   
Once you get over your initial panic, you contact your Global Economy 
professor, who suggests you look at:    
%
\begin{itemize}
\item The Economist Intelligence Unit's various 
\href{http://db.eiu.com/topic_view.asp?pubcode=CP&title=Country+Profile}{reports} 
and 
\href{http://www.countrydata.bvdep.com/cgi/template.dll?product=101&user=ipaddress}
{databases};


\item The OECD's  \href{http://www.oecd.org/topic/0,2686,en_2649_37457_1_1_1_1_37457,00.html} {Employment Resources}; 
% (esp the Employment Outlook)

\item The Bureau of Labor Statistics'
\href{http://www.bls.gov/fls/hcompsupptabtoc.htm}{Foreign Labor Statistics};

\item The World Bank's
\href{http://www.doingbusiness.org/}
{Doing Business} 
and
\href{http://devdata.worldbank.org/dataonline/}
{World Development Indicators}.  

\end{itemize}
%
You quickly turn this data into a series of charts and bullet points 
that highlight the salient differences across these countries.  (35~points) 

%\begin{comment}
Answer.  Consider the following data:  

\begin{center}
\begin{tabular}{lcccc}
\hline\hline
Indicator   & Belarus & France & Ireland  \\
\hline\hline 
1. Unemployment rate & 2.6 & 9.7  &  4.5   \\

2. Employment rate   & & 62.3 & 67.1  \\
3. GDP per worker    & 23,139 & 56,909 & 65,925\\
4. Hours per year    & & 1535 & 1638  \\
5. Hourly direct pay & & 16.93  & 19.86  \\
6. Hourly total cost & & 24.63  &  22.76   \\
7. Difficulty of firing & 40  & 40 & 30  \\
8. Employment rigidity & 27 & 56 & 33  \\
9. Educ of adults    &  &  7.9   &  9.4 \\
10. Scientific literacy &   & 500 & 513  \\
\hline\hline 
\end{tabular}
\end{center}
Sources:  Economist, OECD, 
BLS, Doing Business, and NationMaster.com.  
Most recent year available.   

What do we make of this?  
Belarus remains a mystery;
we'd guess labor is cheap, 
but there's little information available from standard sources. 
For the others:  the major difference 
is in labor market regulations, which make employment 
more rigid in France.  
No one would make a decision on the basis of this information alone, 
but it gives you a sense of what to look for.  
%\end{comment}

% ----------------------------------------------------------------------
\item {\it Trade liberalization.} You have been asked to explain
to an audience of consultants why trade liberalization is win-win:  
why all countries can benefit.  
You realize that they want to
believe in free trade because their clients do, but like many
people around the world they don't quite believe it themselves.
You decide that the best approach is to construct a numerical
example.

You tell a story about the European Union (country 1) and Romania
(country 2), each of which produces cellular phones and shirts by
means of labor.  Their productivity levels in the two activities
are reported in the table below. (One unit of labor produces 5
cell phones in Romania, ...)
%\begin{table}[h]
\begin{center}

\begin{tabular}{||l|c|c||}
\hline\hline%
                     &   Cell Phones       &      Shirts    \\%

\hline\hline%
EU (country 1)       &   $\alpha_{1}=20$   & $\beta_{1}=10$  \\%
\hline%
Romania (country 2)  &   $\alpha_{2}=5$    & $\beta_{2}=8$   \\%
\hline\hline%
\end{tabular}
%\caption{Productivity levels.}\label{tab:prod}
\end{center}
%\end{table}
For the sake of the example, you assume that product and labor
markets are competitive in both countries and each has a total of
100 units of labor.
%
\begin{enumerate}

\item You start by illustrating the case of no trade. You compute
the relative price of shirts to cell phones in each country and
draw their consumption possibilities frontiers. (10~points)

\item Next you show that if the two countries allow trade, each
will benefit from specializing (the EU in cell phones and Romania
in shirts).   
Explain why residents in both countries are better off than with no trade. (10~points)

\item Lobbyists for the textile industry in the EU have been
arguing that competition from developing countries such as Romania
is unfair because it is supported by extremely low wages. 
Show that if wages in Romania approached EU levels, 
trade would stop altogether. 
Show, in addition, that this outcome would leave both
Romanian and EU residents worse off. (10~points)

\end{enumerate}


%\begin{comment}
Answer. Let's use $a$ to denote cell phones and $b$ to denote
shirts.
%
\begin{enumerate}

\item In absence of trade, since markets for the two products are
competitive, it will be the case that in the EU
%
\[ 
    p_{a} \;=\; \frac{w_{1}}{\alpha_{1}} , \;\;\;
    p_{b} \;=\; \frac{w_{1}}{\beta_{1}},
\] 
%
and in Romania
%
\[ 
    p_{a} \;=\; \frac{w_{2}}{\alpha_{2}} , \;\;\;
    p_{b} \;=\; \frac{w_{2}}{\beta_{2}}.
\] 
%
Therefore the relative prices are going to be
${p_{a}}/{p_{b}}= {\beta_{1}}/{\alpha_{1}} = {1}/{2}$ in
the EU and
${p_{a}}/{p_{b}} = {\beta_{2}}/{\alpha_{2}}= {8}/{5}$ in
Romania.

%%%%%%%%%%%%%%%%%%%%%%%%%%%%%%%%%%%%%%%%%%%%%%%%%%%%%%%%%%%%%%%%%%%%%%%%%%%%

\begin{figure}[h]
\begin{center}
\begin{picture}(300,200)(-30,-20)%

\footnotesize%
\put(-30,0){\vector(1,0){300}}%
\put (0,-20){\vector(0,1){200}}%
\put(265,-14){$b$}%
\put(-15,175){$a$}%

%\put (154,57){Possibilities with trade}%
%\put(17,17){Possibilities without trade}%

\put(0,60){\line(5,-4){75}}%
%\qbezier[100](0,150)(100,75)(200,0)%

\put(-25,58){$500$}%
\put(68,-18){$800$}%

\end{picture}
\end{center}
\caption{No trade:  Possibilities in Romania.} \label{fig:gains1}
\end{figure}

%%%%%%%%%%%%%%%%%%%%%%%%%%%%%%%%%%%%%%%%%%%%%%%%%%%%%%%%%%%%%%%%%%%%%%%%%%%%

\begin{figure}[!]
\begin{center}
\begin{picture}(300,200)(-30,-20)%

\footnotesize%
\put(-30,0){\vector(1,0){300}}%
\put (0,-20){\vector(0,1){200}}%
\put(265,-14){$b$}%
\put(-15,175){$a$}%

%\put (154,57){Possibilities with trade}%
%\put(17,17){Possibilities without trade}%

\put(0,150){\line(1,-2){75}}%
%\qbezier[100](0,150)(100,75)(200,0)%

\put(-25,148){$2,000$}%
\put(68,-18){$1,000$}%

\end{picture}
\end{center}
\caption{No trade:  Possibilities in the EU.} \label{fig:gains2}
\end{figure}

%%%%%%%%%%%%%%%%%%%%%%%%%%%%%%%%%%%%%%%%%%%%%%%%%%%%%%%%%%%%%%%%%%%%%%%%%%%%

\item Trade will be mutually beneficial if the relative price
satisfies
%
\[ 
    \frac{1}{2} \;<\; \frac{p_{a}}{p_{b}}<\frac{8}{5}.
\]
%
In fact, if this is the case, both possibility frontiers expand.

If the EU specializes in cell phones and Romania in shirts, the
international prices will be $p_{a}=\frac{w_{1}}{\alpha_{1}}$ and
$p_{b}=\frac{w_{2}}{\beta_{2}}$, respectively. Therefore the above
condition can be rewritten as
%
\[ 
    \frac{1}{2} \;<\; 
    \frac{w_{1}}{w_{2}}\frac{\beta_{2}}{\alpha_{1}}<\frac{8}{5}
\] 
%
or
%
\[ 
    \frac{5}{4} \;<\; \frac{w_{1}}{w_{2}} \;<\; 4.
\] 
%\pagebreak

\item Given the current productivity levels in the two countries,
any legislation that led to a decrease in the ratio
$w_{1} < w_{2}$ below $ {5}/{4}$ would imply that the EU
is better off by producing shirts domestically, rather than import
them from Romania. In turn, this would imply a reversal to
no-trade. Given the above discussion, this would make Romanians
(and EU citizens) worse off.
\end{enumerate}
%\end{comment}


\end{enumerate}


\vfill \centerline{\it \copyright \ \number\year \ NYU Stern
School of Business}

\end{document}
