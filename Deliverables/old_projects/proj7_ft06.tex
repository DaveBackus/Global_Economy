\documentclass[letterpaper,12pt]{article}

\RequirePackage{GE05}
% this inputs graphicx, too
\RequirePackage{comment}
\RequirePackage{hyperref}

\def\ClassName{The Global Economy}
\def\Category{Professor David Backus}
\def\HeadName{Group Project \#7}

\begin{document}
\parindent = 0.0in
\parskip = \bigskipamount
\thispagestyle{empty}%
\Head

\centerline{\large \bf \HeadName}%
\centerline{Revised:  \today}

\medskip
{\it Due March 30 at the start of class. }
\medskip

As the manager of a European bond fund, 
you wonder whether this is a good time to invest in short-term 
Turkish debt (money market securities with maturities under 2 years).  
Your primary concern:  that inflation might reduce the real returns 
on such investments.  
Over most of the 1990s, 
Turkey experienced inflation rates close to 100\% a year.  
In recent times, 
inflation has fallen to single digits, 
yet short-term interest rates have fallen more slowly.  
The question is whether interest rates are high enough to make Turkish 
securities attractive.  

You decide to draft a report for your team focussing on the following
 questions:  
%
\begin{itemize}

\item What were the economic and political 
causes of Turkish inflation in the 1990s? Are they still relevant?

\item How do you see Turkish inflation evolving over the next two years?  

\item What other factors affect your assessment of Turkish debt over the 
next two years?

\item Overall, do you believe that Turkish debt will be a good investment relative to comparable Euro Zone instruments?  

\end{itemize}
%
Not knowing where to start, you contact your Global Economy professor, 
who writes:
\begin{itemize}

\item For an economic and political analysis of Turkey, try the Economist Intelligence Unit's
\href{http://db.eiu.com/index.asp?layout=publicationTypes}{Country Intelligence site}, especially
the Country Profile and the Country Risk Service.

%\item For an analysis of Turkey's fiscal situation, try this
%\href{http://www.olis.oecd.org/olis/2005doc.nsf/43bb6130e5e86e5fc12569fa005d004c/1c9236589dacdf13c1256fa800827a0d/$FILE/JT00178336.PDF}{OECD
%report}.

\item For macroeconomic data, try the EIU's
\href{http://www.countrydata.bvdep.com/cgi/template.dll?product=101&user=ipaddress}{Country Data}
or the International Monetary Fund's
\href{http://ifs.apdi.net/imf/ifsbrowser.aspx?branch=ROOT}{International Financial Statistics}.

\end{itemize}

\vfill
\centerline{\it \copyright \ \number\year \ NYU Stern School of Business}

%\end{document}

% --------------------------------------------------------------------------
\newpage
{\bf Sketch of an answer:}  

This is really a traditional inflation story:  money-financed deficits 
lead to high inflation.
What led to deficits?  A turbulent political situation.  
What led to their end?  A crisis, followed by political agreement 
to get spending under control.  
What do we see for the near future?  
My sense is that things look relatively good:  the government 
has been reasonably disciplined.
One thing that may help:  the possibility of entering the EU, 
which imposes some discipline on what any government can do.  
A long-term issue:  whether populist pressures lead to policies
like we saw in the 1990s.  There's always a chance of a relapse.  

A good answer would make these points, have data to support them, 
and have an informed discussion of Turkish politics.  
The data should include a look at Turkish real interest rates, 
which have been very high (6-8\%), esp compared to 
Euro Zone rates.   

One last thought:  it's not uncommon to see high real interest rates 
following a decline in the inflation rate.  We saw it in the US in the early 1990s, and we see it today in Brazil.  
These are potentially good investment opportunities, but
there's always a risk of a return to inflation.  

\end{document} 