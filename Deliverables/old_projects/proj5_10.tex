\documentclass[letterpaper,12pt]{article}

\RequirePackage{GE05}
% this inputs graphicx, too
\RequirePackage{comment}
\RequirePackage[hypertex]{hyperref}
    \hypersetup{colorlinks=true,urlcolor=blue,linkcolor=red}

\def\ClassName{The Global Economy}
\def\Category{Professor David Backus}
\def\HeadName{Group Project \#5}

\begin{document}
\parindent = 0.0in
\parskip = 0.75\bigskipamount
\thispagestyle{empty}%
\Head

\centerline{\large \bf \HeadName:  Fixed Income Research}%
\centerline{Revised:  \today}

\medskip
{\it Submit via Blackboard by 9am March 30.}
\medskip

You are delighted to have a summer internship in Barclay's  
capital markets group.  
Your first rotation:  Fixed Income Research.  
On your first day, the Managing Director gives you 
a small project to get your feet wet. 
Noting that bond markets are driven largely by macroeconomic news, 
she asks you to write a report summarizing the near-term prospects for the US 
economy, specifically the first three quarters of the year.  

Your report should be a professional business document, no more than 5 pages long, and include:  
%
\begin{itemize}
\item A description of the most recent quarterly growth rates of US GDP 
--- the last ten years, let us say.  

\item A list of what you think are the most useful indicators
of future economic growth 
along with a short description of what they are and why you think they 
are useful.  

\item A qualitative analysis of recent behavior of the same indicators 
and whether they point to higher growth or something else.  

\item An overall assessment of the growth prospects of the US 
 economy for the first three quarters of the year.  
 
%\item A narrative to be shared with clients about the factors 
%affecting the near-term performance of the US economy 
%including, if you wish, a discussion of factors not 
%captured by the indicators described above.  
\end{itemize}
%
She hands you a spreadsheet containing a large number of 
economic indicators and walks away.  

Keep in mind:  
\begin{itemize} 
\item If you do not recall what a professional business document
looks like, review Group Project \#2.
\item Convenient sources of US data include the St Louis Fed's 
\href{http://research.stlouisfed.org/fred2/}{FRED}, 
the \href{http://www.census.gov/briefrm/staticindipage.html}
{Census Bureau}, 
and the 
\href{http://www.federalreserve.gov/econresdata/releases/statisticsdata.htm}
{Federal Reserve}.
\end{itemize} 


\vfill \centerline{\it \copyright \ \number\year \ 
NYU Stern School of Business}


\end{document}
% --------------------------------------------------------------------------
\newpage
{Sketch of an answer.  
The project is a loosely structured real-world problem;
it does not lend itself to a simple mechanical answer.
Here's a sketch of what I'd do: }  
\begin{itemize}
\item List of indicators.  
We've looked at many, take your pick.  
Among the most popular:  
housing starts and building permits, 
stock market, 
new claims for unemployment insurance, 
industrial production, term spreads.  
The challenge here is that a 6-month forecast is very demanding:
you need a substantial lead in anything you use.  

\item Statistical analysis.  
I think cross-correlation functions are useful ways to get across leads and lags, but you can get across similar 
information in different ways --- 
regressions with lags, for example,
or scatterplots of variables with different leads and lags.  

\item Combination.  
I'd use a multivariate regression, with the forward-looking 
6-month growth rate of employment as the dependent variable
and several indicators as independent variables.
I'd include lags of the indicators and the dependent 
variable as additional independent variables.
You'd have to take a course in time series statistics 
to understand this fully, 
but the idea is that they allow more flexibility in the dynamic relations
among the variables.  

\item Narrative.  This would depend on what you find.
If you find that (say) housing variables are weak,
and that this weakness translates (through the regression)
into a forecast of weak growth, 
you could join Professor Roubini and many others and talk about that.
Almost everyone predicts weakness right now, 
the question is how much.  

\end{itemize}

\end{document} 
