\documentclass[12pt]{article}

\RequirePackage{GE05}
% this inputs graphicx, too
\RequirePackage{comment}
\RequirePackage[hypertex]{hyperref}
\hypersetup{colorlinks=true,urlcolor=blue,linkcolor=red}
    
% ****************************************************************************
\def\ClassName{The Global Economy}
\def\Category{Professor David Backus}
\def\HeadName{Group Project \#7: US Debt and Deficits}

\begin{document}
\parindent = 0.0in
\parskip = 0.75\bigskipamount
\thispagestyle{empty}%
\Head

\centerline{\large \bf \HeadName}%
\centerline{Revised:  \today}

\medskip
{\it Submit via Blackboard by 9am April 20.}
\medskip

You are a junior member of the team at JP Morgan Chase 
that advises sovereign wealth funds:  
large investment funds managed by governments.  
You have been asked to prepare an analysis of 
the fiscal position of the US federal government 
for a potential client, the China Investment Corporation.  
The CIC is reported to be a major investor in US Treasury securities.  
From past experience, you know that your report should 
be a professional business document, 
no more than 5 pages long.
You are asked to include:  
%
\begin{itemize} 

\item A description of the US federal budget situation since 1990.  

\item The Congressional Budget Office (CBO) 
forecast of how the budget deficit and  government debt 
are likely to evolve over the next 20 years.  

\item An assessment of the CBO forecast.  
This should include a discussion of the most important inputs 
into the CBO's forecast 
and of changes in circumstances and/or policies 
that would make future deficits larger or smaller.  

\item A discussion of (i)~how future deficits might affect
the interest rates on US Treasuries and (ii)~the ability and willingness
of the US government to honor its debt, especially its debt to foreign 
investors.  

\item Your judgment about whether Treasury securities 
with maturities above 10 years 
are a good investment right now for the CIC.  
\end{itemize}
%

One of your new colleagues suggests you start with these sources:  
%
\begin{itemize}
\item The  
\href{http://www.cbo.gov/publications/}
{Congressional Budget Office} (CBO).  
Look for relevant reports.  

\item Gale and Auerbach's 
\href{http://www.brookings.edu/papers/2009/06_fiscal_crisis_gale.aspx}
{fiscal analysis} 
for Brookings.  

\item The Economist Intelligence Unit's
\href{http://db.eiu.com/index.asp?layout=publicationTypes}
{Country Intelligence site},  
including the Country Report and Country Risk Service.  

\item For data, see 
\href{http://research.stlouisfed.org/fred2/}{FRED} or the 
\href{http://www.bea.gov/National/nipaweb/Index.asp}{BEA}.
\end{itemize}

\vfill \centerline{\it \copyright \ \number\year \
 NYU Stern School of Business}

\end{document}
% ****************************************************************************

2008 version

\def\ClassName{The Global Economy}
\def\Category{Professor David Backus}
\def\HeadName{Group Project \#7: Fiscal Policy in India}

\begin{document}
\parindent = 0.0in
\parskip = \bigskipamount
\thispagestyle{empty}%
\Head

\centerline{\large \bf \HeadName}%
\centerline{Revised:  \today}

\medskip
{\it Submit via Blackboard by 9am April 20 (??).}
\medskip

You are a junior analyst in Booz \& Company's emerging markets group, 
which advises companies on opportunities in developing economies.  
Your team's latest project:  assess the risks for a 
large international retailer thinking of entering the Indian market.  
Your role:  fiscal policy.  
From past experience, you know that your report should 
be a professional business document, 
no more than 5 pages long,  
and include
(i)~an analysis of taxes the company will pay and 
(ii)~an assessment of the risk of a fiscal crisis in the near future.  


Here are some topics you might want to cover:  
%
\begin{itemize}

\item The most important taxes the company will pay, their rates, 
and the difficulty of complying with local tax laws.   

\item A description of India's overall fiscal situation, 
including government revenue, spending, budget balance, 
government debt (all expressed as ratios to GDP).  

\item An analysis of the government debt dynamics.  
How do you expect the debt-to-GDP ratio evolve over the next 5 years?  

\item A qualitative discussion of factors  
that could lead to large increases in government expenditure 
in the future.  

\end{itemize}
%
And here are some sources of information and data:  
%
\begin{itemize}
\item The  
\href{http://www.doingbusiness.org/ExploreTopics/PayingTaxes/}
{paying taxes}
component of the World Bank's Doing Business indicators.  

\item India's official  
\href{http://indiabudget.nic.in/}{budget}
and 
\href{http://www.indianembassy.org/newsite//doing_business_in_india/fiscal_taxation_system_in_india.asp}
{tax} 
web sites.  

\item The Economist Intelligence Unit's
\href{http://db.eiu.com/index.asp?layout=publicationTypes}
{Country Intelligence site},  
including the Country Commerce Report, 
and the EIU's
\href{http://www.countrydata.bvdep.com/cgi/template.dll?product=101&user=ipaddress}
{Country Data}.  

\end{itemize}




From Amit: 

Hard to typify projects - there can be a huge spectrum, anything from country entry to a general country study. fiscal component - absolutely. please feel free to make one up.
 
some examples where we help clients with -
 
- do general country assessment for business climate
- do a country entry strategy for a specific product/service or general entry opportunity assessment (would include entry options like JVs, etc, and any regulatory constraints
- develop a business case for specific entry options and frame different scenarios by changing macro and micro variables (including regulatory and tax changes that may be coming up in the near future)
- assess the current limited footprint of a company in an emerging economy and develop a growth strategy given the business and policy climate
  
happy to continue the discussion and please let me know if there is anything i can do to help. perhaps other sources of information can be some people (e.g., some govt. official, partner in a consulting firm, CEO of a tech firm, etc.) in India and students can do video conference calls with them and have a set of questions to further inform their research efforts.
 
rgds,
Amit

2004 version:  

Germany experienced a record-high federal government budget deficit of 3.9\% of GDP in 2004. You
have been hired as a consultant by the manager of a European equity fund, who has asked you to
outline the prospects for the deficit over the next 2-5 years and its likely impact on the German
economy and stock market.  Your report should include:
%
\begin{itemize}
\item A description of the major categories of government revenue and spending.

\item A review of the events that led to the current deficit.  This should include a discussion of
the role of the European Union's stability pact.

\item An assessment of how the deficit and the government debt are likely to evolve, and how any
changes will affect the economy and asset returns.

\end{itemize}
%
Not knowing where to start, you contact your Global Economy professor, who suggests the following:
%
\begin{itemize}
\item For online sources in general, try NYU's
\href{http://www.nyu.edu/library/bobst/vbl/}{Virtual Business Library} and Nouriel Roubini's
\href{http://www.stern.nyu.edu/globalmacro/}{Global Macro site}  (click on ``Countries").

\item For an economic and political analysis of Germany, try the Economist Intelligence Unit's
\href{http://db.eiu.com/index.asp?layout=publicationTypes}{Country Intelligence site}, especially
the Country Profile of Germany, which includes a budget overview.

\item For a discussion of the European Union's Growth and Stability Pact, see the EIU's
\href{http://db.eiu.com/report_dl.asp?mode=pdf&eiu_issue_id=238104023}{Country Forecast for
Europe}.

\item For macroeconomic data, try the EIU's
\href{http://www.countrydata.bvdep.com/cgi/template.dll?product=101&user=ipaddress}{Country Data}
or the International Monetary Fund's
\href{http://ifs.apdi.net/imf/ifsbrowser.aspx?branch=ROOT}{International Financial Statistics}.

\item For current analysis, try investment bank reports (access varies, but use your contacts).
\end{itemize}

\vfill \centerline{\it \copyright \ \number\year \ NYU Stern School of Business}


\end{document}
