\documentclass[letterpaper,12pt]{article}

\RequirePackage{GE05}
% this inputs graphicx, too
\RequirePackage{comment}
\RequirePackage[hypertex]{hyperref}
    \hypersetup{colorlinks=true,urlcolor=blue,linkcolor=red}

\def\ClassName{The Global Economy}
\def\Category{Professor David Backus}
\def\HeadName{Group Project \#6}

\begin{document}
\parindent = 0.0in
\parskip = \bigskipamount
\thispagestyle{empty}%
\Head

\centerline{\large \bf \HeadName:  European Monetary Policy}%
\centerline{Revised:  \today}

\medskip
{\it Submit via Blackboard by 9am April 14.}
\medskip

As Chief Economist for Euro Zone Markets in Deutsche Bank's London office,
you have been asked to address your New York colleagues 
about monetary policy and interest rates in the Euro Zone:
the countries of the European Union that have adopted 
the euro as their currency.  
Your US counterpart asks specifically that you comment on these issues:   
%
\begin{itemize}
\item The Euro Zone:  what countries does it include, 
and how do they collaborate in setting monetary policy?  

\item The European Central Bank (ECB):  how do its mission 
and political mandate differ from that of the Federal Reserve?  

\item Economic conditions in the Euro Zone:   
inflation, employment, and output growth.
NB:  description only --- no statistical work called for.  

\item The recent policy stance of the ECB:  
your sense of the likely impact of changes in inflation 
and growth on  near-term ECB policy.  
The Taylor rule might be helpful here.  

\item Prospects for interest rates in the Euro Zone over the next 6-12 months. 
\end{itemize}
%
Your report (5-page maximum) or 
slides (10-slide maximum) 
should reflect its professional audience in both content and presentation. 
   
You start by putting together a collection of sources and 
links that might be helpful to your New York audience: 
%
\begin{itemize}

\item Descriptions and comparisons of central bank policies and procedures:
%
\begin{itemize}
\item From the ECB:  
\href{http://www.ecb.int/mopo/html/index.en.html}{Monetary policy}. 


\item From the St Louis Fed: 
\href{http://research.stlouisfed.org/publications/review/03/01/Pollard.pdf}
{A look inside two central banks} 

\item From the BIS:  
\begin{itemize}
\item \href{http://www.bis.org/publ/bispap09.htm}{Comparing monetary policy procedures}
\item \href{http://papers.ssrn.com/sol3/papers.cfm?abstract_id=856944}
{The Taylor rule in the Euro Zone} 
%\item \href{http://www.bis.org/publ/bppdf/bispap12s.pdf}
%{The ECB and money markets} 

\end{itemize}

\end{itemize}

\item Data and commentary:  
\begin{itemize}
\item The ECB has a 
\href{http://www.ecb.int/stats/html/index.en.html}{data warehouse}.

\item Data for the Euro Zone, as well as its members, 
is available from Eurostat, the EIU, and the OECD.  
NYU's 
\href{http://library.nyu.edu/vbl/}{Virtual Business Library} 
has links to each.  


\item From Professor Roubini's 
\href{http://www.rgemonitor.com/}{RGE Monitor}:  current commentary (free registration with NYU IP address).

\end{itemize}

\end{itemize}


\vfill \centerline{\it \copyright \ \number\year \ 
NYU Stern School of Business}


\end{document}

% --------------------------------------------------------------------------
\newpage
Outline of an answer.

\end{document} 
