\documentclass[letterpaper,12pt]{article}

\RequirePackage{GE05}
% this inputs graphicx, too
\RequirePackage{comment}
\RequirePackage{hyperref}

\def\ClassName{The Global Economy}
\def\Category{Professor David Backus}
\def\HeadName{Group Project \#8}

\begin{document}
\parindent = 0.0in
\parskip = \bigskipamount
\thispagestyle{empty}%
\Head

\centerline{\large \bf \HeadName}%
\centerline{Revised:  \today}

\medskip
{\it Due April 6 at the start of class. }
\medskip

As the lead analyst on the  Japan desk of the International Monetary Fund, 
you have been asked to write an assessment of Japan's fiscal policy, where 
government debt has more than doubled in recent times.   
The challenge is to design a policy that  
brings the budget into balance and fosters good economic performance.  
You recall the history:  
After decades of fiscal discipline, Japan reacted to the collapse 
of its equity and real estate markets in the early 1990s with 
aggressive fiscal policy,   
using cutting taxes and increasing spending on huge public works projects 
to return the economy to prosperity.  
Prosperity failed to materialize, but (gross) public debt increased from 
70\% of GDP in 1990 to 160\% in 2004. 
What should Japan do now? 

From long experience, you know that your report should include:  
%
\begin{itemize}
\item A graphical summary of Japan's fiscal situation 
from 1980 to the present, 
including the government deficit and government debt (both expressed 
as ratios to GDP).  

\item A ``sustainability analysis'' of the current situation.  
If the primary deficit continued at its current level, how large would 
the government debt be in 10 years?  

\item A qualitative discussion of the impact of an aging population.  

\item Recommendations for changes in Japan's fiscal policy 
designed to foster economic performance without letting 
government debt explode.   

\end{itemize}
%
You turn to the usual sources of information and data:  
%
\begin{itemize}
\item For economic and political analysis, 
you use the Economist Intelligence Unit's
\href{http://db.eiu.com/index.asp?layout=publicationTypes}{Country Intelligence site} 
and the OECD's most recent 
\href{http://www.oecd.org/document/61/0,2340,en_2649_22054704_34274621_1_1_1_1,00.html}
{Economic Survey} (click on ``SourceOECD'' for the complete report).  

\item For macroeconomic data, you use the EIU's
\href{http://www.countrydata.bvdep.com/cgi/template.dll?product=101&user=ipaddress}{Country Data}
or the International Monetary Fund's
\href{http://ifs.apdi.net/imf/ifsbrowser.aspx?branch=ROOT}{International Financial Statistics}.

\end{itemize}

\vfill \centerline{\it \copyright \ \number\year \ NYU Stern School of Business}



\end{document}


2004 version:  

Germany experienced a record-high federal government budget deficit of 3.9\% of GDP in 2004. You
have been hired as a consultant by the manager of a European equity fund, who has asked you to
outline the prospects for the deficit over the next 2-5 years and its likely impact on the German
economy and stock market.  Your report should include:
%
\begin{itemize}
\item A description of the major categories of government revenue and spending.

\item A review of the events that led to the current deficit.  This should include a discussion of
the role of the European Union's stability pact.

\item An assessment of how the deficit and the government debt are likely to evolve, and how any
changes will affect the economy and asset returns.

\end{itemize}
%
Not knowing where to start, you contact your Global Economy professor, who suggests the following:
%
\begin{itemize}
\item For online sources in general, try NYU's
\href{http://www.nyu.edu/library/bobst/vbl/}{Virtual Business Library} and Nouriel Roubini's
\href{http://www.stern.nyu.edu/globalmacro/}{Global Macro site}  (click on ``Countries").

\item For an economic and political analysis of Germany, try the Economist Intelligence Unit's
\href{http://db.eiu.com/index.asp?layout=publicationTypes}{Country Intelligence site}, especially
the Country Profile of Germany, which includes a budget overview.

\item For a discussion of the European Union's Growth and Stability Pact, see the EIU's
\href{http://db.eiu.com/report_dl.asp?mode=pdf&eiu_issue_id=238104023}{Country Forecast for
Europe}.

\item For macroeconomic data, try the EIU's
\href{http://www.countrydata.bvdep.com/cgi/template.dll?product=101&user=ipaddress}{Country Data}
or the International Monetary Fund's
\href{http://ifs.apdi.net/imf/ifsbrowser.aspx?branch=ROOT}{International Financial Statistics}.

\item For current analysis, try investment bank reports (access varies, but use your contacts).
\end{itemize}

\vfill \centerline{\it \copyright \ \number\year \ NYU Stern School of Business}


\end{document}
