\documentclass[letterpaper,12pt]{article}

\RequirePackage{GE05}
% this inputs graphicx, too
\RequirePackage{comment}
\RequirePackage[hypertex]{hyperref}
    \hypersetup{colorlinks=true,urlcolor=blue,linkcolor=red}

\def\ClassName{The Global Economy}
\def\Category{Professor David Backus}
\def\HeadName{Group Project \#4}

\begin{document}
\parindent = 0.0in
\parskip = \bigskipamount
\thispagestyle{empty}%
\Head

\centerline{\large \bf \HeadName:  Labor and Trade}%
\centerline{Revised:  \today}

\medskip
{\it Submit via Blackboard by 9am March 4.}

%{\it Answer TWO (2) of the following questions.}

\begin{enumerate}

\item {\it Labor market theory.\/} As background to a
consulting assignment, you have been asked to work through an
example to illustrate the impact of
minimum wage legislation and payroll taxes on employment and
unemployment. You decide to adapt the framework of your 
{Global Economy} class notes, 
working through the demand for labor
of a single firm producing widgets according to the production
function $Y= A K^{1/3}L^{2/3}$. For simplicity, you assume that
$A=K=1$ and that the price of widgets is 1, too.  The supply of
labor rises with the wage $w$: $L^{s}=w^{2}$.
%
\begin{enumerate}

\item Describe, first, how the labor market might work in this
economy if there were no labor market regulation.  
What is the wage?
Employment? Unemployment?  (15~points)

\item Now consider the effect of introducing a minimum wage
$w_{m}$.  What are employment and unemployment rates if $w_{m}=1$?
If $w_{m}=0.5$?  How does this market differ from the one above?
How much would an unemployed person be willing to pay a recruiter
to find a job?  
(15~points)

\item Suppose workers differ in productivity, 
with some contributing effectively 1.5 units of labor 
per unit paid, and others contributing only 0.5 units.  
Describe qualitatively how  you would expect a minimum wage of 0.75 
to affect the two types.  
(10~points)

\end{enumerate}

%\begin{comment}
Answer. The firm's optimal level of labor (its labor demand) is
such that the gain of adding one more worker is equal to its cost.
The marginal gain is the derivative of sales with respect to
labor: $\frac{2}{3}pAK^{1/3}L^{-1/3}$. The marginal cost is simply
the wage $w$. Therefore, since $p=K=A=1$, the demand for labor is
\[
    L^{d} \;=\; \left(\frac{3}{2}w\right)^{-3}.
\] 

\begin{enumerate}
\item The equilibrium is reached when demand equals supply. That
is:
\[ 
    \left(\frac{3}{2}w\right)^{-3} \;=\; w^{2}.
\] 
Solving for $w$, we obtain that the equilibrium wage is
\[ 
    w^{*} \;=\; \left(\frac{3}{2}\right)^{-3/5}=.78.
\] 
The level of employment is
\[ 
    L^{*} \;=\; \left(\frac{3}{2}w^{*}\right)^{-3}={w^{*}}^{2}=.61.
\] 
The unemployment rate is $0$. Since there is no distortion, all
individuals that would like to work at the equilibrium wage rate
get a job.

\item If $w_{m}=1$, the demand for labor is
$L^{d}=\left(\frac{3}{2}\times 1\right)^{-3}=.30$. Instead the
supply is $L^{s}=1^{2}=1$. Therefore the level of employment is
$.30$. The unemployment rate is $\frac{1-.3}{1}=70\%$. Every
unemployed will be willing to pay a recruiter up to the difference
between the ongoing wage ($1$) and his/her reservation wage. The
reservation wage is different across unemployed, and can be read
off the labor supply curve.

%%%%%%%%%%%%%%%%%%%%%%%%%%%%%%%%%%%%%%%%%%%%%%%%%%%%%%%%%%%%%%%%%%%%%%%%%%%%

\begin{figure}[h]
\begin{center}
\begin{picture}(300,200)(-30,-20)%

\footnotesize%
\put(-30,0){\vector(1,0){300}}%
\put (0,-20){\vector(0,1){200}}%
\put(265,-14){$L$}%
\put(-15,175){$w$}%
\put (154,157){$L^{s}(w)$}%
\put(44,157){$L^{d}(w)$}%

% Demand Curve
\qbezier(45,145)(90,41)(165,10)%

% Supply Curve
\qbezier(45,25)(80,41)(175,140)%

\qbezier[60] (0,92)(65,92)(128,92)%
\qbezier[40] (71,0)(71,46)(71,93)%
\qbezier[40] (128,0)(128,46)(128,93)%

\qbezier[40] (0,62)(50,62)(95,62)%
\qbezier[30] (95,0)(95,31)(95,62)%

\put(-38,60){$w^{*}=.76$}%
\put(-33,90){$w_{m}=1$}%

\put(95,-18){$.61$}%

\put(65,-18){$.30$}%
\put(130,-18){$1$}%

\end{picture}
\end{center}
\caption{With and without unemployment.} \label{fig:minimum1}
\end{figure}

%%%%%%%%%%%%%%%%%%%%%%%%%%%%%%%%%%%%%%%%%%%%%%%%%%%%%%%%%%%%%%%%%%%%%%%%%%%%

The wage $w_{m}=.5$ is below the equilibrium rate. This means that
the minimum wage is not binding. The outcome in terms of wage and
employment is going to be the same as in part (a).

\item You might guess that the minimum wage would  affect 
the low-skilled workers more than the high-skilled. 
Why?  Because the high-skilled are worth more than the minimum 
wage, so they will be hired (presumably) something close to their 
productivity.
But if the productivity of the low-skilled is below the minimum wage, 
they won't be hired. 
The result is that the unemployed will consist predominantly 
(perhaps entirely in this setup) 
on the low-skilled. 

\end{enumerate}
%\end{comment}


% ----------------------------------------------------------------------
\item {\it Labor market practice.\/}  
Your first day on the job at General Electric, 
you are given 4 hours to prepare
a 5-minute presentation to your group summarizing the 
labor market issues a manufacturer would face in 
Brazil, the Czech Republic, and Hong Kong.    
Once you get over your initial panic, you contact your Global Economy 
professor, who suggests you look at:    
%
\begin{itemize}

\item The Bureau of Labor Statistics'
\href{http://www.bls.gov/fls/}{Foreign Labor Statistics};


\item The World Bank's
\href{http://www.doingbusiness.org/}
{Doing Business};   

\item The Economist Intelligence Unit's various 
\href{http://db.eiu.com/topic_view.asp?pubcode=CP&title=Country+Profile}{reports} 
and 
\href{http://www.countrydata.bvdep.com/cgi/template.dll?product=101&user=ipaddress}
{databases}.  

\end{itemize}
%
You quickly turn this information into a series of charts and bullet points 
that highlight the salient differences across these countries.  (40~points) 

%\begin{comment}
Answer.  Here's an example of the kind of data you
might collect:  

\begin{center}
\begin{tabular}{lcccc}
\hline\hline
Indicator   & Brazil & Czech Rep & Hong Kong  \\
\hline\hline 
Hourly direct pay (USD) & 3.30 & 4.96   & 5.29  \\
Hourly total comp (USD) & 4.91 & 6.77   & 5.78  \\
Non-wage cost (\%)       &  37  &   35   &  5    \\
Difficulty of firing (index)   &  0   & 20     &  0  \\
Employment rigidity (index)    & 46   & 31     &  0  \\
Education of adults (yrs) &  5.1  &  9.6  & --- \\
\hline\hline 
\end{tabular}
\end{center}
Sources:  EIU, BLS, Doing Business, and NationMaster.com.  
Most recent year available.   

What do we make of this?  
Labor is cheap in all of these countries relative 
to the US (total comp is \$23.82).  
Note that the non-wage component of total compensation 
changes the ranking of Hong Kong; 
it's a reminder that taxes and benefits are an important expense.  
Education seems to be higher in Czecho.  
Labor markets are most flexible in HK.
In a real report, you'd want to flesh out what these factors mean
to GE.  
%\end{comment}


% ----------------------------------------------------------------------
\item {\it Trade liberalization.} You have been asked to explain
to an audience of Latin American business 
consultants why trade liberalization is win-win:  
why all countries can benefit.  
After talking to a few of them, you realize that their major concerns 
are the impact of trade on  their clients' businesses.
What can you say to persuade them that intra-American
trade could be good for business?  
Use examples where possible.  
Two pages maximum.     
(20~points) 

%\begin{comment}
Answer.  
You could answer this a variety of ways, ranging from working 
through a numerical example, as we did in class, 
to describing how specific clients might benefit. 
%\end{comment}

\end{enumerate}


\vfill \centerline{\it \copyright \ \number\year \ NYU Stern
School of Business}

\end{document}
