\documentclass[letterpaper,12pt]{article}

\usepackage{comment}
\usepackage{hyperref}
\RequirePackage{GE05}
% this inputs graphicx, too

\def\ClassName{The Global Economy}
\def\Category{Professor David Backus}
\def\HeadName{Group Presentations}

\begin{document}
\parindent = 0.0in
\parskip = \bigskipamount
\thispagestyle{empty}%
\Head

\centerline{\large \bf \HeadName}%
\centerline{Revised:  \today}

%\bigskip
\subsubsection*{Overview}

As noted in the Syllabus, each group will give a presentation to the class on a topic related to the course.  
Your presentation is an opportunity to apply the tools you've learned in the course to a business problem 
and to hone your presentation skills.
%We will use your
%previously assigned groups. If your group is three or less, you
%can merge with another group as long as the total size of the new
%group is no larger than five.


Presentations will be held in class on April 17, April 19, and April 24.  
Every member of your group must participate.
Each group will be randomly assigned one of these dates, but you are free to switch with another group as long as you
let me know in advance. 
Presentations will last 10 minutes, plus 5 for questions from the
class. 
Take the time limit seriously.  
If you violate it by a significant margin, 
I will deduct points from your grade.  

The following intermediate steps are recommended:  
%
\begin{itemize}
\item After spring break:  Come see me  
to discuss the topic(s) you are considering.  
I'll do my best to give you good advice on the topic and 
how to find useful information related to it.   

\item By April 5:  Submit your proposal via BlackBoard.  
Your proposal should be 1-2 pages of professional-looking 
text that includes the following elements 
(see below for an explanation):
%
\begin{itemize}
\item List of group members
\item Topic 
\item Context
\item Your role
\item Audience
\item Tools you intend to use
\item Outline (subject to change, of course, once you've completed your analysis)
\item Time:  whether you plan for present in the morning or afternoon 
class (this for groups that include students from both) 
\end{itemize}

\item By April 12:  Submit your draft slides via Blackboard.  
\end{itemize}
Grades will be based on all three components, with 
10\% allocated to the proposal, 10\% to the draft slides, 
and 80\% to the presentation.  



\subsubsection*{Choosing a topic}

Your first task is to choose a topic.  
The requirements are (i) that the topic be interesting to me and your fellow students  
and (ii) that you apply one or more of the tools developed in class.  
Beyond that, feel free to use your imagination.   
Some of the best presentations are based on realistic business problems:
the challenges of starting a credit card business in [country]; 
the opportunities in [country] for selling consumer products; 
the opportunities for sourcing production in [country]; 
and so on.  


Once you've chosen a topic, 
you must think about the context and audience for your presentation.
Who are you?  Who are you addressing?  What is your mission? 
Examples could be:  outside consultants suggesting opportunities 
to a company's management; 
management describing an opportunity to a company's board; and so on.  


Some concrete examples follow, but feel free to design your own. 
We'll mention more in class. 
%
\begin{itemize}

\item {\bf Long-run analysis}

\underline{Example} \\
\newline {\it Context\/}:  consumer products firm looking for new markets.
\newline {\it Your role\/}:  outside consultants.
\newline {\it Audience\/}:  company's senior executives.
\newline {\it Tools\/}:  growth accounting, measures of institutional quality,
 data on income and demographics as needed.
\newline {\it Outline\/}:  [Country] has grown rapidly over the last 20 years.
Growth accounting suggests that most of this growth comes from increases in the
capital stock, with little increase in TFP.
Institutional indicators suggest that the country lags behind others with comparable GDP per capita.
We are concerned, therefore, that growth may slow down until these institutional factors
change to provide an environment conducive to faster TFP growth.  In the meantime, look elsewhere.

\item {\bf Labor market analysis}

\underline{Example} \\
\newline {\it Context\/}:  manufacturer interested in building a plant in [country name].
\newline {\it Your role\/}: inside management.
\newline {\it Audience\/}:  company's board of directors.
\newline {\it Tools\/}:  labor market analysis, indicators.  
\newline {\it Outline\/}:  [Country] has very low labor costs relative to the US.
Indicators suggest strong unions, but flexibility in hiring and firing workers
and a high level of education relative to other countries with similar labor costs.
Corruption is an ongoing concern.
Despite obvious risks, we think building and operating a plant is a good bet.
Similar companies have had some success.
Membership in the EU, should it occur, would reduce the risk considerably.
%\newline Variants:  examine other relevant regulations, including those affecting international
%trade;  substitute service provider (call center?) for manufacturer.

\item {\bf Short Run Analysis}

\underline{Example} \\
\newline {\it Context\/}:  Fortune 500 company interested in issuing debt.
\newline {\it Your role\/}:  investment banker.
\newline {\it Audience\/}:  company's CFO.
\newline {\it Tools\/}:  interest rates, yield curve, monetary policy.
\newline {\it Outline\/}:  Short-term interest rates are low in [country], but the yield curve is relatively steep.
Our analysis suggests that short-term rates will rise rapidly 
in the next 2-5 years.
Our estimated Taylor rule suggests that this prediction is not supported by forecasts of inflation or output growth,
both of which are low.  We therefore suggest that short-term rates will continue to be low,
and recommend that our clients issue short or floating rate debt to take advantage of the differential between
short and long-term interest rates.


\item {\bf Policy Analysis} 

You illustrate a public policy challenge and then go on to describe
a possible solution that has been adopted elsewhere in the world. 
The nature of the policy
challenge must be such that different solutions have the potential
of generating very different outcomes in terms of business
conditions (read: very different outcomes for companies' bottom
lines). Below I provide a few examples of what I mean by policy
challenge, along with a succinct illustration of how each of them
was tackled in one or more countries. 
You are more than welcome to come up with your own.

\begin{itemize}

\item {\bf Social Security.} Over the years, most developed countries have put in place
comprehensive Social Security programs. In several of these
countries, the expected discounted value of outlays is now much
greater than the value of contributions to the program. In the case
of the United States, one proposal is a (partial) shift from 
an unfunded pay--as--you--go system to a funded system based on 
individual private accounts.  Chile and Sweden are two notable examples of
the latter.  


\item {\bf Public provision of health services.} Most societies
care about the availability of health services to all individuals.
In most countries, public and private provision co--exist, although
in somewhat different proportions. Canada is the only country in
which the private sector is banned from this market. The United
States is one of the few countries in which, historically,
businesses have been directly responsible for their employees' 
health care bills.

\item {\bf Capital accumulation.} For a variety of reasons, governments in many
countries provide citizens with incentives to increase their savings
and also with incentives to direct them to particular vehicles. For
example, the deductability of mortgage interest payments makes
investment in real estate more appealing to American
households. Another example is given by Singapore's mandated savings
program: in 1955, the Singaporean government established a
compulsory national social security savings plan. At one point, its
rate of contribution reached a whopping 50\% of gross wages.  

\item {\bf International trade agreements.} In the last twenty years
or so, trade flows have increased considerably across the board.
However, different countries have followed different paths to trade
liberalization: some have proceeded by signing bilateral trade
agreements, others have joined free-trade areas such as Mercosur,
Nafta, and the European Union. Others have ``simply" joined, or have
applied for membership in, the WTO. 

\item {\bf Government-owned enterprises.} In certain countries,
governments produce cars, aircraft, carbonated drinks and
cookies, manage the transportation networks, energy production, and
the entire education system, ... In other countries, instead,
government is much smaller and leaves production of most goods and
services to private concerns.  Different degrees of state
intervention in the economy may have very different consequences for
the price and the availability of goods and services and ultimately
for the overall level of TFP.

\end{itemize}

%\pagebreak % ??
\underline{Example} \\
\newline {\it Context}: apparel manufacturer 
with manufacturing facilities in [country].
\newline {\it Your role}: outside consultant.
\newline {\it Audience}: senior management.
\newline {\it Tools}:  data on trade quotas and on human capital investment.
\newline {\it Outline}:  how should we move in expectation of [country]'s application for WTO membership being accepted?
[Country] is currently subject to tight quotas on apparel exports to
the US. Will these quotas disappear with WTO membership? Is it
absolutely clear that in the medium run [country]'s comparative
advantage is in the production of clothing? Should we relocate? If
yes, where?

\underline{Example}\\
\newline {\it Context}:  large insurance company 
with assets invested in [country].
\newline {\it Your Role}: investment analyst.
\newline {\it Audience}: investment strategy team.
\newline {\it Tools}:  data on fiscal policy, capital flows.
\newline {\it Outline}: how should we move in expectation of [country]'s
government phasing out the pay-as-you go social security program
in favor of a defined contribution, private-account model? In
particular, do we expect this change to have a sizeable impact on
long term interest rates? If yes, when? Why?  

\end{itemize}

{\bf Remember:  the heart of your presentation is your analysis.  You must describe a meaningful
application of one or more of the tools of macroeconomic analysis introduced in this course.}




\subsubsection*{Presentation guide}

This guide is designed to remind you of some basic delivery skills.
It was prepared with the help of Stern's 
Management Communication faculty.  
It is not concerned with the content of your presentation,
which obviously should be well researched and organized logically, with clear transitions and conclusions.

The suggested {\bf audience should determine the content and style} of your presentation. Do not
turn it into an economic theory lecture!

You must be audible:  {\bf If the class can't hear you, you have
wasted everyone's time.} When you practice, have a group member
listen to you in a large room and have that person alert you if
you cannot be heard easily.

{\bf Make eye contact:}  When you rehearse
your presentation, practice scanning the room and looking into people's
eyes.  

The norm at Stern tends to be that presenters use PowerPoint
slides as their notes.  That's OK as long as you {\bf don't turn
your back on the audience}.  To deal with this potential problem,
some people use note cards to keep them facing front, on track,
and as a backup in case of technology failure.  A general rule is
that {\bf your presentation should include more than what the audience
sees on the slides}.  Otherwise, they will simply read the slides
and ignore you.

{\bf Keep your slides spare.}  Otherwise your audience will spend more time reading than listening
to you.  Your slides should support what you say, not compete with you for the audience's attention.

{\bf Do not read your presentation.}  Reading is the best way to lose your audience.  Speak
naturally so that you don't sound as if you have memorized your talk.

{\bf Many of us speak too fast.}  If you are one of them, build
into your presentation pauses and reminders to speak more slowly.
Some people find it helpful to include at selected points in their
notes big red marks saying ``BREATHE!" or ``PAUSE" or ``SCAN
AUDIENCE" or ``SLOW DOWN."

{\bf If English is not your first language}, or if you have a strong accent, {\bf speak slowly and
deliberately at the beginning} of your presentation to allow the audience time to get used to your
speech patterns.

{\bf Time yourself.}  Cut appropriately.  Include the crucial information but skip everything
else.  Save interesting but inessential material for the question and answer period.  Remember
that less is more:  if you add too many details, your audience may lose track of your main points.

{\bf Make yourself comfortable:}  Many people have trouble standing still
when they give a talk.  Practice finding a comfortable position.

Since you are presenting in a group format, try to {\bf present a
united front} when you make your presentation.  For example, using
``we" instead of ``I" makes it sound as if you worked together.
Showing that you are listening when others talk is good, too.

{\bf Practice}, preferably in front of other people who can give you honest and useful feedback.
As they say about Carnegie Hall:  practice, practice, practice.

Finally, try to {\bf enjoy yourself}.  If you look like you're
having fun, your audience will, too.


\subsubsection*{Communication skills}

Communication skills are extremely important to most people in their careers.  If you'd like to
work on your skills (and career), 
we recommend the Management Communication course ``Advanced Topics in
Management Communication'' (B45.2103).  Stern is blessed with some of the best communication
faculty anywhere, and they are ready, willing, and able to share their insights with you.

% ---------------------------------------------------------------------------
\pagebreak
\thispagestyle{empty}%
\Head
\centerline{\large \bf Presentation Evaluation}%

\bigskip
{\bf Group:}

{\bf Topic:}

{\bf Informativeness:  How much did we learn from your presentation? } 


 
\vspace*{1.25in}
{\bf Analysis:  Did you apply the lessons of the course effectively?  
Suggest a novel insight into a problem or issue?}


\vspace*{1.25in}
{\bf Organization and delivery:  Was your presentation clear and compelling?  
Were the slides effective?  Did you speak clearly?}


\vspace*{1.25in}
{\bf Grade:}
\begin{itemize}
\item Proposal (10~points): 
\item Draft slides (10~points):
\item Presentation (80~points):  
\end{itemize}

\end{document}
