\documentclass[letterpaper,12pt]{article}

\RequirePackage{comment}
\RequirePackage{GE05} % this inputs graphicx, too
\RequirePackage[hypertex]{hyperref}
    \hypersetup{colorlinks=true,urlcolor=blue,linkcolor=red}

\def\ClassName{The Global Economy}
\def\Category{Professor David Backus}
\def\HeadName{Final Group Project}

\begin{document}
\parindent = 0.0in
\parskip = \bigskipamount
\thispagestyle{empty}%
\Head

\centerline{\large \bf \HeadName}%
\centerline{Revised:  \today}

\medskip
{\it Submit via Blackboard by 9am Apr 29.  Submit ``term sheet'' by 9am April 15.}
\medskip

The final group project is a chance to apply what you've learned to 
a realistic business issue.  
Two templates are given below.    
If you would like to depart from the templates, 
you must come talk to me about it no later than April 11.
In either case, {you must submit a term sheet by 
9am April 15\/} outlining your business proposal 
and the evidence, including data sources,
that you plan to collect to support it.  
The term sheet should be 1-2 pages long and 
will account for 20\% of the final project grade.     
The project itself can be either a report or a set of slides.
Either one should document both your argument and the rationale 
for it.  


% --------------------------------------------------------------------------

{\it Template 1 (international business opportunity).\/} 
As an international business consultant, 
you identify promising new markets for corporate clients, 
typically large firms based in  developed countries.
Your experience includes consumer products, 
financial products, 
and real estate development for both recreational 
and manufacturing purposes.  
Your mission:  to identify a country and product that you think 
represent an attractive opportunity for one of your clients.  
Your recommendation is to take the following form:   
%
\begin{itemize}%

\item A one-page executive summary outlining the business opportunity 
and your reasons for recommending it.   

\item An in-depth analysis of 
both the potential benefits and the risks involved 
in pursuing this opportunity.    
Each should be supported by 
data and analysis as far as possible.  

\item Further analysis of related products and locations, 
along with supporting evidence of their suitability for your client.  

\end{itemize}
%
You understand that success depends on both the macroeconomic 
environment and the microeconomic conditions specific to 
the product.      
In addition to the usual sources on the Global Economy 
%\href{http://www.stern.nyu.edu/eco/b012303/Backus/ge_resources_db.htm}
{resource page},
you check several of the following:  
the Economist Intelligence Unit's 
%\href{http://db.eiu.com/index.asp?layout=AllTitles}
{Country Intelligence},
the World Bank's 
%\href{http://www.doingbusiness.org/}
{Doing Business},
the United Nations Population Information Network, 
% http://www.un.org/popin/  
MarketResearch.com, 
Business Insights,   
and Euromonitor's Global Market Information Database.  
Links to most of these are available through NYU's 
\href{http://library.nyu.edu/vbl/}
{Virtual Business Library}.  


% --------------------------------------------------------------------------

{\it Template 2 (emerging market investment opportunity).\/} 
As a principal in an international equity fund, 
you have been asked to address New York University's Board of Trustees 
about possible new investments.
Although NYU is well known for its conservative investment philosophy, 
they tell you that the apparent opportunities in emerging markets 
and NYU's global ambitions have led them to reconsider.
They ask you to make a recommendation to the Board 
on how they might invest up to \$500 million in a portfolio 
that will be managed on a one-time buy-and-hold basis with a 10-year
investment horizon.  
Your recommendation should focus on 
an emerging market country you think offers unusual promise 
and specific sectors and companies in that country 
that are likely to benefit from that promise.  

Your recommendation is to take the following form:   
%
\begin{itemize}%

\item A one-page executive summary outlining your country, 
sector, and company recommendations, and your rationale for each.  

\item An in-depth analysis of the risks and opportunities 
represented by the country you recommend.  
These risk and opportunities should be supported by 
data and analysis as far as possible.  

\item Further analysis of specific sectors and countries, 
along with supporting evidence of their suitability
as investments for NYU.  

\end{itemize}
%
Your experience to date has been primarily in developed countries, 
but you understand that emerging markets are largely ``macro'' markets, 
in the sense that stocks do well if the countries do.  
Accordingly, you look at a range of possible target countries 
and examine their economic fundamentals.  
Once you select a country, you use a combination
of macro- and microeconomic analysis  
to guide your search for appropriate sectors and companies.   
In addition to the usual sources on the Global Economy 
%\href{http://www.stern.nyu.edu/eco/b012303/Backus/ge_resources_db.htm}
{resource page},
you check several of the following:  
the Economist Intelligence Unit's 
%\href{http://db.eiu.com/index.asp?layout=AllTitles}
{Country Intelligence},
the World Bank's 
%\href{http://www.doingbusiness.org/}
{Doing Business},  
the United Nations Population Information Network, 
% http://www.un.org/popin/  
ISI's %\href{http://site.securities.com/}
{Emerging Markets}, 
Institutional Investor, 
Pensions and Investments Age, 
and the Financial Times.  
Links to most of these are available through NYU's 
\href{http://library.nyu.edu/vbl/}
{Virtual Business Library}.


\vfill \centerline{\it \copyright \ \number\year \  
NYU Stern School of Business}

\end{document}

