\documentclass[letterpaper,12pt]{article}

\RequirePackage{comment}
\RequirePackage{GE05}
% this inputs graphicx, too
\RequirePackage[hypertex]{hyperref}
    \hypersetup{colorlinks=true,urlcolor=blue,linkcolor=red}

\newcommand{\POP}{\mbox{\it POP\/}}

\def\ClassName{The Global Economy}
\def\Category{Professor David Backus}
\def\HeadName{Group Project \#3}

\begin{document}
\parindent = 0.0in
\parskip = \bigskipamount
\thispagestyle{empty}%
\Head

\centerline{\large \bf \HeadName:  Sources of Country Performance}%
\centerline{Revised:  \today}

\medskip
{\it Submit via Blackboard by 9am February 19.}


\begin{enumerate}

\item {\it Japan and Korea. }   
Japan and Korea are remarkable success stories, 
having experienced unusually rapid economic growth over long periods of time.
In Japan, per capita GDP rose from 30\% of the US in 1870 
(similar to Mexico today) to about 75\% today.  
In Korea, GDP per capita rose from about 10\% of the US in 1960 
(similar to the Philippines at the time) 
to over 50\% today.
Your mission:  to determine the sources of the remaining differences
between the two countries --- capital, productivity, and anything else 
that you think is relevant.  

The data for 2003 
(the most recent year for which all of these numbers are available):
\begin{center}
%\tabcolsep = 0.12in
\begin{tabular}{lcccccc}
\hline\hline
        &  POP  &  Y/POP  &  L/POP  &  K/Y     &  Education &  Hours \\
\hline\hline
Japan   & 127.7m &  24,036 &  0.534 & 4.003     & \phantom{1}9.6 & 1801 \\%
Korea   &  \phantom{1}47.5m 
                 &  17,600 &  0.494 & 1.608    & 11.0  &  2434  \\%
\hline\hline
\end{tabular}
\end{center}
POP is population (millions), Y is GDP (2000 US dollars), 
K is capital (2000 US dollars), Education is years of school, 
and Hours is annual hours worked.  
Y and K are PPP-adjusted.  
Education and Hours are from the OECD's {\it Employment Outlook\/};
the other variables are from the Penn World Tables.  

\begin{enumerate}
\item Compute output per capita, output per worker, and output per
hour worked.  How do they differ? (15~points)

\item What are the primary sources of the difference in output per
capita? (15~points)

\item Using whatever resources you think appropriate, 
comment on the likely performance of the two countries over the next 
decade.  
What do you see as the major opportunities?  Challenges?  (20~points) 
\end{enumerate}


%\newpage
%\begin{comment}
Answer.
Per capita output is higher in Japan than Korea, but why?
Is the Japanese economy more productive?  Other reasons?  
What would it take for Korea to catch up?  
\begin{enumerate}
\item Calculations are reported below.  
The most striking difference in the numbers:  
a large difference in average hours worked means 
that Japan's 26\% advantage in output per worker 
increases to 71\% in output per hour.  
In words:  Korea's gap in GDP per worker would be substantially 
larger if they didn't work so much.  
\begin{center}
\begin{tabular}{lccc}
\hline\hline
Country   &  GDP Per Capita &  GDP Per Worker &  GDP Per Hour \\
\hline\hline
Japan       &  24,036  & 45,011 &  24.99 \\
Korea       &  17,600  & 35,628 &  14.64 \\
Japan/Korea &  1.37 &  1.26 &  1.71 \\
\hline\hline
\end{tabular}
\end{center}

\item The idea is to dig a little deeper, 
introducing (for example) capital and productivity.  
You could do this in a variety of ways; 
one is to use 
\[
    Y \;=\; A K^\alpha (h L)^{1-\alpha} 
\]
to account for the difference in output per hour worked.  
A little multiplying and dividing gives us 
\[
    Y/\POP \;=\; (Y/L) (L/\POP) \;=\; (L/\POP) A (K/L)^\alpha h^{1-\alpha} .
\]
This isn't the only possible line of attack on this problem, but
it's a reasonable one.  It tells us we can attribute differences
in per capita GDP to differences in:  participation ($L/\POP$),
total factor productivity ($A$), capital per worker ($K/L$), 
and average hours per worker ($h$). 
The math implies
%
\begin{eqnarray*}
    \frac{(Y/\POP)_{J}}{(Y/\POP)_{K}} &=&
             \left[ \frac{(L/\POP)_J}{(L/\POP)_{K}} \right]
             \left[ \frac{A_J}{A_{K}} \right]
             \left[ \frac{(K/L)_{J}}{(K/L)_{K}} \right]^\alpha
             \left[ \frac{h_{J}}{h_{K}} \right]^{1-\alpha} \\
             &=& \left[ \frac{.534}{.494} \right]
                 \left[ \frac{5.38}{5.11} \right]
                 \left[ \frac{180.2}{57.3} \right]^{1/3}
                 \left[ \frac{1801}{2434} \right]^{2/3}_{\phantom{\sum_i}} \\
      1.37   &=& (1.08)(1.05) (1.47) (0.82) .
\end{eqnarray*}
%
Where does the 37\% difference in GDP per person come from?
Most of the answer is capital: 
Japan has only an 8\% TFP advantage, 
but its higher capital per worker would (on its own) 
lead to a 47\% output difference.  
The main reason the difference in per capita GDP isn't larger than 37\% is that hours worked, as we've seen, goes the other way:  
higher in Korea.  

Education also plays a role here.  
Since average years of school are higher in Korea, this will increase the difference
in TFP (what's left).  
One approach (mentioned in the notes) is to measure $H$ by $S$.  Then a 
term $(H_J/H_K)^{2/3}$ will be added above.  
Or you could approximate $H$ by $ \exp(0.07 \times S)$, which would give 
you a somewhat different answer.  
Depending on your choice, the TFP difference will change, 
since TFP is (by construction) whatever's left.  


\item This part is up to you.  
I'd like to see some measures of institutional quality, 
and a discussion of the role you think they'll play in 
the future.
Institutions, on the whole, are good indicators of the future, 
since they tend not to change very quickly.  


\end{enumerate}
%\end{comment}


%\end{enumerate} \end{document}
% ===============================================================================
\item {\it Argentina and Chile. } 
Argentina and Chile have both experienced dramatic economic growth
and painful reversals over the last century.  
Argentina was one of the richest countries in the world in 1900, 
but mixed economic performance since then has 
dropped it back into the world of emerging markets.  
Chile suffered a traumatic change in government in the 1970s, 
but emerged a decade later as a fast-growing democracy with 
the highest per capita GDP in Latin America.  
\begin{center}
%\tabcolsep = 0.12in
\begin{tabular}{lccccc}
\hline\hline
Country   &  Year   &  POP   &  Employment &  Y/POP &  K/POP  \\
\hline\hline
Argentina \hspace*{0.1in} &  1960   &  20.6m &  \phantom{1}8.1m   
                &  \phantom{1}7,838  &  12,713  \\
Argentina &  2004   &  39.1m &  16.2m             &  10,939 &  24,343  \\
Chile   &  1960  & 7.59m &  2.55m   
                &  \phantom{1}5,086   &   16,666  \\
Chile     &  2004   & 15.67m\phantom{1} &  6.57m   &  12,678  &  29,437  \\
\hline\hline
\end{tabular}
\end{center}
POP is population (millions), Employment is the number of people working 
(millions), Y is GDP (2000 US dollars), 
and K is capital (2000 US dollars).  
Y and K are PPP-adjusted.  
Data are from the Penn World Tables.  


\begin{enumerate}
\item Compute the (average annual continuously compounded) 
growth rates of GDP per capita and GDP per worker in the two countries.  (10~points)

\item Use our growth accounting methodology to allocate growth in
output per worker to TFP and capital per worker.
Which factor has been most important over the last 40+ years? (15~points)

\item Speculate on the differences in the two countries 
that led to the observed differences in their performance.  (5~points)

%\item Institutions??  

\item Using whatever resources you think appropriate, 
comment on the likely performance of the two countries over the next 
decade.  
What do you see as the major opportunities?  Challenges?  (20~points) 

\end{enumerate}


%\begin{comment}
Answer.  
We've seen that GDP per capita in Chile has gone 
from 35\% below Argentina in 1960 to 16\% above in 2004.
How did this happen?
Most of this problem is concerned with aggregate data:
how much comes from capital, TFP, etc?  
The last part is an opportunity to look further 
into features of the two economies that might account 
for what we find in the data analysis.  
%
\begin{enumerate}
\item The (continuously compounded annual) 
growth rate of GDP per capita for Argentina is 
\[
    \gamma \;=\; \log (10,939/7,838) / (2004 - 1960) \;=\; 0.76\%. (!) 
\]
Refer to the discussion of growth rates in ``Sources of Growth''
if you're not sure why this works.
A similar calculation gives us 2.08\% for Chile.  
GDP per worker is GDP per capita times the population then
divided by the number of workers.  
Its growth rates are 0.64\% (Argentina) and 1.57\% (Chile).  
Bottom line:  faster growth in Chile.  


\item We use the production function 
\[
    Y/L \;=\; A (K/L)^{1/3} ,
\]
so TFP is $ A = (Y/L) / (K/L)^{1/3}  $. 
In growth rates, this translates into 
\begin{eqnarray*}
    \gamma_{Y/POP} &=&  \gamma_{L/POP}  + \gamma_A 
                + \alpha \gamma_{K/L}  ,
\end{eqnarray*}
which allows us to divide growth in per capita GDP into 
components due to the employment rate, productivity, 
and capital per worker.  
A few calculations give us 
%
\begin{center}
%\tabcolsep = 0.12in
\begin{tabular}{lcccc}
\hline\hline
            &  $Y/\POP$      &  $L/\POP$    &   $A$   &  $K/L$     \\
\hline\hline
\multicolumn{5}{l}{\it Argentina} \\
1960        &    7,838    &  0.393  &   626  &  32,332      \\
2004        &    10,939\phantom{1}   
                          &  0.414  &   679  &  58,754    \\
Growth rate (\%) &  0.76  &  0.12   &   0.19 &  1.36   \\
Contribution to growth (\%) 
                &   0.76  &  0.12   &   0.19 &  0.45  \\
\hline                 
\multicolumn{5}{l}{\it Chile} \\
1960        &    5,086    &  0.336   &  412  & 49,616      \\
2004        &    12,678\phantom{1}
                          &  0.419   &  733  & 70,210    \\
Growth rate (\%) &  2.08  &  0.50   &   1.31 &   0.79   \\
Contribution to growth (\%) 
                &   2.08  &  0.50   &  1.31  &   0.26  \\
\hline\hline
\end{tabular}
\end{center}
Here ``growth rate'' is computed as in (a).
``Contribution'' modifies these growth
rates as they occur in the formula:  the growth rate of $K/L$ gets
multiplied by 1/3 (its exponent in the production function).

The numbers tell us that almost all of the difference in growth 
comes from TFP.  

\item Almost all of the institutional indicators are stronger for Chile
than Argentina, so perhaps it's not surprising that economic performance
has been better.  
Some of this has come in the last 25 years, as Chile has reestablished a 
stable democracy with sensible economic policies, no matter who's in power.
Argentina continues to fluctuate between policies that favor long-term 
performance and those that focus on the short term.  

\item Your call.  My own opinion is that the strong economic and political 
foundations Chile has laid will continue to generate higher growth 
than we'll see in Argentina. 
For one thing, we've had different parties in power over the last 20 years, 
yet the same basic economic policies are followed.  
One that I find intriguing is a privatized social security system, 
which helped build a domestic capital market.  
A temporary factor is copper prices, which are unusually high right now. 
It's a tricky measurement issue, but this tends to show up as high GDP 
and therefore high TFP.  
History tells us that probably won't last.

\end{enumerate}
%\end{comment}

\end{enumerate} 

\vfill \centerline{\it \copyright \ \number\year \  
NYU Stern School of Business}

\end{document}

\begin{enumerate}

\item {\it Japan and Korea. }   
Japan and Korea are remarkable success stories, 
having experienced unusually rapid economic growth over long periods of time.
In Japan, per capita GDP rose from 30\% of the US in 1870 
(similar to Mexico today) to about 75\% today.  
In Korea, GDP per capita rose from about 10\% of the US in 1960 
(similar to the Philippines at the time) 
to over 50\% today.
Your mission:  to determine the sources of the remaining differences
between the two countries --- capital, productivity, and anything else 
that you think is relevant.  

The data for 2003 
(the most recent year for which all of these numbers are available):
\begin{center}
%\tabcolsep = 0.12in
\begin{tabular}{lcccccc}
\hline\hline
        &  POP  &  Y/POP  &  L/POP  &  K/Y     &  Education &  Hours \\
\hline\hline
Japan   & 127.7m &  24,036 &  0.534 & 4.003     & \phantom{1}9.6 & 1801 \\%
Korea   &  \phantom{1}47.5m 
                 &  17,600 &  0.494 & 1.608    & 11.0  &  2434  \\%
\hline\hline
\end{tabular}
\end{center}
POP is population (millions), Y is GDP (2000 US dollars), 
K is capital (2000 US dollars), Education is years of school, 
and Hours is annual hours worked.  
Y and K are PPP-adjusted.  
Education and Hours are from the OECD's {\it Employment Outlook\/};
the other variables are from the Penn World Tables.  


\begin{enumerate}
\item Compute output per capita, output per worker, and output per
hour worked.  How do they differ? (15~points)

\item What are the primary sources of the difference in output per
capita? (15~points)

\item Using whatever resources you think appropriate, 
comment on the likely performance of the two countries over the next 
decade.  
What do you see as the major opportunities?  Challenges?  (20~points) 
\end{enumerate}


%\newpage
%\begin{comment}
Answer.
\begin{enumerate}
\item As we know from class, GDP per capita, per worker, and per
hour worked are simply the ratios of GDP to population, number of
workers, and total hours worked, respectively. The numbers are:
\begin{center}
\begin{tabular}{lccc}
\hline\hline
Country   &  GDP Per Capita &  GDP Per Worker &  GDP Per Hour Worked \\
\hline\hline
France  &  26.673  & 65.01 &  0.0447 \\
US      &  35.356  & 73.19 &  0.0408 \\
France/US &  0.754 &  0.888 &  1.095 \\
\hline\hline
\end{tabular}
\end{center}
Note that the ratios (the bottom line) increase as we move to the
right:  the difference in GDP per person is due to differences in the amount of work (workers and hours).

\item A natural version of the production function in this case is
\[
    Y \;=\; A K^\alpha (H h L)^{1-\alpha} ,
\]
where $h$ is average hours worked and $L$ is the number of workers
(employment).  We find $A$ by substituting in for all the other
numbers:  $ A =Y / [ K^\alpha (H h L)^{1-\alpha}] $.  How do we
use this to make sense of output per capita? If POP is the
population, then
\[
    Y/\POP \;=\; (Y/L) (L/\POP) \;=\; (L/\POP) A (K/L)^\alpha (H
    h)^{1-\alpha} .
\]
?? ** do more on education ??
This isn't the only possible line of attack on this problem, but
it's a reasonable one.  It tells us we can attribute differences
in per capita GDP to differences in:  participation ($L/\POP$),
total factor productivity ($A$), capital per worker ($K/L$), human
capital ($H$), and average hours per worker ($h$). The math
implies
%
\begin{eqnarray*}
    \frac{(Y/\POP)_{F}}{(Y/\POP)_{US}} &=&
             \left[ \frac{(L/\POP)_F}{(L/\POP)_{US}} \right]
             \left[ \frac{A_F}{A_{US}} \right]
             \left[ \frac{(K/L)_{F}}{(K/L)_{US}} \right]^\alpha
             \left[ \frac{H_{F}}{H_{US}} \right]^{1-\alpha}
             \left[ \frac{h_{F}}{h_{US}} \right]^{1-\alpha} \\
             &=& \left[ \frac{.410}{.483} \right]
                 \left[ \frac{.0158}{.0147} \right]
                 \left[ \frac{232.7}{201.3} \right]^{1/3}
                 \left[ \frac{11.9}{13.8} \right]^{2/3}
                 \left[ \frac{1453}{1792} \right]^{2/3}_{\phantom{\sum_i}} \\
             &=& (0.849)(1.074) (1.050) (0.862) (0.870) \;=\;
             0.754 .
\end{eqnarray*}
%
Where does the 25\% difference in GDP per person come from?
In order of importance:  ratio of employment to population, education, 
and hours worked.  
Capital per worker and TFP go the other way ---
they're higher in France. Obviously there are many caveats to this
analysis.  A major one is about education, which we have measured
with average numbers of completed years of schooling for the
population older than 15. Obviously, this is not the only way. For
example, we could try to correct this indicator for the quality of
education. France boasts the lowest number of pupils per teacher
in the world. If we accounted for this, the differences in human
capital between the two countries would not be as stark.
Another is that a lower employment rate may lead to a higher 
average skill level, with the least skilled not working.  

\item Evidently labor market outcomes are very different in the
two countries. Our conjecture is that differences in the
institutions and legislation that regulates the functioning of
labor markets play a major role. The US is probably a better place
to work, at least if you want to work long hours.  
We still love France as a place to live and eat, though.

\end{enumerate}
%\end{comment}

%\end{enumerate} \end{document}

% ===============================================================================
\item {\it Argentina and Chile. } 
Argentina and Chile have both experienced dramatic economic growth
and painful reversals over the last century.  
Argentina was one of the richest countries in the world in 1900, 
but mixed economic performance since then has 
dropped it back into the developing world.  
Chile suffered a traumatic change in government in the 1970s, 
but emerged a decade later as a fast-growing democracy with 
the highest per capita GDP in Latin America.  
\begin{center}
%\tabcolsep = 0.12in
\begin{tabular}{lccccc}
\hline\hline
Country   &  Year   &  POP   &  Employment &  Y/POP &  K/POP  \\
\hline\hline
Argentina \hspace*{0.1in} &  1960   &  20.6m &  \phantom{1}8.1m   
                &  \phantom{1}7,838  &  12,713  \\
Argentina &  2004   &  39.1m &  16.2m             &  10,939 &  24,343  \\
Chile   &  1960  & 7.59m &  2.55m   
                &  \phantom{1}5,086   &   16,666  \\
Chile     &  2004   & 15.67m\phantom{1} &  6.57m   &  12,678  &  29,437  \\
\hline\hline
\end{tabular}
\end{center}
POP is population (millions), Employment is the number of people working 
(millions), Y is GDP (2000 US dollars), 
and K is capital (2000 US dollars).  
Y and K are PPP-adjusted.  
Data are from the Penn World Tables.  


\begin{enumerate}
\item Compute the (average annual continuously compounded) 
growth rates of GDP per capita and GDP per worker in the two countries.  (10~points)

\item Use our growth accounting methodology to allocate growth in
output per worker to TFP and capital per worker.
Which factor has been most important over the last 40+ years? (15~points)

\item Speculate on the differences in the two countries 
that led to the observed differences in their performance.  (5~points)

%\item Institutions??  

\item Using whatever resources you think appropriate, 
comment on the likely performance of the two countries over the next 
decade.  
What do you see as the major opportunities?  Challenges?  (20~points) 

\end{enumerate}


%\begin{comment}
Answer.  NB:  the numbers used in class are based on more recent
data for education and differ slightly from those reported in the
assignment.
\begin{enumerate}
\item GDP per worker is GDP per capita times the population then
divided by the number of workers:  4,576 in 1961 and 36,851 in
2000.  Its continuously compounded growth rate is
\[
    \gamma \;=\; \log (36,851/4,576) / 39 \;=\; 5.35\%.
\]
Refer to the discussion of growth rates in ``Sources of Growth''
if you're not sure why this works.

\item We use the production function in output per worker form,
\[
    Y/L \;=\; A (K/L)^{1/3} H^{2/3} .
\]
TFP is then $ A = (Y/L) /[ (K/L)^{1/3} H^{2/3}]  $. The necessary
data for growth accounting is
%
\begin{center}
%\tabcolsep = 0.12in
\begin{tabular}{lcccc}
\hline\hline
Year        &  $Y/L$      &   $A$   &  $K/L$    &  $H$   \\
\hline\hline
1961        &    4,576    &  769    &   11,676  & 4.25      \\
2000        &    36,851   &  1,600  &  103,947  & 10.84     \\
Growth rate (\%) &  5.35  &  1.88   &   5.61    &   2.40   \\
Contribution to growth (\%) &   5.35   &  1.88  &  1.87   &   1.60  \\
 \hline\hline
\end{tabular}
\end{center}
The third row (``growth rate'') is computed as we described in
(a):  the logarithm of the ratio of the 2000 number to the 1961
number, divided by 39 (the number of years between 1961 and 2000).
The next row (``contribution to growth'') modifies these growth
rates as they occur in the formula:  the growth rate of $K/L$ gets
multiplied by 1/3 (its exponent in the production function) and
the growth rate of $H$ gets multiplied by 2/3 (its exponent).

Where does growth in output per worker come from?  The 5.35\%
average growth in output per worker has three components of
comparable size:  TFP (1.88), capital per worker (1.87), and human
capital (1.60).

\item Per capita GDP grew by 5.98\% a year.  The difference of
0.63\% reflects growth in the ratio of workers to population,
which rose sharply in Korea over this period.
\end{enumerate}
%\end{comment}

\end{enumerate} \end{document}

% ===============================================================================
\item {\it China. } 
China's remarkable economic growth and large
population have created one of the world's largest economies. In
2004, GDP in China was slightly greater than half the size of US
GDP, but growing twice as fast. The question is how this is likely
to change in the coming years. Will China become the world's
largest economy?

Your mission is to estimate the economic magnitudes of the US and
China in 2020 using the Solow model and some inspired estimates of
the parameters. In 2004, the economies had the following
characteristics:
%
\begin{center}
\tabcolsep = 0.17in
\begin{tabular}{lccc}
\hline\hline
Country   &  GDP &  Capital &  Employment  \\
\hline\hline
China   &  5,592 &  22,276   &   747.36   \\%
US      & 10,761 &  31,672   &   141.93  \\%
\hline\hline
\end{tabular}
\end{center}
%
GDP and capital are reported in billions of 2000 US dollars,
employment and population in millions.  Your economic consultants
tell you that the parameter values are
%
\begin{center}
%\tabcolsep = 0.1in
\begin{tabular}{lcccc}
\hline\hline
Country \hspace{0.1in}  &  Saving &  Depreciation &  Empl. Growth &  TFP Growth    \\
\hline\hline
China   &  22\% &  6\%   &   1.0\%  & 4\%  \\%
US      &  20\% &  6\%   &   0.5\%  & 2\%  \\%
\hline\hline
\end{tabular}
\end{center}
%
\begin{enumerate}
\item Compute TFP for each country in 2004 using the basic
production function, $Y = A K^\alpha L^{1-\alpha}$. (5~points)

\item Compute time paths for GDP in both countries starting in
2004 and ending in 2020. How large is China relative to the US in
2020? (25~points)

\item Comment (briefly) on the strengths and weaknesses of your
analysis.  What parameter values are you least certain about? What
features of the world does the model miss? (10~points)
\end{enumerate}
Data sources. GDP and Employment are from the World Bank. Capital is
our estimate, based on data on investment from the PWT and the World
Bank.

%\begin{comment}
Answer.
\begin{enumerate}
\item A variant of the usual calculation:  $A = Y /(K^{\alpha}
L^{1-\alpha})$. (We don't have education data, so its impact will 
be inherited by $A$ --- ``everything else.'')  
The numbers are 2.413 for China and 12.500 for the US.

\item This is a dynamic simulation much like the one described in
``Solow's Model of Economic Growth.''  The major difference is the
addition of growth in TFP.

We simulate the model like this.  At each date, we compute
employment $L$ and productivity $A$ from their values at the
previous date by using their (assumed constant) growth rates:
\begin{eqnarray*}
    L_{t+1} &=& (1+n) L_t \\
    A_{t+1} &=& (1+a) A_t,
\end{eqnarray*}
where $n$ and $a$ are the growth rates of employment and TFP,
respectively. Capital evolves as in the Solow model:
\begin{eqnarray*}
    K_{t+1} &=& s A_t K_t^\alpha L_t^{1-\alpha} + (1 - \delta) K_t .
\end{eqnarray*}
All we need are the parameters, which are given to us in the
problem. 
%
The complete solution is
\begin{center}
\tabcolsep = 0.2in
\begin{tabular}{lcc}
\hline\hline Year   &  China   &  US  \\
\hline\hline
    2004  &  5,592   &  10,761  \\
    2005  &  5,845   &  11,042  \\
    2006  &  6,115   &  11,335  \\
    2007  &  6,404   &  11,640  \\
    2008  &  6,712   &  11,958  \\
    2009  &  7.043   &  12,289  \\
    2010  &  7,396   &  12,634  \\
    2011  &  7,774   &  12,992  \\
    2012  &  8,179  &  13,366  \\
    2013  &  8,612  &  13,755  \\
    2014  &  9,077  &  14,159  \\
    2015  &  9,575  &  14,579  \\
    2016  &  10,109  &  15,016  \\
    2017  &  10,681  &  15,471  \\
    2018  &  11,295  &  15,943  \\
    2019  &  11,593  &  16,433  \\
    2020  &  12,659  &  16,943  \\
\hline\hline
\end{tabular}
\end{center}

In this scenario, China overcomes about half the difference 
between the two countries by 2020.  
If you keep running the model, it surpasses the US in 2030.  
In per capita terms, there remain large differences
between the two countries even then.  


\item All of the parameters could be questioned --- take your
pick. For example, in a related exercise described in the Goldman
Sachs report entitled ``Dreaming of BRICs,'' it is argued that the
rate of TFP growth will fall as China approaches the world's
``technology frontier.'' Another issue is foreign-financed capital
formation. As you recall, in the context of the Solow model
capital can grow faster only if the domestic saving rate
increases. In reality, this is not true. Foreign residents can
lend their savings to domestic companies, providing a further
source of financing for capital expenditures. In the exercise we
somewhat dealt with this by using numbers for the parameters $s$
which are actually closer to the average investment rates, rather
than the saving rates. Still, you you might imagine that changes
in the climate for foreign investment could lead to more rapid
accumulation of capital than the model allows for.

\end{enumerate}
%\end{comment} 

\end{enumerate}


\vfill \centerline{\it \copyright \ \number\year \  NYU Stern
School of Business}


\end{document}
