\documentclass[letterpaper,12pt]{article}

\RequirePackage{GE05}
% this inputs graphicx, too
\RequirePackage{comment}
\RequirePackage[hypertex]{hyperref}
    \hypersetup{colorlinks=true,urlcolor=blue,linkcolor=red}

\def\ClassName{The Global Economy}
\def\Category{Professor David Backus}
\def\HeadName{Effective Business Writing}

\begin{document}
\parindent = 0.0in
\parskip = 0.25\bigskipamount
\thispagestyle{empty}%
\Head

\centerline{\large \bf \HeadName}%
\centerline{Revised:  \today}

\medskip
One of the most useful skills in the business world and beyond 
is the ability to write and communicate effectively.  
Many of you will experience this in your internships
this coming summer, where a final report or presentation
is the norm.  
Several of our projects have been designed as opportunities to develop
this skill and will be graded on both content
and whether they convey the content in an effective professional manner. 

There's no definitive best model for business writing,
whether it be reports, memos, flip charts, slides, or something else.  
My favorite examples have the following elements:  
%
\begin{itemize}
\item Reader-friendly.  
The important information should be easy to find and read.  
If you don't know what the important information is, 
sit down and think before writing. 
If you do, you should make sure it's apparent to even a casual reader.
Remember:  everyone's busy these days, make it easy for them.  

\item Executive summary.  Start with a short focused summary.  Could be (say) 3-5 sentences, bullet points, whatever you think works 
     --- and looks --- best.  
     
\item Concise.  I will generally give you page guidelines, 
but shorter is better for two reasons:  it's easier on the reader and 
it forces you to decide what's important.  
Do not include everything you know, only what's relevant to your purpose.
If you have extra related material, try to put it in an appendix, 
which people can read or not as they like.  
    
\item Effective graphics.  Attractive and informative graphics 
can be extremely helpful in getting your ideas across.
Try to make them understandable
on their own without reference to the text.  A good title helps.  

\item Clear headings.  If done well, the headings should 
give you an outline of the argument.  
You don't want to overdo bold and italics, 
but selective use can help here.  

\end{itemize}
%
Last piece of advice:  when you read business (or other) writing, 
pay attention to format and structure.  
If you see good ideas, copy them.  

For further information, see:   \\ 
Samples:  \url{http://www.stern.nyu.edu/eco/B012303/Backus/Writing_samples/} \\
Guide to business reports:  \url{http://www.businessballs.com/writing.htm} 


\vfill \centerline{\it \copyright \ \number\year \ 
NYU Stern School of Business}


\end{document}
