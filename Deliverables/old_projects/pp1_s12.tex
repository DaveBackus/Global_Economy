\documentclass[12pt]{exam}

\usepackage{ge05}
\usepackage{comment}
\usepackage[dvipdfm]{hyperref}
\urlstyle{rm}   % change fonts for url's (from Chad Jones)
\hypersetup{
    colorlinks=true,        % kills boxes
    allcolors=blue,
    pdfsubject={NYU Stern course GB 2303, Global Economy},
    pdfauthor={Dave Backus @ NYU},
    pdfstartview={FitH},        
%    pdfnewwindow=true,      % links in new window
%    linkcolor=blue,         % color of internal links
%    citecolor=blue,         % color of links to bibliography
%    filecolor=blue,         % color of file links
%    urlcolor=blue           % color of external links
% see:  http://www.tug.org/applications/hyperref/manual.html
}

% for ge05.sty
\def\ClassName{The Global Economy}
\def\Category{Professor David Backus}
\def\HeadName{Problem Set \#1}
\newcommand{\phm}{\phantom{--}}
\newcommand{\NX}{\mbox{\it NX\/}}

\noprintanswers
%\printanswers

\begin{document}
\parindent = 0.0in
\parskip = \bigskipamount
\thispagestyle{empty}%
\Head

\centerline{\large \bf \HeadName: Macroeconomic Data}
\centerline{Revised:  \today}

\medskip
{\it You may do this assignment in a group of up to give people.  
What you hand in should be the group's work.}

\begin{questions}
% --------------------------------------------------------------------
\question National accounts in Margaritaville (40 points).
Jimmy Buffett has decided to apply for membership in the European Union 
on behalf of his newly sovereign nation, Margaritaville. 
As part of his application, he must provide the EU
technocrats with a complete set of national accounts.  
You have been hired as the Chief National Accountant. 
Your first day on the job, 
you receive an official Coral Reefer Crew{\texttrademark} t-shirt
and the following information about local economic activity:
%
\begin{itemize}
\item Local Cheeseburger in Paradise{\texttrademark} cafes
sold \$50,000 worth of cheeseburgers to local consumers.  
Their expenses were:  imported beef and sesame seeds (\$7,000), 
locally produced catsup (\$10,000), 
wages and benefits (\$20,000), and rent (\$3,000).  
Hint: you will need to compute the profit earned by the cafes.  

\item Local tomato growers sold \$8,000 worth of tomatoes to domestic
catsup producers and exported another \$2,000 to the US.  
They paid land rent (\$1,000) and wages (\$9,000).  

\item Local producers of the Margaritaville Frozen Concoction Maker{\texttrademark }
sold \$100,000 worth of blenders.
40\% were exported to Europe, the remainder to local consumers.  
Their expenses were \$15,000 worth of imported metal, 
\$15,000 for a new CNC machine imported from Germany, 
and \$70,000 in wages.  

\item The domestic catsup industry sold \$10,000 worth of product to local 
cafes.  
They purchased \$8,000 worth of tomatoes from domestic growers 
and paid \$2,000 in wages.  


\item The newly-formed government collected \$10,000 in taxes from its citizens
and paid \$10,000 to government regulators, who oversee food and beverage safety.  
\end{itemize}
%
You mission is to use this raw data to construct 
national income and product accounts for Magaritaville.  
Specifically:
%
\begin{parts}
\part Compute GDP and its expenditure components (consumption, 
investment, government purchases of goods and services, 
exports, and imports).  
(10~points) 

\part What are saving and investment?  Why are they different?
Where does the difference go?
(10~points)

\part Compute the contribution of each production unit to GDP.  
(10~points) 

\part Jimmy looks over your calculation in (a) and is worried 
that you made a mistake. 
Over a couple Land Shark Lagers\texttrademark 
you explain to him that GDP can be computed three different ways:
the sum of expenditure components, 
the sum of value-added across production units, 
and the sum of payments to labor and capital.  
You do the remaining one, payments to labor and capital,
and show him that you get the same answer.  
He buys you a margarita to show his appreciation.  
(10~points) 
\end{parts}

% --------------------------------------------------------------------
\question Inputs and outputs (20 points).  
Specify the most likely direct impact of each of the following
on the production function:  
%
\begin{parts}
\part A new office building in Wuhan, China.  (5~points) 
\part A reduction in the minimum wage that leads more people to work. (5~points)
\part A more efficient air-conditioning system in the Kaufman Management Center. (5~points)
\part A reduction in tariffs in Brazil on imported computer equipment. (5~points)
\end{parts}

% --------------------------------------------------------------------
\question Saving in China and India (40~points).  
China and India have both grown rapidly in the recent past, 
but there are a number of differences.  
One of the saving rate, which is much higher in China.  
Our mission here is simply to document the facts.  

Using one of the sources mentioned below, 
download the expenditure components of GDP (measured at current prices)
for both countries for the period 1990-present.  Use them to
%
\begin{parts}
\part Graph saving, investment, and net exports as ratios to GDP.
Define saving here as $S = Y - C - G $.
How does saving compare in the two countries?  
Investment?  
(20~points)

\item How important are foreign sources of funds to investment
in the two countries?  
(10~points) 

\part Why do you think the two countries show such different patterns?
(10~points)
\end{parts}


Data guide.  One good source is the EIU's CountryData, 
which you can access with these steps:  
\begin{itemize}
\item Go to NYU's  
\href{http://library.nyu.edu/vbl/}{http://library.nyu.edu/vbl/}, 
then click on 
Country Information, 
EIU Country Data, and   
CountryData.  
\item Once there, click on Data Selection to choose the countries,
time period, and series of interest.
\end{itemize}
The IMF has a new interface, too:   
\begin{itemize}
\item Go to 
\href{http://elibrary-data.imf.org/}{eLibrary}.
\item Look for Query Builder, 
click on Country (China: Mainland), 
then under Concept choose National Accounts.
Looks for the series you need.  
\end{itemize}  
Let me know which one you like best and I'll tell everyone.  
\end{questions}

\vfill \centerline{\it \copyright \ \number\year \
NYU Stern School of Business}

\end{document}

