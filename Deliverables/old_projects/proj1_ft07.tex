\documentclass[letterpaper,12pt]{article}

\RequirePackage{GE05}
% this inputs graphicx, too
\RequirePackage{comment}
\RequirePackage[hypertex]{hyperref}
%\hypersetup{dvips}
%\hypersetup{backref,colorlinks=false,urlcolor=cyan,linkcolor=blue}
%\hypersetup{pdfpagemode=None,pdfstartview=FitH}

\def\ClassName{The Global Economy}
\def\Category{Professor David Backus}
\def\HeadName{Group Project \#1}

\begin{document}
\parindent = 0.0in
\parskip = 0.6\bigskipamount
\thispagestyle{empty}%
\Head

\centerline{\large \bf \HeadName}%
\centerline{Revised:  \today}

\medskip 
{\it Submit via Blackboard by 9am January 25.}
\medskip

It's the first day of your summer internship at DaimlerChrysler USA.
You are thrilled by the opportunity to market Mercedes to American consumers, 
but somewhat nervous about your new job.  
Your boss enters your office with your first assignment:  
summarize the macroeconomic conditions most likely to affect 
the US market for Mercedes over the next 12 months.  
She explains that growth, inflation, and interest rates 
have a direct impact on consumer spending, 
and that exchange rates affect your cost of production 
as well as that of your competitors.  
She asks, specifically, for you to read a number of 
industry forecasts and come up with your own synthesis for 
%
\begin{enumerate}
\topsep=0.05\bigskipamount
\itemsep=0.05\bigskipamount

\item Economic growth:  
the year-on-year growth rate of real GDP through the fourth quarter.  

\item Employment:  the year-on-year 
change in nonfarm payroll employment 
through December.  

\item Inflation:  
the year-on-year inflation rate through December 
(consumer price index).  

\item Interest rates:  the ``Fed Funds target rate'' reported 
after the March 27/28 meeting of the FOMC
(Federal Open Market Committee, 
the group that sets US monetary policy).  

\item Foreign exchange rates:  
the rates reported by the 
\href{http://www.federalreserve.gov/releases/h10/update/}
{Federal Reserve} at year-end for 
\begin{itemize}
\item the dollar price of one euro and 
\item the yen price of one dollar.  
\end{itemize}

\end{enumerate}
Each item should include a brief rationale 
and an assessment of its likely impact on the production, pricing, and positioning of Mercedes in the US.  
% target:  5 pages + tables/figures
%Do you think they would affect the company's Chrysler brands 
%the same way?  


You barely remember your Global Economy course, 
but turn immediately to the course's 
\href{http://www.stern.nyu.edu/eco/b012303/Backus/ge_resources_db.htm}
{resource page} for help,
starting with the Economist Intelligence Unit's country data and country 
intelligence, 
%the Bloomberg economic calendar, 
the Cleveland Fed's {\it Economics Trends\/}, 
and the Philadelphia Fed's 
{\it Survey of Professional Forecasters\/}.
You also call a couple classmates and ask for their banks' forecasts.  


\vfill \centerline{\it \copyright \ \number\year \ NYU Stern
School of Business}

%\end{document}

% --------------------------------------------------------------------------
\newpage
{\bf We  discussed this at some length in class, 
so what follows is extremely brief:  }  
\begin{enumerate}
\item GDP growth:  the numbers floating around are typically between 2.0 and 2.5\% (below average), indicating growth somewhat below average.    
Commonly mentioned sources of concern:  
flat yield curve, current account deficit, 
weak housing market, 
dozens of other short-run indicators.
Impact on:
\begin{itemize}
\item Production:  car sales are very cyclical, 
luxury brands possibly less so. 
In this case, you might suggest a modest decline in production
if the economy slows, as some expect.   
\item Pricing:  expect soft pricing in a soft market. 
You might expect this to be less so for a luxury brand, 
and might mention differences across models.    

\item Positioning:  Brands have to be coherent, but there's often room to 
change emphasis as market conditions change.  
In a boom, stress luxury, in a weak market maybe safety and reliability.  
\end{itemize}

\item Employment: another indicator of growth (employment typically goes up and down with GDP, and is closely watched on Wall Street).  
Relevant evidence:  recent job growth has been modest. 
Implications:  same as above.  

\item Inflation:  typical numbers are 2.0-2.5\% annually.  
Relevant information:  
Actual inflation was been in the mid-3s last year, 
largely due to a spike in energy prices.  
The drop in oil prices should lead to a reversal if it continues.  
Many analysts would look at both inflation 
and core inflation (``all items less food and energy'' in the report),
which excludes two volatile components.  
Impact on:
\begin{itemize}
\item Production:  little.  
Could effect future wages, hence cost of production.
\item Pricing:  you'd expect to be able to raise prices by the inflation 
rate, on average.  
Change in energy prices could affect demand for cars, esp those that 
are not fuel-efficient.  
\item Positioning:  little.  
\end{itemize}

\item Interest rates:  the Fed has kept the Fed funds rate 
(an overnight interest rate that the Fed targets) 
at 5.25\% for several meetings of its monetary policy group.  
Most observers expect this rate to fall by the end of the year.  
Impact on:
\begin{itemize}
\item Production:  little, although a fall in rates
could make cars more affordable.  
\item Pricing:  might make cars more affordable, 
but this is less likely with a luxury brand than (say) a 
mid-line Chrysler.  
\item Positioning:  little.  
\end{itemize}

\item Foreign exchange:  the simplest reasonable guess is no change, 
but there are lots of different views out there.  
If you can predict exchange rates consistently, 
you should either start a hedge fund or head to Vegas.  

Impact on:
\begin{itemize}
\item Production and pricing:  since Mercs are produced (in part) in Europe, 
an increase in the value of the euro would raise costs for dollar buyers.  
The reverse, of course, for a decrease.  
An increase would squeeze margins, leaving you with the choice of 
lower margins or smaller market share vis a vis non-European competitors.  
The yen has a similar effect on Japanese producers.  
If the yen falls and the euro rises, you get a double whammy:  
your costs have gone up, theirs have gone down.  
\item Positioning:  little.  
\end{itemize}

\end{enumerate}



\end{document}


Comments on market impact from a 2000 alum:

> Prof. Backus,
>
> Although there is no one universal answer for each case due to seond-order
> effects, common sense and a little bit of macro economic analysis would
> yield the following:
>
> 1. Higher inflation: buy 10-year inflation protected treasuries, sell
> 10-year nominal treasuries (higher inflation erodes the value of nominal
- Hide quoted text -
> future cashflows, while raises the coupon on inflation protected
> securities).
>
> Alternative answer: sell 2-year treasury, buy 10-year treasury (higher
> inflation may result in a fed funds rate increase, which impacts the 2-year
> yield more than the 10-year yield, resulting in a flattening of the yield
> curve)
>
> 2. Higher GDP: Buy 'AA' corporate bond, sell treasuries (higher GDP suggest
> a more robust business activity and higher corporate profits, which reduces
> risk premium and tightens corporate bond spreads to treasuries).
>
> 3. Higher employment - a combination of 1 and 2 - higher employment may
> suggest hiring trend due to business cycle strength, but may also spark a
> demand-side inflation pressure.
>
> 4. Stronger euro - buy bunds, sell treasuries. (euro-denominated future
> casflows are more valuable in a strengthening euro scenario).
>
> Please feel free to contact me with any questions.
>
> Itai Benosh
> Freddie Mac
> MBA Class of 2000
>
> ----Original Message Follows----
> From: David Backus <david.backus@gmail.com>
> Reply-To: dbackus@stern.nyu.edu
> To: Itai Benosh <ibenosh@hotmail.com>
> Subject: Re: Comments on Project #1
> Date: Wed, 16 Mar 2005 11:00:55 -0500
>
> Itai,
>
> I've been thinking about your idea and wonder if I could impose
> further and get a suggested answer.  Could you tell me how you think
> the various assets would respond to your question if you forecast
> higher than consensus:  inflation, output growth, employment?
>
> Thanks again for your help.
>
> Dave
>
> On Mon, 14 Mar 2005 23:44:57 -0500, Itai Benosh <ibenosh@hotmail.com> wrote:
> >
> > Prof. Backus,
> >
> > Please accept my apologies for the late response. In order to add some
> "real
> > life" experience to this project, I would add the following question:
> >
> > Compare your estimates to the average estimates of your study group. (or
> to
> > consensus estimates on Bloomberg). Assuming the markets fully reflect the
> > average (or the consensus), and given your estimates, which of the
> following
> > assets would you recommend that your clients buy (or sell), and why?
> > - 2 Year US Treasury
> > - 10 Year US Treasury
> > - 10 Year German Bund
> > - 10 Year inflation protected US Treasury
> > - 5 Year 'AA' rated corporate bond.
> >
> > Hope this helps. Please let me know if you need a speaker on issues such
> as
> > Debt Markets or Global Funding.  
