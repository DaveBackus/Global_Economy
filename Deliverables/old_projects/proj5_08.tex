\documentclass[letterpaper,12pt]{article}

\RequirePackage{GE05}
% this inputs graphicx, too
\RequirePackage{comment}
\RequirePackage[hypertex]{hyperref}
    \hypersetup{colorlinks=true,urlcolor=blue,linkcolor=red}

\def\ClassName{The Global Economy}
\def\Category{Professor David Backus}
\def\HeadName{Group Project \#5}

\begin{document}
\parindent = 0.0in
\parskip = \bigskipamount
\thispagestyle{empty}%
\Head

\centerline{\large \bf \HeadName:  Fixed Income Research}%
\centerline{Revised:  \today}

\medskip
{\it Submit via Blackboard by 9am April 1.}
\medskip

You are delighted to have a summer internship in Lehman's capital markets group.  
Your first rotation:  Fixed Income Research.  
On your first day, the Managing Director gives you 
a small project to get your feet wet. 
Noting that bond markets are driven largely by macroeconomic news, 
she asks you to write a report summarizing the prospects for the US 
economy over the coming year --- specifically, 
the change in employment (nonfarm payroll) between now and, respectively, 
1 and 6 months from now. 
Your report should include:  
%
\begin{itemize}

\item A list of what you think are the most useful indicators
of future employment 
along with a short description of what they are and why you think they 
are useful.  

\item A statistical analysis of what each of these indicators implies  
for employment over the next 1-6 months.  

\item A combination forecast that aggregates  --- in some way --- 
the information in each of your chosen indicators into a 
forecast of the change in employment.  

\item A narrative to be shared with clients 
about the likely evolution of the US economy over the next 6 months.  
\end{itemize}
%
She hands you a spreadsheet containing a large number of 
economic indicators and walks away.  

Notes: 
\begin{itemize}
\item Your mission is to forecast the change in employment, 
not the growth rate.
If you run a forecasting regression with the growth rate, 
be sure to convert the forecast to the change before reporting.  
Why?  Because that's the industry standard.   
\item If you find your forecasting regression has an $R^2$ over 0.3, 
you should check to make sure you're really forecasting --- that you
have the timing right in the regression.
It's generally easier to forecast the future with future data
than with current data, but it's not really a forecast. 
\end{itemize}


\newpage
{\it Thoughts about effective business writing}

You'll see a wide range of formats and structures 
for business communications:
memos, reports, slides, flip charts, and so on.
There's no definitive best model,
but my favorite examples have the following elements:  
%
\begin{itemize}
\item Reader-friendly.  
The important information should be easy to find and read.  
If you don't know what the important information is, 
sit down and think before writing. 
If you do, you should make sure it's apparent to even a casual reader.
Remember:  everyone's busy these days, make it easy for them.  

\item Executive summary.  Start with a short focused summary.  Could be (say) 3-5 sentences, bullet points, whatever you think works 
     --- and looks --- best.  
    
\item Effective graphics.  Attractive and informative graphics 
can be extremely helpful in getting your ideas across.
Try to make them understandable
on their own without reference to the text.  A good title helps.  

\item Clear headings.  If done well, the headings should 
give you an outline of the argument.  
You don't want to overdo bold and italics, 
but selective use can help here.  

\end{itemize}
%
Last piece of advice:  when you read business (or other) writing, 
pay attention to format and structure.  
If you see good ideas, copy them.  

For further information, see:   \\ 
Samples:  \url{http://www.stern.nyu.edu/eco/B012303/Backus/Writing_samples/} \\
Guide to business reports:  \url{http://www.businessballs.com/writing.htm} 


\vfill \centerline{\it \copyright \ \number\year \ 
NYU Stern School of Business}


%\end{document}
% --------------------------------------------------------------------------
\newpage
{Sketch of an answer.  
The project is a loosely structured real-world problem;
it does not lend itself to a simple mechanical answer.
Here's a sketch of what I'd do: }  
\begin{itemize}
\item List of indicators.  
We've looked at many, take your pick.  
Among the most popular:  
housing starts and building permits, 
stock market, 
new claims for unemployment insurance, 
industrial production, term spreads.  
The challenge here is that a 6-month forecast is very demanding:
you need a substantial lead in anything you use.  

\item Statistical analysis.  
I think cross-correlation functions are useful ways to get across leads and lags, but you can get across similar 
information in different ways --- 
regressions with lags, for example,
or scatterplots of variables with different leads and lags.  

\item Combination.  
I'd use a multivariate regression, with the forward-looking 
6-month growth rate of employment as the dependent variable
and several indicators as independent variables.
I'd include lags of the indicators and the dependent 
variable as additional independent variables.
You'd have to take a course in time series statistics 
to understand this fully, 
but the idea is that they allow more flexibility in the dynamic relations
among the variables.  

\item Narrative.  This would depend on what you find.
If you find that (say) housing variables are weak,
and that this weakness translates (through the regression)
into a forecast of weak growth, 
you could join Professor Roubini and many others and talk about that.
Almost everyone predicts weakness right now, 
the question is how much.  

\end{itemize}

\end{document} 
