\documentclass[letterpaper,12pt]{exam}

%\usepackage[hypertex]{hyperref}
\usepackage{amsmath}
\usepackage{natbib}
%\usepackage[dvips]{graphicx}
\usepackage{comment}
\RequirePackage{GE05}
% this inputs graphicx, too

\newcommand{\GDP}{\mbox{\em GDP\/}}
\newcommand{\NDP}{\mbox{\em NDP\/}}
\newcommand{\GNP}{\mbox{\em GNP\/}}
\newcommand{\NX}{\mbox{\em NX\/}}
\newcommand{\NY}{\mbox{\em NY\/}}
\newcommand{\CA}{\mbox{\em CA\/}}
\newcommand{\NFA}{\mbox{\em NFA\/}}
\newcommand{\Def}{\mbox{\em Def\/}}
\newcommand{\CPI}{\mbox{\em CPI\/}}
\newcommand{\phm}{\phantom{--}}

\def\ClassName{The Global Economy}
\def\Category{Professor David Backus}
\def\HeadName{Final Examination}

\begin{document}
\parindent = 0.0in
\parskip = \bigskipamount
\thispagestyle{empty}%
\Head

\centerline{\large \bf \HeadName}%
\centerline{Revised:  \today}

\bigskip
You have 100 minutes to complete this exam.  Please answer each question in the space provided.
You may consult one page of notes and a calculator, but devices capable of wireless transmission
are prohibited.

I understand that the honor code applies: I will not lie, cheat, or
steal to gain an academic advantage, or tolerate those who do.

\begin{flushright}
\rule{4in}{0.5pt} \\ (Name, section, and signature)
\end{flushright}


\begin{enumerate} 
% ---------------------------------------------------------
\item {\it Stimulus in China (30 points).\/}
China has responded to the current global crisis by 
implementing a massive program of government spending 
on infrastructure. 
Your mission is to outline the argument for or against such a program
using the aggregate supply and demand (AS/AD) framework.  
%
\begin{enumerate}

\item Over the last year, output growth and inflation have both fallen in China.  Would you say this comes from a shift in supply or demand?  
    Illustrate your answer with the appropriate diagram.  
    (10~points)

\item Describe the impact of 
a large increase in government spending on infrastructure projects.
What is the likely impact on output?  Inflation?  
(10~points) 

\item What are the traditional goals of macroeconomic policy, 
expressed in terms of aggregate supply and demand?  
Does the Chinese spending program move them closer to these goals?  
(10~points) 

\end{enumerate}

%\begin{comment}
Answer.
\begin{enumerate}
\item Shifts in demand move output and prices in the same direction, 
shifts in supply move them in opposite directions.
(By longstanding tradition, we interpret output as output growth
and prices and inflation.)  
Since they both fell, we would interpret this as a shift left in demand.  

Grading:  5 points for graph, 5 more for showing that 
a demand shock reduces output growth and inflation.  

\item This is a purchase of goods, therefore affects demand. 
A shift right in demand increases output growth and inflation.  

Grading:  5 points for noting this is a shift right in demand, 
5 more for the impact on output and inflation.  

\item The goals are (i) output equal to the long-run aggregate supply curve AS$^*$ and (ii)~stable prices.  
    The answer depends where you start:  are we to the left of AS$^*$ prior to the stimulus?  If so, then the stimulus program moves 
output in the right direction.  
Ditto with inflation:  if we start with stable prices, 
the stimulus 

Grading:  5 points for noting the goals, 5 more for noting
the appropriate policy and the importance of the staring point.  

\end{enumerate}
%\end{comment}

%\pagebreak \phantom{bla} \pagebreak %\phantom{bla} \pagebreak
% ---------------------------------------------------------
%
\begin{table}[h]
\vspace{1em}%
\centering%
\hspace{-6cm}%
\begin{minipage}
{0.52\textwidth}%
\begin{center}{\small
\begin{tabular}{lrrrrrrrr}%
\vspace{-0.6cm}\\
\hline%
\vspace{-.3cm}\\
    & 2002 & 2003 & 2004 & 2005 & 2006 &  2007 &  2008 & 2009 \\%
\vspace{-.3cm}\\
\hline%
\vspace{-.2cm}\\
GDP growth  & 4.3 & 3.0 & 3.8 & 2.8 & 2.8 & 4.0 & 2.1 
        & --1.6 \\
Inflation 
        & 3.0 & 2.8 & 2.3 & 2.7 & 3.5 & 2.3 & 4.4 & 1.2 \\
Interest rate:  short
        & 4.6 & 4.8 & 5.3 & 5.5 & 5.8 & 6.4 & 6.7 & 2.8 \\
Interest rate:  long 
        & 5.8 & 5.4 & 5.6 & 5.3 & 5.6 & 6.0 & 5.8 & 3.3 \\            
Investment rate    
        & 24.1 & 25.2 & 25.5 & 26.5 & 26.8 & 27.8 & 28.7 &  \\
Saving rate          
        & 20.1 & 20.4 & 20.0 & 21.3 & 21.2 & 21.9 & 24.3 &  \\
Current account 
        & --3.8 & --5.5 & --6.1 & --5.8 & --5.5 
        & --6.4 & --4.2 & --3.5 \\
Govt budget:  total
        & 1.3 & 1.8 & 1.1 & 1.5 & 1.5 & 1.6 & 1.8 & --3.3 \\
Govt budget:  primary 
        & 2.9 & 3.2 & 2.4 & 2.7 & 2.6 & 2.6 & 2.7 & --2.6 \\
Govt debt
        & 20.1 & 18.5 & 17.5 & 17.0 & 16.4 & 15.4 & 13.9  \\
Exchange rate 
        & 1.84 & 1.54 & 1.36 & 1.31 & 1.33 & 1.20 & 1.19 \\        
Real exchange rate 
        & 100 &  113 & 121 & 125 & 125 & 133 & 132 \\
FX reserves (USD) 
        & 21 & 32 & 36 & 42 & 53 & 25 & 31 & \\
FX reserves (months) 
        &  2.9 & 3.7 & 3.3 & 3.4 & 4.0 & 1.6 & 1.7 \\        
Net foreign assets  
        & --68 & --68 & --65 & --71 & --77 & --89 & --103 \\ 
\vspace{-3mm} \\
\hline 
\end{tabular}
}
\end{center}
\end{minipage}
\caption{Economic indicators for Australia.  
Notes:  
(i)~Investment, saving, current account, government budget, 
government debt, and net foreign assets
are expressed as percentages of GDP (ratios multiplied by 100).  
(ii)~The exchange rate is the Aussie dollar (AUD) price 
of one US dollar;
high numbers indicate that foreign currency is expensive.  
The real exchange rate is a weighted average across trading partners.  
The convention is the inverse of the exchange rate:  
high numbers indicate that local goods are expensive relative to foreign 
goods.  
(iii)~Foreign exchange reserves are expressed, first, 
in billions of USD, second, 
as a ratio to monthly imports. 
Thus the number 2.9 means that reserves are 2.9 times one
month's imports.
(iv)~2009 numbers are estimates.}
\label{tab:auz} 
\end{table}

\item {\it Deficits down under (40 points).\/} 
As a European investor in short-term Australian securities, 
you have made a fair amount of money over the last decade
betting that Australia's high interest rates would not 
be offset by declines in the value of its currency.  
You wonder, however, whether it's time to take your money 
and run.  

Having some experience with such situations, 
you check the Economist Intelligence Unit's Country Data, 
summarized in Table \ref{tab:auz}, 
and Country Risk Service, 
which reports:  
%
\begin{itemize}
\item The exchange rate is flexible, and could move either way against
the euro.  
\item Australia's large net foreign liability position reflects 
a combination of direct investment in Australian businesses, 
notably mining, 
and the carry trade, 
in which investors purchase AUD-denominated assets 
in order to benefit from relatively high local interest rates.  
\item The banking system faces higher financing costs due to global 
credit conditions, but is stable.  Government guarantee of deposits
has reduced the risk of failures. 
\end{itemize}

With this information in hand, you go through your checklist:  
%
\begin{enumerate}
\item Fiscal policy 1.  
Why are the total and primary government balances different?  
What would you estimate for the debt-to-GDP ratio 
at the end of 2009?  
(10~points) 

\item Fiscal policy 2.  
Are Australia's government debt and deficit large enough to 
concern you?
Why or why not? 
(10~points) 

\item What other factors would you consider in 
assessing the risks to Australia's economy?  
Which ones do you regard as troubling?  
Why?  
(20~points)

\end{enumerate}

%\begin{comment}
Answer.
\begin{enumerate}
\item The difference is interest on government debt.  
The government budget looks like (you'll have to look 
up the notation)
\begin{eqnarray*}
    G_t + V_t - T_t &=& B_t - (1+i)B_{t-1} .
\end{eqnarray*}
The expression on the left is the primary deficit.
On the right, interest in the debt is part of the second
term:  $ i B_{t-1}$.  
The numbers tell us this is expected to be 0.7\% of GDP
(the difference between the primary and total deficits).  
Since debt at the end of the last year was 13.9\% of GDP,
the interest rate was about 5\%.  

The debt-to-GDP ratio evolves like this:  
\begin{eqnarray*}
    B_t/Y_t  &=&  [(1+i)/(1+g)] B_{t-1}/Y_{t-1} + D_t/Y_t .
\end{eqnarray*}
The numbers we're given suggest:  $i = 3.3$
and $ g = -1.6 + 1.2 = -0.4 $.
There's some question about the exact numbers;
it's critical, though, that you understand that $g$ 
is {\it nominal\/} GDP growth.  If $B_{t-1}/Y_{t-1} = 13.9\%$, 
that gives us $B_t/Y_t = 17\% $ at the end of this year.  

Grading:  5 points for noting the difference between the two
budget numbers (interest), 5 for the calculation of 
the new debt-to-GDP ratio.  

\item You have some flexibility here, but I would expect you to 
go through the checklist:  
fiscal policy (done), 
exchange rate (overvalued) and reserves (not many), 
current account and NFA, 
banking system (IEU says ok), and politics/institutions.  

Three points seem essential:  
the debt to GDP ratio is very low, 
net foreign assets is a huge negative,
and Australia  is a developed country with good institutions.  
The first and third suggest no cause for concern.  
The second is an ``it depends'' issue.  Given that these
are primarily private transactions 
(the public deficit is too small for it to be otherwise), 
I wouldn't be worried about that either.  

Grading:  20 points for a lucid answer something like this.  

\end{enumerate}
%\end{comment}


%\pagebreak \phantom{bla} \pagebreak \phantom{bla} \pagebreak
% ---------------------------------------------------------
\item {\it Miscellany (30 points).}
%
\begin{enumerate} 

\item If the inflation rate rises, 
how would a central bank following a Taylor rule respond?  
Why?  
(10~points) 

\item If Hungary's central bank sells euros to citizens 
who want them,  
what is the impact on Hungary's money supply?  
How might it offset this impact?  
(10~points)

\item If the unemployment rate rises, 
is that bad news for future economic growth?  Why or why not?  
(10~points) 

\end{enumerate}

%\begin{comment}
Answers.
\begin{enumerate}
\item The Taylor rule indicates that a central bank would raise 
the short-term interest rate

Grading:  8 points for noting the interest rate rises, 
1 for noting it rises more than one for one, 
and 1 for noting that the rationale is to keep inflation down 
(higher interest rate corresponds to reduced money supply).  

\item We're talking about a central bank selling euros, collecting
local currency in return.  The latter reduces the amount of currency
in private circulation:  reduces the money supply, in other words.  
To reverse that, it could purchase government securities 
(the usual way to increase the money supply).  
In this context, the transaction would be called sterilization. 

The two transactions might look like this on the 
central bank's balance sheet.
At the start:  
%
\begin{center}
\begin{tabular}{lrclr}
               Assets  &     &&     Liabilities                     \\  
               \hline 
               FX Reserves &  100 &&     Monetary Base &  200   \\    
               Bonds   & 100 && \\
\end{tabular}
\end{center}
%
After selling euros:  
%
\begin{center}
\begin{tabular}{lrclr}
               Assets  &     &&     Liabilities                     \\ 
               \hline 
               FX Reserves &  75 &&     Monetary Base &  175   \\    
               Bonds   & 100 && \\
\end{tabular}
\end{center}
%
After purchasing securities:  
%
\begin{center}
\begin{tabular}{lrclr}
               Assets  &     &&     Liabilities                     \\  
               \hline 
               FX Reserves &   75 &&     Monetary Base &  200   \\    
               Bonds   & 125 && \\
\end{tabular}
\end{center}

Grading:  10 points for similar description of 
how the balance sheet changes.
It's not necessary to show the balance sheets explicitly
as long as the logic is clear.  
    
\item I would say no:  the unemployment rate is a lagging indicator, 
so it's telling us about bad news that's already happened.  
In that sense, the unemployment rate is bad news about the past, 
not the future.  

Grading:  10 points for understanding that the unemployment rate
is a lagging indicator.  
Partial credit of 5 or 6 points for anything sensible.  
\end{enumerate}
%\end{comment}

\end{enumerate}

%\pagebreak \phantom{bla} %\pagebreak \phantom{bla} %\pagebreak

\vfill \centerline{\it \copyright \ \number\year \ 
NYU Stern School of Business}


\end{document}

