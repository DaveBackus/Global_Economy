\documentclass[letterpaper,12pt]{exam}

\usepackage{ge13}
\usepackage{comment}
\usepackage{booktabs}
\usepackage{hyperref}
\urlstyle{rm}   % change fonts for url's (from Chad Jones)
\hypersetup{
    colorlinks=true,        % kills boxes
    allcolors=blue,
    pdfsubject={NYU Stern course GB 2303, Global Economy},
    pdfauthor={Dave Backus @ NYU},
    pdfstartview={FitH},
    pdfpagemode={UseNone},
%    pdfnewwindow=true,      % links in new window
%    linkcolor=blue,         % color of internal links
%    citecolor=blue,         % color of links to bibliography
%    filecolor=blue,         % color of file links
%    urlcolor=blue           % color of external links
% see:  http://www.tug.org/applications/hyperref/manual.html
}

\newcommand{\GDP}{\mbox{\em GDP\/}}
\newcommand{\NDP}{\mbox{\em NDP\/}}
\newcommand{\GNP}{\mbox{\em GNP\/}}
\newcommand{\NX}{\mbox{\em NX\/}}
\newcommand{\NY}{\mbox{\em NY\/}}
\newcommand{\CA}{\mbox{\em CA\/}}
\newcommand{\NFA}{\mbox{\em NFA\/}}
\newcommand{\Def}{\mbox{\em Def\/}}
\newcommand{\CPI}{\mbox{\em CPI\/}}
\newcommand{\phm}{\phantom{--}}

\def\ClassName{The Global Economy}
\def\Category{Backus \& Cooley}
\def\HeadName{Practice Final Examination 1}

%\printanswers

\begin{document}
\parindent = 0.0in
\parskip = \bigskipamount
\thispagestyle{empty}%
\Head

\centerline{\large \bf \HeadName}%
\centerline{Revised:  \today}

\bigskip
You have 120 minutes to complete this exam.  Please answer each
question in the space provided. You may consult one page of notes
and a calculator, but devices capable of wireless transmission are
prohibited.

I understand that the honor code applies: I will not lie, cheat,
or steal to gain an academic advantage, or tolerate those who do.

\begin{flushright}
\rule{4in}{0.5pt} \\ (Name and Signature)
\end{flushright}

%\bigskip
{\it Note:  These questions come from old exams,
so the topics and numbers may be out of date.
But be assured:  good analysis lasts forever.}

\begin{questions}
% ************************************************************************
\question {\it US monetary policy (25 points).}
When the US Federal Open Market Committee (FOMC) met in April 2007,
inflation was close to 3\% and GDP growth was about 2\%.
After the meeting,
their statement said, in part:
%
\begin{quote}
Recent indicators have been mixed. ...
Nevertheless, the economy seems likely to continue
to expand at a moderate pace over coming quarters.
%
Recent readings on core inflation have been somewhat elevated.
Although inflation pressures seem likely to moderate over time,
the high level of resource utilization has the potential
to sustain those pressures.
%
In these circumstances, the Committee's predominant policy concern remains the risk that inflation will fail to moderate as expected. Future policy adjustments will depend on the evolution of the outlook for both inflation and economic growth, as implied by incoming information.
\end{quote}
%
Comment:  this is a time when it's helpful to
see the AS/AD model lurking behind the words.
The following questions should help you work your
way through this process.
%
\begin{parts}
\part How would you interpret the Fed's statement in terms
of its goals of low inflation and full employment?
Do you agree that the evidence, as presented,
suggests that inflation is
the ``predominant policy concern''?
(10~points)

\part Use the aggregate supply and demand framework
to illustrate how the FOMC might think about monetary policy.
Consider each of the following questions in turn:
(i)~How would an increase in the money supply affect
output and prices in the short run?
(ii)~In the long run?
(10~points)

\part Given the Fed's assessment of current economic conditions,
how would you expect it to respond?
Would you expect interest rates to rise or fall?
(10~points)
\end{parts}

\begin{solution}
\begin{parts}
\part
The statement suggests that inflation is above target
(``core inflation elevated'') and output is at the target
(``high level of resource utilization'') or possible beyond it.
With output at or above target, inflation is the appropriate focus.
GDP growth points the other way, so there's a question about this assessment.

\part Draw a diagram with AS, AD, and AS$^*$.
%Note in particular where the short-run equilibrium (where AS and AD cross)
%is relative to AS$^*$.
Assume for the time being that we start at a long-run equilibrium
(where AS crosses AD and AS$^*$ at the same time).
(i)~Short-run impact of an increase in the money supply.
AD shifts right/up, and we move to where AS and AD cross (and
ignore AS$^*$). This raises output and prices (inflation).
(ii)~In the long-run, AS also shifts, sending us to the point where AD and AS$^*$
cross. Why? Because the sticky wages that give AS its slope
eventually adjust. The result:
prices rise, but output stays the same, relative to our starting
point. This illustrates the difference between the short-run and
long-run effects of monetary policy:  In the long run, all we get is
inflation.  In the short run, we get a combination of higher prices
and higher output. Which sends us back to (a):  one of the questions
the Fed must address is how large the short-run increase in output
is.  That, in turn, depends on how steep the AS curve is:  the
steeper the curve, the smaller the increase in output. [For
practice, contrast this analysis with one where we start to the left
of AS$^*$.]

\part The comment about utilization tells you that
the FOMC sees output as at or beyond the long-run supply curve AS$^*$.
To reduce inflation, the Fed would reduce the money supply,
shifting aggregate demand left.
In practice, they would do this by raising the interest rate.
The short-run impact is to lower inflation and output.
If output is above target, that's ok on both fronts.
If it's at target, the Fed will have to decide whether output or inflation
is more important.
In this case, we have the Fed's own statement that inflation is the
``predominant policy concern,''
so this is the policy they have in mind.
We would therefore expect the Fed to raise the Fed funds rate.

\end{parts}
\end{solution}
%\end{questions}\end{document}

% ************************************************************************
%
\begin{center}
{\small
\begin{tabular}{lrrrrrrrr}%
\toprule
%\vspace{-.3cm}\\
    & 2002 & 2003 & 2004 & 2005 & 2006 &  2007 &  2008 & 2009 \\%
%\vspace{-.3cm}\\
\midrule
Real GDP growth  & 4.3 & 3.0 & 3.8 & 2.8 & 2.8 & 4.0 & 2.1
        & --1.6 \\
Inflation
        & 3.0 & 2.8 & 2.3 & 2.7 & 3.5 & 2.3 & 4.4 & 1.2 \\
Interest rate:  short
        & 4.6 & 4.8 & 5.3 & 5.5 & 5.8 & 6.4 & 6.7 & 2.8 \\
Interest rate:  long
        & 5.8 & 5.4 & 5.6 & 5.3 & 5.6 & 6.0 & 5.8 & 3.3 \\
Investment rate
        & 24.1 & 25.2 & 25.5 & 26.5 & 26.8 & 27.8 & 28.7 &  \\
Saving rate
        & 20.1 & 20.4 & 20.0 & 21.3 & 21.2 & 21.9 & 24.3 &  \\
Current account
        & --3.8 & --5.5 & --6.1 & --5.8 & --5.5
        & --6.4 & --4.2 & --3.5 \\
Govt budget:  total
        & 1.3 & 1.8 & 1.1 & 1.5 & 1.5 & 1.6 & 1.8 & --3.3 \\
Govt budget:  primary
        & 2.9 & 3.2 & 2.4 & 2.7 & 2.6 & 2.6 & 2.7 & --2.6 \\
Govt debt
        & 20.1 & 18.5 & 17.5 & 17.0 & 16.4 & 15.4 & 13.9  \\
Exchange rate
        & 1.84 & 1.54 & 1.36 & 1.31 & 1.33 & 1.20 & 1.19 \\
Real exchange rate
        & 100 &  113 & 121 & 125 & 125 & 133 & 132 \\
FX reserves (USD)
        & 21 & 32 & 36 & 42 & 53 & 25 & 31 & \\
FX reserves (months)
        &  2.9 & 3.7 & 3.3 & 3.4 & 4.0 & 1.6 & 1.7 \\
\bottomrule
\end{tabular}
}
\end{center}
{Economic indicators for Australia.
Notes:
(i)~Investment, saving, current account, government budget,
government debt, and net foreign assets
are expressed as percentages of GDP (ratios multiplied by 100).
(ii)~The exchange rate is the Aussie dollar (AUD) price
of one US dollar;
high numbers indicate that foreign currency is expensive.
The real exchange rate is a weighted average across trading partners.
The convention is the inverse of the exchange rate:
high numbers indicate that local goods are expensive relative to foreign
goods.
(iii)~Foreign exchange reserves are expressed, first,
in billions of USD, second,
as a ratio to monthly imports.
Thus the number 2.9 means that reserves are 2.9 times one
month's imports.
(iv)~2009 numbers are estimates.}

\question {\it Deficits down under (40 points).\/}
As a European investor in short-term Australian securities,
you have made a fair amount of money over the last decade
betting that Australia's high interest rates would not
be offset by declines in the value of its currency.
You wonder, however, whether it's time to take your money
and run.

Having some experience with such situations,
you check the Economist Intelligence Unit's Country Data,
summarized above,
and Country Risk Service,
which reports:
%
\begin{itemize}
\item The exchange rate is flexible, and could move either way against
the euro.
\item Australia's large net foreign liability position reflects
a combination of direct investment in Australian businesses,
notably mining,
and the carry trade,
in which investors purchase AUD-denominated assets
in order to benefit from relatively high local interest rates.
\item The banking system is stable.
\item Australian political institutions are widely regarded to be
of high quality and stable.
\end{itemize}

With this information in hand, you go through your checklist:
%
\begin{parts}
\part Fiscal policy 1.
Why are the total and primary government balances different?
(10~points)

\part Fiscal policy 2.
What would you estimate for the debt-to-GDP ratio
at the end of 2009?
(20~points)

\part Considering fiscal policy and other risk factors,
how do you see the risks to the Australian economy
over the next two years or so?
(10~points)
\end{parts}

\begin{solution}
\begin{parts}
\part The difference is interest on government debt.
The government budget looks like (you'll have to look
up the notation)
\begin{eqnarray*}
        D_t &=&  B_t - (1+i)B_{t-1} .
\end{eqnarray*}
The expression on the left is the primary deficit.
On the right, interest in the debt is part of the second
term:  $ i B_{t-1}$.
The numbers tell us this is expected to be 0.7\% of GDP
(the difference between the primary and total deficits).
Since debt at the end of the last year was 13.9\% of GDP,
the interest rate was about 3.0\%.

\part The debt-to-GDP ratio evolves like this:
\begin{eqnarray*}
   \Delta (B_t/Y_t)  &=&  (i - \pi)(B_{t-1}/Y_{t-1})
                - g (B_{t-1}/Y_{t-1}) + D_t/Y_t .
\end{eqnarray*}
The numbers we're given suggest
$r = i-\pi = 1.8\%$.
That gives us
\begin{eqnarray*}
   \Delta (B_t/Y_t)  &=&  0.3 + 0.2 + 2.6
            \;\;=\;\; 3.1,
\end{eqnarray*}
so the debt-to-GDP ratio rises to 17.0\%.

\part You have some flexibility here, but I would expect you to
go through the checklist:
fiscal policy (done),
exchange rate (overvalued but flexible, with limited reserves),
banking system (EIU says ok), and politics/institutions.

Two points seem essential:
the debt to GDP ratio is very low,
and Australia  is a developed country with good institutions.
This suggests  no particular cause for concern.

[In hindsight, some of the same might have been said about the US,
with somewhat higher debt and deficits,
but disaster struck anyway.
It's a good reminder:  these things are hard to predict.
Nevertheless, it's helpful to survey the territory,
see if there are any obvious landmines lying around.]

\end{parts}
\end{solution}
%\end{questions}\end{document}

% ***************************************************************************
\question {\it Miscellany (50 points).}

\begin{parts}
\part {\it Chinese crisis?\/}
An analyst suggested that China may suffer a currency crisis
along the lines of Mexico in 1994-95,
in which the peso fell sharply when the Banco de Mexico
ran out of foreign currency reserves.
Do you find this scenario likely or unlikely?  Why?
 (10~points)

\part {\it Government deficits.\/}
Can a country run a fiscal (government) deficit forever?
Why or why not?
 (10~points)


\part {\it Canadian inflation.\/}
In Canada over the last year, inflation has been 2.3\%
and money growth  has been 11.8\%.
Do you find the difference between the two numbers surprising?
Why or why not?  (10~points)

\part {\it Leading indicators.\/}
Explain what a leading indicator is
and give an example for the US.  (10~points)

\part {\it Monetary policy mechanics.\/}
Use the central bank's balance sheet to describe
how it
maintains the short-term interest rate at a specific level.
(10~points)
\end{parts}

\begin{solution}
\begin{parts}
\part
(i)~China has enormous foreign exchange
reserves:  they won't run out any time soon.
(ii)~The renminbi seems to be undervalued:
people want to buy it,
not sell it, which results in the central bank accumulating reserves,
not losing them.

\part
The present value of future primary surpluses has to equal
the current debt.  Thus past deficits must be balanced by future
surpluses --- you can't run a primary deficit forever. The key word
is primary:  you can run a primary surplus and an overall deficit at
the same time, as we see in (for example) Turkey.

\part
Under the quantity theory, inflation equals money growth
minus real GDP growth. Unless real GDP growth is 9\%, something's
wrong. What's wrong is that this relation is well known not to work
in the short run.  Over periods of several years, however, it
typically works pretty well.

\part
A leading indicator (of the economy)
is an observable economic variable whose ups and
downs precede those in (say) real GDP.
You can see this in the cross-correlation function, for example.
Common examples:  housing starts, stock market indexes,
interest rate spread (long minus short).

\part
Central banks manage short-term interest rates through
``open market operations'': buying and selling government
securities. Selling securities, for example, reduces the amount of
currency in private circulation, which generally increases
short-term interest rates.  [Insert T-accounts here.] The story we
tell is that this reduces the liquidity of capital markets by
reducing the quantity of currency in circulation.

\end{parts}
\end{solution}

\end{questions}
%\end{document}
%\pagebreak \phantom{bla} \pagebreak \phantom{bla}


\vfill \centerline{\it \copyright \ \number\year \
NYU Stern School of Business}



%**********************************************************************
%**********************************************************************
\newpage
\def\HeadName{Practice Final Examination 2}
\parindent = 0.0in
\parskip = \bigskipamount
\setcounter{page}{1} \thispagestyle{empty}
\Head

\centerline{\large \bf \HeadName}%
\centerline{Revised:  \today}

\bigskip
You have 120 minutes to complete this exam.  Please answer each
question in the space provided. You may consult one page of notes
and a calculator, but devices capable of wireless transmission are
prohibited.

I understand that the honor code applies: I will not lie, cheat,
or steal to gain an academic advantage, or tolerate those who do.

\begin{flushright}
\rule{4in}{0.5pt} \\ (Name and Signature)
\end{flushright}

\begin{questions}
% ************************************************************************
\question {\it Chinese foreign exchange intervention (30 points).}
Joseph Yam, Chief Executive of the Hong Kong Monetary Authority, wrote about the foreign exchange activities of the People's Bank of China (PBOC) in January 2007:
%
\begin{quote}
The accumulation of foreign exchange reserves
involves the PBOC buying foreign assets through creating renminbi.
As a result, the monetary base is increased, creating a need to ``sterilise.''
This is done by issuing paper [short-term notes] to the banks.
The PBOC obviously has to pay interest on the money borrowed.
Currently the yield of, for example, three-month paper issued by the PBOC is about 2.5\%.
This is lower than the yield on foreign assets held as reserves ---
the yield on US treasuries is about 4 to 5\% --- so theoretically
reserve accumulation can be profitable.
The problem, however, is the continuing appreciation of the renminbi,
which gradually reduces the value of those foreign assets in renminbi terms.
\end{quote}

\begin{parts}
\part Use the central bank's balance sheet to show how purchases of
foreign currency increase the monetary base (think:  supply of currency).
(10~points)

\part Show how sterilization can be used to reverse the impact on
the supply of currency.
(10~points)

\part You may note that interest rates have now flipped, with Chinese interest rates above US interest rates.
    What does this imply for the returns on the PBOC's balance sheet?
    How might it avoid this outcome?
(10~points)
\end{parts}

\begin{solution}
\begin{parts}
\part A typical central bank balance sheet looks something like this:
%
\begin{center}
\begin{tabular}{lrclr}
               Assets  &     &&     Liabilities                     \\
               \hline
               FX Reserves &  100 &&     Monetary Base &  200   \\
               Bonds   & 100 && \\
\end{tabular}
\end{center}
%
Monetary base is (roughly) another term for currency.
Suppose, now, that the PBOC has to purchase another 100 worth of
foreign currency (or foreign-currency denominated bonds)
and issues currency in return.
Then the balance sheet becomes
%
\begin{center}
\begin{tabular}{lrclr}
               Assets  &     &&     Liabilities                     \\
               \hline
               FX Reserves &  200 &&     Monetary Base &  300   \\
               Bonds   & 100 && \\
\end{tabular}
\end{center}
%
This is the impact of the accumulation of reserves on the monetary base described by Yam.


\part To undo the increase in the monetary base, the PBOC will issue 100 worth of bonds and accept money in return:
%
\begin{center}
\begin{tabular}{lrclr}
               Assets  &     &&     Liabilities                     \\
               \hline
               FX Reserves &  200 &&     Monetary Base &  200   \\
               Bonds   & 0 && \\
\end{tabular}
\end{center}
%
This operation is referred to as sterilization.
For the PBOC, this process has continued to the extent that they
have had to issue bonds as liabilities, so that their balance sheet now looks something like this:
%
\begin{center}
\begin{tabular}{lrclr}
               Assets  &     &&     Liabilities                     \\
               \hline
               FX Reserves &  800 &&     Monetary Base &  200   \\
               Bonds   &  0 &&    Bonds  &  600 \\
\end{tabular}
\end{center}
%
The bond liabilities (so-called ``sterilization bonds'')
are issued primarily to Chinese commercial banks,
as outlined above.
Note, as Yam does, that assets and liabilities are equal.

\part The PBOC is now likely losing money on its portfolio
for two reasons:
(i)~because the rate it's paying on liabilities is greater than
the rate received on assets and
(ii)~continued increases in the value of the renminbi raise
the value of its liabilities relative to its assets.
In a sense, they have the wrong side of the carry trade.

How could it avoid this outcome?
Let the currency float and stop buying foreign currency.
This is entirely the byproduct of managing the exchange rate:
that leads them to buy foreign currency, with the results
we've just seen.
\end{parts}
\end{solution}
%\end{questions}\end{document}

%\pagebreak \phantom{bla} \pagebreak \phantom{bla} \pagebreak
% ---------------------------------------------------------
\question {\it Globalization and inflation (20 points).\/}
Fed Chairman Bernanke said recently (March 2007):
\begin{quote}
As national markets become increasingly integrated and open, sellers of goods, services, and labor may face more competition and have less market power than in the past.
These linkages suggest that, at least in the short run, globalization and trade
may affect the course of domestic inflation.
\end{quote}
%
\begin{parts}
\part Use aggregate supply and demand to
describe how expansionary monetary policy
affects output and inflation in the short run.
(10~points)

\part Back to Bernanke:
What do you see as the impact of globalization
(think: imports from China)
on domestic inflation?
How would you represent this
in the aggregate supply and demand diagram?
How does globalization change the impact of expansionary monetary
policy in this model?
Do you find the model persuasive in this respect?
(10~points)
\end{parts}


\begin{solution}
\begin{parts}
\part This would increase both output and prices (inflation,
loosely speaking).  Your answer should show a diagram with
AS, AD, and AS$^*$.
AD shifts out, with the stated result.

\part The typical argument is that AS has become flatter,
since attempts to raise prices will meet with strong foreign competition.
As a result, expansionary monetary policy has a
smaller impact on inflation and larger impact on output.
\end{parts}
\end{solution}
%\end{questions}\end{document}

%\pagebreak \phantom{bla} \pagebreak

\question {\it Miscellany (50 points).}

\begin{parts}
\part {\it Exchange rates.\/}
{\it The Economist\/} reports that a Big Mac costs \$2.90 in the US, \$3.28 in the eurozone,
and \$2.33 in Japan.  (These prices are averages for the various regional markets, expressed in US
dollars using current spot exchange rates.)  What does this suggest about the likely change in
value of the euro and yen v. the dollar over the coming 6 months?  6 years? (10~points)

\part {\it Inflation.\/}
Milton Friedman once said:  inflation is always and everywhere a monetary phenomenon.
Do you agree or disagree?  Why or why not?  (10~points)

\part {\it Employment report.\/}
At 8:30 am on April 6, 2006,
the US Bureau of Labor Statistics released
its closely-watched employment report, {\it The Employment Situation\/}.
Firms reported an increase of 180,000 jobs in March,
well above the consensus of 135,000.
Treasury yields immediately rose 5-10 basis points
for maturities from 2 to 30 years.
Why?
(10~points)

\part {\it ECB policy.\/}
The European Central Bank has kept short-term interest rates
in the Euro Zone well above those in the US.
Why?  (10~points)

\part {\it Cross-correlation function.\/}
Describe the cross-correlation function and show how it can
be used to identify promising leading indicators.
(10~points)
\end{parts}

\begin{solution}
\begin{parts}

\part   PPP suggests that exchange rates will adjust to make prices the same across countries. In
this case, that means the dollar will rise against the euro, fall against the yen.  Is this right?
Over short periods of time, exchange rates are close to unpredictable by any means, PPP included.
Over longer periods of time (5-20 years) PPP is a reasonable indicator.

\part It has lots of truth in it, but I'd disagree for two reasons.
First, it's an incomplete statement for high inflations:
it's true, but high money growth itself typically stems from
a government deficit.
Second, over short periods of time, the quantity theory doesn't work
that well.
It's entirely possible, as our AS/AD analysis implies, that
money can have only a modest short-run impact on inflation,
and that other demand and supply factors play a role, too.

\part  I'd start with the Taylor rule:  indicators of high output
lead to high interest rates.
The deeper question is why this shows up in long yields.
Certainly it will take some time to affect the Fed's choice of
target interest rate, but the impact on the very long end is a
typical, if somewhat mysterious, result.

\part One reason is that the ECB, by design,
places greater weight on inflation:  its primary goal is price stability.
Another is that they have not had the kind of financial turmoil
that has afflicted US markets and driven the Fed to
largely abandon its own devotion to stable prices.

\part The cross-correlation function is a plot of the correlation
of the correlation of two variables at different leads and lags
against the lead or lag.
Formally, the ccf for two variables $x$ and $y$ is a plot
of
\[
    \mbox{ccf}(k) \;=\;  \mbox{\it corr\/} (x_t,y_{t-k}) .
\]
against $k$.
\end{parts}
\end{solution}

\end{questions}

%\pagebreak \phantom{bla} \pagebreak \phantom{bla}


\vfill \centerline{\it \copyright \ \number\year \
NYU Stern School of Business}

%**********************************************************************
%**********************************************************************
\newpage
\def\HeadName{Practice Final Examination 3}
\parindent = 0.0in
\parskip = \bigskipamount
\thispagestyle{empty}%
\Head

\centerline{\large \bf \HeadName}%
%\centerline{March 9, 2005}
\centerline{Revised:  \today}

\bigskip
You have 120 minutes to complete this exam.  Please answer each
question in the space provided and show all of your work.
You may consult one page of notes and a calculator,
but devices capable of wireless transmission are prohibited.

I understand that the honor code applies: I will not lie, cheat,
or steal to gain an academic advantage, or tolerate those who do.

\begin{flushright}
\rule{4in}{0.5pt} \\ (Name and Signature)
\end{flushright}

%\pagebreak
\begin{questions}
% ======================================================================
\question {\it Risk and opportunity in Ghana.\/}
You have been asked to prepare a report on crisis risk in
the West African country of Ghana.
You recall from your Global Economy class that
Ghana is a former British colony that has been growing rapidly
in recent years after a period of unusually stable politics.
The Economist Intelligence Unit refers to it as a ``robust democracy.''
The World Economic Forum ranked Ghana 114th (of 133)
in their Global Competitiveness Report.
They continue:
``The country continues to display strong public institutions and
governance indicators,
particularly in regional comparison.''

The EIU's Country Risk Report adds:
\begin{itemize}
\item The December 2012 elections are expected to be close.
The president, John Atta Mills, came to power promising accountability
and transparency, but  has struggled to maintain party unity
while evidence emerges of financial impropriety of some government ministers.
\item The victor faces a challenging policy environment, particularly
the fiscal situation.
\item Expectations among the population are high as production
starts at the offshore Jubilee oil field.
\item The government's decision to allow use of 70\% of future
oil revenue as collateral for borrowing is a cause for concern
if the revenue is not managed properly.
\item The Bank of Ghana (the central bank) faces the twin goals of
containing inflation and fostering growth.
\item The currency --- the cedi --- floats with occasional heavy intervention.
\end{itemize}
%
Your mission is to assess the risks to Ghana  using
the information above, the data in Table \ref{tab:ghana},
and your own good judgement and analytical skills.

\begin{parts}
\part  You decide to start with a fiscal assessment.
What trend do you see in government revenues and expenses?
(5~points)

\part You notice that neither the primary deficit nor
interest expenses are reported separately.
How would you estimate them from the numbers in the table?
What are their values for 2011?
{\it Warning: this was harder than intended. If you get stuck, 
just make up primary deficit numbers somehow.\/}
(10~points)

\part Using what you know about government debt dynamics,
compute the ratio of government debt to GDP for 2011.
What factors contribute the most to the change from 2010?
(10~points)

\part Overall, how would you assess the risks to Ghana's economy over
the next couple of years?
(10~points)
\end{parts}

\begin{table}
\centering
%\tabcolsep = 0.2in
\begin{tabular}{lrrrrrr}
\toprule
        & 2007 & 2008 & 2009 & 2010 & 2011 & 2012 \\
\midrule
GDP growth (\%) & 6.5 & 8.4 & 4.0 & 7.7 & 13.6 & 7.4 \\
Inflation (\%)  & 12.7& 18.1 & 16.0 & 8.6 & 8.6 & 8.5 \\
Interest rate (\%) & 14.5 & 20.8 & 28.8 & 22.7 & 20.5 & 20.6  \\
Govt revenue (\% of GDP)  & 17.5 & 16.0 & 16.5 & 19.1 & 23.4 & 22.2 \\
Govt spending (\% of GDP) & 23.1 & 24.5 & 22.3 & 25.5 & 27.6 & 27.7 \\
Govt budget balance (\% of GDP) & --5.6 & --8.5 & --5.8 & --6.5& --4.2 & --5.5\\
Govt debt (\% of GDP) & 30.4 & 30.6 & 33.3 & 33.9 & {\bf } & {\bf } \\ %& 36.8 & 41.6\\
Real exchange rate (index) & 85.2 & 81.7 & 76.3 & 81.8 & 78.1 & 74.5\\
FX reserves (USD billions) & 2.6 & 1.8 & 2.9 & 4.3 & 4.4 & 4.8 \\
\bottomrule
\end{tabular}
\caption{Macroeconomic data for Ghana.
Data from EIU CountryData.
The government budget balance is a surplus if positive, deficit if negative.
The real exchange rate is the price of goods in Ghana relative to the rest
of the world;
the larger the number, the more expensive goods are in Ghana.
The numbers for 2011 and 2012 are estimates.
}
\label{tab:ghana}
\end{table}


\begin{solution}
\begin{parts}
\part Trends include:
(i)~revenues and spending both rising,
(ii)~spending still ahead of revenue (there's a deficit),
and (as a direct result)
(iii)~ratio of debt to GDP rising a little
(more on that to come).

Grading:  5 points for noting (i) and (ii).

\part Remember that interest payments in year $t$ are $ i_t B_{t-1}/Y_t$,
in which the timing of the components doesn't match.
We get what we want from:
\begin{eqnarray*}
    i_t B_{t-1}/ Y_{t} &=& i_t (B_{t-1}/ Y_{t-1}) (Y_{t-1}/ Y_{t})
            \;\;=\;\; i_t (B_{t-1}/Y_{t-1}) /(1+g_t + \pi_t) .
\end{eqnarray*}
That gives us interest payments in 2011 of 5.7\% of GDP and
a primary deficit of $-1.5$\% (that is, a surplus).

We can't compute the numbers for 2012 until we have the debt to GDP ratio
for 2011, so that's no longer part of the question.

Grading:  10 points for correct calculations.

\part The key relation is this one:
\begin{eqnarray*}
    \Delta ({B_{t}}/{Y_{t}})
            &=&
                (i-\pi) ({B_{t-1}}/{Y_{t-1}})
                - g ({B_{t-1}}/{Y_{t-1}})
             +    ({D_{t}}/{Y_{t}})  .
\end{eqnarray*}
We refer to the components on the right as A, B, and C.
Calculations in the spreadsheet give us

\begin{center}
\begin{tabular}{lrrr}
\toprule
        &  2010 & 2011 & 2012  \\
\midrule
Interest payments  &  &  4.0 & 3.9 \\
Component A (interest)  &  &  4.0 & 3.9 \\
Component B (growth)            &   & --4.6 &--2.4 \\
Component C (primary deficit)   &   & --1.5 & 1.4 \\
Total change in $B/Y$       &       &--2.1 & 2.4 \\
Public debt (\% of GDP)     &  33.9 & 31.8 & 33.2 \\
\bottomrule
\end{tabular}
\end{center}

Over this period, the ratio of debt to GDP fell by 0.7\%.
The components contributed:
interest +7.9, growth --7.0, and the primary deficit --1.7.

One thing you might note is that impact of the enormous GDP growth
rate in 2011.

Grading:  10 points for correct calculations,
including the components.

\part This is a call to look at the checklist:
\begin{itemize}
\item Government debt and deficits:  we have deficits, but there's not
much sign yet of a growth debt to GDP ratio.
One future concern might be the possibility of borrowing now against future oil
revenue.  Will any debts incurred be spent wisely?
Will the oil revenue show up?
\item Banking system.  No information on that provided.
\item Exchange rate and reserves.  Reserves are modest,
but with the exchange rate floating there shouldn't be much
concern about that.
\item Politics.  Always an issue,
especially with a contentious election coming
and the promise of money from oil revenue.
It's an odd fact but a true one that revenue
from natural resources is more likely to cause problems than solve them.
\end{itemize}

Grading:  10 points for this list of issues and a sensible
discussion of each one.



\end{parts}
\end{solution}


%\pagebreak \phantom{xx} \pagebreak \phantom{xx} \pagebreak
% ======================================================================
\question {\it Aggregate implications of employer-provided health insurance.\/}
By an accident of history, health insurance in the US is generally
provided by employers.
Suppose a sharp rise in healthcare costs leads firms to hire fewer workers.
\begin{parts}
\part How would you represent this in an aggregate supply and demand diagram?
Which curve shifts?  In which direction?
(10~points)
\part What is the new short-run equilibrium?  Long-run equilibrium?
What happens to inflation and output?
(10~points)
\part How should the central bank respond?
Be specific about its goals and how it would accomplish them.
(10~points)
\end{parts}


\begin{solution}
\begin{parts}
\part Since we're talking about firms and production,
this must involve the supply side of the model.
We shift AS and AS$^*$ to the left, both by the same amount.
See the figure below.

\begin{center}
\setlength{\unitlength}{0.075em}
\begin{picture}(250,200)(0,0)
%\footnotesize
\thicklines

% horizontal axis
\put(-30,0){\vector(1,0){300}}
\put(255,-16){$Y$}
\put(142,-16){$Y^*$}
\put(102,-16){$Y^{*\prime}$}

% vertical axis
\put(0,-20){\vector(0,1){200}}
\put(-15,155){$P$}

% demand
\put(25,165){\line(4,-3){200}}\put(230,10){AD}
%\put(65,165){\line(4,-3){200}}\put(270,10){AD$'$}

% supply
\put(65,13){\line(4,3){200}} \put(270,160){AS}
\put(25,13){\line(4,3){200}} \put(230,160){AS$'$}
\put(146.4,0){\line(0,1){170}} \put(138,175){AS$^*$}
\put(106.4,0){\line(0,1){170}} \put(98,175){AS$^{*\prime}$}

% equilibrium labels
\put(150,55){\footnotesize A}
\put(122,75){\footnotesize B}
\put(95,94){\footnotesize C}
\put(95,76){\footnotesize D}
%\put(95,54){\footnotesize D}
% dotted lines
%\qbezier[31]{(133,0)(133,46)(133,92)}
%\qbezier[45]{(0,92)(67,92)(133,92)}
%\qbezier[45]{(0,72)(67,72)(133,72)}

\end{picture}
\end{center}
\bigskip\bigskip

Grading:  10 points for noting that it's a supply shift
and reproducing the figure above.

\part We started at A.
After the shift, we move to a new short-run equilibrium at B,
where the new AS crosses AD.
Evidently output falls and prices rise.

Eventually we move to a new long-run equilibrium at C,
where AD crosses the new AS$^*$.
At this point, output has fallen more and prices have risen more.

Grading:  7 points for telling us which curves determine
 short-run and long-run equilibria and 3 points for reporting
 the implications for output and prices.

\part The central bank has two goals:  stable prices and output
at its long-run equilibrium.
Here we've moved from A to C.
We're ok at C on the second goal:  output fell,
but that's the long-run equilibrium so there's nothing monetary policy
can do about that.
(We could consider other policies, but they're not the job of the central bank.)

Where C is bad is with respect to price stability:  prices are higher.
So the central bank could shift AD to the left, giving us the same
long-run output but lower prices.
The central bank would accomplish this by reducing the money supply,
which it might do by targeting a higher interest rate.

Grading:  5 points for listing the goals and 5 for showing
what they imply in this situation.

\end{parts}
\end{solution}


%\pagebreak \phantom{xx} \pagebreak %\phantom{xx} \pagebreak
% ======================================================================
\question {\it True/false/uncertain.\/}
Please explain why each statement is true, false, or uncertain.
The explanation is essential.
%
\begin{parts}
\part The New York Times reports that Apple paid a worldwide corporate
tax rate of about 10\%, while Walmart paid about 24\%.
This difference in tax rates is good for the global economy,
because Apple is part of the high-growth technology sector.
(10~points)

\part The unemployment rate is a leading countercyclical indicator
of economic growth.
(10~points)

\part If a central bank is purchasing foreign currency,
sterilization would consist of selling bonds.
(10~points)

\part Purchases of household furniture are more cyclical
than purchases of toothpaste.
(10~points)

\part If year-on-year GDP growth rises 1 percent, you
would expect the fed funds rate to go up by roughly 1.5 percent.
(10~points)

\part Since 1983, Hong Kong has run a hard peg, with
the Hong Kong Dollar trading between 7.75 and 7.85 per US dollar.
As a result, Hong Kong inherits US monetary policy.
(10~points)
\end{parts}

\begin{solution}
\begin{parts}
\part False.  A sound principle of tax policy is to apply the same rate
to (in this case) both firms:  this is what maximizes the surplus,
minimizes the cost of distortions.
And in this case, it's arguable that Walmart has done more for aggregate
productivity than Apple.

Grading:  7 points for saying that the two firms should be taxed at the same rate,
3 for noting that this comes from some kind of welfare analysis.

\part True.  Furniture is a durable good, hence likely to be more cyclical
than a nondurable good like toothpaste.

Grading:  10 points for noting the durability issue.

\part Half true.  The unemployment rate is countercyclical,
but it's a lagging indicator.

Grading:  5 points for each part.

\part True.  Here's how that might look.
The central bank starts out with the balance sheet

\begin{center}
\begin{tabular}{lr|lr}
Assets  &&  Liabilities \\
\midrule
Bonds &  10    & Money & 20 \\
FX reserves & 10
\end{tabular}
\end{center}

If it purchases foreign currency (say 5 worth) with domestic currency,
that changes to

\begin{center}
\begin{tabular}{lr|lr}
Assets  &&  Liabilities \\
\midrule
Bonds &  10    & Money & 25 \\
FX reserves & 15
\end{tabular}
\end{center}

So the money supply has gone up.
To undo that, they would sell 5 worth of bonds, taking back money in return:

\begin{center}
\begin{tabular}{lr|lr}
Assets  &&  Liabilities \\
\midrule
Bonds &  5    & Money & 20 \\
FX reserves & 15
\end{tabular}
\end{center}

Grading:  5 points for a verbal argument,
5 more if they include balance sheets like those above.

\part False.
It's an allusion to the Taylor rule,
which says (in the form we used it in class) that an increase in GDP growth
of 1\% would lead to an increase in the fed funds rate of 0.5\%.

Grading:  5 points if they note this refers to the Taylor rule,
5 more for noting the correct answer.

\part True.
It's a reference to the trilemma, where you get to choose no more than two of:
(i) fixed exchange rate, (ii)~free flows of capital, and (iii)~independent monetary policy. Hong Kong has chosen (i) and (ii), given up on (iii).

Grading:  7 points for noting the connection to the trilemma,
3 more for raising the issue of capital mobility,
which wasn't stated in the question.
It's ok if they don't know whether HK has free capital mobility,
but they need to state it as an issue.

\end{parts}
\end{solution}

\end{questions}

%\pagebreak \phantom{xx} %\pagebreak \phantom{xx}

\vfill \centerline{\it \copyright \ \number\year \ NYU Stern
School of Business}

\end{document}



\end{document}

