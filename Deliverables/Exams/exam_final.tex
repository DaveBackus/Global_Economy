\documentclass[letterpaper,12pt]{article}

%\usepackage[hypertex]{hyperref}
\usepackage{comment}
\RequirePackage{GE05}
% this inputs graphicx, too

\newcommand{\GDP}{\mbox{\em GDP\/}}
\newcommand{\NDP}{\mbox{\em NDP\/}}
\newcommand{\GNP}{\mbox{\em GNP\/}}
\newcommand{\NX}{\mbox{\em NX\/}}
\newcommand{\NY}{\mbox{\em NY\/}}
\newcommand{\CA}{\mbox{\em CA\/}}
\newcommand{\NFA}{\mbox{\em NFA\/}}
\newcommand{\Def}{\mbox{\em Def\/}}
\newcommand{\CPI}{\mbox{\em CPI\/}}
\newcommand{\phm}{\phantom{--}}

\def\ClassName{The Global Economy}
\def\Category{Professor David Backus}
\def\HeadName{Final Exam}

\begin{document}
\parindent = 0.0in
\parskip = \bigskipamount
\thispagestyle{empty}%
\Head

\centerline{\large \bf \HeadName}%
\centerline{Revised:  \today}

\bigskip
You have 100 minutes to complete this exam.  Please answer each question in the space provided.
You may consult one page of notes and a calculator, but devices capable of wireless transmission
are prohibited.

I understand that the honor code applies: I will not lie, cheat, or
steal to gain an academic advantage, or tolerate those who do.

\begin{flushright}
\rule{4in}{0.5pt} \\ (Name, block, and signature)
\end{flushright}


%\bigskip
\begin{enumerate}

\item {\it Real and nominal interest rates (25 points).} 
Forward rates on inflation-indexed bonds can be used to infer future real interest rates, 
just as forward rates on traditional nominal bonds 
can be used to infer future nominal interest rates.  
Since 1997, the US government has issued bonds with coupons and principle indexed to the consumer price index, 
and other issuers have become increasingly common.  

Consider these rates reported on March 31, 2006:   
\begin{center}
\begin{tabular}{ccccc}
\hline
    & \multicolumn{2}{c}{Nominal} & \multicolumn{2}{c}{Real}  \\
    Maturity  &  Forward &  Mean &  Forward &  Mean  \\
\hline
     1      &   4.82     &    3.14    &  1.95  &  1.10    \\
     2      &   4.77     &    3.70    &  2.17  &  1.72  \\
     3      &   4.74     &    4.23    &  2.35  &  2.29  \\
\hline      
\end{tabular}
\end{center}
Source:  Huston McCulloch's web site.  
Maturity is measured in years, rates are annual percentages.  
% http://www.econ.ohio-state.edu/jhm/ts/ts.html

\begin{enumerate} 
\item Compute and describe the paths of future one-year interest rates
implied by real and nominal forward rates.  (15~points) 
\item What pattern of future inflation and GDP growth would be consistent 
with such paths?  (10~points)
\end{enumerate}


Answer.  Expected future one-year rates look like this: 
\begin{center}
\begin{tabular}{ccccc}
\hline
    & \multicolumn{2}{c}{Nominal} & \multicolumn{2}{c}{Real}  \\
  Maturity  &  Risk Prem &  Future Short &  Risk Prem 
    &  Future Short  \\
\hline
     1   &  0.00  & 4.82  & 0.00  & 1.95  \\
     2   &  0.56  & 4.21  & 0.62  & 1.55  \\
     3   &  1.09  & 3.65  & 1.19  & 1.16  \\
\hline 
\end{tabular}
\end{center}

\begin{enumerate} 
\item The usual expectations hypothesis logic suggests that 
both real and nominal rates will fall over the next three years.   
\item We have the usual issue with a falline real yield curve: 
is this a prediction of slower growth, or is something else going on?  
On the nominal yield curve:  this is expected to fall more than 
the real yield curve (117 bps v 79 bps), so the market 
seems to expect lower inflation (the difference between the two). 
One possible reason:  increases in oil prices have led to 
higher inflation in the recent past.  
If we expect oil prices to stabilize or fall, 
over inflation would likely fall, too.  
\end{enumerate}


%\pagebreak \phantom{bla} \pagebreak \phantom{bla} \pagebreak 
\item {\it Prospects for Turkey (25 points).} 
During a job interview with a European portfolio manager, 
you are given  the following macroeconomic indicators for Turkey:  
%\tabcolsep=0.25in
\begin{center}
\begin{tabular}{lcc}
  {\it Indicator}  &  {\it Value}   & {\it Date}     \\
  GDP growth (real) &  9.5\%    & 2005 Q4     \\
  Inflation  &  8.2\%     & March 2006    \\
  Short-term interest rate &  14\%  &  April 2006 \\
  Fiscal balance:  total  (ratio to GDP) &  --3\%  & 2006 est\\
  Fiscal balance:  primary  (ratio to GDP) &  +6\%  & 2006 est \\
  Government debt (ratio to GDP) &  64\%  &  2006 est \\
  Current account balance (ratio to GDP)&  --5\%  &  2006 est \\
  Net foreign income (ratio to GDP)&  --1\%  &  2006 est \\
  Foreign debt (ratio to GDP)      &    43\%   &  2006 est    \\
  Net foreign assets (ratio to GDP) &   --30\%   &   2006 est    \\
  Foreign reserves (ratio to GDP) &  29\%  & 2006 est \\
\end{tabular}
\end{center}
Source:  Economist Intelligence Unit.  

The recruiter asks you:  
\begin{enumerate}
\item Is Turkey's fiscal deficit a cause of concern for
an investor in Turkish government debt?   (10~points)
\item Is Turkey's current account deficit a cause for concern 
for an investor in high-grade private Turkish debt?    (10~points)
\item What is your overall assessment of the 
macroeconomic risks to Turkish debt?  
(5~points) 
\end{enumerate}
You understand that you will be assessed on both the rigor of your logic
and your ability to convey it clearly.  


Answer.  You have some room for creativity here, but one possibility is:  
\begin{enumerate}
\item Turkish government debt is 64\% of GDP, which is enough to think 
about but not enormous.  
But high interest rate means that interest on the debt eats up 9\% of GDP.  
We can get a clearer picture if we look at debt dynamics:  
\[
    \frac{B_{t+1}}{Y_{t+1}} \;=\; \left( \frac{1+i}{1+g} \right) 
     \frac{B_{t}}{Y_{t}} 
     + (1+g)^{-1} \frac{D_{t}}{Y_{t}} ,
\]
With $i$ about 14\% and $g$ about 17\%, we can see that the debt to GDP ratio
will fall slightly over the next year on its own --- largely the effect of 
high GDP growth. 
The primary surplus of 6\% will further reduce the debt to GDP ratio.
In short, the country is in good shape right now.  

\item Turkish NFA is 43\%, which again is enough to think about 
but not enormous.  
Ditto the current account deficit.  
Net foreign income is --1\%, which seems modest.  
A more formal analysis of NFA dynamics:  
\[
    \frac{\mbox{\em NFA}_{t+1}}{Y_{t+1}} \;=\; \left( \frac{1+i}{1+g} \right) 
     \frac{\mbox{\em NFA}_{t}}{Y_{t}} + (1+g)^{-1} \frac{\mbox{\em NX}_{t}}{Y_{t}} .
\]
Again, the term $[(1+i)/(1+g)]$ is a little less than one, so that's reassuring, 
but the current account will result in a (roughly) 2\% increase in the ratio 
of NFA to GDP.  
The issue, really, is whether the economy will continue to grow at this rate. 
If so, it should be able to pay off the debt.  
Questions to ask:  What is the form of foreign claims on Turkish companies?  
Are they debt or equity?  How stable is the political situation?  

\item  Despite modest government and current account deficits, I'd say there were no obvious signs of trouble.  

\end{enumerate}


%\pagebreak \phantom{bla} \pagebreak \phantom{bla} \pagebreak
\item {\it Miscellany (50 points).}
%
\begin{enumerate} 
\item Explain what a leading indicator is and give an example.  
 (10~points)
 
Answer.  Something whose ups and downs precede those of GDP.  
Examples:  term spread (slope of yield curve), hours worked 
in manufacturing, S\^P 500.  

\item Grupo Modelo produces Corona, the leading imported beer in the US.  
As part of a strategy exercise, they are working through the implications of 
large fluctuations in the peso.  
If the peso appreciated by 20\% relative to the dollar, 
by roughly how much should they  change the US price of Corona?  Why?  (10~points)

Answer.  You'll probably raise prices by less than 20\% (10\%?).
Why?  (i)~Only about half of your cost of selling in the US consists
of Mexican production costs; the rest are dollar costs of marketing
and distribution.  
(ii)~You may want to maintain market share if customers are ``sticky.''
See the discussion of Group Project \#9.  

\item An analyst suggested that the Chinese yuan renminbi  
may suddenly collapse in a crisis along the lines of Mexico in 1994-95, 
in which the peso fall sharply when the Banco de Mexico 
ran out of foreign currency reserves.  
Give two reasons why this scenario is likely or unlikely.   
 (10~points)
 
Answer.  (i)~China has enormous reserves:  they won't run out any time soon.  
(ii)~The yuan renminbi seems to be undervalued:  people want to buy it, 
not sell it, which results in the central bank accumulating reserves, 
not losing them.  
 
\item Can a country run a fiscal (government) deficit forever?  Explain why or why not.  
 (10~points)
 
 Answer.  The present value of future primary surpluses has to equal 
 the current debt.  Thus past deficits must be countered by 
 future surpluses --- you can't run a primary deficit forever.
 The key word is primary:  you could run primary surplus and overall
 deficit at the same time, as we see in (for example) Brazil.  

\item Ben Bernanke, Chair of the Federal Reserve Board, has commented 
that while the Fed controls the short-term interest rate, 
it has only indirect influence on long-term interest rates. 
Describe how the Fed might influence the 5-year bond rate.  
(10~points)

Answer.  Basically, through market expectations of future policy. 
If we expect the Fed to raise interest rates in the future, 
that should show up in higher bond yields, including the 5-year rate.  
This past week, for example, there's been a lot of discussion 
of Bernanke's public statements about future increases in the 
Fed Funds rate.  

\end{enumerate}
%\pagebreak \phantom{bla} \pagebreak \phantom{bla} \pagebreak
\end{enumerate}

\end{document} 


% ------------------------------------------------------------------------
%  2005 exam ...
\begin{enumerate}

\item {\bf US economic conditions (30 points).}  How well is the US economy doing?  One source of
information is the forward rate curve.   Forward rates on May 5, 2005, and averages over the last
25 years, were
%
\begin{center}
\begin{tabular}{ccc}
  Maturity ($m$) & Forward Rate ($f_{m-1}$) & Mean Forward Rate \\
     1      &   3.73       &    7.51        \\
     2      &   4.33       &    8.10        \\
     3      &   4.47       &    8.39          \\
     4      &   4.62       &    8.57         \\
     5      &   4.75       &    8.67
\end{tabular}
\end{center}
Maturity is measured in years and forward rates are annually compounded percentages.
%
\begin{enumerate}

\item Compute zero-coupon yields for maturities between 1 and 5 years.  (5~points)

\item Use the forward rate curve to construct forecasts of future 1-year yields.  How do you see
the 1-year rate evolving over the next four years? What does this suggest for output growth?  (15~points)

\item Describe the Taylor Rule.  Under what conditions would it lead to the path of interest rates
you described above?  Do you find these conditions plausible?  (10 points)

\end{enumerate}

%\begin{comment}
%
Answer.  Consider these calculations:
\begin{center}
\begin{tabular}{ccccc}
  Maturity ($m$) &  Bond Price & Yield ($y_m$)  &  Risk Premium  &  Future Short Rate \\
     1   &  96.40   &   3.73       &    0.00   &  3.73  \\
     2   &  92.40   &   4.03       &    0.59   &  3.74  \\
     3   &  88.45   &   4.18       &    0.88   &  3.59  \\
     4   &  84.54   &   4.29       &    1.06   &  3.56  \\
     5   &  80.71   &   4.38       &    1.16   &  3.59
\end{tabular}
\end{center}

\begin{enumerate}
\item See table. \item Ditto.  The flat to decreasing path of expected future short rates suggests
modest to below-average growth of output. %
\item The Taylor rule is
\[
    i_t  \;=\;  r^* + \pi_t + a_1 (\pi_t - \pi^*)  + a_2 \log (Y_t /\bar{Y}_t)  .
\]
It describes how the Fed sets the short-term interest rate.  The flat interest rate path above
suggests flat inflation and output relative to trend, which is consistent with what we've seen
recently.
\end{enumerate}
%\end{comment}

%\pagebreak \phantom{bla} \pagebreak \phantom{bla} \pagebreak

% *************************************************************************************************
\item \textbf{Japanese fiscal policy (20 points).} Japan last year ran a government budget deficit
of close to 7\% of GDP.  In addition, the ratio of public debt to GDP was over 100\%, GDP growth
was below 1\%, inflation was slightly negative, and the current account surplus was 3.5\% of GDP.
%
\begin{enumerate}

\item Why might a government deficit be associated with a current account surplus, when we see the
opposite pattern in the US (government and current account deficits)?
%[Hint:  Use the flow identity.]
(10~points)

\item The Japanese government is said to be considering three approaches to the government
deficit: (i)~do nothing and hope it goes away, (ii)~raise tax rates, and (iii)~reduce purchases of
goods and services.  How would each be likely to affect the Japanese economy?  Which one would you
suggest? (10 points)

\end{enumerate}

Answer.
\begin{enumerate}
\item One version of the flow identity is:
\[
    S \;=\; I + \mbox{\em Def} + \mbox{\em CA} .
\]
In Japan, $\mbox{\em Def}$  (govt deficit) and $\mbox{\em CA}$ (current account balance) are both
positive, so $S$ (personal saving) must be significantly greater than $I$ (investment).  In the
US, $\mbox{\em CA}$ is negative:  $S$ is significantly smaller than $I$.

\item The list:
%
\begin{enumerate}
\item [(i)] Do nothing.  Debt will continue to build up, adding interest expense to the budget.
The longer this goes on, the larger the debt burden will be.

\item [(ii)]  Raise tax rates.  This may have adverse incentive effects, but it's not clear how
large those are.  Japan still has a relatively small tax burden overall.

\item[(iii)] Reduce spending. Depends what it is, but some cuts in spending are probably needed.
\end{enumerate}
\end{enumerate}

%\pagebreak \phantom{bla} \pagebreak


\item \textbf{Miscellany (50 points).}

\begin{enumerate}

\item {\it The Economist\/} reports that a Big Mac costs \$2.90 in the US, \$3.28 in the eurozone,
and \$2.33 in Japan.  (These prices are averages for the various regional markets, expressed in US
dollars using current spot exchange rates.)  What does this suggest about the likely change in
value of the euro and yen v. the dollar over the coming 6 months?  6 years? (10 points)

Answer.  PPP suggests that exchange rates will adjust to make prices the same across countries. In
this case, that means the dollar will rise against the euro, fall against the yen.  Is this right?
Over short periods of time, exchange rates are close to unpredictable by any means, PPP included.
Over longer periods of time (5-20 years) PPP is a fairly reliable guide.


\item Over the last year, Mexico had GDP growth of 5\%, a government budget deficit of less than
1\% of GDP, and a current account deficit of \$100b (about 2\% of GDP).  Foreign exchange reserves
are now about \$64b.  Do these indicators suggest possible difficulties in the Mexican economy? Be
specific.  (10 points)

Answer.  The ``Roubini indicators'' of trouble include a large current account deficit and a small
quantity of reserves.  In this case, reserves are modest (less than one year's current account),
but the current account deficit is small, esp for a country growing so rapidly.   In short: little
sign of difficulty.

\item In some countries, fear of collapse of the banking system has led people to hold more cash
and fewer bank deposits.  For a given level of the monetary base, what is the impact on the money
supply?  (10 points)

Answer.  The money supply falls:  the currency-deposit ratio rises, so that a given monetary base
leads to a smaller monetary aggregate.

\item In Canada over the last year, inflation has been 2.3\% and money growth (M1) has been
11.8\%.  Do you find the difference between the two numbers surprising? Why or why not?  (10
points)

Answer.  Not surprising because the short-run relation between money growth and inflation is weak.

\item Contrast labor markets in South Africa, Pakistan, and Ireland.  Which, in your view, offers
the most flexible market at the present time?  Why?  (10 points)

Answer.  Much more flexible in Ireland.  Pakistan is a morass of regulations that inhibit job
growth.  South Africa is burdened by (among other things) racial quasi-quotas.  History tells us
why they're there, but that makes them no less onerous for employers.  [Comment:  These countries
were the subjects of presentations.]

\end{enumerate}


\end{enumerate}

%\pagebreak \phantom{bla} \pagebreak \phantom{bla}


\vfill \centerline{\it \copyright \ \number\year \ NYU Stern School of Business}

\end{document}
