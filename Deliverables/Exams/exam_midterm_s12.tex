\documentclass[letterpaper,12pt]{exam}

\usepackage{ge05}
\usepackage{comment}
\usepackage{booktabs}
\usepackage[dvipdfm]{hyperref}
\urlstyle{rm}   % change fonts for url's (from Chad Jones)
\hypersetup{
    colorlinks=true,        % kills boxes
    allcolors=blue,
    pdfsubject={NYU Stern course GB 2303, Global Economy},
    pdfauthor={Dave Backus @ NYU},
    pdfstartview={FitH},
    pdfpagemode={UseNone},
%    pdfnewwindow=true,      % links in new window
%    linkcolor=blue,         % color of internal links
%    citecolor=blue,         % color of links to bibliography
%    filecolor=blue,         % color of file links
%    urlcolor=blue           % color of external links
% see:  http://www.tug.org/applications/hyperref/manual.html
}

\newcommand{\NX}{\mbox{\em NX\/}}
\newcommand{\POP}{\mbox{\em POP\/}}

\def\ClassName{The Global Economy}
\def\Category{David Backus}
\def\HeadName{Midterm Examination}

\printanswers

\begin{document}
\parindent = 0.0in
\parskip = \bigskipamount
\thispagestyle{empty}%
\Head

\centerline{\large \bf \HeadName}%
%\centerline{March 9, 2005}
\centerline{Revised:  \today}

\bigskip
You have 90 minutes to complete this exam.  Please answer each
question in the space provided and show all of your work.
You may consult one page of notes and a calculator,
but devices capable of wireless transmission are prohibited.

I understand that the honor code applies: I will not lie, cheat,
or steal to gain an academic advantage, or tolerate those who do.

\begin{flushright}
\rule{4in}{0.5pt} \\ (Name and Signature)
\end{flushright}


\begin{questions}
% ======================================================================
\question {\it Indonesia.\/}
Indonesia is one of the world's most populous countries,
but it remains a poor one,
with GDP per capita of about 6 thousand US dollars.
Its recent trajectory, however, has been strong,
with average GDP growth over 5\% between 2000 and 2011
and a barely perceptible impact from the global financial crisis.

From EIU reports,
we find that Indonesia's recent success comes after a tumultuous history.
Following independence from the Dutch after World War II,
it had several decades of authoritarian rule.
The bloody transition from Sukarno to Suharto in 1965 is vividly portrayed in
Peter Weir's 1982 film, ``The Year of Living Dangerously.''
Economic performance improved under Suharto, but dissatisfaction
with authoritarian rule peaked after the Asian Crisis of 1997,
when the currency fell by 80\% against the dollar
and real GDP  fell 14\%.

After the crisis, Indonesia made a rapid transition to
multi-party democracy,
with the first democratic elections in 34 years in 1999
and several more since then.

Your mission is to examine the economic roots of recent success using
the data in Table \ref{tab:indonesia}.

\begin{parts}

\part What is the average annual growth rate of GDP per capita between
2000 and 2011?
GDP per worker?
(Here and elsewhere in this question,
growth rates are understood to be continuously-compounded.)
(10~points)

\part What was total factor productivity in 2000 and 2011?
Its average annual growth rate?
(10~points)

\part What are the other sources of growth?
What factors account for the growth rate of GDP per worker
you computed in (a)?
GDP per capita?
(10~points)
\end{parts}

\begin{table}
    \centering
    \tabcolsep = 0.2in
    \begin{tabular}{lrrrr}
    \toprule
    Year     &  $ \POP $   &  $Y/\POP$   &  $Y/L$  &  $K/L$  \\
    \midrule
    2000 &   220.0   &   4,151  & 8,828  & 21,408  \\
    2011 &   245.6   &   6,209  & 12,672 & 23,471  \\
    \bottomrule
    \end{tabular}
    \caption{Indonesia: aggregate data on output and inputs.
    Population is in millions.  The other numbers are 2005 US dollars
    (PPP adjusted, from Penn World Tables and EIU CountryData).}
    \label{tab:indonesia}
\end{table}


\begin{solution}
Short answers follow, see the accompanying spreadsheet for specific calculations.
\begin{parts}
\part The growth rate of GDP per capita is
\begin{eqnarray*}
    \gamma &=& \log (6.209/4.151)/(2011-2000) \;\;=\;\; 3.660\%.
\end{eqnarray*}
As always, $\log$ means the natural logarithm,
sometimes denoted {\tt LN}.
I moved the decimal point for convenience, but it has no affect on the growth rate.
Using the same method, the growth rate of GDP per worker is 3.286\%.

Grading:  5 points for correctly computing each number.

\part Productivity we find indirectly from
$ A = (Y/L)/(K/L)^\alpha$.
In 2000, we find
\begin{eqnarray*}
    A &=& (Y/L)/(K/L)^{1/3} \;\;=\;\; 8.828/21.408^{1/3}
    \;\;=\;\; 3.179 .
\end{eqnarray*}
If you don't move the decimal point, your numbers will be multiplied by 100.
In 2011, the same calculation gives us $A = 4.426$.
The growth rate is 3.007\%.

Grading:  4 points for each TFP number, 2 for its growth rate.

\part Growth in GDP per worker has these components:
\begin{eqnarray*}
    \gamma_{Y/L} &=&   \gamma_A  + \alpha \gamma_{K/L}  \\
    3.286     &=& 3.007  + 0.279 .
\end{eqnarray*}
Similarly, growth in GDP per capita is
\begin{eqnarray*}
    \gamma_{Y/POP} &=&  \gamma_{L/POP}  + \gamma_A
                + \alpha \gamma_{K/L}  \\
    3.660     &=& 0.374 + 3.007  + 0.279 .
\end{eqnarray*}
It's evident in both cases that productivity is the primary force
behind economic growth.
The bigger question, of course, is where the productivity came from.
That wasn't part, but it's not hard to imagine some improvement in institutions.
Some of that is evident in the next question.

Grading:  6 points for growth in GDP per worker and its components,
4 for the same with GDP per capita.
\end{parts}
\end{solution}


%\pagebreak \phantom{xx} \pagebreak \phantom{xx} \pagebreak
% ======================================================================
\question {\it Indonesia and Kazakhstan.\/}
As the junior member of a consulting team,
you have been asked to collect information on the pros and cons
of building a small manufacturing operation
in Indonesia or Kazakhstan.
The plant would produce toys aimed at the growing Asian market.
Both countries have shown recent signs of economic progress.
Both are actively recruiting foreign manufacturers,
Indonesia to continue its growth, Kazakhstan to diversify beyond its resource-based economy.

A collection of institutional indicators is given in Table \ref{tab:i-v-k}.
In addition, the political situations are quite different.
Indonesia is an emerging democracy.
The EIU describes Kazakhstan's political structure as authoritarian:
%
\begin{quote}
Nursultan Nazarbayev, the current president and formerly the first secretary of
the Communist Party of the Kazakh Soviet Socialist Republic, has ruled
Kazakhstan since independence.
He has steadily increased his control over Kazakhstan's political
structures, which has allowed him to secure re-election several times, the
most recent presidential election being in December 2005. Parliament
approved amendments that pave the way for
him to remain president for life.
His party, Nur Otan (Light-Fatherland),
won every seat available for
election in the new parliament.
\end{quote}
As a result, the political situation is thought to be stable.


\begin{table}[t]
\centering
\begin{tabular}{lrrrr}
\toprule
        &  \multicolumn{2}{c}{Indonesia} & \multicolumn{2}{c}{Kazakhstan}  \\
Indicator   &  1996   & 2010    & 1996  & 2010 \\
\midrule
\multicolumn{2}{l}{\it Governance} \\
Political stability (percentile)  &  13.5  &  18.9 & 28.4 & 61.8 \\
Govt effectiveness (percentile)   &  40.0  & 47.8  & 13.2 & 44.5 \\
Control of corruption (percentile) & 30.7  & 27.3  & 9.3  & 15.3 \\
\midrule
\multicolumn{2}{l}{\it Labor} \\
Minimum wage (ratio to average) & & 0.41 & & 0.13 \\
Severance after 10 years (weeks of pay) & & 56 & & 4 \\
Mandatory vacation (days per year)  && 0 & & 13 \\
Flexible hours?  (yes, no)  && yes & & yes \\
%Educational quality (percentile) && 69 && 25 \\
\midrule
\multicolumn{2}{l}{\it Transportation infrastructure} \\
Overall quality (percentile)  &&  42 && 40 \\
\midrule
\multicolumn{2}{l}{\it International trade} \\
Documents required (number) & & 4 & & 9 \\
Delay (days) &  & 17 & & 76 \\
Cost (USD per container) & & 644 & & 3130 \\
\bottomrule
\end{tabular}
\caption{Institutional indicators for Indonesia and Kazakhstan.}
\label{tab:i-v-k}
\end{table}


\begin{parts}
\part Which of these indicators are most important to your venture?
How do the two countries compare on them?
(10~points)
\part Which country or countries would you recommend to your clients?
What are the challenges they would face?
(10~points)
\end{parts}

\begin{solution}
\begin{parts}
\part All of these matter somewhat.  I'd say
labor is important in a manufacturing operation,
and also infrastructure and trade,
because you plan to export your product.
And political stability is important because
you want to know the climate won't quickly change for the worse.

Grading:  10 points for a clear logical argument that connects
the institutions to the demands of the business,
either this one or something else.
Partial credit for part thereof.

\part One thing you might do is rate the two countries along all
of these dimensions,
then come up with an overall grade based on your weighting of their importance.
A quick summary might be:
\begin{itemize}
\item Governance.  Both look ok.  The numbers aren't great,
but that's the challenge of operating in an emerging market.
(The benefit, of course, is low price.)
Curiously, Kazakhstan gets a better grade on political stability.
It's a bit worse, though, on corruption.

\item Labor.  Both countries have reasonably flexible
labor markets, although severance is higher in Indonesia.

\item Infrastructure and trade.
The big issue here is the delay in exporting (76 days!)
and cost of shipping a container --- both worse for Kazakhstan.
\end{itemize}
You could go either way, but I lean toward Indonesia,
which has become something
of a darling among emerging markets.

The World Economic Forum says:
``Indonesia remains one of the best-performing countries
within the developing Asia region.
Sound fiscal management has brought the budget deficit and
public debt down to very low levels, attributes that
contribute to further upgrading the country�s credit rating.
The situation is also improving,
albeit from a much lower base,
in the area of physical infrastructure.''
They also note that ``the quality of port facilities remains alarming''
and ``the electricity supply continues to be unreliable and scarce.''
As usual, some good, some bad.
They rate Indonesia 46th (of 142 countries), Kazakhstan 76th.
They're using information that goes beyond the question,
but I include it for background.

Grading:  10 points for a clear logical argument,
either this one or something else,
partial credit for part thereof.

\end{parts}
\end{solution}

%\pagebreak \phantom{xx} \pagebreak %\phantom{xx} \pagebreak
% ======================================================================
\question {\it True/false.\/}
Please explain why each statement is true, false, or uncertain.
The explanation is essential.
%
\begin{parts}
\part If a product is made in the Mexico but sold to consumers in the US,
it is not included in Mexican GDP.
(10~points)

\part If the unemployment rate falls, employment has risen.
(10~points)

\part Firms find it costly to search for workers with the right skills.
For that reason, regulations that discourage labor turnover are good for the economy.
(10~points)

\part A tax on labor tends to reduce employment.
(10~points)

\part In Ricardo's model, free trade is good for consumers but bad for workers.
(10~points)
\end{parts}

\begin{solution}
\begin{parts}
\part False.  GDP measures production in a country.  It doesn't matter who buys it.
\part Uncertain.  The issue here is that there are three categories:
employed, unemployed, and not in the labor force.
A lot of the action is in the last category.
So if some of the unemployed get jobs,
that lowers the unemployment rate and raises employment and the statement is true.
But if some of the unemployed leave the labor force,
we could see employment flat and the statement is false.
Or employment could fall, too, if the workers
leave the labor force without ever being unemployed.
In short, employment and unemployment can (and do) point in
different directions.


\part False.
Since the cost is borne by firms, they're in the best position
to act accordingly.
But reducing turnover by regulation will force firms to retain
workers even when that cost is outweighed by the benefits of a flexible workforce.

Think of the example of Spain where high severance and related requirements
discourages firms from hiring during expansions. This leads to a lower
level of employment on average.
It also leads to greater use of temporary workers who are
not subject to this requirement but must leave their jobs when their contracts expire.
In this case, raising the cost of firing workers actually increases turnover.

\part True.  Think of a supply/demand setup.
Consider a tax on labor paid by firms
(that's not essential, but it's helpful to be precise).
That will shift the demand curve down by the amount of the tax.
If demand slopes down and supply slopes up,
we'll see a decline in employment.
This isn't all that different from other products:
we tax cigarettes, for example, because we want to reduce the quantity.
This is the same logic.

\part False.
It's good for everyone.
More to the point:  workers and consumers are the same people,
both in Ricardo's model and the real world.

\end{parts}
%
Grading:  10 points for something like the answers above.
\end{solution}

\end{questions}

%\pagebreak \phantom{xx} %\pagebreak \phantom{xx}

\vfill \centerline{\it \copyright \ \number\year \ NYU Stern
School of Business}

\end{document} 