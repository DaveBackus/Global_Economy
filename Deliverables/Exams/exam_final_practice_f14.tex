\documentclass[letterpaper,12pt]{exam}

\usepackage{../../LaTeX/ge14}
\input{../../LaTeX/preamble.tex}

% for gexx.sty
\def\ClassName{The Global Economy}
\def\Category{David Backus}
\def\HeadName{Practice Final \#1}

% prints answers -- or not, if commented out
\printanswers

\begin{document}
\parindent = 0.0in
\parskip = \bigskipamount
\thispagestyle{empty}%
\Head

\centerline{\large \bf \HeadName}%
\centerline{December 2013}
%\centerline{Revised:  \today}

\bigskip
{\it You have 120 minutes to complete this exam.  Please answer each
question in the space provided and show all of your work.
You may consult one page of notes and a calculator,
but devices capable of wireless transmission are prohibited.

I understand that the honor code applies: I will not lie, cheat,
or steal to gain an academic advantage, or tolerate those who do.}

\bigskip
\begin{flushright}
\rule{4in}{0.5pt} \\ (Name and Signature)
\end{flushright}


\begin{questions}
% ======================================================================
\begin{table}[h]
\centering
\tabcolsep=0.1in
\begin{tabular}{lrrrrr}
\toprule
                & 2010 & 2011 & 2012 & 2013 \\
\midrule
Official exchange rate (pesos per USD)  & 3.90 & 4.11 & 4.54 & 5.46  \\
Inflation (\%)              & 22.9 & 24.4 & 25.3 & 20.6 \\
Foreign currency reserves (USD billions) & 52.2 & 46.4 & 43.2 & 32.2 \\
Real GDP growth (\%)        & 9.2 & 8.9 & 1.9 & 5.2  \\
Govt revenue (\% of GDP)    & 24.3 & 23.6 & 25.4 & 27.3 \\
Govt spending (\% of GDP)   & 24.1 & 25.3 & 28.0 & 30.5  \\
Public sector surplus (\% of GDP) & 0.2 & --1.7 & --2.6 & --3.2 \\
Primary surplus (\% of GDP) &  1.7 & 0.3 & --0.2 & --0.8   \\
Govt debt (yearend, \% of GDP)  & & & 44.8\\
Interest rate paid on debt (\%) & 4.0 & 5.5 & 6.7 & 6.5  \\
Money market interest rate (\%) & 9.1 & 10.0 & 9.8 & 12.7 \\
\bottomrule
\end{tabular}
\label{tab:portugal}
\caption{Economic indicators for Argentina.  Source:  EIU.}
\end{table}

\item  {\it Don't Cry for Me Argentina (40~points).\/}
Argentina is a seemingly endless source of entertainment to economists,
yet its economy has done well in the recent past.
GDP growth fell to 0.9\% in 2009, during the global financial crisis,
but averaged over 9\% the next two years.
Most analysts attribute this success to
favorable commodity prices and strong global demand for Argentina's commodity exports.

At the same time, the government of President Cristina Fernandez de Kirchner
continues to adopt policies that befuddle outside observers, including:
taking over private pension funds,
restricting imports and purchases of foreign currency,
attacking the press,
nationalizing the Spanish-owned oil company YPF,
imposing price controls on electricity, natural gas, and public transportation,
and subsidizing energy consumption.

The Economist Intelligence Unit reports:
\begin{itemize}
\item A US court case may eventually leave
Argentina with the unpalatable choice of repaying the ``holdouts'' (creditors that
did not participate in the 2005 or 2010 restructurings) in full --- something that it
has sworn never to do --- or falling into technical default to avoid repaying
current creditors in a US jurisdiction.

\item According to official data, consumer price inflation remains among the highest
in emerging markets, at 10.5\% in April 2013. However, the official data are
widely discredited, and we are now using estimates produced by PriceStats,
which estimates that inflation in 2012 was 25\%.

\item Double-digit inflation has generated real peso appreciation.
Foreign-exchange controls have failed to prevent an erosion of the reserves cushion,
heightening the risk of an eventual devaluation.

\item The Argentine peso floats in principle, but the central bank intervenes to limit
the peso's depreciation.
In addition, foreign currency transactions are subject to a variety of controls.
For the past couple of years, the government has been gradually tightening the `clamp,'
an unofficial policy of discouraging purchases of dollars.
As a result, the peso's official decline has been modest,
but the unofficial ``blue market'' price of the peso is considerably lower.

\item The (bad) banking sector risk rating reflects weak economic activity, expansionary
monetary policies that contribute to credit risk, high risk of exchange-rate
and interest-rate volatility, and increased currency convertibility risk.

\item The ruling party fared badly in the October midterm election,
 leaving the president without enough support in Congress
 to change the constitution and run for re-election.
 Focus will now shift rapidly to the 2015 presidential race.
 The president remains alienated from almost all of the country's most influential groups,
including the unions, the media, the Catholic Church and the traditional
leaders of the Peronist party. In this context, risks to political stability will be
high. An additional risk to stability is the president's health.
\end{itemize}
%
The question is what happens next:  Could another crisis be on the way,
or has Argentina put its problematic past to rest?
Use the information provided, and your own experience and good judgement,
to assess the risks to the Argentina economy over the next 2-3 years.
%
\begin{parts}
\item By ``real appreciation'' we mean an increase in the price
of local goods relative to foreign goods ---
what is sometimes called a decline in the real exchange rate.
Use the numbers in the table to demonstrate (or disprove) real appreciation
of the peso.
(10~points)

\item Why do you think the central bank's foreign exchange reserves have declined?
(5~points)

\item How do you see government debt evolving?
Compute, in particular, the ratio of government debt to GDP at year-end 2013.
What factors contribute the most to the change in the ratio?
(10~points)

\item Overall, how would you rate the risk of a macroeconomic crisis in Argentina?
What are the biggest sources of concern?
(15~points)
\end{parts}

\begin{solution}
Answers follow.
See the spreadsheet for calculations in (a) and (c)
(download this pdf file, open it with the Adobe Reader or the equivalent,
and click on the pushpin):
\attachfile{exam_final_f13_answerkey.xlsx}

\begin{parts}
\item
One way to think about this (not the only one) is with the real exchange rate
$ \mbox{\em RER\/} = eP^*/P$, where $e$ is the exchange rate, $P$ is the price of Argentine goods,
and $P^*$ is the price of American goods.
So how is the real exchange rate changing?
Inflation is the rate of increase in $P$, so we see the price of Argentine goods
is going up rapidly, roughly 20\% a year.
In contrast, $eP^*$ is going up less:
$P^*$ is roughly flat (1-2\% inflation in the US)
and $e$ is rising (if we compute its rate of change)
5\% in 2011 and 10\% in 2012.
Thus {\it RER\/} is falling, as Argentine goods get relatively more expensive.

In words:  the combination of high inflation and
more modest currency depreciation has made Argentine goods expensive.

Grading:  5 points for the basic idea, 5 for a calculation that compares
inflation rates and the change in the exchange rate.

\item Evidently people want dollars, not pesos, and the central bank supplies
them to maintain a relatively stable exchange rate.
One possible reason:  Argentine prices are rising,
and a substantial depreciation is one way to get that.
That makes pesos less attractive, since you'd lose (relative to dollars)
if the peso falls in value.

Grading:  5 points for something along these lines.

See also \href{http://focoeconomico.org/2013/12/22/el-drenaje-de-reservas-del-bcra-una-amenaza-al-sistema-financiero/}
{this link}.

\item The debt dynamics equation is
\begin{eqnarray*}
   \Delta (B_t/Y_t)  &=&  (i_t - \pi_t)(B_{t-1}/Y_{t-1})
                - g_t (B_{t-1}/Y_{t-1}) + D_t/Y_t .
\end{eqnarray*}
The three terms are
\begin{eqnarray*}
    (i_t - \pi_t)(B_{t-1}/Y_{t-1}) &=&  -6.3  \\
    - g_t (B_{t-1}/Y_{t-1})   &=&  -2.3 \\
    D_t/Y_t  &=&  0.8 .
\end{eqnarray*}
Their total is --7.8, so the ratio of debt to GDP will fall to 37.0.
Note for later the negative contribution of the real interest rate:
they're getting a very good deal on their debt, hard to believe that will
continue.

Grading:  3 points for each component done right, 1 more for summing them.

\item This is a call for the checklist:
\begin{itemize}
\item Debt and deficits.
(i)~The calculation shows the debt ratio is falling.
But the US court case could lead to a technical default,
which isn't a good thing.
And the negative real interest rate is unlikely to continue.
If they paid a modest 2\% real rate on debt, the debt ratio would
go up about 4\% this year.

\item Banks.
The EIU suggests that banks could suffer from a weak economy.

\item Exchange rates and reserves.
They're losing reserves as they try to keep the peso
from depreciating.
Either the peso depreciates more or they continue to lose reserves.

\item Politics.  Always an issue in Argentina.
There's some uncertainty given the president's lame duck status and health.
On the other hand, a change could make things better.
\end{itemize}
%
The fiscal situation, including the court case, the exchange rate and reserve position,
the banking system, and the political situation all shows signs of trouble.
Overall, I'd say they'll probably muddle through,
but there's a chance of serious trouble.

Grading:  10 points for the checklist and its components,
5 points for a sensible argument assessing the various risks.
\end{parts}
\end{solution}


%\pagebreak \phantom{x} \pagebreak
% ======================================================================
\item  {\it The supply and demand of Abenomics (30~points).\/}
Shinzo Abe was elected Prime Minister of Japan in December 2012
after two decades of slow growth and falling prices.
He pledged dramatic policy changes to revive the Japanese economy,
dubbed the ``three arrows'' of ``Abenomics.''
We consult the Economist Intelligence Unit for specifics:
%
\begin{itemize}
\item Fiscal stimulus.  A sizeable economic stimulus package was passed by parliament in
February 2013, and a smaller one in October.
This is expected to produce a budget deficit of 8\% in 2013.
\item Monetary stimulus. A plan to double Japan's
money supply within two years was implemented in April 2013 to help to achieve the Bank of Japan's
target of 2\%  inflation.
\item Structural reform.
This is less clearly articulated, but some observers hope for a range of micro-based reforms,
including loosening product-market regulations that reduce productivity,
tightening corporate requirements for funding pensions,
creating a more flexible labor market,
and reducing subsidies to an inefficient agricultural sector.
\end{itemize}
%
Your mission is to explore the impact of the three arrows using the aggregate supply and demand
framework.
\begin{parts}
\item Explain, for each ``arrow,'' whether it affects supply or demand.
Which way does each one shift the appropriate curve(s)?
(15~points)
\item Compare the short- and long-term impact on output of the three policies.
Which are likely to have the greatest impact in the short term?
In the long term?
(15~points)
\end{parts}

\begin{solution}

This continues to be topical.
Here's a
\href{http://www.bloomberg.com/news/2013-12-12/abe-pushes-biggest-farm-revamp-since-macarthur-broke-landlords.html}
{recent comment} about agricultural policy.

\begin{parts}
\item We have:
\begin{itemize}
\item Fiscal stimulus. This shifts aggregate demand to the right.
\item Monetary stimulus. Same.
\item Structural reform. This shifts both aggregate supply curves to the right.
\end{itemize}
Grading:  5 points for each bullet done correctly.
The graphs need not be included, but if they are they should be correct.

\item Fiscal and monetary stimulus will raise output in the short run.
They have no long-run impact on output.

Structural reform, on the other hand, raises output both short-term and long-term.
In this respect, it's likely the most important of the arrows.
(Also, unfortunately, the one that's been executed least effectively.)

Grading: 5 points for each one.
Also:  the original question asks for the short and long-term impact, and doesn't
specify output, so answers might include comments about the impact on prices.
\end{parts}
\end{solution}


%\pagebreak \phantom{x} \pagebreak
% ======================================================================
\needspace{2\baselineskip}
\item  {\it Short answers (40 points).\/}
\begin{parts}
\item  How sensitive to the business cycle would you expect demand for Rolex watches to be?
Why?
(10~points)

\item If US inflation jumped to 5\%, how would you expect 
the Fed to change the fed funds rate? 
(10~points)

\item  Given what you know about global economic conditions,
how would you expect the US dollar to perform over the next year versus the euro?  Why?
(10~points)

\item Consider the statement:  ``Tax deductions are good, because they save taxpayers money.''
Do you agree or disagree?  Why?
(10~points)
\end{parts}

\begin{solution}
\begin{parts}
\item We would expect it to be very cyclical for two reasons:
it's a durable good, and it's a luxury.
Both categories are very cyclical.

Grading:  5 points each for durable and luxury.

\item This calls for a rough-and-ready Taylor rule calculation:
\begin{eqnarray*}
    i &=& r^* + \pi + 0.5 (\pi - \pi^*) + 0.5 (g - g^*) .
\end{eqnarray*}
With any conceivable inputs, you'd see a sharp increase in the fed funds rate,
and therefore in other interest rates.

Grading:  5 points for noting that this calls for the Taylor rule,
5 for suggesting it calls for an increase in the interest rate.

\item For most exchange rates, our best guess over periods less than 5 years is no change.

Grading:  10 points for something like this.

\item One feature of a good tax system is that it applies a low tax rate to a broad
base.
Given the overall level of spending, a tax deduction means tax rates must be higher
on other things, which violates this principle.

Grading:  10 points for something like this.
\end{parts}
\end{solution}

\end{questions}


\input{../../LaTeX/footer.tex}

%**********************************************************************
%**********************************************************************
\newpage
\def\HeadName{Practice Final \#2}
\parindent = 0.0in
\parskip = \bigskipamount
\thispagestyle{empty}%
\Head

\centerline{\large \bf \HeadName}%
\centerline{May 2013}
%\centerline{Revised:  \today}

\bigskip
{\it You have 120 minutes to complete this exam.  Please answer each
question in the space provided and show all of your work.
You may consult one page of notes and a calculator,
but devices capable of wireless transmission are prohibited.

I understand that the honor code applies: I will not lie, cheat,
or steal to gain an academic advantage, or tolerate those who do.}

\bigskip
\begin{flushright}
\rule{4in}{0.5pt} \\ (Name and Signature)
\end{flushright}

\begin{figure}[h]
\begin{center}
\setlength{\unitlength}{0.075em}
%\setlength{\unitlength}{0.1em}
\begin{picture}(280,200)(-20,0)
%\footnotesize
\thicklines

% horizontal axis
\put(-30,0){\vector(1,0){300}}
\put(255,-16){$Y$}

% vertical axis
\put(0,-20){\vector(0,1){200}}
\put(-15,155){$P$}

% demand
\put(25,165){\line(4,-3){200}}\put(230,10){AD}
%\put(65,165){\line(4,-3){200}}\put(270,10){AD}

% supply
\put(25,13){\line(4,3){200}} \put(230,160){AS}
%\put(65,13){\line(4,3){200}} \put(270,160){AS$'$}
\put(126.4,0){\line(0,1){170}} \put(118,175){AS$^*$}\put(122,-16){$Y^*$}

% equilibrium labels
%\put(105,85){\footnotesize B}
%\put(150,115){\footnotesize A}
%\put(138,64){\footnotesize C}
% dotted lines
%\qbezier[31]{(133,0)(133,46)(133,92)}
%\qbezier[45]{(0,92)(67,92)(133,92)}
%\qbezier[45]{(0,72)(67,72)(133,72)}

\end{picture}
\end{center}
\caption{Aggregate supply and demand diagram}
\label{fig:asad}
\end{figure}

\begin{questions}
% ======================================================================
\question {\it Two views of monetary policy (40~points).\/}
The goal is to connect some of the things we've learned
about monetary policy,
starting with the diagram in Figure \ref{fig:asad}.

\begin{parts}
\item If we increase the money supply, what happens to the curves in the diagram?
Which ones shift?  Why?
(5~points)
\item What is the short-run impact on prices and output?  The long-run impact?
Illustrate both in the diagram.
(10~points)
\item How does the impact compare to the traditional goals of monetary policy?
Is the policy of increasing the money supply a good one in the context of the figure?
(10~points)
\item Now consider the same thing from the perspective of the quantity theory equation.
If velocity is constant, what is the impact of an increase in the money supply?
(10~points)
\item How do your answers to (b) and (d) compare?
Is your analysis in (b) consistent with the quantity theory?
(5~points)
\end{parts}

\begin{solution}
We'll refer to the diagram:
\begin{center}
\setlength{\unitlength}{0.075em}
%\setlength{\unitlength}{0.1em}
\begin{picture}(280,200)(-10,-15)
%\footnotesize
\thicklines

% horizontal axis
\put(-30,0){\vector(1,0){300}}
\put(255,-16){$Y$}

% vertical axis
\put(0,-20){\vector(0,1){200}}
\put(-15,155){$P$}

% demand
\put(25,165){\line(4,-3){200}}\put(230,10){AD}
\put(65,165){\line(4,-3){200}}\put(270,10){AD$'$}

% supply
\put(25,13){\line(4,3){200}} \put(230,160){AS}
%\put(65,13){\line(4,3){200}} \put(270,160){AS$'$}
\put(126.4,0){\line(0,1){170}} \put(118,175){AS$^*$}\put(122,-16){$Y^*$}

% equilibrium labels
\put(145,112){\footnotesize B}
\put(105,85){\footnotesize A}
\put(115,110){\footnotesize C}
% dotted lines
%\qbezier[31]{(133,0)(133,46)(133,92)}
%\qbezier[45]{(0,92)(67,92)(133,92)}
%\qbezier[45]{(0,72)(67,72)(133,72)}

\end{picture}
\end{center}


\begin{parts}
\item An increase in the money supply shifts aggregate demand to the right.
The new curve is represented in the diagram by AD$'$.

\item In the short run, we move from A to B.  Prices and output both rise.
In the long run, we move to C, where we see (relative to A) that the only impact is on prices.

\item The traditional goals are (i) output equal to $Y^*$ and
(ii) stable prices.
We miss on both in the short run and on stable prices in the long run.
The best policy is to stay at A.

\item The quantity theory equation is
\begin{eqnarray*}
    M V &=& P Y ,
\end{eqnarray*}
where $M$ is the money supply, $V$ is velocity,
$P$ is the price level, and $Y$ is real output.
If $V$ is constant, an increase in $M$ increases some combination
of $P$ and $Y$.

Grading:  5 points for the equation, 5 points for noting its consequences.

\item We found the same in our AS/AD analysis,
but in the latter
we also had a way to determine how much of each,
and how that changes over different time horizons.
\end{parts}
\end{solution}

%\pagebreak \phantom{x} \pagebreak
% ======================================================================
\begin{table}[h]
\centering
{\small
\begin{tabular}{lrrrrrr}
\toprule
        & 2008 & 2009 & 2010 & 2011 & 2012 & 2013 \\
\midrule
Nominal GDP (\euro billions) & 172.0 & 168.5 & 172.8 & 171.0 & 165.4 & 156.7 \\
Real GDP growth (\%) & 0.0 & --2.9 & 1.9 & --1.6 & --3.2 & --3.0 \\
Inflation (\%) & 2.6 & --0.8 & 1.4 & 3.7 & 2.8 & 0.4 \\
Govt revenue (\% of GDP)  & 41.1 & 39.6 & 41.4 & 45.0 & 41.0 & 40.7 \\
Govt spending (\% of GDP) & 44.8 & 49.8 & 51.3 & 49.4 & 47.4 & 46.7  \\
Public sector balance (\% of GDP) & --3.7 & --10.2 & --9.9 & --4.4 & --6.4 & --6.0\\
Primary surplus (\% of GDP)  & --1.0 & --7.6 &  --7.1 & --0.6 & --2.4 & --0.5   \\
Govt debt (yearend, \% of GDP) & 71.7 & 83.4 & 91.0 & 98.9 & 118.2 \\
Interest rate paid on debt (\%) &  & 3.6 & 3.5 & 4.0 & 3.6	& 4.2 \\
Market rate on debt (\%)        & 4.5 & 4.2 & 5.4 & 10.2 & 10.5 & 6.3 \\
\bottomrule
\end{tabular}
}
\label{tab:portugal}
\caption{Economic indicators for Portugal.  Source:  EIU reports.}
\end{table}

\question {\it Disaster and opportunity in Portugal (40~points).\/}
As an investor in distressed debt,
you know well that economic disasters can be great opportunities.
You wonder whether Portugal is one now, specifically
the debt of the Portuguese government.

You take a look at the Economist Intelligence Unit's reports
and find:
\begin{itemize}
\item Portugal is currently operating with deficit financing
provided by the ``troika'' (the EU, ECB, and IMF).
One of the conditions is that the deficit be reduced.
The \euro 80b in ``bailout loans'' have more attractive interest rates
than debt issued in the public market.

\item The average maturity of government debt is currently 7.5 years.

\item Although 80\% of the deficit reduction plans involved increases in
tax revenue, the Constitutional Court ruled against roughly half the proposed
cuts in spending.

\item Portuguese banks face rising loan losses as the economy endures a prolonged
contraction.

\item There is risk the government will absorb large
``contingent liabilities'' from insolvent banks and state-owned companies.

\item The exchange rate is fixed within the Euro Zone.
The central bank has no foreign currency reserves.

\item The EIU expects the governing center-right coalition
of the Social Democratic Party and the Popular Party
to hold together through 2015, when elections are due.
However, continued austerity could weaken public support for the government
amid a prolonged recession.
\end{itemize}

Using all of the information above, and your own experience and good judgement,
assess the risks to an investor in Portuguese government debt.
%
\begin{parts}
\item Use what you know about debt dynamics to compute the ratio
of government debt to GDP at year-end 2013.
What factors contribute to this number?
(10~points)
\item How important is the interest rate paid on debt?
How would your analysis change if the rate rose by 2\%?
Fell by 2\%?
Which is more likely, in your view?
(10~points)
\item Overall, how would you assess the risks to investors?
What are the biggest risks?
Would you buy Portuguese debt now?
(20~points)
\end{parts}

\begin{solution}
\begin{parts}
\item The debt dynamics equation is
\begin{eqnarray*}
   \Delta (B_t/Y_t)  &=&  (i_t - \pi_t)(B_{t-1}/Y_{t-1})
                - g_t (B_{t-1}/Y_{t-1}) + D_t/Y_t .
\end{eqnarray*}
The three terms are
\begin{eqnarray*}
    (i_t - \pi_t)(B_{t-1}/Y_{t-1}) &=& (0.042 - 0.004) (118.2) \;\;=\;\; 4.5  \\                - g_t (B_{t-1}/Y_{t-1})   &=& - (-0.030) (118.2) \;\;=\;\; 3.5 \\
    D_t/Y_t  &=&  0.5 .
\end{eqnarray*}
Their total is 8.5, so the ratio of debt to GDP will rise from 118.2 to 126.7.

\item This involves the first term above.  If we increase the interest rate paid
by 2\%, that adds $ 0.02 \times 118.2 = 2.4$,
so you get an additional increase in debt of 2.4\% of GDP.
Similarly, a decline of 2\% leads to a decline in debt of the same amount.
Given the large difference between the interest rate paid on the debt
and the current market rate, it seems more likely the rate will go up
than go down.

\item This is a call for the checklist:
\begin{itemize}
\item Debt and deficits.
(i)~The calculation shows us debt is still rising, although a substantial part
comes from a shrinking economy.
(ii)~Contingent liabilities.

\item Banks.
The EIU suggests that banks could be collateral damage
from a bad economy.

\item Exchange rates and reserves.  Not relevant here,
Portugal is part of the Euro Zone and does not have any control over
its exchange rate.

\item Politics.  Always an issue, but the EIU thinks
it's mostly under control.
\end{itemize}
The biggest risk seems to be that the government is unable to get
the budget under control.
If it does, the debt could be a good deal.
Should you buy some? It's a risky bet, for sure.
To me, the fiscal situation is bad enough that I'd stay away.
But there's some upside if they turn it around,
and if they continue to get favorable terms from the troika.

\end{parts}
\end{solution}

%\pagebreak \phantom{x} \pagebreak
% ======================================================================
\question {\it Short answers (40 points).\/}
\begin{parts}
\item Draw the cross-correlation function for a countercyclical lagging indicator.
(10~points)
\item In the Euro Zone, GDP growth is now $-0.5$\%
and inflation is 1.7\%.
The ECB recently reduced its target interest rate to 0.5\%.
Use the Taylor rule to assess the suitability of this policy.
(10~points)
\item Describe how China's purchases of foreign currency affect
the balance sheet of the central bank, the People's Bank of China.
How would it be affected by ``sterlization.''
(10~points)
\item The Economist reports that Big Macs cost about the same in Canada
and the US in 2008, but are now 25\% more expensive in Canada.
What does this suggest about the ``dollar-dollar'' exchange rate over the next 6 months?
(10~points)
\end{parts}

\begin{solution}
\begin{parts}
\item The largest correlation should be in the lower-right quadrant,
like the ccf for the unemployment rate in Figure 11.3 of the book.
Your figure should be labeled correctly, with leads to the left and
lags to the right.

\item The Taylor rule calculation is something like this:
\begin{eqnarray*}
    i &=& r^* + \pi + 0.5 (\pi - \pi^*) + 0.5 (g - g^*) .
\end{eqnarray*}
We need some inputs to put this to work, your call what that might be.  
The US numbers are
$ r^* = 2$, $\pi^* = 2$, and $g^* = 3$.
If we use the same, we get $i = 1.8$, well above the current rate of 0.5.

\item Let's say the PBOC's balance sheet looks something like this:
%
\begin{center}
\begin{tabular}{lr|lr}
               Assets  &     &     Liabilities                     \\
               \hline
               FX Reserves &  100 &     Money &  200   \\
               Bonds   & 100 & \\
\end{tabular}
\end{center}
%
If the PBOC purchases 50 worth of foreign currency,
it changes to
%
\begin{center}
\begin{tabular}{lr|lr}
               Assets  &     &     Liabilities                     \\
               \hline
               FX Reserves &  150 &     Money  &  250   \\
               Bonds   & 100 & \\
\end{tabular}
\end{center}
%
Note that (the supply of) money has gone up:  when it bought foreign currency,
it issued local currency in return.
The PBOC can undo this by selling 50 of bonds, taking money
in exchange:
%
\begin{center}
\begin{tabular}{lr|lr}
               Assets  &     &     Liabilities                     \\
               \hline
               FX Reserves &  150 &     Money  &  200   \\
               Bonds   & 50 & \\
\end{tabular}
\end{center}
%
This transaction is referred to as ``sterilization.''

\item This is a PPP-like calculation:  goods are expensive in Canada,
so the Canadian dollar is overvalued.
We know from experience that this has no predictive value in the short term.
\end{parts}
\end{solution}

\end{questions}
%\pagebreak \phantom{xx} %\pagebreak \phantom{xx}


\input{../../LaTeX/footer.tex}

%**********************************************************************
%**********************************************************************
\newpage
\def\HeadName{Practice Final \#3}
\parindent = 0.0in
\parskip = \bigskipamount
%\setcounter{page}{1} 
\thispagestyle{empty}
\Head

\centerline{\large \bf \HeadName}%
\centerline{May 2010}

\bigskip
You have 120 minutes to complete this exam.  Please answer each
question in the space provided. You may consult one page of notes
and a calculator, but devices capable of wireless transmission are
prohibited.

I understand that the honor code applies: I will not lie, cheat,
or steal to gain an academic advantage, or tolerate those who do.

\begin{flushright}
\rule{4in}{0.5pt} \\ (Name and Signature)
\end{flushright}

%\bigskip
{\it Note:  These questions come from old exams,
so the topics and numbers may be out of date.
But be assured:  good analysis lasts forever.}

\begin{questions}
% ************************************************************************
\question {\it US monetary policy (25 points).}
When the US Federal Open Market Committee (FOMC) met in April 2007,
inflation was close to 3\% and GDP growth was about 2\%.
After the meeting,
their statement said, in part:
%
\begin{quote}
Recent indicators have been mixed. ...
Nevertheless, the economy seems likely to continue
to expand at a moderate pace over coming quarters.
%
Recent readings on core inflation have been somewhat elevated.
Although inflation pressures seem likely to moderate over time,
the high level of resource utilization has the potential
to sustain those pressures.
%
In these circumstances, the Committee's predominant policy concern remains the risk that inflation will fail to moderate as expected. Future policy adjustments will depend on the evolution of the outlook for both inflation and economic growth, as implied by incoming information.
\end{quote}
%
Comment:  this is a time when it's helpful to
see the AS/AD model lurking behind the words.
The following questions should help you work your
way through this process.
%
\begin{parts}
\item How would you interpret the Fed's statement in terms
of its goals of low inflation and full employment?
Do you agree that the evidence, as presented,
suggests that inflation is
the ``predominant policy concern''?
(10~points)

\item Use the aggregate supply and demand framework
to illustrate how the FOMC might think about monetary policy.
Consider each of the following questions in turn:
(i)~How would an increase in the money supply affect
output and prices in the short run?
(ii)~In the long run?
(10~points)

\item Given the Fed's assessment of current economic conditions,
how would you expect it to respond?
Would you expect interest rates to rise or fall?
(10~points)
\end{parts}

\begin{solution}
\begin{parts}
\item The statement suggests that inflation is above target
(``core inflation elevated'') and output is at the target
(``high level of resource utilization'') or possible beyond it.
With output at or above target, inflation is the appropriate focus.
GDP growth points the other way, so there's a question about this assessment.

\item Draw a diagram with AS, AD, and AS$^*$.
%Note in particular where the short-run equilibrium (where AS and AD cross)
%is relative to AS$^*$.
Assume for the time being that we start at a long-run equilibrium
(where AS crosses AD and AS$^*$ at the same time).
(i)~Short-run impact of an increase in the money supply.
AD shifts right/up, and we move to where AS and AD cross (and
ignore AS$^*$). This raises output and prices (inflation).
(ii)~In the long-run, AS also shifts, sending us to the point where AD and AS$^*$
cross. Why? Because the sticky wages that give AS its slope
eventually adjust. The result:
prices rise, but output stays the same, relative to our starting
point. This illustrates the difference between the short-run and
long-run effects of monetary policy:  In the long run, all we get is
inflation.  In the short run, we get a combination of higher prices
and higher output. Which sends us back to (a):  one of the questions
the Fed must address is how large the short-run increase in output
is.  That, in turn, depends on how steep the AS curve is:  the
steeper the curve, the smaller the increase in output. [For
practice, contrast this analysis with one where we start to the left
of AS$^*$.]

\item The comment about utilization tells you that
the FOMC sees output as at or beyond the long-run supply curve AS$^*$.
To reduce inflation, the Fed would reduce the money supply,
shifting aggregate demand left.
In practice, they would do this by raising the interest rate.
The short-run impact is to lower inflation and output.
If output is above target, that's ok on both fronts.
If it's at target, the Fed will have to decide whether output or inflation
is more important.
In this case, we have the Fed's own statement that inflation is the
``predominant policy concern,''
so this is the policy they have in mind.
We would therefore expect the Fed to raise the Fed funds rate.

\end{parts}
\end{solution}
%\end{questions}\end{document}

% ************************************************************************
%
\begin{center}
{\small
\begin{tabular}{lrrrrrrrr}%
\toprule
%\vspace{-.3cm}\\
    & 2002 & 2003 & 2004 & 2005 & 2006 &  2007 &  2008 & 2009 \\%
%\vspace{-.3cm}\\
\midrule
Real GDP growth  & 4.3 & 3.0 & 3.8 & 2.8 & 2.8 & 4.0 & 2.1
        & --1.6 \\
Inflation
        & 3.0 & 2.8 & 2.3 & 2.7 & 3.5 & 2.3 & 4.4 & 1.2 \\
Interest rate:  short
        & 4.6 & 4.8 & 5.3 & 5.5 & 5.8 & 6.4 & 6.7 & 2.8 \\
Interest rate:  long
        & 5.8 & 5.4 & 5.6 & 5.3 & 5.6 & 6.0 & 5.8 & 3.3 \\
Investment rate
        & 24.1 & 25.2 & 25.5 & 26.5 & 26.8 & 27.8 & 28.7 &  \\
Saving rate
        & 20.1 & 20.4 & 20.0 & 21.3 & 21.2 & 21.9 & 24.3 &  \\
Current account
        & --3.8 & --5.5 & --6.1 & --5.8 & --5.5
        & --6.4 & --4.2 & --3.5 \\
Govt budget:  total
        & 1.3 & 1.8 & 1.1 & 1.5 & 1.5 & 1.6 & 1.8 & --3.3 \\
Govt budget:  primary
        & 2.9 & 3.2 & 2.4 & 2.7 & 2.6 & 2.6 & 2.7 & --2.6 \\
Govt debt
        & 20.1 & 18.5 & 17.5 & 17.0 & 16.4 & 15.4 & 13.9  \\
Exchange rate
        & 1.84 & 1.54 & 1.36 & 1.31 & 1.33 & 1.20 & 1.19 \\
Real exchange rate
        & 100 &  113 & 121 & 125 & 125 & 133 & 132 \\
FX reserves (USD)
        & 21 & 32 & 36 & 42 & 53 & 25 & 31 & \\
FX reserves (months)
        &  2.9 & 3.7 & 3.3 & 3.4 & 4.0 & 1.6 & 1.7 \\
\bottomrule
\end{tabular}
}
\end{center}
{Economic indicators for Australia.
Notes:
(i)~Investment, saving, current account, government budget,
government debt, and net foreign assets
are expressed as percentages of GDP (ratios multiplied by 100).
(ii)~The exchange rate is the Aussie dollar (AUD) price
of one US dollar;
high numbers indicate that foreign currency is expensive.
The real exchange rate is a weighted average across trading partners.
The convention is the inverse of the exchange rate:
high numbers indicate that local goods are expensive relative to foreign
goods.
(iii)~Foreign exchange reserves are expressed, first,
in billions of USD, second,
as a ratio to monthly imports.
Thus the number 2.9 means that reserves are 2.9 times one
month's imports.
(iv)~2009 numbers are estimates.}

\question {\it Deficits down under (40 points).\/}
As a European investor in short-term Australian securities,
you have made a fair amount of money over the last decade
betting that Australia's high interest rates would not
be offset by declines in the value of its currency.
You wonder, however, whether it's time to take your money
and run.

Having some experience with such situations,
you check the Economist Intelligence Unit's Country Data,
summarized above,
and Country Risk Service,
which reports:
%
\begin{itemize}
\item The exchange rate is flexible, and could move either way against
the euro.
\item Australia's large net foreign liability position reflects
a combination of direct investment in Australian businesses,
notably mining,
and the carry trade,
in which investors purchase AUD-denominated assets
in order to benefit from relatively high local interest rates.
\item The banking system is stable.
\item Australian political institutions are widely regarded to be
of high quality and stable.
\end{itemize}

With this information in hand, you go through your checklist:
%
\begin{parts}
\item Fiscal policy 1.
Why are the total and primary government balances different?
(10~points)

\item Fiscal policy 2.
What would you estimate for the debt-to-GDP ratio
at the end of 2009?
(20~points)

\item Considering fiscal policy and other risk factors,
how do you see the risks to the Australian economy
over the next two years or so?
(10~points)
\end{parts}

\begin{solution}
\begin{parts}
\item The difference is interest on government debt.
The government budget looks like (you'll have to look
up the notation)
\begin{eqnarray*}
        D_t &=&  B_t - (1+i)B_{t-1} .
\end{eqnarray*}
The expression on the left is the primary deficit.
On the right, interest in the debt is part of the second
term:  $ i B_{t-1}$.
The numbers tell us this is expected to be 0.7\% of GDP
(the difference between the primary and total deficits).

\item The debt-to-GDP ratio evolves like this:
\begin{eqnarray*}
   \Delta (B_t/Y_t)  &=&  (i_t - \pi_t)(B_{t-1}/Y_{t-1})
                - g_t (B_{t-1}/Y_{t-1}) + D_t/Y_t .
\end{eqnarray*}
The numbers we're given suggest
$i-\pi = 1.8\%$.
(I'm taking a rough average of the two interest rates and using 3.0.)
That gives us
\begin{eqnarray*}
   \Delta (B_t/Y_t)  &=&  0.3 + 0.2 + 2.6
            \;\;=\;\; 3.1,
\end{eqnarray*}
so the debt-to-GDP ratio rises to 17.0\%.

\item You have some flexibility here, but I would expect you to
go through the checklist:
fiscal policy (done),
exchange rate (overvalued but flexible, with limited reserves),
banking system (EIU says ok), and politics/institutions.

Two points seem essential:
the debt to GDP ratio is very low,
and Australia  is a developed country with good institutions.
This suggests  no particular cause for concern.

[In hindsight, some of the same might have been said about the US,
with somewhat higher debt and deficits,
but disaster struck anyway.
It's a good reminder:  these things are hard to predict.
Nevertheless, it's helpful to survey the territory,
see if there are any obvious landmines lying around.]

\end{parts}
\end{solution}
%\end{questions}\end{document}

% ***************************************************************************
\question {\it Miscellany (50 points).}

\begin{parts}
\item {\it Chinese crisis?\/}
An analyst suggested that China may suffer a currency crisis
along the lines of Mexico in 1994-95,
in which the peso fell sharply when the Banco de Mexico
ran out of foreign currency reserves.
Do you find this scenario likely or unlikely?  Why?
 (10~points)

\item {\it Government deficits.\/}
Can a country run a fiscal (government) deficit forever?
Why or why not?
 (10~points)


\item {\it Canadian inflation.\/}
In Canada over the last year, inflation has been 2.3\%
and money growth  has been 11.8\%.
Do you find the difference between the two numbers surprising?
Why or why not?  (10~points)

\item {\it Leading indicators.\/}
Explain what a leading indicator is
and give an example for the US.  (10~points)

\item {\it Monetary policy mechanics.\/}
Use the central bank's balance sheet to describe
how it
maintains the short-term interest rate at a specific level.
(10~points)
\end{parts}

\begin{solution}
\begin{parts}
\item (i)~China has enormous foreign exchange
reserves:  they won't run out any time soon.
(ii)~The renminbi seems to be undervalued:
people want to buy it,
not sell it, which results in the central bank accumulating reserves,
not losing them.

\item The present value of future primary surpluses has to equal
the current debt.  Thus past deficits must be balanced by future
surpluses --- you can't run a primary deficit forever. The key word
is primary:  you can run a primary surplus and an overall deficit at
the same time, as we see in (for example) Turkey.

\item Under the quantity theory, inflation equals money growth
minus real GDP growth. Unless real GDP growth is 9\%, something's
wrong. What's wrong is that this relation is well known not to work
in the short run.  Over periods of several years, however, it
typically works pretty well.

\item A leading indicator (of the economy)
is an observable economic variable whose ups and
downs precede those in (say) real GDP.
You can see this in the cross-correlation function, for example.
Common examples:  housing starts, stock market indexes,
interest rate spread (long minus short).

\item Central banks manage short-term interest rates through
``open market operations'': buying and selling government
securities. Selling securities, for example, reduces the amount of
currency in private circulation, which generally increases
short-term interest rates.  [Insert T-accounts here.] The story we
tell is that this reduces the liquidity of capital markets by
reducing the quantity of currency in circulation.

\end{parts}
\end{solution}

\end{questions}
%\end{document}
%\pagebreak \phantom{bla} \pagebreak \phantom{bla}



\input{../../LaTeX/footer.tex}

%**********************************************************************
%**********************************************************************
\newpage
\def\HeadName{Practice Final \#4}
\parindent = 0.0in
\parskip = \bigskipamount
%\setcounter{page}{1} 
\thispagestyle{empty}
\Head

\centerline{\large \bf \HeadName}%
\centerline{May 2009}

\bigskip
You have 120 minutes to complete this exam.  Please answer each
question in the space provided. You may consult one page of notes
and a calculator, but devices capable of wireless transmission are
prohibited.

I understand that the honor code applies: I will not lie, cheat,
or steal to gain an academic advantage, or tolerate those who do.

\begin{flushright}
\rule{4in}{0.5pt} \\ (Name and Signature)
\end{flushright}

\begin{questions}
% ************************************************************************
\question {\it Chinese foreign exchange intervention (30 points).}
Joseph Yam, Chief Executive of the Hong Kong Monetary Authority, wrote about the foreign exchange activities of the People's Bank of China (PBOC) in January 2007:
%
\begin{quote}
The accumulation of foreign exchange reserves
involves the PBOC buying foreign assets through creating renminbi.
As a result, the monetary base is increased, creating a need to ``sterilise.''
This is done by issuing paper [short-term notes] to the banks.
The PBOC obviously has to pay interest on the money borrowed.
Currently the yield of, for example, three-month paper issued by the PBOC is about 2.5\%.
This is lower than the yield on foreign assets held as reserves ---
the yield on US treasuries is about 4 to 5\% --- so theoretically
reserve accumulation can be profitable.
The problem, however, is the continuing appreciation of the renminbi,
which gradually reduces the value of those foreign assets in renminbi terms.
\end{quote}

\begin{parts}
\item Use the central bank's balance sheet to show how purchases of
foreign currency increase the monetary base (think:  supply of currency).
(10~points)

\item Show how sterilization can be used to reverse the impact on
the supply of currency.
(10~points)

\item You may note that interest rates have now flipped, with Chinese interest rates above US interest rates.
    What does this imply for the returns on the PBOC's balance sheet?
    How might it avoid this outcome?
(10~points)
\end{parts}

\begin{solution}
\begin{parts}
\item A typical central bank balance sheet looks something like this:
%
\begin{center}
\begin{tabular}{lrclr}
               Assets  &     &&     Liabilities                     \\
               \hline
               FX Reserves &  100 &&     Monetary Base &  200   \\
               Bonds   & 100 && \\
\end{tabular}
\end{center}
%
Monetary base is (roughly) another term for currency.
Suppose, now, that the PBOC has to purchase another 100 worth of
foreign currency (or foreign-currency denominated bonds)
and issues currency in return.
Then the balance sheet becomes
%
\begin{center}
\begin{tabular}{lrclr}
               Assets  &     &&     Liabilities                     \\
               \hline
               FX Reserves &  200 &&     Monetary Base &  300   \\
               Bonds   & 100 && \\
\end{tabular}
\end{center}
%
This is the impact of the accumulation of reserves on the monetary base described by Yam.


\item To undo the increase in the monetary base, the PBOC will issue 100 worth of bonds and accept money in return:
%
\begin{center}
\begin{tabular}{lrclr}
               Assets  &     &&     Liabilities                     \\
               \hline
               FX Reserves &  200 &&     Monetary Base &  200   \\
               Bonds   & 0 && \\
\end{tabular}
\end{center}
%
This operation is referred to as sterilization.
For the PBOC, this process has continued to the extent that they
have had to issue bonds as liabilities, so that their balance sheet now looks something like this:
%
\begin{center}
\begin{tabular}{lrclr}
               Assets  &     &&     Liabilities                     \\
               \hline
               FX Reserves &  800 &&     Monetary Base &  200   \\
               Bonds   &  0 &&    Bonds  &  600 \\
\end{tabular}
\end{center}
%
The bond liabilities (so-called ``sterilization bonds'')
are issued primarily to Chinese commercial banks,
as outlined above.
Note, as Yam does, that assets and liabilities are equal.

\item The PBOC is now likely losing money on its portfolio
for two reasons:
(i)~because the rate it's paying on liabilities is greater than
the rate received on assets and
(ii)~continued increases in the value of the renminbi raise
the value of its liabilities relative to its assets.
In a sense, they have the wrong side of the carry trade.

How could it avoid this outcome?
Let the currency float and stop buying foreign currency.
This is entirely the byproduct of managing the exchange rate:
that leads them to buy foreign currency, with the results
we've just seen.
\end{parts}
\end{solution}
%\end{questions}\end{document}

%\pagebreak \phantom{bla} \pagebreak \phantom{bla} \pagebreak
% ---------------------------------------------------------
\question {\it Globalization and inflation (20 points).\/}
Fed Chairman Bernanke said recently (March 2007):
\begin{quote}
As national markets become increasingly integrated and open, sellers of goods, services, and labor may face more competition and have less market power than in the past.
These linkages suggest that, at least in the short run, globalization and trade
may affect the course of domestic inflation.
\end{quote}
%
\begin{parts}
\item Use aggregate supply and demand to
describe how expansionary monetary policy
affects output and inflation in the short run.
(10~points)

\item Back to Bernanke:
What do you see as the impact of globalization
(think: imports from China)
on domestic inflation?
How would you represent this
in the aggregate supply and demand diagram?
How does globalization change the impact of expansionary monetary
policy in this model?
Do you find the model persuasive in this respect?
(10~points)
\end{parts}


\begin{solution}
\begin{parts}
\item This would increase both output and prices (inflation,
loosely speaking).  Your answer should show a diagram with
AS, AD, and AS$^*$.
AD shifts out, with the stated result.

\item The typical argument is that AS has become flatter,
since attempts to raise prices will meet with strong foreign competition.
As a result, expansionary monetary policy has a
smaller impact on inflation and larger impact on output.
\end{parts}
\end{solution}
%\end{questions}\end{document}

%\pagebreak \phantom{bla} \pagebreak

\question {\it Miscellany (50 points).}

\begin{parts}
\item {\it Exchange rates.\/}
{\it The Economist\/} reports that a Big Mac costs \$2.90 in the US, \$3.28 in the eurozone,
and \$2.33 in Japan.  (These prices are averages for the various regional markets, expressed in US
dollars using current spot exchange rates.)  What does this suggest about the likely change in
value of the euro and yen v. the dollar over the coming 6 months?  6 years? (10~points)

\item {\it Inflation.\/}
Milton Friedman once said:  inflation is always and everywhere a monetary phenomenon.
Do you agree or disagree?  Why or why not?  (10~points)

\item {\it Employment report.\/}
At 8:30 am on April 6, 2006,
the US Bureau of Labor Statistics released
its closely-watched employment report, {\it The Employment Situation\/}.
Firms reported an increase of 180,000 jobs in March,
well above the consensus of 135,000.
Treasury yields immediately rose 5-10 basis points
for maturities from 2 to 30 years.
Why?
(10~points)

\item {\it ECB policy.\/}
The European Central Bank has kept short-term interest rates
in the Euro Zone well above those in the US.
Why?  (10~points)

\item {\it Cross-correlation function.\/}
Describe the cross-correlation function and show how it can
be used to identify promising leading indicators.
(10~points)
\end{parts}

\begin{solution}
\begin{parts}

\item   PPP suggests that exchange rates will adjust to make prices the same across countries. In
this case, that means the dollar will rise against the euro, fall against the yen.  Is this right?
Over short periods of time, exchange rates are close to unpredictable by any means, PPP included.
Over longer periods of time (5-20 years) PPP is a reasonable indicator.

\item It has lots of truth in it, but I'd disagree for two reasons.
First, it's an incomplete statement for high inflations:
it's true, but high money growth itself typically stems from
a government deficit.
Second, over short periods of time, the quantity theory doesn't work
that well.
It's entirely possible, as our AS/AD analysis implies, that
money can have only a modest short-run impact on inflation,
and that other demand and supply factors play a role, too.

\item  I'd start with the Taylor rule:  indicators of high output
lead to high interest rates.
The deeper question is why this shows up in long yields.
Certainly it will take some time to affect the Fed's choice of
target interest rate, but the impact on the very long end is a
typical, if somewhat mysterious, result.

\item One reason is that the ECB, by design,
places greater weight on inflation:  its primary goal is price stability.
Another is that they have not had the kind of financial turmoil
that has afflicted US markets and driven the Fed to
largely abandon its own devotion to stable prices.

\item The cross-correlation function is a plot of the correlation
of the correlation of two variables at different leads and lags
against the lead or lag.
Formally, the ccf for two variables $x$ and $y$ is a plot
of
\[
    \mbox{ccf}(k) \;=\;  \mbox{\it corr\/} (x_t,y_{t-k}) .
\]
against $k$.
\end{parts}
\end{solution}

\end{questions}


\input{../../LaTeX/footer.tex}

%**********************************************************************
%**********************************************************************
\newpage
\def\HeadName{Practice Final \#5}
\parindent = 0.0in
\parskip = \bigskipamount
\thispagestyle{empty}%
\Head

\centerline{\large \bf \HeadName}%
\centerline{December 2012}
%\centerline{Revised:  \today}

\bigskip
You have 120 minutes to complete this exam.  Please answer each
question in the space provided and show all of your work.
You may consult one page of notes and a calculator,
but devices capable of wireless transmission are prohibited.

I understand that the honor code applies: I will not lie, cheat,
or steal to gain an academic advantage, or tolerate those who do.

\begin{flushright}
\rule{4in}{0.5pt} \\ (Name and Signature)
\end{flushright}

%\pagebreak
\begin{questions}
% ======================================================================
\question {\it Risk and opportunity in Ghana.\/}
You have been asked to prepare a report on crisis risk in
the West African country of Ghana.
You recall from your Global Economy class that
Ghana is a former British colony that has been growing rapidly
in recent years after a period of unusually stable politics.
The Economist Intelligence Unit refers to it as a ``robust democracy.''
The World Economic Forum ranked Ghana 114th (of 133)
in their Global Competitiveness Report.
They continue:
``The country continues to display strong public institutions and
governance indicators,
particularly in regional comparison.''

The EIU's Country Risk Report adds:
\begin{itemize}
\item The December 2012 elections are expected to be close.
The president, John Atta Mills, came to power promising accountability
and transparency, but  has struggled to maintain party unity
while evidence emerges of financial impropriety of some government ministers.
\item The victor faces a challenging policy environment, particularly
the fiscal situation.
\item Expectations among the population are high as production
starts at the offshore Jubilee oil field.
\item The government's decision to allow use of 70\% of future
oil revenue as collateral for borrowing is a cause for concern
if the revenue is not managed properly.
\item The Bank of Ghana (the central bank) faces the twin goals of
containing inflation and fostering growth.
\item The currency --- the cedi --- floats with occasional heavy intervention.
\end{itemize}
%
Your mission is to assess the risks to Ghana  using
the information above, the data in Table \ref{tab:ghana},
and your own good judgement and analytical skills.

\begin{parts}
\item  You decide to start with a fiscal assessment.
What trend do you see in government revenues and expenses?
(5~points)

\item You notice that neither the primary deficit nor
interest expenses are reported separately.
How would you estimate them from the numbers in the table?
What are their values for 2011?
{\it Warning: this was harder than intended. If you get stuck,
just make up primary deficit numbers somehow.\/}
(10~points)

\item Using what you know about government debt dynamics,
compute the ratio of government debt to GDP for 2011.
What factors contribute the most to the change from 2010?
(10~points)

\item Overall, how would you assess the risks to Ghana's economy over
the next couple of years?
(10~points)
\end{parts}

\begin{table}
\centering
%\tabcolsep = 0.2in
\begin{tabular}{lrrrrrr}
\toprule
        & 2007 & 2008 & 2009 & 2010 & 2011 & 2012 \\
\midrule
GDP growth (\%) & 6.5 & 8.4 & 4.0 & 7.7 & 13.6 & 7.4 \\
Inflation (\%)  & 12.7& 18.1 & 16.0 & 8.6 & 8.6 & 8.5 \\
Interest rate (\%) & 14.5 & 20.8 & 28.8 & 22.7 & 20.5 & 20.6  \\
Govt revenue (\% of GDP)  & 17.5 & 16.0 & 16.5 & 19.1 & 23.4 & 22.2 \\
Govt spending (\% of GDP) & 23.1 & 24.5 & 22.3 & 25.5 & 27.6 & 27.7 \\
Govt budget balance (\% of GDP) & --5.6 & --8.5 & --5.8 & --6.5& --4.2 & --5.5\\
Govt debt (\% of GDP) & 30.4 & 30.6 & 33.3 & 33.9 & {\bf } & {\bf } \\ %& 36.8 & 41.6\\
Real exchange rate (index) & 85.2 & 81.7 & 76.3 & 81.8 & 78.1 & 74.5\\
FX reserves (USD billions) & 2.6 & 1.8 & 2.9 & 4.3 & 4.4 & 4.8 \\
\bottomrule
\end{tabular}
\caption{Macroeconomic data for Ghana.
Data from EIU CountryData.
The government budget balance is a surplus if positive, deficit if negative.
The real exchange rate is the price of goods in Ghana relative to the rest
of the world;
the larger the number, the more expensive goods are in Ghana.
The numbers for 2011 and 2012 are estimates.
}
\label{tab:ghana}
\end{table}


\begin{solution}
Answers follow.
See the spreadsheet for calculations
(download this pdf file, open it with the Adobe Reader or the equivalent,
and click on the pushpin):
\attachfile{exam_final_s12_answerkey.xlsx}

\begin{parts}
\item Trends include:
(i)~revenues and spending both rising,
(ii)~spending still ahead of revenue (there's a deficit),
and (as a direct result)
(iii)~ratio of debt to GDP rising a little
(more on that to come).

\item {\it Comment:  this was more difficult than intended.}
Remember that interest payments in year $t$ are $ i_t B_{t-1}$.
Expressed as a ratio to GDP we have $ i_t B_{t-1}/Y_t$.
This is a little involved, but we can get what we want from
\begin{eqnarray*}
    i_t B_{t-1}/ Y_{t} &=& i_t (B_{t-1}/ Y_{t-1}) (Y_{t-1}/ Y_{t})
            \;\;=\;\; i_t (B_{t-1}/Y_{t-1}) /(1+g_t + \pi_t) .
\end{eqnarray*}
That gives us interest payments in 2011 of 5.7\% of GDP and
a primary deficit of $-0.2$\% (that is, a surplus).

\item The key relation is this one:
\begin{eqnarray*}
    \Delta ({B_{t}}/{Y_{t}})
            &=&
                (i_t-\pi_t) ({B_{t-1}}/{Y_{t-1}})
                - g_t ({B_{t-1}}/{Y_{t-1}})
             +    ({D_{t}}/{Y_{t}})  .
\end{eqnarray*}
We refer to the components on the right as A, B, and C.
Calculations in the spreadsheet give us

\medskip
\begin{center}
\begin{tabular}{lrrr}
\toprule
        &  2010 & 2011 & 2012  \\
\midrule
Interest payments  &  &  4.0 & 3.9 \\
Component A (interest)  &  &  4.0 & 3.9 \\
Component B (growth)            &   & --4.6 &--2.4 \\
Component C (primary deficit)   &   & --1.5 & 1.4 \\
Total change in $B/Y$       &       &--2.1 & 2.4 \\
Public debt (\% of GDP)     &  33.9 & 31.8 & 33.2 \\
\bottomrule
\end{tabular}
\end{center}
\medskip

Over this period, the ratio of debt to GDP fell by 0.7\%.
The components contributed:
interest +7.9, growth --7.0, and the primary deficit --1.7.

One thing you might note is that impact of a very high interest rate
and similarly high real GDP growth.

\item This is a call to look at the checklist:
\begin{itemize}
\item Government debt and deficits:  we have deficits, but there's not
much sign yet of a growth debt to GDP ratio.
One future concern might be the possibility of borrowing now against future oil
revenue.  Will any debts incurred be spent wisely?
Will the oil revenue show up?
\item Banking system.  No information on that provided.
\item Exchange rate and reserves.  Reserves are modest,
but with the exchange rate floating there shouldn't be much
concern about that.
\item Politics.  Always an issue,
especially with a contentious election coming
and the promise of money from oil revenue.
It's an odd fact but a true one that revenue
from natural resources is more likely to cause problems than solve them.
\end{itemize}

\end{parts}
\end{solution}


%\pagebreak \phantom{xx} \pagebreak \phantom{xx} \pagebreak
% ======================================================================
\question {\it Aggregate implications of employer-provided health insurance.\/}
By an accident of history, health insurance in the US is generally
provided by employers.
Suppose a sharp rise in healthcare costs leads firms to hire fewer workers.
\begin{parts}
\item How would you represent this in an aggregate supply and demand diagram?
Which curve shifts?  In which direction?
(10~points)
\item What is the new short-run equilibrium?  Long-run equilibrium?
What happens to inflation and output?
(10~points)
\item How should the central bank respond?
Be specific about its goals and how it would accomplish them.
(10~points)
\end{parts}


\begin{solution}
\begin{parts}
\item Since we're talking about firms and production,
this must involve the supply side of the model.
We shift AS and AS$^*$ to the left, both by the same amount.
See the figure below.

\begin{center}
\setlength{\unitlength}{0.075em}
\begin{picture}(250,200)(0,0)
%\footnotesize
\thicklines

% horizontal axis
\put(-30,0){\vector(1,0){300}}
\put(255,-16){$Y$}
\put(142,-16){$Y^*$}
\put(102,-16){$Y^{*\prime}$}

% vertical axis
\put(0,-20){\vector(0,1){200}}
\put(-15,155){$P$}

% demand
\put(25,165){\line(4,-3){200}}\put(230,10){AD}
%\put(65,165){\line(4,-3){200}}\put(270,10){AD$'$}

% supply
\put(65,13){\line(4,3){200}} \put(270,160){AS}
\put(25,13){\line(4,3){200}} \put(230,160){AS$'$}
\put(146.4,0){\line(0,1){170}} \put(138,175){AS$^*$}
\put(106.4,0){\line(0,1){170}} \put(98,175){AS$^{*\prime}$}

% equilibrium labels
\put(150,55){\footnotesize A}
\put(122,75){\footnotesize B}
\put(95,94){\footnotesize C}
\put(95,76){\footnotesize D}
%\put(95,54){\footnotesize D}
% dotted lines
%\qbezier[31]{(133,0)(133,46)(133,92)}
%\qbezier[45]{(0,92)(67,92)(133,92)}
%\qbezier[45]{(0,72)(67,72)(133,72)}

\end{picture}
\end{center}
\bigskip\bigskip

Grading:  10 points for noting that it's a supply shift
and reproducing the figure above.

\item We started at A.
After the shift, we move to a new short-run equilibrium at B,
where the new AS crosses AD.
Evidently output falls and prices rise.

Eventually we move to a new long-run equilibrium at C,
where AD crosses the new AS$^*$.
At this point, output has fallen more and prices have risen more.

Grading:  7 points for telling us which curves determine
 short-run and long-run equilibria and 3 points for reporting
 the implications for output and prices.

\item The central bank has two goals:  stable prices and output
at its long-run equilibrium.
Here we've moved from A to C.
We're ok at C on the second goal:  output fell,
but that's the long-run equilibrium so there's nothing monetary policy
can do about that.
(We could consider other policies, but they're not the job of the central bank.)

Where C is bad is with respect to price stability:  prices are higher.
So the central bank could shift AD to the left, giving us the same
long-run output but lower prices.
The central bank would accomplish this by reducing the money supply,
which it might do by targeting a higher interest rate.

\end{parts}
\end{solution}


%\pagebreak \phantom{xx} \pagebreak %\phantom{xx} \pagebreak
% ======================================================================
\question {\it True/false/uncertain.\/}
Please explain why each statement is true, false, or uncertain.
The explanation is essential.
%
\begin{parts}
\item The New York Times reports that Apple paid a worldwide corporate
tax rate of about 10\%, while Walmart paid about 24\%.
This difference in tax rates is good for the global economy,
because Apple is part of the high-growth technology sector.
(10~points)

\item The unemployment rate is a leading countercyclical indicator
of economic growth.
(10~points)

\item If a central bank is purchasing foreign currency,
sterilization would consist of selling bonds.
(10~points)

\item Purchases of household furniture are more cyclical
than purchases of toothpaste.
(10~points)

\item If year-on-year GDP growth rises 1 percent, you
would expect the fed funds rate to go up by roughly 1.5 percent.
(10~points)

\item Since 1983, Hong Kong has run a hard peg, with
the Hong Kong Dollar trading between 7.75 and 7.85 per US dollar.
As a result, Hong Kong inherits US monetary policy.
(10~points)
\end{parts}

\begin{solution}
\begin{parts}
\item False.  A sound principle of tax policy is to apply the same rate
to (in this case) both firms:  this is what maximizes the surplus,
minimizes the cost of distortions.
And in this case, it's arguable that Walmart has done more for aggregate
productivity than Apple.

\item True.  Furniture is a durable good, hence likely to be more cyclical
than a nondurable good like toothpaste.

\item Half true.  The unemployment rate is countercyclical,
but it's a lagging indicator.

\item True.  Here's how that might look.
The central bank starts out with the balance sheet

\begin{center}
\begin{tabular}{lr|lr}
Assets  &&  Liabilities \\
\midrule
Bonds &  10    & Money & 20 \\
FX reserves & 10
\end{tabular}
\end{center}

If it purchases foreign currency (say 5 worth) with domestic currency,
that changes to

\begin{center}
\begin{tabular}{lr|lr}
Assets  &&  Liabilities \\
\midrule
Bonds &  10    & Money & 25 \\
FX reserves & 15
\end{tabular}
\end{center}

So the money supply has gone up.
To undo that, they would sell 5 worth of bonds, taking back money in return:

\begin{center}
\begin{tabular}{lr|lr}
Assets  &&  Liabilities \\
\midrule
Bonds &  5    & Money & 20 \\
FX reserves & 15
\end{tabular}
\end{center}

\item False.
It's an allusion to the Taylor rule,
which says (in the form we used it in class) that an increase in GDP growth
of 1\% would lead to an increase in the fed funds rate of 0.5\%.

\item True.
It's a reference to the trilemma, where you get to choose no more than two of:
(i) fixed exchange rate, (ii)~free flows of capital, and (iii)~independent monetary policy. Hong Kong has chosen (i) and (ii), given up on (iii).

\end{parts}
\end{solution}

\end{questions}

\input{../../LaTeX/footer.tex}

\end{document}




