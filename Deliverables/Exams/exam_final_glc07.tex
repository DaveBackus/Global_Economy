\documentclass[letterpaper,12pt]{exam}

\usepackage{amsmath}
\usepackage{natbib}
\usepackage[dvips]{graphicx}

\newcommand{\GDP}{\mbox{\em GDP\/}}
\newcommand{\CPI}{\mbox{\em CPI\/}}
\newcommand{\GNP}{\mbox{\em GNP\/}}
\newcommand{\NX}{\mbox{\em NX\/}}
\newcommand{\NY}{\mbox{\em NY\/}}
\newcommand{\CA}{\mbox{\em CA\/}}
\newcommand{\NFA}{\mbox{\em NFA\/}}
\newcommand{\Def}{\mbox{\em Def\/}}
\newcommand{\NIPA}{\mbox{\em NIPA\/}}
\newcommand{\TFP}{\mbox{\em TFP\/}}

\addtolength{\oddsidemargin}{-15pt} \evensidemargin\oddsidemargin
\addtolength{\topmargin}{-25pt} \addtolength{\textheight}{250pt}
\addtolength{\textwidth}{45pt}

\newtheorem{theorem}{Theorem}
\newtheorem{acknowledgement}[theorem]{Acknowledgement}
\newtheorem{algorithm}[theorem]{Algorithm}
\newtheorem{axiom}[theorem]{Axiom}
\newtheorem{case}[theorem]{Case}
\newtheorem{claim}[theorem]{Claim}
\newtheorem{conclusion}[theorem]{Conclusion}
\newtheorem{condition}[theorem]{Condition}
\newtheorem{conjecture}[theorem]{Conjecture}
\newtheorem{corollary}[theorem]{Corollary}
\newtheorem{criterion}[theorem]{Criterion}
\newtheorem{definition}[theorem]{Definition}
\newtheorem{example}{Example}[section]
\newtheorem{exercise}[theorem]{Exercise}
\newtheorem{lemma}[theorem]{Lemma}
\newtheorem{notation}[theorem]{Notation}
\newtheorem{problem}[theorem]{Problem}
\newtheorem{proposition}{Proposition}
\newtheorem{remark}[theorem]{Remark}
%\newtheorem{solution}[theorem]{Solution}
\newtheorem{summary}[theorem]{Summary}
\newenvironment{proof}[1][Proof]{\textbf{#1.} }{\ \rule{0.5em}{0.5em}}
%\citestyle{chicago}

\RequirePackage{GE05}
% this inputs graphicx, too

\def\ClassName{The Global Economy}
\def\Category{Gian Luca Clementi}
\def\HeadName{Practice Final}

\begin{document}

% Headline

\thispagestyle{empty}%
\Head

\vspace{15pt}

\centerline{\Large \bf Final Exam}%
\vspace{12pt}%
%\centerline{\large \bf Thomas Cooley and Gian Luca Clementi}


\vspace{.3cm}

\setlength{\baselineskip}{15pt}

%%%%%%%%%%%%%%%%%%%%%%%%%%%%%%%%%%%%%%%%%%%%%%%%%%%%%%%%%%%%%%%%%%%%%%%%%%%%%
%%%%%%%%%%%%%%%%%%%%%%%%%%%%%%%%%%%%%%%%%%%%%%%%%%%%%%%%%%%%%%%%%%%%%%%%%%%%%


%%%%%%%%%%%%%%%%%%%%%%%%%%%%%%%%%%%%%%%%%%%%%%%%%%%%%%%%%%%%%%%%%%%%%%%%%%%%%
%%%%%%%%%%%%%%%%%%%%%%%%%%%%%%%%%%%%%%%%%%%%%%%%%%%%%%%%%%%%%%%%%%%%%%%%%%%%%

\noindent Name \vspace{0.4cm} \noindent\rule{5.5in}{0.1mm} %

\noindent \textbf{Instructions.} You have one hour and fifteen
minutes to complete this exam. Please read the whole text carefully
and then write legibly in the allotted space. I wish you good luck.
I hope you will do well.\vspace{.2in}

\begin{questions}
\question \textbf{Spain.} In recent years, with its larger neighbors
struggling, Spain's economy grew at remarkable rates. Economists and
finance analysts have been investigating the causes of this outcome,
with the main purpose of understanding whether we should expect
similar achievements in the near future. Here are some recent
macroeconomic data about Spain.

\begin{table}[h]
%\caption{Long--Horizon Regressions of Equity Returns and Dividend Growth} \label{tab:long_hor_equity_div}%
\vspace{1em}%
\centering%
\hspace{-3cm}%
\begin{minipage}
{0.52\textwidth}%
\begin{center}{\small
\begin{tabular}{lccccccc}%
\vspace{-0.6cm}\\
%\multicolumn{2}{c}{Seedorf Bank} \\%
\hline%
\vspace{-.3cm}\\
                     & 2000  &  2001  &  2002   & 2003  & 2004 &  2005  &  2006 \\%
\vspace{-.3cm}\\
\hline%
\vspace{-.2cm}\\
GDP Growth           &  5.0   &   3.6  &   2.7   &  3.0  &  3.3  &   3.5  &  3.9 \\%
Population Growth    &  0.6   &   1.1  &    1.2  &  2.0  &  2.2  &   2.3  &  2.2 \\
Investment Rate      &  25.9  &   26.2 &   26.2  &  27.1 &  27.4 &   28.4 &  29.1\\%
Saving Rate          &  22.3  &   22.4 &   23.4  &  23.9 &  23.0 &   22.2 &  21.8\\%
TFP Growth           &  --0.1 &  --0.4 &   --0.4 &  --0.7&  --0.5&   --1.1&  --0.2\\%
Budget Balance       &  --0.9 &  --0.5 &   --0.3 &  --0.1&  --0.2&    1.1 &  1.4 \\%
Public Debt          &  59.2  &  55.6  &   52.5  &  48.8 &  46.2 &   43.1 &  39.8\\%
Current Account      &  --4.0 &  --3.9 &   --3.2 &  --3.5&  --5.3&   --7.4&  --8.8\\%
Foreign Debt         &  74.9  &  77.1  &   92.4  &  97.2 &  90.1 &   86.3 &  81.3 \\%
\hline%
\vspace{-3mm}\\
\end{tabular}
}
\end{center}
\end{minipage}
\end{table}

[Note: {\it GDP Growth} is yearly growth in GDP in constant Euros;
the {\it Investment Rate} is the ratio of gross domestic investment
to GDP; the {\it Saving Rate} is national savings divided by GDP,
where national savings are the sum of government, household, and
corporate savings; the {\it Budget Balance} is net Government'
savings divided by GDP; the {\it Public Debt} is the ratio of
Government Debt to GDP; the {\it Current Account} is the actual
current account divided by GDP; similarly, for the country's {\it
Foreign Debt}.]

\begin{parts}

\part In spite of a continuing drop in TFP, Spain's GDP and per capita GDP have been
growing throughout the period. Which, according to you, are the most
likely causes of such growth? \underline{Why}? (10 points)

%{\bf Answer.}

\vspace{0.4cm} \noindent\rule{5.5in}{0.1mm}%

\vspace{0.4cm} \noindent\rule{5.5in}{0.1mm}%

\vspace{0.4cm} \noindent\rule{5.5in}{0.1mm}%

\vspace{0.4cm} \noindent\rule{5.5in}{0.1mm}%

\vspace{0.4cm} \noindent\rule{5.5in}{0.1mm}%

\vspace{0.4cm} \noindent\rule{5.5in}{0.1mm}%

\part What is the outlook for the Spanish economy in case TFP growth keeps on being negative in the
foreseeable future? \underline{Why}? (10 points)

%{\bf Answer.}

\vspace{0.4cm} \noindent\rule{5.5in}{0.1mm}%

\vspace{0.4cm} \noindent\rule{5.5in}{0.1mm}%

\vspace{0.4cm} \noindent\rule{5.5in}{0.1mm}%

\vspace{0.4cm} \noindent\rule{5.5in}{0.1mm}%

\vspace{0.4cm} \noindent\rule{5.5in}{0.1mm}%

\vspace{0.4cm} \noindent\rule{5.5in}{0.1mm}%

\part Spain's current account deficit is now among the largest in the entire world. Since
it mostly reflects a trade balance deficit, many analysts have
argued that it is likely due to lack of competitiveness of Spanish
products on international markets. The same analysts conjecture that
this is an early sign of the decline that is about to come. Do you
agree with this view? \underline{Why}? (10 points)

%{\bf Answer.}

\vspace{0.4cm} \noindent\rule{5.5in}{0.1mm}%

\vspace{0.4cm} \noindent\rule{5.5in}{0.1mm}%

\vspace{0.4cm} \noindent\rule{5.5in}{0.1mm}%

\vspace{0.4cm} \noindent\rule{5.5in}{0.1mm}%

\vspace{0.4cm} \noindent\rule{5.5in}{0.1mm}%

\vspace{0.4cm} \noindent\rule{5.5in}{0.1mm}%

\vspace{0.4cm} \noindent\rule{5.5in}{0.1mm}%

\vspace{0.4cm} \noindent\rule{5.5in}{0.1mm}%


\end{parts}

\question \textbf{Hyperinflation in Zimbabwe.} Please, read the
enclosed article entitled ``Bags of bricks," which appeared on the
print edition of ``The Economist" on August 24, 2006.

\begin{parts}

\part Do you expect that, by exchanging 1,000 Zimbabwean dollars
for 1 new Zimbabwean dollar, the Central Back had (or will have) any
impact on inflation? \underline{Why}? Will the same policy have any
positive impact whatsoever on Zimbabwe's economy? \underline{Why}?
If yes, how long--lasting you think will it be? (10 points)

%{\bf Answer.}

\vspace{0.4cm} \noindent\rule{5.5in}{0.1mm}%

\vspace{0.4cm} \noindent\rule{5.5in}{0.1mm}%

\vspace{0.4cm} \noindent\rule{5.5in}{0.1mm}%

\vspace{0.4cm} \noindent\rule{5.5in}{0.1mm}%

\vspace{0.4cm} \noindent\rule{5.5in}{0.1mm}%

\vspace{0.4cm} \noindent\rule{5.5in}{0.1mm}%

\part The article mentions the project of creating a monetary union in Africa by 2018,
modeled on the Euro--zone. Please, provide one reason for and one
reason against this proposal. (10 points)

%{\bf Answer.}

\vspace{0.4cm} \noindent\rule{5.5in}{0.1mm}%

\vspace{0.4cm} \noindent\rule{5.5in}{0.1mm}%

\vspace{0.4cm} \noindent\rule{5.5in}{0.1mm}%

\vspace{0.4cm} \noindent\rule{5.5in}{0.1mm}%

\vspace{0.4cm} \noindent\rule{5.5in}{0.1mm}%

\vspace{0.4cm} \noindent\rule{5.5in}{0.1mm}%


\part Assume that Zimbabwe's government decides to shut down the
Central Bank and adopt the US dollar as official currency. What do
you expect the effects on inflation and on seignorage revenues to
be? \underline{Why}? (10 points)

%{\bf Answer.}

\vspace{0.4cm} \noindent\rule{5.5in}{0.1mm}%

\vspace{0.4cm} \noindent\rule{5.5in}{0.1mm}%

\vspace{0.4cm} \noindent\rule{5.5in}{0.1mm}%

\vspace{0.4cm} \noindent\rule{5.5in}{0.1mm}%

\vspace{0.4cm} \noindent\rule{5.5in}{0.1mm}%


\end{parts}


\question \textbf{The Amateur Economist.}

It is quite common, in particular in New York, to end up at dinner
with someone that believes to master economics and finance. To have
figured it all out. Here are statements that may have been uttered
in front of you by one of these self--proclaimed experts. For each
one of them, please explain why you agree or disagree.

\begin{parts}

\part Most technological progress is labor--saving. We now produce one unit of every given product using a small
fraction of the man--hours that were needed only a few years ago. It
is obvious that such innovations increase the share of aggregate
income accruing to capital, at the expense of the quota accruing to
labor. (10 points)

%{\bf Answer.}

\vspace{0.4cm} \noindent\rule{5.5in}{0.1mm}%

\vspace{0.4cm} \noindent\rule{5.5in}{0.1mm}%

\vspace{0.4cm} \noindent\rule{5.5in}{0.1mm}%

\vspace{0.4cm} \noindent\rule{5.5in}{0.1mm}%

\vspace{0.4cm} \noindent\rule{5.5in}{0.1mm}%

\part For small, developing countries, liberalizing capital flows with the rest of the
world is always a detrimental policy. If the economy grows at a
healthy rate, foreigners will go in and buy all profitable assets
for very cheap, therefore ending up expropriating locals for
generations to come. If, on the other hand, the economy is afflicted
by structural problems that severely limit its growth, we will
witness a capital flight that will dramatically impoverish the
country. (10 points)

%{\bf Answer.}

\vspace{0.4cm} \noindent\rule{5.5in}{0.1mm}%

\vspace{0.4cm} \noindent\rule{5.5in}{0.1mm}%

\vspace{0.4cm} \noindent\rule{5.5in}{0.1mm}%

\vspace{0.4cm} \noindent\rule{5.5in}{0.1mm}%

\vspace{0.4cm} \noindent\rule{5.5in}{0.1mm}%

\vspace{0.4cm} \noindent\rule{5.5in}{0.1mm}%

\part While most people understand the perils of monopoly in a
variety of markets, very few appreciate how dangerous it is to
entrust the Fed alone with the process of money creation. All
commercial banks should be authorized to create money. (10 points)

%{\bf Answer.}

\vspace{0.4cm} \noindent\rule{5.5in}{0.1mm}%

\vspace{0.4cm} \noindent\rule{5.5in}{0.1mm}%

\vspace{0.4cm} \noindent\rule{5.5in}{0.1mm}%

\vspace{0.4cm} \noindent\rule{5.5in}{0.1mm}%

\vspace{0.4cm} \noindent\rule{5.5in}{0.1mm}%

\vspace{0.4cm} \noindent\rule{5.5in}{0.1mm}%

\part There is at least one reason why eliminating Government debt
would be detrimental to the nation: the Fed would no longer be able
to conduct open market operations. (10 points)

%{\bf Answer.}

\vspace{0.4cm} \noindent\rule{5.5in}{0.1mm}%

\vspace{0.4cm} \noindent\rule{5.5in}{0.1mm}%

\vspace{0.4cm} \noindent\rule{5.5in}{0.1mm}%

\vspace{0.4cm} \noindent\rule{5.5in}{0.1mm}%

\vspace{0.4cm} \noindent\rule{5.5in}{0.1mm}%

\vspace{0.4cm} \noindent\rule{5.5in}{0.1mm}%


\end{parts}

\end{questions}


\vfill \centerline{\it \copyright \ \number\year \ NYU Stern School
of Business}

\end{document}
