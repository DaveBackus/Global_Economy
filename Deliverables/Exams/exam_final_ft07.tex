\documentclass[letterpaper,12pt]{article}

%\usepackage[hypertex]{hyperref}
\usepackage{comment}
\RequirePackage{GE05}
% this inputs graphicx, too

\newcommand{\GDP}{\mbox{\em GDP\/}}
\newcommand{\NDP}{\mbox{\em NDP\/}}
\newcommand{\GNP}{\mbox{\em GNP\/}}
\newcommand{\NX}{\mbox{\em NX\/}}
\newcommand{\NY}{\mbox{\em NY\/}}
\newcommand{\CA}{\mbox{\em CA\/}}
\newcommand{\NFA}{\mbox{\em NFA\/}}
\newcommand{\Def}{\mbox{\em Def\/}}
\newcommand{\CPI}{\mbox{\em CPI\/}}
\newcommand{\phm}{\phantom{--}}

\def\ClassName{The Global Economy}
\def\Category{Professor David Backus}
\def\HeadName{Final Exam}

\begin{document}
\parindent = 0.0in
\parskip = \bigskipamount
\thispagestyle{empty}%
\Head

\centerline{\large \bf \HeadName}%
\centerline{Revised:  \today}

\bigskip
You have 100 minutes to complete this exam.  Please answer each question in the space provided.
You may consult one page of notes and a calculator, but devices capable of wireless transmission
are prohibited.

I understand that the honor code applies: I will not lie, cheat, or
steal to gain an academic advantage, or tolerate those who do.

\begin{flushright}
\rule{4in}{0.5pt} \\ (Name, section, and signature)
\end{flushright}


%\bigskip
\begin{enumerate}

\item {\it Prospects for Spain (35 points).\/} 
While much of Europe struggles, Spain has been a continuing economic 
success, with growth rates well above the EU average.  
As an analyst in Deutsche Bank's London office, 
you wonder whether this trend is likely to continue.  
You look at the data:  

\begin{table}[h]
\vspace{1em}%
\centering%
\hspace{-3cm}%
\begin{minipage}
{0.52\textwidth}%
\begin{center}{\small
\begin{tabular}{lccccccc}%
\vspace{-0.6cm}\\
%\multicolumn{2}{c}{Seedorf Bank} \\%
\hline%
\vspace{-.3cm}\\
                     & 2000  &  2001  &  2002   & 2003  & 2004 &  2005  &  2006 \\%
\vspace{-.3cm}\\
\hline%
\vspace{-.2cm}\\
Real GDP Growth      &  5.0   &   3.6  &   2.7   &  3.0  &  3.3  &   3.5  &  3.9 \\%
Population Growth    &  0.6   &   1.1  &    1.2  &  2.0  &  2.2  &   2.3  &  2.2 \\
TFP Growth           &  --0.1 &  --0.4 &   --0.4 &  --0.7&  --0.5&   --1.1&  --0.2\\%
Interest rate (3m)   &  4.1 &  4.4  &  3.3  &  2.3  & 2.0   &  2.1  &  2.8 \\
Inflation            &  3.4 &  3.6  &  3.5  &  3.0  &  3.0  &  3.4  &   3.5 \\
Investment*      &  25.9  &   26.2 &   26.2  &  27.1 &  27.4 &   28.4 &  29.1\\%
Saving*          &  22.3  &   22.4 &   23.4  &  23.9 &  23.0 &   22.2 &  21.8\\%
Budget Balance*       &  --0.9 &  --0.5 &   --0.3 &  --0.1&  --0.2&    1.1 &  1.4 \\%
Primary Balance*      &  2.1 &    2.2 &   2.1  &  2.0 &  1.7  &  2.7 &  2.8  \\%
Public Debt*          &  59.2  &  55.6  &   52.5  &  48.8 &  46.2 &   43.1 &  39.8\\%
Current Account*      &  --4.0 &  --3.9 &   --3.2 &  --3.5&  --5.3&   --7.4&  --8.8\\%
Net Foreign Assets*   & --74.9  & --77.1  & --92.4  & --97.2 & --90.1 & --86.3 &  --81.3 \\%
\hline%
\vspace{-3mm}\\
\end{tabular}
}
\end{center}
\end{minipage}
\end{table}

Note: Variables marked by * are ratios to GDP (nominal to nominal).  

\begin{enumerate}

\item What do you see as the likely source(s) of GDP growth 
over the recent past?  
(5~points)

\item Does the budget balance concern you?  
Why or why not?  (10~points)

\item The current account deficit is currently one of the largest 
among developed countries.
Does that concern you?  Be as specific as possible in your answer. 
(10~points)

\item What is your overall assessment of the prospects for Spain
for the next 2-3 years?  
Please mention any issues that you think might call for a closer look.  
(10~points)

\end{enumerate}

%\begin{comment}
Answer.
\begin{enumerate}
\item Lack of TFP growth is a concern.  
Growth must be coming from increases in $K$ and/or $L$.  
\item The government is running a surplus, with the result that 
the debt-to-GDP ratio has fallen dramatically.  
You could formalize this, but the answer is clear.  
\item This is worth a closer look:  the deficit and net foreign assets
are both large.  
Are they getting worse?  
Consider the relation, 
\[
    \frac{\mbox{\em NFA}_{t+1}}{Y_{t+1}} \;=\; \left( \frac{1+i}{1+g} \right)
     \frac{\mbox{\em NFA}_{t}}{Y_{t}} + (1+g)^{-1} \frac{\mbox{\em NX}_{t}}{Y_{t}} .
\]
We don't have net exports, but we can try the closely related current account.
With $ i = 2.8\%$ and $ g = 2.9 + 3.5 = 7.4\%$, 
the numbers tell us that next year's ratio of net foreign 
assets to GDP will fall to about:
\[
    - (1.028/1.074) \times 81.3 - 8.8/1.074 \;=\; -86.0 .
\]
In short, foreign borrowing is going up. 
(If we had NX data, the change would probably be less, because 
the current account includes interest payments.) 
The question is why. 
A deeper analysis would look into the kinds of borrowing 
(equity or debt?  firms or housing?) and its origins.  

\item Personally, I think they're good.  
However, lack of TFP growth is an obvious concern.  
Ditto the current account deficit.
Either of these could reflect good or bad news, 
depending on the source. 

\end{enumerate}
%\end{comment}


%\pagebreak \phantom{bla} \pagebreak %\phantom{bla} \pagebreak
% ---------------------------------------------------------
\item {\it Globalization and inflation (20 points).\/}
Fed Chairman Bernanke said recently (March 2007):  
\begin{quote}
As national markets become increasingly integrated and open, sellers of goods, services, and labor may face more competition and have less market power than in the past. 
These linkages suggest that, at least in the short run, globalization and trade may affect the course of domestic inflation.
\end{quote}
%
\begin{enumerate}
\item Use aggregate supply and demand to 
describe how expansionary monetary policy 
affects output and inflation in the short run.  
(10~points)

\item Back to Bernanke:  How would you represent the impact of 
globalization (think:  imports from China) 
in the aggregate supply and demand diagram?  
How does globalization change the impact of expansionary monetary 
policy in this model?
Do you find the model persuasive in this respect?  
(10~points)
\end{enumerate}


\begin{comment}
Answer.
\begin{enumerate}
\item This would increase both output and prices (inflation, 
loosely speaking).  Your answer should show a diagram with 
AS, AD, and AS$^*$.
AD shifts out, with the stated result.  

\item The typical argument is that AS has become flatter.
As a result, expansionary monetary policy has a 
smaller impact on inflation, and larger impact on output. 

\end{enumerate}
\end{comment}


%\pagebreak \phantom{bla} \pagebreak %\phantom{bla} \pagebreak
% ---------------------------------------------------------
\item {\it Miscellany (45 points).}
%
\begin{enumerate} 

\item {\it Exchange rates.\/}
In the most recent data, a Big Mac costs \$3.22 in the US but only
\$2.31 in Japan.
Short-term interest rates are 5.23\% in the US, 0.57\% in Japan.
What do each of these comparisons suggest for the future value of the 
yen-dollar exchange rate?  
(15~points) 

Answer.  Big Mac:  PPP suggests that the yen will appreciate, 
but we know that this tendency only shows up over long periods of time.  
Interest differential:  on average, currencies with low interest rates
fall in value in the short run.
That's what fuels the infamous carry trade:  borrow in yen, 
lend in dollars, collect the interest differential, 
and hope the yen doesn't rise enough to wipe out your gains.  

\item {\it Inflation.\/}
Milton Friedman once said:  inflation is always and everywhere a monetary phenomenon. 
Do you agree or disagree?  Why or why not?  (15~points)

Answer.  It has lots of truth in it, but I'd disagree for two reasons. 
First, it's an incomplete statement for high inflations: 
it's true, but high money growth itself typically stems from 
a government deficit.
Second, over short periods of time, the quantity theory doesn't work 
that well.  
It's entirely possible, as our AS/AD analysis implies, that 
money can have only a modest short-run impact on inflation, 
and that other demand and supply factors play a role, too.  

\item {\it Employment report.\/} 
At 8:30 am on April 6, the US Bureau of Labor Statistics released
its closely-watched employment report, {\it The Employment Situation\/}.
Firms reported an increase of 180,000 jobs in March, 
well above the consensus of 135,000.  
Treasury yields immediately rose 5-10 basis points 
for maturities from 2 to 30 years.  
Why?  
(15~points)

Answer.  I'd start with the Taylor rule:  indicators of high output 
lead to high interest rates.
The deeper question is why this shows up in long yields.
Certainly it will take some time to affect the Fed's choice of 
target interest rate, but the impact on the very long end is a 
typical, if somewhat mysterious, result.  

\end{enumerate}

\end{enumerate}

%\pagebreak \phantom{bla} %\pagebreak \phantom{bla} %\pagebreak

\vfill \centerline{\it \copyright \ \number\year \ 
NYU Stern School of Business}


\end{document}

