\documentclass[letterpaper,12pt]{exam}

\usepackage{ge13}
\usepackage{comment}
\usepackage{booktabs}
\usepackage{eurosym}
\usepackage{hyperref}
\urlstyle{rm}   % change fonts for url's (from Chad Jones)
\hypersetup{
    colorlinks=true,        % kills boxes
    allcolors=blue,
    pdfsubject={NYU Stern course GB 2303, Global Economy},
    pdfauthor={Dave Backus @ NYU},
    pdfstartview={FitH},
    pdfpagemode={UseNone},
%    pdfnewwindow=true,      % links in new window
%    linkcolor=blue,         % color of internal links
%    citecolor=blue,         % color of links to bibliography
%    filecolor=blue,         % color of file links
%    urlcolor=blue           % color of external links
% see:  http://www.tug.org/applications/hyperref/manual.html
}

% list spacing
\usepackage{enumitem}
\setitemize{leftmargin=*, topsep=0pt}
\setenumerate{leftmargin=*, topsep=0pt}

\usepackage{attachfile}
    \attachfilesetup{color=0.5 0 0.5}

\usepackage{needspace}
% example:  \needspace{4\baselineskip} makes sure we have four lines available before pagebreak

\def\ClassName{The Global Economy}
\def\Category{David Backus}
\def\HeadName{Final Examination}

\printanswers

\begin{document}
\parindent = 0.0in
\parskip = \bigskipamount
\thispagestyle{empty}%
\Head

\centerline{\large \bf \HeadName}%
%\centerline{March 9, 2005}
\centerline{Revised:  \today}

\bigskip
{\it You have 120 minutes to complete this exam.  Please answer each
question in the space provided and show all of your work.
You may consult one page of notes and a calculator,
but devices capable of wireless transmission are prohibited.

I understand that the honor code applies: I will not lie, cheat,
or steal to gain an academic advantage, or tolerate those who do.}

\bigskip
\begin{flushright}
\rule{4in}{0.5pt} \\ (Name and Signature)
\end{flushright}


\begin{questions}
% ======================================================================
\begin{table}[h]
\centering
\tabcolsep=0.1in
\begin{tabular}{lrrrrr}
\toprule
                & 2010 & 2011 & 2012 & 2013 \\
\midrule
Official exchange rate (pesos per USD)  & 3.90 & 4.11 & 4.54 & 5.46  \\
Inflation (\%)              & 22.9 & 24.4 & 25.3 & 20.6 \\
Foreign currency reserves (USD billions) & 52.2 & 46.4 & 43.2 & 32.2 \\
Real GDP growth (\%)        & 9.2 & 8.9 & 1.9 & 5.2  \\
Govt revenue (\% of GDP)    & 24.3 & 23.6 & 25.4 & 27.3 \\
Govt spending (\% of GDP)   & 24.1 & 25.3 & 28.0 & 30.5  \\
Public sector surplus (\% of GDP) & 0.2 & --1.7 & --2.6 & --3.2 \\
Primary balance (\% of GDP) &  1.7 & 0.3 & --0.2 & --0.8   \\
Govt debt (yearend, \% of GDP)  & & & 44.8\\
Interest rate paid on debt (\%) & 4.0 & 5.5 & 6.7 & 6.5  \\
Money market interest rate (\%) & 9.1 & 10.0 & 9.8 & 12.7 \\
\bottomrule
\end{tabular}
\label{tab:portugal}
\caption{Economic indicators for Argentina.  Source:  EIU.}
\end{table}

\item  {\it Don't Cry for Me Argentina (40~points).\/}
Argentina is a seemingly endless source of entertainment to economists,
yet its economy has done well in the recent past.
GDP growth fell to 0.9\% in 2009, during the global financial crisis,
but averaged over 9\% the next two years.
Most analysts attribute this success to
favorable commodity prices and strong global demand for Argentina's commodity exports.

At the same time, the government of President Cristina Fernandez de Kirchner
continues to adopt policies that befuddle outside observers, including:
taking over private pension funds,
restricting imports and purchases of foreign currency,
attacking the press,
nationalizing the Spanish-owned oil company YPF,
imposing price controls on electricity, natural gas, and public transportation,
and subsidizing energy consumption.

The Economist Intelligence Unit reports:
\begin{itemize}
\item A US court case may eventually leave
Argentina with the unpalatable choice of repaying the ``holdouts'' (creditors that
did not participate in the 2005 or 2010 restructurings) in full --- something that it
has sworn never to do --- or falling into technical default to avoid repaying
current creditors in a US jurisdiction.

\item According to official data, consumer price inflation remains among the highest
in emerging markets, at 10.5\% in April 2013. However, the official data are
widely discredited, and we are now using estimates produced by PriceStats,
which estimates that inflation in 2012 was 25\%.

\item Double-digit inflation has generated real peso appreciation.
Foreign-exchange controls have failed to prevent an erosion of the reserves cushion,
heightening the risk of an eventual devaluation.

\item The Argentine peso floats in principle, but the central bank intervenes to limit
the peso's depreciation.
In addition, foreign currency transactions are subject to a variety of controls.
For the past couple of years, the government has been gradually tightening the `clamp,'
an unofficial policy of discouraging purchases of dollars.
As a result, the peso's official decline has been modest,
but the unofficial ``blue market'' price of the peso is considerably lower.

\item The (low) banking sector risk rating reflects weak economic activity, expansionary
monetary policies that contribute to credit risk, high risk of exchange-rate
and interest-rate volatility, and increased currency convertibility risk.

\item The ruling party fared badly in the October midterm election,
 leaving the president without enough support in Congress
 to change the constitution and run for re-election.
 Focus will now shift rapidly to the 2015 presidential race.
 The president remains alienated from almost all of the country's most influential groups,
including the unions, the media, the Catholic Church and the traditional
leaders of the Peronist party. In this context, risks to political stability will be
high. An additional risk to stability is the president's health.
\end{itemize}
%
The question is what happens next:  Could another crisis be on the way,
or has Argentina put its problematic past to rest?
Use the information provided, and your own experience and good judgement,
to assess the risks to the Argentina economy over the next 2-3 years.
%
\begin{parts}
\item By ``real appreciation'' we mean an increase in the price
of local goods relative to foreign goods ---
what is sometimes called a decline in the real exchange rate.
Use the numbers in the table to demonstrate (or disprove) real appreciation
of the peso.
(10~points)

\item Why do you think the central bank's foreign exchange reserves have declined?
(5~points)

\item How do you see government debt evolving?
Compute, in particular, the ratio of government debt to GDP at year-end 2013.
What factors contribute the most to the change in the ratio?
(10~points)

\item Overall, how would you rate the risk of a macroeconomic crisis in Argentina?
What are the biggest sources of concern?
(15~points)
\end{parts}

\begin{solution}
Answers follow.  
See the spreadsheet for calculations in (a) and (c) 
(download this pdf file, open it with the Adobe Reader or the equivalent,
and click on the pushpin):
\attachfile{exam_final_f13_answerkey.xlsx}

\begin{parts}
\item 
One way to think about this (not the only one) is with the real exchange rate 
$ \mbox{\em RER\/} = eP^*/P$, where $e$ is the exchange rate, $P$ is the price of Argentine goods,
and $P^*$ is the price of American goods.  
So how is the real exchange rate changing?  
Inflation is the rate of increase in $P$, so we see the price of Argentine goods
is going up rapidly, roughly 20\% a year.  
In contrast, $eP^*$ is going up less:
$P^*$ is roughly flat (1-2\% inflation in the US)
and $e$ is rising (if we compute its rate of change)
5\% in 2011 and 10\% in 2012.
Thus {\it RER\/} is rising, as Argentine goods get relatively more expensive.  

In words:  the combination of high inflation and 
more modest currency depreciation has made Argentine goods expensive. 

Grading:  5 points for the basic idea, 5 for a calculation that compares
inflation rates and the change in the exchange rate.  

\item Evidently people want dollars, not pesos, and the central bank supplies
them to maintain a relatively stable exchange rate.  
One possible reason:  Argentine prices are rising, 
and a substantial depreciation is one way to get that. 
That makes pesos less attractive, since you'd lose (relative to dollars)
if the peso falls in value.  

Grading:  5 points for something along these lines.  

\item The debt dynamics equation is
\begin{eqnarray*}
   \Delta (B_t/Y_t)  &=&  (i_t - \pi_t)(B_{t-1}/Y_{t-1})
                - g_t (B_{t-1}/Y_{t-1}) + D_t/Y_t .
\end{eqnarray*}
The three terms are
\begin{eqnarray*}
    (i_t - \pi_t)(B_{t-1}/Y_{t-1}) &=&  -6.3  \\
    - g_t (B_{t-1}/Y_{t-1})   &=&  2.3 \\
    D_t/Y_t  &=&  0.8 .
\end{eqnarray*}
Their total is --3.2, so the ratio of debt to GDP will fall to 41.6. 
Note for later the negative contribution of the real interest rate:
they're getting a very good deal on their debt, hard to believe that will 
continue.   

Grading:  3 points for each component done right, 1 more for summing them.

\item This is a call for the checklist:
\begin{itemize}
\item Debt and deficits.
(i)~The calculation shows the debt ratio is falling.
But the US court case could lead to a technical default, 
which isn't a good thing.  
And the negative real interest rate is unlikely to continue. 
If they paid a modest 2\% real rate on debt, the debt ratio would 
go up about 4\% this year.  

\item Banks.
The EIU suggests that banks could suffer from a weak economy.  

\item Exchange rates and reserves.  
They're losing reserves as they try to keep the peso 
from depreciating.  
Either the peso depreciates more or they continue to lose reserves.  

\item Politics.  Always an issue in Argentina.  
There's some uncertainty given the president's lame duck status and health.
On the other hand, a change could make things better.  
\end{itemize}
% 
The fiscal situation, including the court case, the exchange rate and reserve position, 
the banking system, and the political situation all shows signs of trouble.  
Overall, I'd say they'll probably muddle through, 
but there's a chance of serious trouble.

Grading:  10 points for the checklist and its components,
5 points for a sensible argument assessing the various risks.
\end{parts}
\end{solution}


%\pagebreak \phantom{x} \pagebreak
% ======================================================================
\item  {\it The supply and demand of Abenomics (30~points).\/}
Shinzo Abe was elected Prime Minister of Japan in December 2012
after two decades of slow growth and falling prices.
He pledged dramatic policy changes to revive the Japanese economy,
dubbed the ``three arrows'' of ``Abenomics.''
We consult the Economist Intelligence Unit for specifics:
%
\begin{itemize}
\item Fiscal stimulus.  A sizeable economic stimulus package was passed by parliament in
February 2013, and a smaller one in October.
This is expected to produce a budget deficit of 8\% in 2013.
\item Monetary stimulus. A plan to double Japan's
money supply within two years was implemented in April 2013 to help to achieve the Bank of Japan's
target of 2\%  inflation.
\item Structural reform.
This is less clearly articulated, but some observers hope for a range of micro-based reforms,
including loosening product-market regulations that reduce productivity,
tightening corporate requirements for funding pensions,
creating a more flexible labor market,
and reducing subsidies to an inefficient agricultural sector.
\end{itemize}
%
Your mission is to explore the impact of the three arrows using the aggregate supply and demand
framework.
\begin{parts}
\item Explain, for each ``arrow,'' whether it affects supply or demand.
Which way does each one shift the appropriate curve(s)?
(15~points)
\item Compare the short- and long-term impact on output of the three policies.
Which are likely to have the greatest impact in the short term?
In the long term?
(15~points)
\end{parts}

\begin{solution}

This continues to be topical.  
Here's a 
\href{http://www.bloomberg.com/news/2013-12-12/abe-pushes-biggest-farm-revamp-since-macarthur-broke-landlords.html}
{recent comment} about agricultural policy.  

\begin{parts}
\item We have:
\begin{itemize}
\item Fiscal stimulus. This shifts aggregate demand to the right.
\item Monetary stimulus. Same.
\item Structural reform. This shifts both aggregate supply curves to the right.  
\end{itemize}
Grading:  5 points for each bullet done correctly.
The graphs need not be included, but if they are they should be correct.  

\item Fiscal and monetary stimulus will raise output in the short run. 
They have no long-run impact on output.  

Structural reform, on the other hand, raises output both short-term and long-term.  
In this respect, it's likely the most important of the arrows.  
(Also, unfortunately, the one that's been executed least effectively.) 

Grading: 5 points for each one.  
Also:  the original question asks for the short and long-term impact, and doesn't
specify output, so answers might include comments about the impact on prices.  
\end{parts}
\end{solution}


%\pagebreak \phantom{x} \pagebreak
% ======================================================================
\needspace{2\baselineskip}
\item  {\it Short answers (40 points).\/}
\begin{parts}
\item  How sensitive to the business cycle would you expect demand for Rolex watches to be?
Why?
(10~points)

\item If US inflation jumped to 5\%, how would you expect
interest rates to respond?  Why?
(10~points)

\item  Given what you know about global economic conditions,
how would you expect the US dollar to perform over the next year versus the euro?  Why?
(10~points)

\item Consider the statement:  ``Tax deductions are good, because they save taxpayers money.''
Do you agree or disagree?  Why?
(10~points)
\end{parts}

\begin{solution}
\begin{parts}
\item We would expect it to be very cyclical for two reasons:
it's a durable good, and it's a luxury.
Both categories are very cyclical.  

Grading:  5 points each for durable and luxury. 

\item This calls for a rough-and-ready Taylor rule calculation:
\begin{eqnarray*}
    i &=& r^* + \pi + 0.5 (\pi - \pi^*) + 0.5 (g - g^*) .
\end{eqnarray*}
With any conceivable inputs, you'd see a sharp increase in the fed funds rate, 
and therefore in other interest rates.  

Grading:  5 points for noting that this calls for the Taylor rule,
5 for suggesting it calls for an increase in the interest rate.  

\item For most exchange rates, our best guess over periods less than 5 years is no change.

Grading:  10 points for something like this. 

\item One feature of a good tax system is that it applies a low tax rate to a broad
base.  
Given the overall level of spending, a tax deduction means tax rates must be higher
on other things, which violates this principle. 

Grading:  10 points for something like this. 
\end{parts}
\end{solution}

\end{questions}

%\pagebreak \phantom{xx} %\pagebreak \phantom{xx}

\vfill \centerline{\it \copyright \ \number\year \ NYU Stern
School of Business}

\end{document}

\begin{comment}
\begin{figure}[h]
\begin{center}
\setlength{\unitlength}{0.075em}
%\setlength{\unitlength}{0.1em}
\begin{picture}(280,200)(-20,0)
%\footnotesize
\thicklines

% horizontal axis
\put(-30,0){\vector(1,0){300}}
\put(255,-16){$Y$}

% vertical axis
\put(0,-20){\vector(0,1){200}}
\put(-15,155){$P$}

% demand
\put(25,165){\line(4,-3){200}}\put(230,10){AD}
%\put(65,165){\line(4,-3){200}}\put(270,10){AD}

% supply
\put(25,13){\line(4,3){200}} \put(230,160){AS}
%\put(65,13){\line(4,3){200}} \put(270,160){AS$'$}
\put(126.4,0){\line(0,1){170}} \put(118,175){AS$^*$}\put(122,-16){$Y^*$}

% equilibrium labels
%\put(105,85){\footnotesize B}
%\put(150,115){\footnotesize A}
%\put(138,64){\footnotesize C}
% dotted lines
%\qbezier[31]{(133,0)(133,46)(133,92)}
%\qbezier[45]{(0,92)(67,92)(133,92)}
%\qbezier[45]{(0,72)(67,72)(133,72)}

\end{picture}
\end{center}
\caption{Aggregate supply and demand diagram}
\label{fig:asad}
\end{figure}
\end{comment}
