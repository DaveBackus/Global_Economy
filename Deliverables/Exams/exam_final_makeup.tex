\documentclass[letterpaper,12pt]{exam}

%\usepackage[hypertex]{hyperref}
\usepackage{amsmath}
\usepackage{natbib}
\usepackage{booktabs}
%\usepackage[dvips]{graphicx}
\usepackage{comment}
\RequirePackage{GE05}
% this inputs graphicx, too

\newcommand{\GDP}{\mbox{\em GDP\/}}
\newcommand{\NDP}{\mbox{\em NDP\/}}
\newcommand{\GNP}{\mbox{\em GNP\/}}
\newcommand{\NX}{\mbox{\em NX\/}}
\newcommand{\NY}{\mbox{\em NY\/}}
\newcommand{\CA}{\mbox{\em CA\/}}
\newcommand{\NFA}{\mbox{\em NFA\/}}
\newcommand{\Def}{\mbox{\em Def\/}}
\newcommand{\CPI}{\mbox{\em CPI\/}}
\newcommand{\phm}{\phantom{--}}

\def\ClassName{The Global Economy}
\def\Category{Professor David Backus}
\def\HeadName{Makeup Final Examination}

\printanswers 

\begin{document}
\parindent = 0.0in
\parskip = \bigskipamount
\thispagestyle{empty}%
\Head

\centerline{\large \bf \HeadName}%
\centerline{Revised:  \today}

\bigskip
You have 120 minutes to complete this exam.  Please answer each question in the space provided.
You may consult one page of notes and a calculator, but devices capable of wireless transmission
are prohibited.

I understand that the honor code applies: I will not lie, cheat, or
steal to gain an academic advantage, or tolerate those who do.

\begin{flushright}
\rule{4in}{0.5pt} \\ (Name, section, and signature)
\end{flushright}

\begin{questions} 
\question {\it Is Spain one of the PIIGS? (40 points).\/} 

\begin{center}
\begin{tabular}{lrrrrrrr}
\toprule 
         &  2004  &  2005  &  2006   & 2007  & 2008 &  2009  &  2010 \\%
\midrule 
GDP growth  & 3.3 & 3.6 & 4.0 & 3.6 & 0.9 & --3.7 & --0.3 \\
Inflation   & 3.0 & 3.4 & 3.5 & 2.7 & 4.1 & --0.3 & 0.3  \\
%Population growth & 1.1 & 2.1 & 1.4 & 1.1 & 0.7 & 0.5 & 0.3 \\
TFP growth  & --0.5 & --1.0 & 0.0 & 0.2 & 0.3 & 1.2 & 2.5  \\ 
Investment rate & 28.0 & 29.4 & 30.6 & 30.7 & 28.8 & 24.4 & 24.0\\
Saving rate     & 23.0 & 22.1 & 22.0 & 21.0 & 19.6 & 19.8 & 19.2 \\
Current account & --5.2 & --7.4 & --9.0 & --10.0 & --9.6 & --5.1 
            & --4.7 \\
%Trade balance   &             
Budget balance  & --0.4 & 1.0 & 2.0 &  1.9 & --4.0 & --11.4 & --11.5 \\ 
Primary balance & 1.5 & 2.5 & 3.3 & 3.0 & --3.0 & --10.1 & --9.5 \\
Public debt      & 46.2 & 43.0 & 39.6 & 36.1 & 39.7 & 55.2 & \\
%Net foreign assets   &  \\%
Interest rate:  short & 2.1 & 2.2 & 3.1 & 4.3 & 4.6 & 1.2 & 0.7 \\
Interest rate:  long  & 4.1 & 3.4 & 3.8 & 4.3 & 4.4 & 4.0 & 4.3 \\
Real exchange rate & 99 & 100 & 102 & 104 & 107 & 106 & 106 \\
Reserves            &  20 & 17 & 19 & 19 & 20 & 28 \\
\bottomrule 
\end{tabular}
\end{center}
Economic indicators for Spain.  
(i)~Investment, saving, current account, 
(government) budget balance, primary, 
and public debt are expressed as 
percentages of GDP (ratio to GDP multiplied by 100).  
(ii)~The real exchange rate is a weighted average across trading partners;
high numbers indicate that Spanish goods are expensive relative to 
a weighted average of prices in other countries, with weights
tied to the amount of trade between Spain.  
(iii)~Foreign exchange reserves are expressed in billions of USD.  
(iv)~2010 numbers are estimates.

You have been asked to write a short report summarizing 
the economic prospects for Spain over the next 2-3 years.  
You understand that in the recent past, 
Spain's economy has grown rapidly 
even as many its neighbors struggled.  
However, the global financial crisis hit Spain hard, 
particularly its booming housing sector.  
Is this a temporary setback or something more serious?  

Having some experience with such situations, 
you check the Economist Intelligence Unit's Country Data, 
summarized above, 
and Country Risk Service, 
which reports:  
%
\begin{itemize}
\item Spain is part of the Euro Zone.  
\item Spain's net foreign asset position is roughly --80\% of GDP.  
The largest categories are portfolio investment
(primarily debt) in Spanish firms and loans to Spanish banks.  
%Very little of it is government debt.  
\item The government's budget is deeply in deficit
as a result of the crisis.  
Austerity is planned but uncertain.  
\item Spanish banks came through the crisis in good shape.  
\item The ruling Spanish Socialist Workers' Party (PSOE)
was elected to a second four-year term in 2008 
but faces popular disenchantment. 
\end{itemize}

With this information in hand, you start to sketch out your report:  
\begin{parts}
\part Describe Spain's fiscal situation.  
What is your estimate of the debt-to-GDP ratio 
at year-end 2010?  
How would your answer change if interest rates rose sharply, 
as they have in Greece?  
(10~points) 

\item Is the exchange rate a source of concern?  
Why or why not?  
(10~points) 

\part Spain's current account deficit is one of the largest
in the world.  
Do you see it as a source of concern?  Why or why not?  
(10~points)

\part Overall, what do you see as the ``red flags'' for the Spanish economy?
Are you optimistic about future growth?  
Why or why not?  
(10~points)
\end{parts}


\pagebreak \phantom{bla} \pagebreak %\phantom{bla} \pagebreak
% ---------------------------------------------------------
\question {\it Stimulus in China, 2009 (30 points).\/}
China responded to the global crisis by 
implementing a massive program of government spending 
on infrastructure. 
Your mission is to outline the argument for or against such a program
using the aggregate supply and demand (AS/AD) framework.  
%
\begin{parts}

\part Over the last year, output growth and inflation have both fallen in China.  Would you say this comes from a shift in supply or demand?  
    Illustrate your answer with the appropriate diagram.  
    (10~points)

\part Using the same (or similar) diagram, 
describe the impact on the economy of 
a large increase in government spending on infrastructure projects.  
What is the likely impact on output?  Inflation?  
(10~points) 

\part What are appropriate goals of policy, 
expressed in terms of aggregate supply and demand?  
Does the Chinese spending program move them closer to these goals?  
(10~points) 
\end{parts}

\begin{comment}
Answer.
\begin{enumerate}
\item Inflation is up sharply, output is flat to down. 

\item The combination in (a) suggests a shift up/left in supply.
Why?  Because output and inflation have moved in opposite directions.
Since supply shocks should be accommodated/reinforced, 
the ECB should raise the short-term interest rate.

Grading:  10 for noting shift in supply, 
5 for the comment about policy.  

\item The ECB's primary mission is stable prices, 
so you should see an increase in interest rates.
This could also be expressed in terms of a Taylor rule, 
possibly with a larger coefficient on inflation 
than output growth.  
\end{enumerate}
\end{comment}


\pagebreak \phantom{bla} \pagebreak %\phantom{bla} \pagebreak
% ---------------------------------------------------------
\question {\it Miscellany (30 points).}
%
\begin{parts} 
\part If you see housing starts rise, 
what does that suggest about economic growth in the near-term future?  
(10~points) 

\part If the inflation rate rises, what is the response of 
a central bank following a Taylor rule? 
Why?  
(10~points) 

\part If Hungary's central bank purchases local currency (forints)
from citizens who would would prefer to hold euros, 
what is the impact on Hungary's money supply?  
How might it offset this impact?  
(10~points)
\end{parts}

\begin{comment}
Answers.
\begin{enumerate}
\item Foreign exchange intervention works like this.  If a central bank buys foreign currency, it collects foreign currency and 
    issues domestic currency in return. 
    The latter is an increase in the domestic money supply.   
    Suppose, for example, the central bank
     starts with the balance sheet
    %
\begin{center}
\begin{tabular}{lrclr}
               Assets  &     &&     Liabilities                     \\  
               \hline 
               FX Reserves &  100 &&     Monetary Base &  200   \\    
               Bonds   & 100 && \\
\end{tabular}
\end{center}
%
The purchase of 25 worth of foreign currency 
changes the balance sheet to
%
\begin{center}
\begin{tabular}{lrclr}
               Assets  &     &&     Liabilities                     \\ 
               \hline 
               FX Reserves &  125 &&     Monetary Base &  225   \\    
               Bonds   & 100 && \\
\end{tabular}
\end{center}
%
Grading:  10 points for similar description of 
how balance sheet changes.
    
\item The easiest way is to calculate the cross-correlation function
for housing starts and a measure of economic activity --- say
industrial production.
Since (the growth rate of) housing starts is strongly    
correlated with future industrial production (also growth), 
then it's  a leading indicator.  

Grading:  10 points for something like this.  

\item  Purchasing power parity is a long-run ``anchor'' for the 
exchange rate:  if prices of goods and services in the Euro Zone
are higher than those in the US, when expressed in a common currency, 
we'd expect the euro to fall in value relative to the dollar 
--- eventually.
This is pretty much useless over a period as short as 6 months,
but has some content over 6 years.
More useful in the short-run is the interest differential.
Since the Euro interest rate is higher, we'd expect the euro to increase in value.
Neither works all that well:  an $R^2$ of 0.05 would be good over
periods of a few months.  

Grading:  4 points for PPP, 4 points for interest rates, 
2 for noting that nothing works all that well in the short run.  

\item Zimbabwe is a classic example of a hyperinflation:
an unstable political situation leads to 
a government deficit.
If and when lenders refuse to buy the government's debt, 
the government is forced to finance the deficit by
printing currency.
As the supply of currency increases, its value falls 
--- namely, inflation.

How do you get out of this situation?
Ultimately you need a political resolution.
In economic terms, you get the government to balance
its budget and give the central bank enough autonomy 
to resist the demand to print money. 

Grading:  7 points for the mechanism, 
3 for a good discussion of the resolution.   
\end{enumerate}
\end{comment}

\end{questions}

\pagebreak \phantom{bla} %\pagebreak \phantom{bla} %\pagebreak

\vfill \centerline{\it \copyright \ \number\year \ 
NYU Stern School of Business}


\end{document}

