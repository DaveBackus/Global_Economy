\documentclass[letterpaper,12pt]{exam}

\usepackage{ge13}
\usepackage{comment}
\usepackage{booktabs}
\usepackage{eurosym}
\usepackage{hyperref}
\urlstyle{rm}   % change fonts for url's (from Chad Jones)
\hypersetup{
    colorlinks=true,        % kills boxes
    allcolors=blue,
    pdfsubject={NYU Stern course GB 2303, Global Economy},
    pdfauthor={Dave Backus @ NYU},
    pdfstartview={FitH},
    pdfpagemode={UseNone},
%    pdfnewwindow=true,      % links in new window
%    linkcolor=blue,         % color of internal links
%    citecolor=blue,         % color of links to bibliography
%    filecolor=blue,         % color of file links
%    urlcolor=blue           % color of external links
% see:  http://www.tug.org/applications/hyperref/manual.html
}

\def\ClassName{The Global Economy}
\def\Category{David Backus}
\def\HeadName{Final Examination}

\printanswers

\begin{document}
\parindent = 0.0in
\parskip = \bigskipamount
\thispagestyle{empty}%
\Head

\centerline{\large \bf \HeadName}%
%\centerline{March 9, 2005}
\centerline{Revised:  \today}

\bigskip
{\it You have 120 minutes to complete this exam.  Please answer each
question in the space provided and show all of your work.
You may consult one page of notes and a calculator,
but devices capable of wireless transmission are prohibited.

I understand that the honor code applies: I will not lie, cheat,
or steal to gain an academic advantage, or tolerate those who do.}

\bigskip
\begin{flushright}
\rule{4in}{0.5pt} \\ (Name and Signature)
\end{flushright}

\begin{figure}[h]
\begin{center}
\setlength{\unitlength}{0.075em}
%\setlength{\unitlength}{0.1em}
\begin{picture}(280,200)(-20,0)
%\footnotesize
\thicklines

% horizontal axis
\put(-30,0){\vector(1,0){300}}
\put(255,-16){$Y$}

% vertical axis
\put(0,-20){\vector(0,1){200}}
\put(-15,155){$P$}

% demand
\put(25,165){\line(4,-3){200}}\put(230,10){AD}
%\put(65,165){\line(4,-3){200}}\put(270,10){AD}

% supply
\put(25,13){\line(4,3){200}} \put(230,160){AS}
%\put(65,13){\line(4,3){200}} \put(270,160){AS$'$}
\put(126.4,0){\line(0,1){170}} \put(118,175){AS$^*$}\put(122,-16){$Y^*$}

% equilibrium labels
%\put(105,85){\footnotesize B}
%\put(150,115){\footnotesize A}
%\put(138,64){\footnotesize C}
% dotted lines
%\qbezier[31]{(133,0)(133,46)(133,92)}
%\qbezier[45]{(0,92)(67,92)(133,92)}
%\qbezier[45]{(0,72)(67,72)(133,72)}

\end{picture}
\end{center}
\caption{Aggregate supply and demand diagram}
\label{fig:asad}
\end{figure}

\begin{questions}
% ======================================================================
\question {\it Two views of monetary policy (40~points).\/}
The goal is to connect some of the things we've learned
about monetary policy,
starting with the diagram in Figure \ref{fig:asad}.

\begin{parts}
\part If we increase the money supply, what happens to the curves in the diagram?
Which ones shift?  Why?
(5~points)
\part What is the short-run impact on prices and output?  The long-run impact?
Illustrate both in the diagram.
(10~points)
\part How does the impact compare to the traditional goals of monetary policy?
Is the policy of increasing the money supply a good one in the context of the figure?
(10~points)
\part Now consider the same thing from the perspective of the quantity theory equation.
If velocity is constant, what is the impact of an increase in the money supply?
(10~points)
\part How do your answers to (b) and (d) compare?
Is your analysis in (b) consistent with the quantity theory?
(5~points)
\end{parts}

\begin{solution}
We'll refer to the diagram:
\begin{center}
\setlength{\unitlength}{0.075em}
%\setlength{\unitlength}{0.1em}
\begin{picture}(280,200)(-10,-15)
%\footnotesize
\thicklines

% horizontal axis
\put(-30,0){\vector(1,0){300}}
\put(255,-16){$Y$}

% vertical axis
\put(0,-20){\vector(0,1){200}}
\put(-15,155){$P$}

% demand
\put(25,165){\line(4,-3){200}}\put(230,10){AD}
\put(65,165){\line(4,-3){200}}\put(270,10){AD$'$}

% supply
\put(25,13){\line(4,3){200}} \put(230,160){AS}
%\put(65,13){\line(4,3){200}} \put(270,160){AS$'$}
\put(126.4,0){\line(0,1){170}} \put(118,175){AS$^*$}\put(122,-16){$Y^*$}

% equilibrium labels
\put(145,112){\footnotesize B}
\put(105,85){\footnotesize A}
\put(115,110){\footnotesize C}
% dotted lines
%\qbezier[31]{(133,0)(133,46)(133,92)}
%\qbezier[45]{(0,92)(67,92)(133,92)}
%\qbezier[45]{(0,72)(67,72)(133,72)}

\end{picture}
\end{center}


\begin{parts}
\part An increase in the money supply shifts aggregate demand to the right.
The new curve is represented in the diagram by AD$'$.

Grading: 5 points for noting which curve to shift, and which direction.

\part In the short run, we move from A to B.  Prices and output both rise.
In the long run, we move to C, where we see (relative to A) that the only impact is on prices.

Grading: 5 points for the short run, 5 points for the long run.

\part The traditional goals are (i) output equal to $Y^*$ and
(ii) stable prices.
We miss on both in the short run and the long run.
The best policy is to stay at A.

Grading:  5 points for a clear statement of the goals,
5 points for noting the difference between them and the policy under
consideration.

\part The quantity theory equation is
\begin{eqnarray*}
    M V &=& P Y ,
\end{eqnarray*}
where $M$ is the money supply, $V$ is velocity,
$P$ is the price level, and $Y$ is real output.
If $V$ is constant, an increase in $M$ increases some combination
of $P$ and $Y$.

Grading:  5 points for the equation, 5 points for noting its consequences.

\part We found the same in our AS/AD analysis,
but in the latter
we also had a way to determine how much of each,
and how that changes over different time horizons.

Grading:  5 points for some commentary along these lines.
\end{parts}
\end{solution}

%\pagebreak \phantom{x} \pagebreak
% ======================================================================
\begin{table}[h]
\centering
\begin{tabular}{lrrrrrr}
\toprule
        & 2008 & 2009 & 2010 & 2011 & 2012 & 2013 \\
\midrule
Nominal GDP (\euro billions) & 172.0 & 168.5 & 172.8 & 171.0 & 165.4 & 156.7 \\
Real GDP growth (\%) & 0.0 & --2.9 & 1.9 & --1.6 & --3.2 & --3.0 \\
Inflation (\%) & 2.6 & --0.8 & 1.4 & 3.7 & 2.8 & 0.4 \\
Govt revenue (\% of GDP)  & 41.1 & 39.6 & 41.4 & 45.0 & 41.0 & 40.7 \\
Govt spending (\% of GDP) & 44.8 & 49.8 & 51.3 & 49.4 & 47.4 & 46.7  \\
Public sector balance (\% of GDP) & --3.7 & --10.2 & --9.9 & --4.4 & --6.4 & --6.0\\
Primary balance (\% of GDP)  & --1.0 & --7.6 &  --7.1 & --0.6 & --2.4 & --0.5   \\
Govt debt (yearend, \% of GDP) & 71.7 & 83.4 & 91.0 & 98.9 & 118.2 \\
Interest rate paid on debt (\%) &  & 3.6 & 3.5 & 4.0 & 3.6	& 4.2 \\
Market rate on debt (\%)        & 4.5 & 4.2 & 5.4 & 10.2 & 10.5 & 6.3 \\
\bottomrule
\end{tabular}
\label{tab:portugal}
\caption{Economic indicators for Portugal.  Source:  EIU reports.}
\end{table}

\question {\it Disaster and opportunity in Portugal (40~points).\/}
As an investor in distressed debt,
you know well that economic disasters can be great opportunities.
You wonder whether Portugal is one now, specifically
the debt of the Portuguese government.

You take a look at the Economist Intelligence Unit's reports
and find:
\begin{itemize}
\item Portugal is currently operating with deficit financing
provided by the ``troika'' (the EU, ECB, and IMF).
One of the conditions is that the deficit be reduced.
The \euro 80b in ``bailout loans'' have more attractive interest rates
than debt issued in the public market.

\item The average maturity of government debt is currently 7.5 years.

\item Although 80\% of the deficit reduction plans involved increases in
tax revenue, the Constitutional Court ruled against roughly half the proposed
cuts in spending.

\item Portuguese banks face rising loan losses as the economy endures a prolonged
contraction.

\item There is risk the government will absorb large
``contingent liabilities'' from insolvent banks and state-owned companies.

\item The exchange rate is fixed within the Euro Zone.
The central bank has no foreign currency reserves.

\item The EIU expects the governing center-right coalition
of the Social Democratic Party and the Popular Party
to hold together through 2015, when elections are due.
However, continued austerity could weaken public support for the government
amid a prolonged recession.
\end{itemize}

Using all of the information above, and your own experience and good judgement,
assess the risks to an investor in Portuguese government debt.
\begin{parts}
\part Use what you know about debt dynamics to compute the ratio
of government debt to GDP at year-end 2013.
What factors contribute to this number?
(10~points)
\part How important is the interest rate paid on debt?
How would your analysis change if the rate rose by 2\%?
Fell by 2\%?
Which is more likely, in your view?
(10~points)
\part Overall, how would you assess the risks to investors?
What are the biggest risks?
Would you buy Portuguese debt now?
(20~points)
\end{parts}

\begin{solution}
\begin{parts}
\part The debt dynamics equation is
\begin{eqnarray*}
   \Delta (B_t/Y_t)  &=&  (i_t - \pi_t)(B_{t-1}/Y_{t-1})
                - g_t (B_{t-1}/Y_{t-1}) + D_t/Y_t .
\end{eqnarray*}
The three terms are
\begin{eqnarray*}
    (i_t - \pi_t)(B_{t-1}/Y_{t-1}) &=& (0.042 - 0.004) (118.2) \;\;=\;\; 4.5  \\                - g_t (B_{t-1}/Y_{t-1})   &=& - (-0.030) (118.2) \;\;=\;\; 3.5 \\
    D_t/Y_t  &=&  0.5 .
\end{eqnarray*}
Their total is 8.5, so the ratio of debt to GDP will rise from 118.2 to 126.7.

Grading:  3 points for each component done right, 1 more for summing them.

\part This involves the first term above.  If we increase the interest rate paid
by 2\%, that adds $ 0.02 \times 118.2 = 2.4$,
so you get an additional increase in debt of 2.4\% of GDP.
Similarly, a decline of 2\% leads to a decline in debt of the same amount.
Given the large difference between the interest rate paid on the debt
and the current market rate, it seems more likely the rate will go up
than go down.

Grading: 7 points for the calculation, 3 for noting the difference
between the rate paid and the market.

\part This is a call for the checklist:
\begin{itemize}
\item Debt and deficits.
(i)~The calculation shows us debt is still rising, although a substantial part
comes from a shrinking economy.
(ii)~Contingent liabilities.

\item Banks.
The EIU suggests that banks could be collateral damage
from a bad economy.

\item Exchange rates and reserves.  Not relevant here,
Portugal is part of the Euro Zone and does not have any control over
its exchange rate.

\part Politics.  Always an issue, but the EIU thinks
it's mostly under control.
\end{itemize}
The biggest risk seems to be that the government is unable to get
the budget under control.
If it does, the debt could be a good deal.
Should you buy some? It's a risky bet, for sure.
To me, the fiscal situation is bad enough that I'd stay away.
But there's some upside if they turn it around,
and if they continue to get favorable terms from the troika.

Grading:  15 points for the checklist and its components,
5 points for a sensible argument assessing the various risks.

\end{parts}
\end{solution}

%\pagebreak \phantom{x} \pagebreak
% ======================================================================
\question {\it Short answers (40 points).\/}
\begin{parts}
\part Draw the cross-correlation function for a countercyclical lagging indicator.
(10~points)
\part In the Euro Zone, GDP growth is now $-0.5$\%
and inflation is 1.7\%.
The ECB recently reduced its target interest rate to 0.5\%.
Use the Taylor rule to assess the suitability of this policy.
(10~points)
\part Describe how China's purchases of foreign currency affect
the balance sheet of the central bank, the People's Bank of China.
How would it be affected by ``sterlization.''
(10~points)
\part The Economist reports that Big Macs cost about the same in Canada
and the US in 2008, but are now 25\% more expensive in Canada.
What does this suggest about the ``dollar-dollar'' exchange rate over the next 6 months?
(10~points)
\end{parts}

\begin{solution}
\begin{parts}
\part The largest correlation should be in the lower-right quadrant,
like the ccf for the unemployment rate in Figure 11.3 of the book.
Your figure should be labeled correctly, with leads to the left and
lags to the right.

\part The Taylor rule calculation is something like this:
\begin{eqnarray*}
    i &=& r^* + \pi + 0.5 (\pi - \pi^*) + 0.5 (g - g^*) .
\end{eqnarray*}
The US targets are
$ r^* = 2$, $\pi^* = 2$, and $g^* = 3$.
If we use the same, we get $i = 1.8$, which is well below the current rate of 0.5.
Like the US, the ECB is being more aggressive with monetary policy
than the Taylor rule suggests.

Grading:  5 points for a correct formula for the Taylor rule,
5 for applying it in this way or something close.

\part Let's say the PBOC's balance sheet looks something like this:
%
\begin{center}
\begin{tabular}{lr|lr}
               Assets  &     &     Liabilities                     \\
               \hline
               FX Reserves &  100 &     Money &  200   \\
               Bonds   & 100 & \\
\end{tabular}
\end{center}
%
If the PBOC purchases 50 worth of foreign currency,
it changes to
%
\begin{center}
\begin{tabular}{lr|lr}
               Assets  &     &     Liabilities                     \\
               \hline
               FX Reserves &  150 &     Money  &  250   \\
               Bonds   & 100 & \\
\end{tabular}
\end{center}
%
Note that (the supply of) money has gone up:  when it bought foreign currency,
it issued local currency in return.
The PBOC can undo this by selling 50 of bonds, taking money
in exchange:
%
\begin{center}
\begin{tabular}{lr|lr}
               Assets  &     &     Liabilities                     \\
               \hline
               FX Reserves &  150 &     Money  &  200   \\
               Bonds   & 50 & \\
\end{tabular}
\end{center}
%
This transaction is referred to as ``sterilization.''

Grading:  5 points each for the describing the impact of the two transactions
on the balance sheet.

\part This is a PPP-like calculation:  goods are expensive in Canada,
so the Canadian dollar is overvalued.
We know from experience that this has no predictive value in the short term.

Grading:  5 points for noting the overvaluation, 5 for noting the lack
of predictive content.

\end{parts}
\end{solution}

\end{questions}

%\pagebreak \phantom{xx} %\pagebreak \phantom{xx}

\vfill \centerline{\it \copyright \ \number\year \ NYU Stern
School of Business}

\end{document}
