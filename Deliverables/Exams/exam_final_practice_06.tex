\documentclass[letterpaper,12pt]{article}

\usepackage[hypertex]{hyperref}
\RequirePackage{GE05}
% this inputs graphicx, too
\RequirePackage{comment}

\newcommand{\GDP}{\mbox{\em GDP\/}}
\newcommand{\NDP}{\mbox{\em NDP\/}}
\newcommand{\GNP}{\mbox{\em GNP\/}}
\newcommand{\NX}{\mbox{\em NX\/}}
\newcommand{\NY}{\mbox{\em NY\/}}
\newcommand{\CA}{\mbox{\em CA\/}}
\newcommand{\NFA}{\mbox{\em NFA\/}}
\newcommand{\Def}{\mbox{\em Def\/}}
\newcommand{\CPI}{\mbox{\em CPI\/}}
\newcommand{\phm}{\phantom{--}}

\def\ClassName{The Global Economy}
\def\Category{Professor David Backus}
\def\HeadName{Practice Final Examination}

\begin{document}
\parindent = 0.0in
\parskip = \bigskipamount
\thispagestyle{empty}%
\Head

\centerline{\large \bf \HeadName}%
\centerline{Revised:  \today}

\bigskip
You have 100 minutes to complete this exam.  Please answer each
question in the space provided. You may consult one page of notes
and a calculator, but devices capable of wireless transmission are
prohibited.

I understand that the honor code applies: I will not lie, cheat, 
or steal to gain an academic advantage, or tolerate those who do.

\begin{flushright}
\rule{4in}{0.5pt} \\ (Name and Signature)
\end{flushright}


%\bigskip
{\it Most of these questions come from old exams.  
Most of the data are out of date for that reason. } 

\begin{enumerate}

\item {\it US economic conditions (30 points).}  How well is the US economy doing?  One source of
information is the forward rate curve.   Forward rates on May 5, 2005, and averages over the last
25 years, were
%
\begin{center}
\begin{tabular}{ccc}
  \phantom{xxx}Maturity\phantom{xxx} & Forward Rate  & Mean Forward Rate \\
     1      &   3.73       &    7.51        \\
     2      &   4.33       &    8.10        \\
     3      &   4.47       &    8.39          \\
     4      &   4.62       &    8.57         \\
     5      &   4.75       &    8.67
\end{tabular}
\end{center}
Maturity is measured in years and forward rates are annually compounded percentages.
%
\begin{enumerate}

\item Compute zero-coupon yields for maturities between 1 and 5 years.  (10~points)

\item Use the forward rate curve to construct forecasts of future 1-year yields.  How do you see
the 1-year rate evolving over the next four years? What does this suggest for output growth?  (10~points)

\item Describe the Taylor Rule.  Under what conditions would it lead to the path of interest rates
you described above?  Do you find these conditions plausible?  (10~points)

\end{enumerate}


\begin{comment}
Answer.  Consider these calculations:
\begin{center}
\begin{tabular}{ccccc}
  Maturity  &  Bond Price & Yield  &  Risk Premium  &  Future Short Rate \\
     1   &  96.40   &   3.73       &    0.00   &  3.73  \\
     2   &  92.40   &   4.03       &    0.59   &  3.74  \\
     3   &  88.45   &   4.18       &    0.88   &  3.59  \\
     4   &  84.54   &   4.29       &    1.06   &  3.56  \\
     5   &  80.71   &   4.38       &    1.16   &  3.59
\end{tabular}
\end{center}

\begin{enumerate}
\item See table. 
\item Ditto.  The flat to decreasing path of expected future short rates suggests
modest to below-average growth of output. %
\item The Taylor rule is
\[
    i_t  \;=\;  r^* + \pi_t + a_1 (\pi_t - \pi^*)  + a_2 (y_t - y^*)  .
\]
It approximates how the Fed sets the short-term interest rate.  
The flat interest rate path above suggests flat inflation and output relative to trend, which is consistent with what we've seen recently.
[Of course, developments over the last year are more difficult to 
reconcile.] 
\end{enumerate}
\end{comment}

%\pagebreak \phantom{bla} \pagebreak \phantom{bla} \pagebreak

% *************************************************************************************************
\item {\it Japanese fiscal policy (20 points).} Japan last year ran a government budget deficit
of close to 7\% of GDP.  In addition, the ratio of public debt to GDP was over 100\%, GDP growth
was below 1\%, inflation was slightly negative, and the trade surplus was 3.5\% of GDP.
%
\begin{enumerate}

\item Why might a government deficit be associated with a trade surplus, when we see the
opposite pattern in the US (government and trade deficits)?
%[Hint:  Use the flow identity.]
(10~points)

\item The Japanese government is said to be considering three approaches to the government
deficit: (i)~do nothing and hope it goes away, (ii)~raise tax rates, and (iii)~reduce purchases of
goods and services.  Which one would you suggest? Why?  (10~points)

\end{enumerate}

\begin{comment}
Answer.
\begin{enumerate}
\item The answer revolves around (private) saving and investment, which are much different in 
the two countries.  
One version of the flow identity is:
\[
    S_p + S_g \;=\; I + \mbox{\em NX} .
\]
In Japan, $S_g$  (govt saving or surplus) is negative and $\mbox{\em NX}$ (net exports = trade balance) is positive, so $S_p$ (private saving) must be significantly greater than $I$ (investment).  In the
US, both $\mbox{\em NX}$ and $S_g$ are negative, the former more so, 
therefore $S_p$ is less than $I$.
Roughly speaking, the difference is that investment has been low in Japan, 
which isn't surprising since the economy is growing slowly.  

\item The list:
%
\begin{enumerate}
\item [(i)] Do nothing.  Debt will continue to build up, adding interest expense to the budget.
The longer this goes on, the larger the debt burden will be.
Right now interest rates are very low, so this isn't an immediate concern, but
if interest rates rise the situation could change quickly.  
All of these factors could be discussed using the sustainability  
relation,  
\[
    \frac{B_{t+1}}{Y_{t+1}} \;=\; \left( \frac{1+i}{1+g} \right) 
     \frac{B_{t}}{Y_{t}} 
     + (1+g)^{-1} \frac{D_{t}}{Y_{t}} ,
\]
where $B$ is the government debt and $D$ the primary deficit.  


\item [(ii)]  Raise tax rates.  This may have adverse incentive effects, but it's not clear how
large those are.  Japan still has a relatively small tax burden overall.

\item[(iii)] Reduce spending. Depends what it is, but some cuts in spending are probably needed.
\end{enumerate}
\end{enumerate}
\end{comment}

%\pagebreak \phantom{bla} \pagebreak


\item {\it Miscellany (50 points).}

\begin{enumerate}

\item {\it The Economist\/} reports that a Big Mac costs \$2.90 in the US, \$3.28 in the eurozone,
and \$2.33 in Japan.  (These prices are averages for the various regional markets, expressed in US
dollars using current spot exchange rates.)  What does this suggest about the likely change in
value of the euro and yen v. the dollar over the coming 6 months?  6 years? (10 points)  

\begin{comment}
Answer.  PPP suggests that exchange rates will adjust to make prices the same across countries. In
this case, that means the dollar will rise against the euro, fall against the yen.  Is this right?
Over short periods of time, exchange rates are close to unpredictable by any means, PPP included.
Over longer periods of time (5-20 years) PPP is a reasonable indicator.  
\end{comment}

\item In Canada over the last year, inflation has been 2.3\% and money growth (M1) has been 11.8\%.  Do you find the difference between the two numbers surprising? Why or why not?  (10~points)

\begin{comment}
Answer.  Not surprising because the short-run relation between money growth and inflation is weak.
\end{comment}

\item For each of the following variables, say whether they are leading, lagging, or coincidental:
employment, hours worked, consumption, investment. (10 points)

\begin{comment}
Answer.  Consumption and investment are coincidental, hours leads, and employment lags.
\end{comment}

\item Describe in some detail how the Federal Reserve manages to keep the Federal Funds rate at
its target level. (10 points)

\begin{comment}
Answer.  The Fed increases the monetary base to reduce the Fed Funds rate, the opposite to
increase it.  It does this by purchasing government securities, which credits securities dealers
with deposits at the Fed.  These deposits are part of the monetary base, which therefore rises.  
\end{comment}

\item The US trade deficit is now about 7\% of GDP.  Is the deficit ``sustainable''?  (10~points)

\begin{comment}
Answer.  By sustainable, we mean can it stay this way forever without the ratio of net foreign assets
to GDP exploding.  The answer is no.  Even though the current rate paid on foreign assets is close to zero,
at present, 
adding 7\% to the debt every year will raise the ratio of net foreign assets to GDP by about 1\% 
if nominal GDP grows at 6\%.  If the interest rate rises, net foreign assets will increase more rapidly.      
The relevant equation is 
\[
    \frac{\mbox{\em NFA}_{t+1}}{Y_{t+1}} \;=\; \left( \frac{1+i}{1+g} \right) 
     \frac{\mbox{\em NFA}_{t}}{Y_{t}} + (1+g)^{-1} \frac{\mbox{\em NX}_{t}}{Y_{t}} .
\]
If the economy slows down ($g$ gets smaller), the situation may be worse,  
although slow growth is associated with higher values of $\mbox{\em NX}$.
\end{comment}

\end{enumerate}

\end{enumerate}

%\pagebreak \phantom{bla} \pagebreak \phantom{bla}


\vfill \centerline{\it \copyright \ \number\year \ 
NYU Stern School of Business}

\end{document} 


%  ****************************************************************************************
%  2005 practice exam follows 

\begin{enumerate}

\item \textbf{The US business cycle.} What follows is an excerpt from the article entitled
``Another hangover in the making?" published in the print edition of {\it The Economist\/} on
November 6, 2003:

\textit{Ever since America's stock market bubble burst in 2000, The Economist has argued that
America faced several years of sluggish growth, if not a deep recession, as the economy worked off
the excesses that built up in the late 1990s. Yet the economy has come roaring back, with GDP
rising by 7.2\% at an annual rate in the third quarter -� its fastest sprint for 19 years. Do we
look a bit silly? Indeed. But there are still big risks ahead.}

\textit{Certainly, the recent recovery has been startling. Over the past year, America's GDP grew
by 3.3\%. Growth is becoming more balanced. In the third quarter, consumer spending rose by 6.6\%
at an annual rate, but business investment was also up by an impressive 11\%, housing investment
grew by 20\%, and even exports jumped by 9\%. Corporate profits are rebounding and figures due on
November 6th were widely tipped to show that business productivity grew at an astonishing 8\% or
thereabouts in the third quarter.}

\textit{Many pundits have argued that America's brisk rebound, despite all that it has suffered -�
the bursting of the bubble, September 11th, corporate scandals and the Iraq war -- proves just how
wonderfully flexible and resilient its economy is. Flexibility played a part, but a smaller one
than many believe. The main reason why the economy has held up better than expected is that it has
enjoyed the biggest fiscal and monetary stimulus in decades.}

\textit{No other country has experienced such a huge increase in its structural-budget balance
(i.e., excluding the automatic effect of the economic cycle). Thanks partly to tax cuts, it has
swung from a structural surplus of 1\% of GDP in 2000 to a deficit of 5\% this year. Meanwhile,
historically low interest rates sparked a boom in mortgage refinancing, giving households lots
more extra cash to spend. And spend they have.}

\textit{But the government cannot carry on handing out such largesse for ever. America's
deteriorating fiscal finances rule out further big net tax cuts. Long-term interest rates have
also risen, and as a result mortgage refinancing has fallen by around 80\% from its peak. In
September real consumer spending dipped by 0.6\%. If job lay-offs continue, consumers may also
tighten their belts.}

\textit{Business investment is picking up, but ample spare capacity will continue to discourage
new spending. In September, manufacturing output was running at only 73\% of capacity, well below
the average of 81\% over the past half century. In any case, business investment is too small a
share of the economy to keep it aloft in the absence of robust consumer spending.}

\textit{But the main reason for doubting that America is back on a path of strong, sustainable
growth is that it has failed to purge the excesses of its previous boom. It is, to say the least,
odd that at the beginning of an economic recovery many indicators -- low saving, rampant household
borrowing, record house-building and uncomfortably high stock market p/e ratios to name but a few
-- have more the look of a cycle that is drawing to a close.}

\textit{Debt provides the best example. In the year to the second quarter (the latest figures
available), borrowing by households rose by 11\%, its fastest pace in real terms since 1985.
Economists wearing rose-tinted spectacles point out that interest rates today are low, so
households can afford to borrow more. The main snag with this argument (but by no means the only
one) is that, despite low interest rates, households' debt-service payments are alarmingly high as
a percentage of their income. The Federal Reserve has recently revised its measure of household
debt-service. In contrast to the old figures, this now shows that the debt-service ratio is higher
than at its previous peak in the 1980s.}

\textit{When interest rates inevitably rise to normal levels, debt-service ratios will rise higher
still. But even while interest rates remain low, debt service will eat up a rising share of income
if households keep borrowing more. Sooner or later, consumers will need to save more and spend
less.}

\begin{enumerate}

\item The article alleges that the increase in the federal budget deficit is (at least in part)
responsible for the decline in the household saving rate.  Do you agree? Explain why or why not.
(10 points)

\item The editorial also argues that households are borrowing \textit{too much}:  that debt
service will eat up a rising share of income, so that sooner or later consumers will need to raise
their saving rates.  Do you agree? Explain why or why not. (10 points)

\item It is also argued that ``\textit{business investment is too small a share of the economy to
keep it aloft in the absence of robust investment spending}". Do you agree?  Explain why or why
not. (10 points)

\end{enumerate}

Answer.
\begin{enumerate}
\item There are two issues here. The first is that lower taxes (say) may raise households' current
income and lead them to consume more.  Whether this results in lower saving {\it rate\/} depends
on whether the increase in income is smaller proportionately than the increase in consumption.
Mathematically,
\[
    s \;=\;  (Y - C) / Y \;=\; 1 - C/Y,
\]
where $Y$ in this case is household income. The second issue is more subtle but probably more
important.  Households know (or should know) that a deficit now has to be paid for by running a
surplus later.  (Either that or default on the debt, but we'll assume the US is not considering
that.)  If households expect the government to collect higher tax revenue in the future, then
(other things equal) they might consume less now since the present value of their future income is
lower by that amount of the future increase in tax revenue.  That would raise the current savings
rate.

You might think of it this way.  Suppose you hear that Social Security is in financial difficulty.
(It's probably not, but that's another story.)  Then you might expect your benefits to be lower
than specified by current law and (as a result) save more of your own money for retirement.

\item You need more evidence than you're given to do justice to this, but the issue is whether the
debt could cause trouble for households, causing them to reduce spending in the future.  One way
this could happen is that output falls.  Another is for interest rates to rise, making debt more
expensive and (possibly) asset values to fall.  Either could happen.  What's missing, though, is
that households currently have very high net worth:  the debts are more than balanced by assets.

\item Simple answer:  investment may be a small fraction of output, but it's very volatile, with a
standard deviation of fluctuations 3-4 times that of output as a whole.  Complex answer:  Does
high spending cause the economy to boom, or does a boom generate high spending? It probably works
both ways, but (i) businesses invest when they think they'll make money from it (ie, boom
generates investment) and (ii) households consume when they expect future income to be high
(ditto).  Could it work the other way, too?  Possibly, but the evidence is mixed.

\end{enumerate}

% *************************************************************************************************
\item \textbf{Germany's fiscal policy.} Read carefully the article entitled ``The Keynesian
Temptation," published in the May 6, 2004 print edition of {\it The Economist\/}:

{\it

DEMONSTRATING independence over Iraq paid off handsomely for Gerhard Schroder, the German
chancellor, in the September 2002 election. Without his unwavering opposition to America's
pre-emptive war, he would surely have lost. Might he now be tempted to show a similar
popularity-winning independence over fiscal policy --� even if this kills, once and for all, the
euro's stability-pact limits on budget deficits?

The idea is certainly germinating, even if both Mr Schroder and Hans Eichel, his finance minister,
took pains this week to deny press reports of a Keynesian conversion. These two and others met on
April 28th to discuss saying �farewell to budget tightening�, according to Der Spiegel, which
broke the story. The weekly also ran an interview with Joschka Fischer, the foreign minister, in
which he said that spending cuts alone would not generate badly needed growth.

Early this year, Mr Schroder had what seemed a reasonable game plan. After pushing through his
Agenda 2010 reforms, he hoped to regain popularity on the back of economic revival. Yet last week,
the economics ministry cut its growth forecast for 2004 from 2.25\% to a meagre 1.5\%. This week's
figures also showed a rise in unemployment, which is around 4.4m or over 10\% of the workforce.

Mr Schroder's decision to step down as chairman of the ruling Social Democrats (SPD) and hand over
to Franz Muntefering has also failed to improve his party's poll ratings. The SPD hovers below
30\%, compared with nearly 50\% for the opposition Christian Democrats. This bodes ill for the
party in the 12 local and regional elections this year, and for the European elections on June
13th, the results of which could be disastrous for the SPD. Germany's fiscal situation is also
deteriorating. The federal coffers are short by E18 billion (\$22 billion) this year and E15
billion next, compared with budget projections. And these deficits come on top of already planned
shortfalls of E29 billion in 2004 and E21 billion in 2005. If no action is taken, the government
will have to borrow even more, keeping Germany's budget deficit well above the stability-pact
ceiling of 3\% of GDP again next year (see chart).

What to do next was the main issue at the secret meeting. Mr Schroder reportedly argued that
simply cutting the budget will merely fuel the anxieties of German consumers, and so weaken the
economy further. Instead, he wants to make a �strategic reply�, which could include spending some
real money on his so-called �innovation offensive�. To finance this, the government could sell
more assets, such as stakes in companies and parts of the country's gold reserves�and borrow more,
like George Bush in America. Mr Eichel prefers to stick to budgetary rigour, however.

In contemplating a Keynesian shift, Mr Schroder may be taking a lead from Peter Bofinger, an
economist at Wurzburg University. The government has focused too much on structural reform and not
enough on sustaining demand, argues Mr Bofinger, who was recently appointed to the German council
of economic experts. Yet most other economists make a case that any shift would be harmful. The
country's problem is not growth itself, but low growth potential, says Hans-Werner Sinn, president
of IFO, an economic think-tank.

Whatever the effects of a change in Germany's fiscal policy, the political fallout is predictable.
The stability pact would surely wither if its chief sponsor defied it for a fourth year running.
Ironically, Nicolas Sarkozy, finance minister of the other big pact-breaker, France, this week
promised to do more to cut spending to get back below the ceiling. At home, too, Keynesianism
might come with costs. It would surely seal the fate of Mr Eichel. His departure would trigger a
cabinet reshuffle, which could spark renewed conflict within the ruling coalition. On top of this,
being fiscal conservatives, Germans dislike public debt almost as much as they do wars. Going
Keynesian may tempt Mr Schroder, but it might not pay off.

}

\begin{enumerate}

\item Recent figures show that the German unemployment rate is still rising, and is now over 10\%.
The article suggests that influential cabinet members were contemplating addressing this problem
by employing public funds to hire unemployed individuals. How effective would such a policy be?
Can you think of alternative policy reforms that could successfully tackle the problem? (10
points)

\item In 2002 and 2003, the ratio of Germany's federal budget deficit to GDP exceeded the 3\%
ceiling established in the stability pact signed by the European countries that adopted the Euro.
Which are the main motivations that led to the introduction of such ceiling? (10 points)

\item Do you agree with the German economist Peter Bofinger when he argues that the German
government has focused too much on structural reform and not enough on sustaining demand? Why? (10
points)

\end{enumerate}


Answer.
\begin{enumerate}

\item The central difficulty is that current tax rates make Germany an unattractive place to hire
workers --- not just because of employment protection but because corporate tax rates are high
relative to many other countries in the EU.  Public employment would simply increase the need for
tax revenue, driving tax rates up further.  The alternative is to cut government spending and
taxes on (esp) businesses.  This isn't giving in to the business lobby, it's an economic necessity
given the ready availability of better terms nearby.

\item The ceiling was intended to eliminate the possibility that a government would get into
financial difficulties, leading to pressure on the central bank to inflate or the EU to bail the
country out.  Now that it's more or less gone, we'll see if either of these happens.

\item No!  Jobs come from firms:  you need to create an environment in which they want to hire
workers.  Nothing else works in a market-based system.

\end{enumerate}


\item \textbf{Miscellany.}

\begin{enumerate}

\item For each of the following variables, say whether they are leading, lagging, or coincidental:
employment, hours worked, consumption, investment. (10 points)

Answer.  Consumption and investment are coincidental, hours leads, and employment lags.

\item In which respects does dollarizing (ie,  adopting the US dollar as your official currency)
differ from pegging the exchange rate to the dollar? (10 points)

Answer.  The question is always how to make sure a peg sticks, given how many in the past have
blown up in spectacular fashion.  If you move to dollars, presumably it's less likely you can go
back to having your own currency.

\item Describe in some detail how the Federal Reserve manages to keep the Federal Funds rate at
its target level. (10 points)

Answer.  The Fed increases the monetary base to reduce the Fed Funds rate, the opposite to
increase it.  It does this by purchasing government securities, which credits securities dealers
with deposits at the Fed.  These deposits are part of the monetary base, which therefore rises.  

\item What is the monetary base? Is it different from the money supply?  (10 points)

Answer.  It's the narrowest definition of money:  currency held by the households and firms plus
reserves (cash + deposits) held by financial institutions.

\end{enumerate}

\end{enumerate}


\vfill \centerline{\it \copyright \ \number\year \ NYU Stern
School of Business}

\end{document}
