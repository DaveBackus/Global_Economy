\documentclass[letterpaper,12pt]{article}

\usepackage[hypertex]{hyperref}
\RequirePackage{GE05}
% this inputs graphicx, too
\RequirePackage{comment}

\newcommand{\GDP}{\mbox{\em GDP\/}}
\newcommand{\NDP}{\mbox{\em NDP\/}}
\newcommand{\GNP}{\mbox{\em GNP\/}}
\newcommand{\NX}{\mbox{\em NX\/}}
\newcommand{\NY}{\mbox{\em NY\/}}
\newcommand{\CA}{\mbox{\em CA\/}}
\newcommand{\NFA}{\mbox{\em NFA\/}}
\newcommand{\Def}{\mbox{\em Def\/}}
\newcommand{\CPI}{\mbox{\em CPI\/}}
\newcommand{\phm}{\phantom{--}}

\def\ClassName{The Global Economy}
\def\Category{Professor David Backus}
\def\HeadName{Practice Final Examination}

\begin{document}
\parindent = 0.0in
\parskip = \bigskipamount
\thispagestyle{empty}%
\Head

\centerline{\large \bf \HeadName}%
\centerline{Revised:  \today}

\bigskip
You have 100 minutes to complete this exam.  Please answer each
question in the space provided. You may consult one page of notes
and a calculator, but devices capable of wireless transmission are
prohibited.

I understand that the honor code applies: I will not lie, cheat,
or steal to gain an academic advantage, or tolerate those who do.

\begin{flushright}
\rule{4in}{0.5pt} \\ (Name and Signature)
\end{flushright}

%\bigskip
{\it Most of these questions come from old exams.
They may include outdated data as a result. }

\begin{enumerate}
% ****************************************************************************
\item {\it US monetary policy (25 points).}
After its last meeting,
the US Federal Open Market Committee stated:
\begin{quote}
Recent indicators have been mixed. ...
Nevertheless, the economy seems likely to continue
to expand at a moderate pace over coming quarters.
%
Recent readings on core inflation have been somewhat elevated.
Although inflation pressures seem likely to moderate over time,
the high level of resource utilization has the potential
to sustain those pressures.
%
In these circumstances, the Committee's predominant policy concern remains the risk that inflation will fail to moderate as expected. Future policy adjustments will depend on the evolution of the outlook for both inflation and economic growth, as implied by incoming information.
\end{quote}
%
\begin{enumerate}
\item Do you agree with  the Committee's ``predominant policy concern''?
Why or why not?
(5~points)
\item Use the aggregate supply and demand framework
to illustrate how the FOMC might think about monetary policy.
Consider each of the following questions in turn:
(i)~How would an increase in the money supply affect
output and prices in the short run?
(ii)~In the long run?
(iii)~How does the phrase ``high level of resource utilization''
affect your answer?
(15~points)
\item If future information suggests higher inflation,
what would you expect to happen to interest rates?
(5~points)
\end{enumerate}

%\begin{comment}
Answer.
\begin{enumerate}
\item Inflation is the predominant concern,
although they  mention economic growth
(``expand at a moderate pace'').
%The second question is an opportunity for you to bring in your own
%perspective, possibly including your analysis from Group Project \#6.
My take:  inflation is rightly the Fed's primary concern in most
situations, esp now given the inflation of the last two years. Why?
Because it has limited control over growth, and experience tells us
that if inflation rises substantially, we tend to have below-average
growth.

\item Draw a diagram with AS, AD, and AS$^*$.
%Note in particular where the short-run equilibrium (where AS and AD cross)
%is relative to AS$^*$.
Assume for the time being that we start at a long-run equilibrium
(where AS crosses AD and AS$^*$ at the same time). (i)~AD shifts
right/up. In the short run, we move to where AS and AD cross (and
ignore AS$^*$). This raises output and prices (inflation). (ii)~In
the long-run, AS also shifts, to the point where AD and AS$^*$
cross. Why? Because the sticky wages eventually adjust. The result:
prices rise, but output stays the same, relative to our starting
point. This illustrates the difference between the short-run and
long-run effects of monetary policy:  In the long run, all we get is
inflation.  In the short-run, we get a combination of higher prices
and higher output. Which sends us back to (a):  one of the questions
the Fed must address is how large the short-run increase in output
is.  That, in turn, depends on how steep the AS curve is:  the
steeper the curve, the smaller the increase in output. [For
practice, contrast this analysis with one where we start to the left
of AS$^*$.]
%
(iii)~If you review our earlier answer, you'll see that the long-run
impact depends on where the short-term equilibrium is relative to
AS$^*$. This statement suggests that the FOMC sees the economy as
close to AS$^*$.


\item If inflation rises, the Fed would typically raise
the target fed funds rate. You can see this in the Taylor rule, for
example. This also brings us back to (a), since ultimately the FOMC
policy will depend on both inflation and output news.

\end{enumerate}
%\end{comment}


% ****************************************************************************
\item {\it Risk and reward in Turkey (25 points).}
During a job interview with a European portfolio manager,
you are given  the following macroeconomic indicators for Turkey:
%\tabcolsep=0.25in
\begin{center}
\begin{tabular}{lcc}
  {\it Indicator}  &  {\it Value}   & {\it Date}     \\
  GDP growth (real) &  9.5\%    & 2005 Q4     \\
  Inflation  &  8.2\%     & March 2006    \\
  Short-term interest rate &  14\%  &  April 2006 \\
  Fiscal balance:  total  (ratio to GDP) &  --3\%  & 2006 est\\
  Fiscal balance:  primary  (ratio to GDP) &  +6\%  & 2006 est \\
  Government debt (ratio to GDP) &  64\%  &  2006 est \\
  Current account balance (ratio to GDP)&  --5\%  &  2006 est \\
  Net foreign income (ratio to GDP)&  --1\%  &  2006 est \\
%  Foreign debt (ratio to GDP)      &    43\%   &  2006 est    \\
  Net foreign assets (ratio to GDP) &   --30\%   &   2006 est    \\
  Foreign reserves (ratio to GDP) &  29\%  & 2006 est \\
\end{tabular}
\end{center}
Source:  Economist Intelligence Unit.

The recruiter asks you:
\begin{enumerate}
\item Is Turkey's fiscal deficit a source of concern for
an investor in Turkish government debt?   (10~points)
\item Is Turkey's current account deficit a source of concern
for an investor in high-grade private Turkish debt?    (10~points)
\item What is your overall assessment of the
macroeconomic risks to Turkish debt?
(5~points)
\end{enumerate}
You understand that you will be assessed on both the rigor of your logic
and your ability to convey it clearly.


%\begin{comment}
Answer.  We did a more nuanced version of this in class, but the
standard quantitative analysis would go like this, and then be
modified by subjective judgements about things like the political
situation:
\begin{enumerate}
\item Analysis.
Turkish government debt is 64\% of GDP, which is not enormous but
high for a developing country.  But the high interest rate means
that interest on the debt eats up 9\% of GDP. We can get a clearer
picture if we look at debt dynamics:
\[
    \frac{B_{t+1}}{Y_{t+1}} \;=\; \left( \frac{1+i}{1+g} \right)
     \frac{B_{t}}{Y_{t}}
     + (1+g)^{-1} \frac{D_{t}}{Y_{t}} ,
\]
With $i$ about 14\% and $g$ about 17\%, we can see that the debt to GDP ratio
will fall slightly over the next year on its own --- largely the effect of
high GDP growth.
The primary surplus of 6\% will further reduce the debt to GDP ratio.
In short, the country is in good shape right now.

Pitch.  A close look at the situation suggests that the Turkish
debt-to-GDP ratio, while high, shows no tendency to grow further. In
that sense, government debt looks like a decent bet right now, esp
given the high interest rates.  One pessimistic scenario:  a change
in the political will for fiscal discipline, which could lead to a
sharp increase in interest rates and less attractive debt dynamics.


\item Turkish NFA is --30\%, 
which again is enough to think about but not enormous.
Ditto the current account deficit.
Here's a formal analysis based on the same logic ---
more than you need to answer the question.
Net foreign income is --1\%, which seems modest.
A more formal analysis of NFA dynamics:
\[
    \frac{\mbox{\em NFA}_{t+1}}{Y_{t+1}} \;=\; \left( \frac{1+i}{1+g} \right)
     \frac{\mbox{\em NFA}_{t}}{Y_{t}} + (1+g)^{-1} \frac{\mbox{\em NX}_{t}}{Y_{t}} .
\]
Again, the term $[(1+i)/(1+g)]$ is a little less than one, so that's reassuring,
but the current account will result in a (roughly) 2\% increase in the ratio
of NFA to GDP.


Pitch.  The increase in debt is accompanied by rapid growth. As long
as the economy continues to grow, private debt looks like a good
deal. The pessimistic scenario mentioned above might apply here,
too.

\item  The numbers show no obvious signs of trouble,
but you might want to ask some questions that go beyond the numbers:
What is the form of foreign claims on the government? Short or long?
Local or foreign currency? What is the form of foreign claims on
Turkish companies? Are they debt or equity?  How stable is the
political situation? If (in the worst case) the EU closes the door,
could that undermine successful efforts to establish fiscal
discipline?
\end{enumerate}
%\end{comment}


% ****************************************************************************
\item {\it Miscellany (50 points).}

\begin{enumerate}
\item An analyst suggested that China may suffer a currency crisis
along the lines of Mexico in 1994-95,
in which the peso fell sharply when the Banco de Mexico
ran out of foreign currency reserves.
Give two reasons why this scenario is likely or unlikely.
 (10~points)

%\begin{comment}
Answer.  (i)~China has enormous reserves:  they won't run out any time soon.
(ii)~The yuan renminbi seems to be undervalued:  people want to buy it,
not sell it, which results in the central bank accumulating reserves,
not losing them.
%\end{comment}


\item Can a country run a fiscal (government) deficit forever?
Why or why not?
 (10~points)

%\begin{comment}
Answer.  The present value of future primary surpluses has to equal
the current debt.  Thus past deficits must be balanced by future
surpluses --- you can't run a primary deficit forever. The key word
is primary:  you can run a primary surplus and an overall deficit at
the same time, as we see in (for example) Turkey.
%\end{comment}


\item In Canada over the last year, inflation has been 2.3\%
and money growth  has been 11.8\%.
Do you find the difference between the two numbers surprising?
Why or why not?  (10~points)

%\begin{comment}
Answer. Under the quantity theory, inflation equals money growth
minus real GDP growth. Unless real GDP growth is 9\%, something's
wrong. What's wrong is that this relation is well-known not to work
in the short run.  Over periods of several years, however, it
typically works pretty well.
%\end{comment}

\item Explain what a leading indicator is
and give an example for the US.  (10~points)

%\begin{comment}
Answer.  A leading indicator (of the economy)
is an observable economic variable whose ups and
downs precede those in (say) real GDP.
You can see this in the cross-correlation function, for example.
Common examples:  housing starts, stock market indexes,
interest rate spread (long minus short).
%\end{comment}

\item Describe in some detail what a central bank does to
maintain the short-term interest rate at a specific level.
(10~points)

%\begin{comment}
Answer.  Central banks manage short-term interest rates through
``open market operations'': buying and selling government
securities. Selling securities, for example, reduces the amount of
currency in private circulation, which generally increases
short-term interest rates.  [Insert T-accounts here.] The story we
tell is that this reduces the liquidity of capital markets by
reducing the quantity of currency in circulation.
%\end{comment}

\end{enumerate}
\end{enumerate}

%\pagebreak \phantom{bla} \pagebreak \phantom{bla}


\vfill \centerline{\it \copyright \ \number\year \
NYU Stern School of Business}

\end{document}

\end{enumerate}
