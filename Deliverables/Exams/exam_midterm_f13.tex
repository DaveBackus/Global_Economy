\documentclass[letterpaper,12pt]{exam}

\usepackage{ge13}
\usepackage{comment}
\usepackage{booktabs}
\usepackage{hyperref}
\urlstyle{rm}   % change fonts for url's (from Chad Jones)
\hypersetup{
    colorlinks=true,        % kills boxes
    allcolors=blue,
    pdfsubject={NYU Stern course GB 2303, Global Economy},
    pdfauthor={Dave Backus @ NYU},
    pdfstartview={FitH},
    pdfpagemode={UseNone},
%    pdfnewwindow=true,      % links in new window
%    linkcolor=blue,         % color of internal links
%    citecolor=blue,         % color of links to bibliography
%    filecolor=blue,         % color of file links
%    urlcolor=blue           % color of external links
% see:  http://www.tug.org/applications/hyperref/manual.html
}

\newcommand{\NX}{\mbox{\em NX\/}}
\newcommand{\POP}{\mbox{\em POP\/}}

% list spacing
\usepackage{enumitem}
\setitemize{leftmargin=*, topsep=0pt}
\setenumerate{leftmargin=*, topsep=0pt}

\usepackage{attachfile}
    \attachfilesetup{color=0.5 0 0.5}

\def\ClassName{The Global Economy}
\def\Category{David Backus}
\def\HeadName{Midterm Examination}

%\printanswers

\begin{document}
\parindent = 0.0in
\parskip = 0.75\bigskipamount
\thispagestyle{empty}%
\Head

\centerline{\large \bf \HeadName}%
\centerline{Revised:  \today}

\bigskip
You have 90 minutes to complete this exam.  Please answer each
question in the space provided and show all of your work.
You may consult one page of notes and a calculator,
but devices capable of wireless transmission are prohibited.

I understand that the honor code applies: I will not lie, cheat,
or steal to gain an academic advantage, or tolerate those who do.

\begin{flushright}
\rule{4in}{0.5pt} \\ (Name and Signature)
\end{flushright}

%\medskip
\begin{table}[h]
    \centering
    \tabcolsep = 0.3in
    \begin{tabular}{lrr}
    \toprule
    Year            &  $ Y/L $   &  $K/L$   \\
    \midrule
    1990            &   2.489  &   2.396   \\
    2011            &   6.582  &  24.722   \\
    \bottomrule
    \end{tabular}
    \caption{Aggregate data for Vietnam.
    $Y$ is real GDP, $K$ is the stock of capital, and $L$ is the number
    of people working.
    The numbers are thousands of 2005 US dollars.
    Source:  Penn World Table, Version 8.0.}
    \label{tab:vietnam}
\end{table}



\begin{questions}
% ======================================================================
\question {\it Prospects for Vietnam (40 points).\/}
You have been asked to provide a short overview of Vietnam's economic prospects
for a group of international business leaders who have come to
Washington DC for the annual IMF meetings.
To many in the United States, Vietnam brings up painful memories of war and Oliver Stone movies,
but its economic performance over the past two decades has been extraordinary.
GDP per capita remains low at about 3600 USD,
but its GDP growth rate since 1990 has been over 7 percent per year.

In preparing your report,
you start with some basic data from the Penn World Table,
reported in Table \ref{tab:vietnam}.
Then you turn to the Economist Intelligence Unit,
where you find (quotes from reports, edited lightly for continuity,
with additional data from other sources):
\begin{itemize}
\item The Communist Party remains the dominant political force.
The government, the army, and the bureaucracy are subordinate to it.
\item Vietnam's economic success over the last two decades followed the Party's 1986
commitment to ``economic renovation.''
By the end of the 1980s economic reforms had become part of the new consensus.
%\item Its transportation and communications infrastructure
%have improved markedly, but
%even with substantial investments it has been difficult to keep up with
%rising demand.
\item Regulatory conditions and market access for foreign investors have continued to
improve in recent years, reflecting Vietnam's efforts to fulfil its commitments on
accession to the World Trade Organisation (WTO) in January 2007.

%\item The population is highly literate, but mostly unskilled.
%The average education of the adult population is now 6.3 years of school.
%Primary and secondary school enrollment, however, is close to 100\%.

\item The EIU democracy index puts Vietnam among the most authoratarian
regimes in the world.
In the functioning of government, however, Vietnam fares much better,
and the public's confidence in the government and the party is high.
Vietnam also scores well in the degree of societal consensus and cohesion.
The judiciary is relatively weak and not independent of the
Communist Party.

\item
The World Bank's Doing Business now ranks Vietnam 99th (of 185) on overall ease of doing business, which is below China (91) and above India (132).
The World Bank's Governance Indicators rank Vietnam in the 43rd percentile
on overall government effectiveness and the 39th percentile on rule of law.
The Heritage Foundation rates Vietnam 15 (of 100) on rule of law
but 64 on business freedom and 79 on open markets.

\item A recent slowdown in growth coincided
with growing evidence of corruption
and rising prices gas, food, and utilities,
which led the prime minster to make a public apology.
The trouble stems in large part from the Communist Party's
failure to discipline state-owned enterprises,
which account for 40\% of output, and to clean up bad debts lurking in state-owned banks.
Some state-owned firms are reported to be so cash-strapped
they have stopped paying workers.

\item There has been debate in the party
between conservative hard-liners pushing for stability
and reformist moderates who would like to see further liberalization,
including privatization of state-owned firms.
One Western diplomat says the question now isn't whether
real reform will happen, but how fast.
\end{itemize}
%
\begin{parts}
\item Using the data in the table,
compute the continuously-compounded annual growth rate of GDP per worker
for the period 1990-2011.
(5~points)

\item Identify the sources of growth in Vietnam over the same period.
(This is an indication that you should do the usual growth accounting calculations.
Be sure to include calculations of productivity.)
What is the primary source of growth?
(15~points)

\item Between 1990 and 2011, average years of schooling for adults
rose from 5.3 to 6.3.
By how much would you estimate this increased GDP?
(5~points)

\item Continued high performance in developing countries is often
connected to continued improvement in institutions.
Which institutions are important, in general?
From what you've read above, how does Vietnam rate on them?
Using the information presented, and your own good judgement,
how attractive do you think Vietnam will be over the next ten years
to international firms looking to do business there?
(15~points)
\end{parts}

\begin{solution}
Brief answers follow,
but see also the attached spreadsheet:
download this pdf file, open it with the Adobe Reader or the equivalent,
and click on the pushpin:
\attachfile{exam_midterm_f13_answerkey.xlsx}

The calculations for (a,b) are summarized here:
%
\begin{center}
    \tabcolsep = 0.2in
    \begin{tabular}{lrrr}
    \toprule
    Year            &  $ Y/L $   &  $K/L$  & $A$  \\
    \midrule
    1990            &   2.489  &   2.396  & 1.860  \\
    2011            &   6.582  &  24.722  & 2.259 \\
    Growth rate     &   4.631  &  11.114 & 0.926 \\
    Contribution to growth
                    &   4.631  &  3.705 & 0.926 \\
    \bottomrule
    \end{tabular}
\end{center}
The row labeled ``contribution'' is the same as the previous
one, except that the growth rate of capital per worker has
been multiplied by $\alpha = 1/3$.

\begin{parts}
\item The continuously-compounded annual rate of growth is
\begin{eqnarray*}
    \gamma_{Y/L} &=& [\ln (6.582) - \ln(2.489)]/(2011-1990) \;\;=\;\; 4.631 \%.
\end{eqnarray*}

Grading:  5 points for correct calculation.

\item Growth accounting involves this equation:
\begin{eqnarray*}
    \gamma_{Y/L} &=& (1/3) \gamma_{K/L} + \gamma_{A} .
\end{eqnarray*}
With the numbers above, we have (in percentages)
\begin{eqnarray*}
    4.631 &=& (1/3) 11.114  + 0.926  \;\;=\;\; 3.705  + 0.926.
\end{eqnarray*}
We see, unusually, that most of the growth here has come from
an enormous increase in the capital stock.
The contribution of productivity, which is usually the most important one,
is relatively small.

Looking to later questions, this could be a source of concern:
that we're not seeing the increase in productivity we'd expect to see.

Grading:  3 points for noting the growth accounting equation,
3 for the productivity numbers,
3 for the growth rates of capital per worker and productivity,
3 for multiplying the former by one-third,
and 3 more for interpreting the results sensibly.

\item This is a more difficult question, designed for the aficionados.
At an intuitive level, you might guess that increasing the skill of the labor force
would raise output.
More formally, you might use a version of the production
function that includes ``human capital'' $H$, the term we use for skill:
\begin{eqnarray*}
    Y &=& A K^\alpha (HL)^{1-\alpha} .
\end{eqnarray*}
You could go further and describe the connection between skill and education,
as we do in the text, but that's not necessary here.

Grading:  2 points for noticing that this would raise the skill level of workers,
and therefore raise output.
3 points for a more formal analysis like the one above.

\item The standard list of institutions is governance, property rights, rule of law,
and open competitive markets.
Possible comments include:
%
\begin{itemize}
\item Governance.
The view expressed by Madison is that the government has to be strong enough to
control misbehavior by the governed, but not so strong to encourage misbehavior by itself.
A one-party system provokes some concern about the latter, but the public's high degree
of confidence in the government suggests it has been responsive to their concerns.
The apology for corruption etc is an example.
Ditto the mention of business freedom.

\item Property rights are an issue.  Vietnam ranks poorly here,
a common byproduct of a communist past.

\item Rule of law is a similar source of concern.  Note, for example,
the absence of an independent judiciary.
Despite this, Heritage ranks it highly on business freedom.

\item Open competitive markets.
Heritage ranks it highly on open markets.
The major concern here is with the state-owned firms,
which are likely to have superior access to funding
and perhaps monopoly positions in things like utilities.
The optimistic scenario expressed by the unnamed diplomat
is that reforms will make even these markets
more competitive in the future.

\end{itemize}
Grading:  5 points for the list of institutions, 10 for some sensible assessment of them.
\end{parts}
\end{solution}


%\pagebreak \phantom{xx} \pagebreak \phantom{xx} \pagebreak
\begin{table}[h]
\centering
\tabcolsep = 0.1in
\begin{tabular}{lrrr}
\toprule
Indicator & Zambia &  Botswana & Tanzania \\
\midrule
\multicolumn{2}{l}{\it General} \\
GDP per capita  (2005 USD) &  1690  & 11,300 & 1250  \\
Population (millions)      &   13.5 & 2.0 &  44.9   \\
Doing Business overall (percentile) & 49  & 68 & 27 \\
World Economic Forum overall (percentile) & 37 & 50 & 16\\
\midrule
\multicolumn{2}{l}{\it Governance} \\
Political stability (percentile)  &  66 & 88 & 48 \\
Govt effectiveness (percentile)   &  38 & 68 & 28 \\
Regulatory quality (percentile)   &  37 & 74 & 37 \\
Rule of law (percentile)          &  42 & 75 & 34 \\
Control of corruption (percentile)&  46 & 79 & 22 \\
\midrule
\multicolumn{2}{l}{\it Labor} \\
Minimum wage (USD per month) &   76 & 92 & 52 \\
Severance after 10 years (weeks of pay) & 87 & 36 & 10 \\
Labor market efficiency (percentile) & 37 & 68 & 67  \\
Literacy (percent of adults)        & 71 & 84 & 73 \\
Years of school (adults)        & 6.7 & 9.6 & 5.8  \\
\midrule
\multicolumn{2}{l}{\it Infrastructure and trade} \\
Infrastructure quality (percentile)  & 20 & 36 & 10 \\
%\midrule
%\multicolumn{2}{l}{\it International trade} \\
Import documents required (number) & 8 & 5 & 6\\
Import delay (days) &  56 & 37 & 31  \\
Import cost (USD per container) &  3600 & 3500 & 1600 \\
\bottomrule
\end{tabular}
\caption{Economic and institutional indicators.
Percentiles range from 0 (worst) to 100 (best).
Sources:  Penn World Table, World Economic Forum, World Bank, Doing Business.}
\label{tab:zambia}
\end{table}

% ======================================================================
\question {\it Zambeef looks for opportunities (20~points).\/}
Zambeef, the Zambia-based meat distributor, is looking for new opportunities.
The Economist reports:
``Zambeef operates meat counters at all 20 Shoprite stores across Zambia
as well as its newer outlets in Ghana and Nigeria.
Zambeef also has around 100 shops of its own.
The CEO notes that with markets targeting both low and high-income consumers,
they are able to sell `all of the animal.'
The firm is also vertically integrated;
its `farm-to-fork' model includes farms, retail outlets,
and `cold chain logistics' with its fleet of 78 refrigerated trucks.
The downside of recent expansion, they say, is the demands on its managers.''

Zambeef is now looking to expand, either in Zambia or in nearby countries.
As a consultant based in Johannesburg,
you have been asked to advise Zambeef on the strengths and weaknesses
of neighboring Botswana, Tanzania, and Zambia.
You quickly summarize various measures of institutional quality in the three countries;
see Table \ref{tab:zambia}.
You also turn to the World Economic Forum's Global Competitiveness Report,
which includes a survey of business leaders and tabulates the most commonly
reported problems.
For the three countries, the most common complaints were
\begin{itemize}
\item Zambia:  access to financing, corruption, and inadequate infrastructure.
\item Botswana:  poor work ethic of labor force, inefficient government bureaucracy,
and access to financing.
\item Tanzania:  access to financing, corruption, and inadequate infrastructure.
\end{itemize}


\begin{parts}
\item Describe the features of an economy that are important to
operating a business like Zambeef's.
How do Botswana and Tanzania compare to Zambia on these features?
(10~points)
\item
What issues raise the most concern in each country?
How might you deal with them?
What location(s) would you recommend?
(10~points)
\end{parts}

\begin{solution}
This is a more qualitative question, and a somewhat fuzzy one at that,
but here are some of the things a good answer might mention.
A good answer should put some structure on the analysis,
not simply list what's in the table.
%
\begin{parts}
\item Zambeef is deeply imbedded in any country in which it operates.
Unlike, say, manufacturing for export or business-process outsourcing,
they have to deal with the full range of political and economic conditions.
Moreover, food is typically a regulated business, so they'll have to deal with that.
As a rough guide:
\begin{itemize}
\item Market size.
Tanzania has the largest population by a wide margin.
Zambia is next.
Botswana is smaller, but has much higher income per person.

\item Overall business conditions.
Both DB and the WEF rate the countries in this order:
Botswana $>$ Zambia $>$ Tanzania.
The same for political stability,
government effectiveness, rule of law, and control of corruption.
That falls roughly in line with GDP per capita.
All of these things are important to us.
If, say, meat inspectors are corrupt, that's a problem for the business
--- or at least an extra cost.

\item Labor market.
We have limited data here, but obviously we need to hire people.
The worst labor market seems to be Zambia, where the business seems to operate fine,
so the other countries must be doable.
The minimum wages seem manageable.
It's not clear how important literacy is, but there's not a huge difference
across countries.
One concern is the comment about the poor work ethic in Botswana.

\item Infrastructure.
Botswana is again the strongest of the three and Tanzania the weakest.
Yet you wonder, given that the business works in Zambia,
whether this is something that can be overcome.

\item Trade restrictions.
These seem to be roughly similar, with Zambia the worst.
Curiously, this data suggests it's cheaper to import into
Tanzania than other countries.
You might want to check with Zambeef and see whether that's relevant.
\end{itemize}

Grading:  10 points for a clear list of issues and
a logical argument that connects
the institutions to the demands of the business.
Partial credit for part thereof.

\item If this were real life, we'd have a series of questions we'd want to address
with more information.
Among them:
\begin{itemize}
\item Zambia:  Are there more opportunities?
Or is ``all 20'' at Shoprite the limit?
Are the ``demands on managers'' simply a sign of growth,
or does it reflect something about the supply of the right kind of talent?

\item Botswana:  How concerned should we be about the work ethic?
Does the higher income suggest a different market segment than we're used to?

\item Tanzania:  Is the poor infrastructure manageable?
Should we be concerned about government effectiveness and corruption?
\end{itemize}
All three countries look doable given these numbers,
and the fact that this business works in Zambia.
Two other things we'd want to look at are
the existing competition in each location and whether
Shoprite or another partner might facilitate entry.

Grading:  10 points for a logical argument
that flows from your earlier analysis and identifies the key issues.
\end{parts}
\end{solution}

%\pagebreak \phantom{xx} \pagebreak %\phantom{xx} \pagebreak
% ======================================================================
\question {\it Short questions (50 points).\/}
%
\begin{parts}
% http://research.stlouisfed.org/fred2/series/LNS12300002
\item In 1960, 41\% of women aged 25-54 in the US were employed.
In the most recent data, the number is 69\%.
What is the likely impact on US GDP growth over this period?
What is the likely impact on the US employment rate?
(10~points)

\item Suppose Apple's Irish assembly plant
took 230 (millions of euros)
in parts produced in Asia and hired workers for 80
to assemble products worth 320, which were then sold  in the US.
What is the value-added of the plant?
Which of these transactions appears as an expenditure
in the Irish national accounts?
(10~points)

\item Describe, in Ricardo's model, how --- and why ---
free trade affects productivity.
(10~points)

\item Consider the statement:
``When financial institutions fail, we should let them fail.''
Do you agree or disagree?  Why?
(10~points)

\item Consider the statement:  ``Germany's high investment rate
is supported, in large part, by flows of capital from other countries.''
Do you agree or disagree?  Why?
(10~points)
\end{parts}

\begin{solution}
\begin{parts}
\item An increase in women working should lead to
(i)~more output (via the production function)
and (ii)~an increase in the employment rate
(the ratio of the number of people employed to the adult population).

Grading:  5 points for noting the increase in output, including mention
of the production function;
5 points for noting the increase in the employment rate,
including a statement of what that is.

\item Value-added is $ 320 - 230 = 90$.
Expenditure components are imports (230) and exports (320).

Grading:  5 points for value added, 5 for imports and exports.

\item Free trade changes which good are produced:
we shift people to the most productive sector, which raises overall productivity.

Grading:  10 points for this or the equivalent.

\item We have two opposing objectives:
(i)~create a good set of incentives for financial institutions
(ie, let bad ones fail)
and (ii)~protect the rest of the economy from the collateral damage a
failing financial system would do to them
(ie, prop up financial institutions in times of trouble).
Most financial regulation is an attempt to balance these two things.

Grading:  5 points each for noting (i) and (ii),
or the equivalent.

\item Disagree --- it's opposite.
We saw that Germany's saving rate is greater than its investment rate,
so capital flows out of Germany to other countries, not from them.

Grading:  10 points for this.

\end{parts}
%
\end{solution}

\end{questions}

%\pagebreak \phantom{xx} %\pagebreak \phantom{xx}

\vfill \centerline{\it \copyright \ \number\year \ NYU Stern
School of Business}

\end{document}
