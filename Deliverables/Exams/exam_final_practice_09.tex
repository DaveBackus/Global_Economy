\documentclass[letterpaper,12pt]{article}

\usepackage[hypertex]{hyperref}
\RequirePackage{GE05}
% this inputs graphicx, too
\RequirePackage{comment}

\newcommand{\GDP}{\mbox{\em GDP\/}}
\newcommand{\NDP}{\mbox{\em NDP\/}}
\newcommand{\GNP}{\mbox{\em GNP\/}}
\newcommand{\NX}{\mbox{\em NX\/}}
\newcommand{\NY}{\mbox{\em NY\/}}
\newcommand{\CA}{\mbox{\em CA\/}}
\newcommand{\NFA}{\mbox{\em NFA\/}}
\newcommand{\Def}{\mbox{\em Def\/}}
\newcommand{\CPI}{\mbox{\em CPI\/}}
\newcommand{\phm}{\phantom{--}}

\def\ClassName{The Global Economy}
\def\Category{Professor David Backus}
\def\HeadName{Practice Final Examination 1}

\begin{document}
\parindent = 0.0in
\parskip = \bigskipamount
\thispagestyle{empty}%
\Head

\centerline{\large \bf \HeadName}%
\centerline{Revised:  \today}

\bigskip
You have 100 minutes to complete this exam.  Please answer each
question in the space provided. You may consult one page of notes
and a calculator, but devices capable of wireless transmission are
prohibited.

I understand that the honor code applies: I will not lie, cheat,
or steal to gain an academic advantage, or tolerate those who do.

\begin{flushright}
\rule{4in}{0.5pt} \\ (Name and Signature)
\end{flushright}

%\bigskip
{\it Most of these questions come from old exams and may include old data as a result. }

\begin{enumerate}
% ************************************************************************
\item {\it US monetary policy (25 points).}
After its April 2007 meeting,
the US Federal Open Market Committee stated:
%
\begin{quote}
Recent indicators have been mixed. ...
Nevertheless, the economy seems likely to continue
to expand at a moderate pace over coming quarters.
%
Recent readings on core inflation have been somewhat elevated.
Although inflation pressures seem likely to moderate over time,
the high level of resource utilization has the potential
to sustain those pressures.
%
In these circumstances, the Committee's predominant policy concern remains the risk that inflation will fail to moderate as expected. Future policy adjustments will depend on the evolution of the outlook for both inflation and economic growth, as implied by incoming information.
\end{quote}
%
\begin{enumerate}
\item Do you agree with  the Committee's ``predominant policy concern''?
Why or why not?
(5~points)
\item Use the aggregate supply and demand framework
to illustrate how the FOMC might think about monetary policy.
Consider each of the following questions in turn:
(i)~How would an increase in the money supply affect
output and prices in the short run?
(ii)~In the long run?
(iii)~How does the phrase ``high level of resource utilization''
affect your answer?
(15~points)
\item If future information suggests higher inflation,
what would you expect to happen to interest rates?
(5~points)
\end{enumerate}

\begin{comment}
Answer.
\begin{enumerate}
\item Inflation is the predominant concern,
although they  mention economic growth
(``expand at a moderate pace'').
%The second question is an opportunity for you to bring in your own
%perspective, possibly including your analysis from Group Project \#6.
My take:  inflation is rightly the Fed's primary concern in most
situations, esp now given the inflation of the last two years. Why?
Because it has limited control over growth, and experience tells us
that if inflation rises substantially, we tend to have below-average
growth.

\item Draw a diagram with AS, AD, and AS$^*$.
%Note in particular where the short-run equilibrium (where AS and AD cross)
%is relative to AS$^*$.
Assume for the time being that we start at a long-run equilibrium
(where AS crosses AD and AS$^*$ at the same time). (i)~AD shifts
right/up. In the short run, we move to where AS and AD cross (and
ignore AS$^*$). This raises output and prices (inflation). (ii)~In
the long-run, AS also shifts, to the point where AD and AS$^*$
cross. Why? Because the sticky wages eventually adjust. The result:
prices rise, but output stays the same, relative to our starting
point. This illustrates the difference between the short-run and
long-run effects of monetary policy:  In the long run, all we get is
inflation.  In the short-run, we get a combination of higher prices
and higher output. Which sends us back to (a):  one of the questions
the Fed must address is how large the short-run increase in output
is.  That, in turn, depends on how steep the AS curve is:  the
steeper the curve, the smaller the increase in output. [For
practice, contrast this analysis with one where we start to the left
of AS$^*$.]
%
(iii)~If you review our earlier answer, you'll see that the long-run
impact depends on where the short-term equilibrium is relative to
AS$^*$. This statement suggests that the FOMC sees the economy as
close to AS$^*$.

\item If inflation rises, the Fed would typically raise
the target fed funds rate. You can see this in the Taylor rule, for
example. This also brings us back to (a), since ultimately the FOMC
policy will depend on both inflation and output news.

\end{enumerate}
\end{comment}

\item {\it Prospects for Spain (30 points).\/} 
While much of Europe struggles, Spain has been a continuing economic 
success, with growth rates well above the EU average.  
As an analyst in Deutsche Bank's London office, 
you wonder whether this trend is likely to continue.  
You look at the data:  

\begin{table}[h]
\vspace{1em}%
\centering%
\hspace{-3cm}%
\begin{minipage}
{0.52\textwidth}%
\begin{center}{\small
\begin{tabular}{lccccccc}%
\vspace{-0.6cm}\\
%\multicolumn{2}{c}{Seedorf Bank} \\%
\hline%
\vspace{-.3cm}\\
                     & 2000  &  2001  &  2002   & 2003  & 2004 &  2005  &  2006 \\%
\vspace{-.3cm}\\
\hline%
\vspace{-.2cm}\\
Real GDP Growth      &  5.0   &   3.6  &   2.7   &  3.0  &  3.3  &   3.5  &  3.9 \\%
Population Growth    &  0.6   &   1.1  &    1.2  &  2.0  &  2.2  &   2.3  &  2.2 \\
TFP Growth           &  --0.1 &  --0.4 &   --0.4 &  --0.7&  --0.5&   --1.1&  --0.2\\%
Interest rate (3m)   &  4.1 &  4.4  &  3.3  &  2.3  & 2.0   &  2.1  &  2.8 \\
Inflation            &  3.4 &  3.6  &  3.5  &  3.0  &  3.0  &  3.4  &   3.5 \\
Investment*      &  25.9  &   26.2 &   26.2  &  27.1 &  27.4 &   28.4 &  29.1\\%
Saving*          &  22.3  &   22.4 &   23.4  &  23.9 &  23.0 &   22.2 &  21.8\\%
Budget Balance*       &  --0.9 &  --0.5 &   --0.3 &  --0.1&  --0.2&    1.1 &  1.4 \\%
Primary Balance*      &  2.1 &    2.2 &   2.1  &  2.0 &  1.7  &  2.7 &  2.8  \\%
Public Debt*          &  59.2  &  55.6  &   52.5  &  48.8 &  46.2 &   43.1 &  39.8\\%
Current Account*      &  --4.0 &  --3.9 &   --3.2 &  --3.5&  --5.3&   --7.4&  --8.8\\%
Net Foreign Assets*   & --74.9  & --77.1  & --92.4  & --97.2 & --90.1 & --86.3 &  --81.3 \\%
\hline%
\vspace{-3mm}\\
\end{tabular}
}
\end{center}
\end{minipage}
\end{table}
Note: Variables marked by * are ratios to GDP (nominal to nominal).  

\begin{enumerate}

\item What do you see as the likely source(s) of GDP growth 
over the recent past?  
(5~points)

\item Does the budget balance concern you?  
Why or why not?  (10~points)

\item The current account deficit is currently one of the largest 
among developed countries.
Does that concern you?  Be as specific as possible in your answer. 
(10~points)

\item What is your overall assessment of the prospects for Spain
for the next 2-3 years?  
Please mention any issues that you think might call for a closer look. (5~points)

\end{enumerate}

\begin{comment}
Answer.
\begin{enumerate}
\item Lack of TFP growth is a concern.  
Growth must be coming from increases in $K$ and/or $L$.  
\item The government is running not only a primary surplus, 
but a total surplus, with the result that 
the debt-to-GDP ratio has fallen dramatically.  
You could formalize this with our debt dynamics relation, 
but the answer is clear.  
\item This is worth a closer look:  the deficit and net foreign assets
are both large.  
Are they getting worse?  
Consider the relation, 
\[
    \frac{\mbox{\em NFA}_{t}}{Y_{t}} \;=\; \left( \frac{1+i}{1+g} \right)
     \frac{\mbox{\em NFA}_{t-1}}{Y_{t-1}} 
     + \frac{\mbox{\em NX}_{t}}{Y_{t}} .
\]
We don't have net exports, but we can try the closely related current account.
With $ i = 2.8\%$ and $ g = 3.9 + 3.5 = 7.4\%$, 
the numbers tell us that next year's ratio of net foreign 
assets to GDP will fall to about:
\[
    (1.028/1.074) \times (-81.3) - 8.8 \;=\; -86.6 .
\]
In short, foreign borrowing is going up. 
(If we had NX data, the change would probably be less, because 
the current account includes interest payments.) 
The question is why. 
A deeper analysis would look into the kinds of borrowing 
(equity or debt?  firms or housing?) and its origins.  

\item Personally, I think they're good.  
However, lack of TFP growth is an obvious concern.  
Ditto the current account deficit, 
which could reflect either good or bad news, 
depending on the source. 

\end{enumerate}
\end{comment}


% ****************************************************************************
\item {\it Miscellany (50 points).}

\begin{enumerate}
\item {\it Chinese crisis?\/}
An analyst suggested that China may suffer a currency crisis
along the lines of Mexico in 1994-95,
in which the peso fell sharply when the Banco de Mexico
ran out of foreign currency reserves.
Give two reasons why this scenario is likely or unlikely.
 (10~points)

\item {\it Government deficits.\/}
Can a country run a fiscal (government) deficit forever?
Why or why not?
 (10~points)


\item {\it Canadian inflation.\/}
In Canada over the last year, inflation has been 2.3\%
and money growth  has been 11.8\%.
Do you find the difference between the two numbers surprising?
Why or why not?  (10~points)

\item {\it Leading indicators.\/}
Explain what a leading indicator is
and give an example for the US.  (10~points)

\item {\it Monetary policy mechanics.\/}
Use the central bank's balance sheet to describe 
how it 
maintains the short-term interest rate at a specific level.
(10~points)

\end{enumerate}
\begin{comment}
Answers.
\begin{enumerate}
\item 
(i)~China has enormous foreign exchange 
reserves:  they won't run out any time soon.
(ii)~The renminbi seems to be undervalued:  
people want to buy it,
not sell it, which results in the central bank accumulating reserves,
not losing them.

\item 
The present value of future primary surpluses has to equal
the current debt.  Thus past deficits must be balanced by future
surpluses --- you can't run a primary deficit forever. The key word
is primary:  you can run a primary surplus and an overall deficit at
the same time, as we see in (for example) Turkey.

\item 
Under the quantity theory, inflation equals money growth
minus real GDP growth. Unless real GDP growth is 9\%, something's
wrong. What's wrong is that this relation is well-known not to work
in the short run.  Over periods of several years, however, it
typically works pretty well.

\item 
A leading indicator (of the economy)
is an observable economic variable whose ups and
downs precede those in (say) real GDP.
You can see this in the cross-correlation function, for example.
Common examples:  housing starts, stock market indexes,
interest rate spread (long minus short).

\item 
Central banks manage short-term interest rates through
``open market operations'': buying and selling government
securities. Selling securities, for example, reduces the amount of
currency in private circulation, which generally increases
short-term interest rates.  [Insert T-accounts here.] The story we
tell is that this reduces the liquidity of capital markets by
reducing the quantity of currency in circulation.

\end{enumerate}
\end{comment}

\end{enumerate}


%\pagebreak \phantom{bla} \pagebreak \phantom{bla}


\vfill \centerline{\it \copyright \ \number\year \
NYU Stern School of Business}



%**********************************************************************
%**********************************************************************
\newpage
\def\HeadName{Practice Final Examination 2}
\parindent = 0.0in
\parskip = \bigskipamount
\setcounter{page}{1} \thispagestyle{empty}
\Head

\centerline{\large \bf \HeadName}%
\centerline{Revised:  \today}

\bigskip
You have 100 minutes to complete this exam.  Please answer each
question in the space provided. You may consult one page of notes
and a calculator, but devices capable of wireless transmission are
prohibited.

I understand that the honor code applies: I will not lie, cheat, 
or steal to gain an academic advantage, or tolerate those who do.

\begin{flushright}
\rule{4in}{0.5pt} \\ (Name and Signature)
\end{flushright}

{\it Most of these questions come from old exams and may include old data as a result. }

\begin{enumerate}
% ************************************************************************
\item {\it Chinese foreign exchange intervention (30 points).}
Joseph Yam, Chief Executive of the Hong Kong Monetary Authority, wrote about the foreign exchange activities of the People's Bank of China (PBC) in January 2007:  
%
\begin{quote}
The accumulation of foreign exchange reserves 
involves the PBC buying foreign assets through creating renminbi.
As a result, the monetary base is increased, creating a need to ``sterilise.''
This is done by issuing paper [short-term notes] to the banks.  
The PBC obviously has to pay interest on the money borrowed.  
Currently the yield of, for example, three-month paper issued by the PBC is about 2.5\%.
This is lower than the yield on foreign assets held as reserves ---
the yield on US treasuries is about 4 to 5\% --- so theoretically
reserve accumulation can be profitable.  
The problem, however, is the continuing appreciation of the renminbi, 
which gradually reduces the value of those foreign assets in renminbi terms.
\end{quote}

\begin{enumerate}
\item Use the central bank's balance sheet to show how purchases of
foreign currency increase the monetary base (think:  supply of currency). 
(10~points)  

\item Show how sterilization can be used to reverse the impact on 
the supply of currency.
(10~points)

\item You may note that interest rates have now flipped, with Chinese interest rates above US interest rates.  
    What does this imply for the returns on the PBC's balance sheet?
    How might it avoid this outcome?      
\end{enumerate}

\begin{comment}
Answer. 
\begin{enumerate}
\item A typical central bank balance sheet looks something like this:
%
\begin{center}
\begin{tabular}{lrclr}
               Assets  &     &&     Liabilities                     \\  
               \hline 
               FX Reserves &  100 &&     Monetary Base &  200   \\    
               Bonds   & 100 && \\
\end{tabular}
\end{center}
%
Monetary base is (roughly) another term for currency.  
Suppose, now, that the PBC has to purchase another 100 worth of 
foreign currency (or foreign-currency denominated bonds) 
and issues currency in return.  
Then the balance sheet becomes 
%
\begin{center}
\begin{tabular}{lrclr}
               Assets  &     &&     Liabilities                     \\  
               \hline 
               FX Reserves &  200 &&     Monetary Base &  300   \\    
               Bonds   & 100 && \\
\end{tabular}
\end{center}
%
This is the impact of the accumulation of reserves on the monetary base described by Yam.  


\item To undo the increase in the monetary base, the PBC will issue 100 worth of bonds and accept money in return:  
%
\begin{center}
\begin{tabular}{lrclr}
               Assets  &     &&     Liabilities                     \\  
               \hline 
               FX Reserves &  200 &&     Monetary Base &  200   \\    
               Bonds   & 0 && \\
\end{tabular}
\end{center}
%
This operation is referred to as sterilization.  
For the PBC, this process has continued to the extent that they
have had to issue bonds as liabilities, so that their balance sheet now looks something like this:  
%
\begin{center}
\begin{tabular}{lrclr}
               Assets  &     &&     Liabilities                     \\  
               \hline 
               FX Reserves &  800 &&     Monetary Base &  200   \\    
               Bonds   &  0 &&    Bonds  &  600 \\
\end{tabular}
\end{center}
%
The bond liabilities (so-called ``sterilization bonds'') 
are issued primarily to Chinese commercial banks, 
as outlined above.  
Note, as Yam does, that assets and liabilities are equal.  

\item The PBC is now likely losing money on its portfolio 
for two reasons:
(i)~because the rate it's paying on liabilities is greater than 
the rate received on assets and 
(ii)~continued increases in the value of the renminbi raise 
the value of its liabilities relative to its assets.  
That's one of the reasons they have set up a sovereign wealth fund:
to generate higher returns in assets.  
In a sense, they have the wrong side of the carry trade.  

How could it avoid this outcome?  
Let the currency float and stop buying foreign currency.
This is entirely the byproduct of managing the exchange rate.  

\end{enumerate}
\end{comment}


%\pagebreak \phantom{bla} \pagebreak \phantom{bla} \pagebreak
% ---------------------------------------------------------
\item {\it Globalization and inflation (20 points).\/}
Fed Chairman Bernanke said recently (March 2007):  
\begin{quote}
As national markets become increasingly integrated and open, sellers of goods, services, and labor may face more competition and have less market power than in the past. 
These linkages suggest that, at least in the short run, globalization and trade may affect the course of domestic inflation.
\end{quote}
%
\begin{enumerate}
\item Use aggregate supply and demand to 
describe how expansionary monetary policy 
affects output and inflation in the short run.  
(10~points)

\item Back to Bernanke:  How would you represent the impact of 
globalization (think:  imports from China) 
in the aggregate supply and demand diagram?  
How does globalization change the impact of expansionary monetary 
policy in this model?
Do you find the model persuasive in this respect?  
(10~points)
\end{enumerate}


\begin{comment}
Answer.
\begin{enumerate}
\item This would increase both output and prices (inflation, 
loosely speaking).  Your answer should show a diagram with 
AS, AD, and AS$^*$.
AD shifts out, with the stated result.  

\item The typical argument is that AS has become flatter.
As a result, expansionary monetary policy has a 
smaller impact on inflation, and larger impact on output. 

\end{enumerate}
\end{comment}


%\pagebreak \phantom{bla} \pagebreak

\item {\it Miscellany (50 points).}

\begin{enumerate}

\item {\it Exchange rates.\/}
{\it The Economist\/} reports that a Big Mac costs \$2.90 in the US, \$3.28 in the eurozone,
and \$2.33 in Japan.  (These prices are averages for the various regional markets, expressed in US
dollars using current spot exchange rates.)  What does this suggest about the likely change in
value of the euro and yen v. the dollar over the coming 6 months?  6 years? (10~points)  

\item {\it Inflation.\/}
Milton Friedman once said:  inflation is always and everywhere a monetary phenomenon. 
Do you agree or disagree?  Why or why not?  (10~points)

\item {\it Employment report.\/} 
At 8:30 am on April 6, 2006, 
the US Bureau of Labor Statistics released
its closely-watched employment report, {\it The Employment Situation\/}.
Firms reported an increase of 180,000 jobs in March, 
well above the consensus of 135,000.  
Treasury yields immediately rose 5-10 basis points 
for maturities from 2 to 30 years.  
Why?  
(10~points)

\item {\it ECB policy.\/}
The European Central Bank has kept short-term interest rates
in the Euro Zone well above those in the US.
Why?  (10~points)

\item {\it Cross-correlation function.\/}
Describe the cross-correlation function and show how it can 
be used to identify promising leading indicators.
(10~points) 

\end{enumerate}

\begin{comment}
Answers.
\begin{enumerate}

\item   PPP suggests that exchange rates will adjust to make prices the same across countries. In
this case, that means the dollar will rise against the euro, fall against the yen.  Is this right?
Over short periods of time, exchange rates are close to unpredictable by any means, PPP included.
Over longer periods of time (5-20 years) PPP is a reasonable indicator.  

\item It has lots of truth in it, but I'd disagree for two reasons. 
First, it's an incomplete statement for high inflations: 
it's true, but high money growth itself typically stems from 
a government deficit.
Second, over short periods of time, the quantity theory doesn't work 
that well.  
It's entirely possible, as our AS/AD analysis implies, that 
money can have only a modest short-run impact on inflation, 
and that other demand and supply factors play a role, too.  

\item  I'd start with the Taylor rule:  indicators of high output 
lead to high interest rates.
The deeper question is why this shows up in long yields.
Certainly it will take some time to affect the Fed's choice of 
target interest rate, but the impact on the very long end is a 
typical, if somewhat mysterious, result.  

\item One reason is that the ECB, by design, 
places greater weight on inflation:  its primary goal is price stability.
Another is that they have not had the kind of financial turmoil
that has afflicted US markets and driven the Fed to 
largely abandon its own devotion to stable prices.  

\item The cross-correlation function is a plot of the correlation
of the correlation of two variables at different leads and lags
against the lead or lag.
Formally, the ccf for two variables $x$ and $y$ is a plot 
of 
\[
    \mbox{ccf}(k) \;=\;  \mbox{\it corr\/} (x_t,y_{t-k}) .
\]
against $k$.  


\end{enumerate}
\end{comment}

\end{enumerate}

%\pagebreak \phantom{bla} \pagebreak \phantom{bla}


\vfill \centerline{\it \copyright \ \number\year \ 
NYU Stern School of Business}

\end{document} 

