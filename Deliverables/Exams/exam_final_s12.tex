\documentclass[letterpaper,12pt]{exam}

\usepackage{ge05}
\usepackage{comment}
\usepackage{booktabs}
\usepackage[dvipdfm]{hyperref}
\urlstyle{rm}   % change fonts for url's (from Chad Jones)
\hypersetup{
    colorlinks=true,        % kills boxes
    allcolors=blue,
    pdfsubject={NYU Stern course GB 2303, Global Economy},
    pdfauthor={Dave Backus @ NYU},
    pdfstartview={FitH},
    pdfpagemode={UseNone},
%    pdfnewwindow=true,      % links in new window
%    linkcolor=blue,         % color of internal links
%    citecolor=blue,         % color of links to bibliography
%    filecolor=blue,         % color of file links
%    urlcolor=blue           % color of external links
% see:  http://www.tug.org/applications/hyperref/manual.html
}

\newcommand{\NX}{\mbox{\em NX\/}}
\newcommand{\POP}{\mbox{\em POP\/}}

\def\ClassName{The Global Economy}
\def\Category{David Backus}
\def\HeadName{Final Examination}

\printanswers

\begin{document}
\parindent = 0.0in
\parskip = \bigskipamount
\thispagestyle{empty}%
\Head

\centerline{\large \bf \HeadName}%
%\centerline{March 9, 2005}
\centerline{Revised:  \today}

\bigskip
You have 120 minutes to complete this exam.  Please answer each
question in the space provided and show all of your work.
You may consult one page of notes and a calculator,
but devices capable of wireless transmission are prohibited.

I understand that the honor code applies: I will not lie, cheat,
or steal to gain an academic advantage, or tolerate those who do.

\begin{flushright}
\rule{4in}{0.5pt} \\ (Name and Signature)
\end{flushright}


%\pagebreak
\begin{questions}
% ======================================================================
\question {\it Risk and opportunity in Ghana.\/}
You have been asked to prepare a report on crisis risk in
the West African country of Ghana.
You recall from your Global Economy class that
Ghana is a former British colony that has been growing rapidly
in recent years after a period of unusually stable politics.
The Economist Intelligence Unit refers to it as a ``robust democracy.''
The World Economic Forum ranked Ghana 114th (of 133)
in their Global Competitiveness Report.
They continue:
``The country continues to display strong public institutions and
governance indicators,
particularly in regional comparison.''

The EIU's Country Risk Report adds:
\begin{itemize}
\item The December 2012 elections are expected to be close.
The president, John Atta Mills, came to power promising accountability
and transparency, but  has struggled to maintain party unity
while evidence emerges of financial impropriety of some government ministers.
\item The victor faces a challenging policy environment, particularly
the fiscal situation.
\item Expectations among the population are high as production
starts at the offshore Jubilee oil field.
\item The government's decision to allow use of 70\% of future
oil revenue as collateral for borrowing is a cause for concern
if the revenue is not managed properly.
\item The Bank of Ghana (the central bank) faces the twin goals of
containing inflation and fostering growth.
\item The currency --- the cedi --- floats with occasional heavy intervention.
\end{itemize}
%
Your mission is to assess the risks to Ghana  using
the information above, the data in Table \ref{tab:ghana},
and your own good judgement and analytical skills.

\begin{parts}
\part  You decide to start with a fiscal assessment.
What trend do you see in government revenues and expenses?
(5~points)

\part You notice that neither the primary deficit nor
interest expenses are reported separately.
How would you estimate them from the numbers in the table?
What are their values for 2011?
(10~points)

\part Using what you know about government debt dynamics,
compute the ratio of government debt to GDP for 2011.
What factors contribute the most to the change from 2010?
(10~points)

\part Overall, how would you assess the risks to Ghana's economy over
the next couple of years?
(10~points)
\end{parts}

\begin{table}
\centering
%\tabcolsep = 0.2in
\begin{tabular}{lrrrrrr}
\toprule
        & 2007 & 2008 & 2009 & 2010 & 2011 & 2012 \\
\midrule
GDP growth (\%) & 6.5 & 8.4 & 4.0 & 7.7 & 13.6 & 7.4 \\
Inflation (\%)  & 12.7& 18.1 & 16.0 & 8.6 & 8.6 & 8.5 \\
Interest rate (\%) & 14.5 & 20.8 & 28.8 & 22.7 & 20.5 & 20.6  \\
Govt revenue (\% of GDP)  & 17.5 & 16.0 & 16.5 & 19.1 & 23.4 & 22.2 \\
Govt spending (\% of GDP) & 23.1 & 24.5 & 22.3 & 25.5 & 27.6 & 27.7 \\
Govt budget balance (\% of GDP) & --5.6 & --8.5 & --5.8 & --6.5& --4.2 & --5.5\\
Govt debt (\% of GDP) & 30.4 & 30.6 & 33.3 & 33.9 & {\bf } & {\bf } \\ %& 36.8 & 41.6\\
Real exchange rate (index) & 85.2 & 81.7 & 76.3 & 81.8 & 78.1 & 74.5\\
FX reserves (USD billions) & 2.6 & 1.8 & 2.9 & 4.3 & 4.4 & 4.8 \\
\bottomrule
\end{tabular}
\caption{Macroeconomic data for Ghana.
Data from EIU CountryData.
The government budget balance is a surplus if positive, deficit if negative.
The real exchange rate is the price of goods in Ghana relative to the rest
of the world;
the larger the number, the more expensive goods are in Ghana.
The numbers for 2011 and 2012 are estimates.
}
\label{tab:ghana}
\end{table}


\begin{solution}
\begin{parts}
\part Trends include:
(i)~revenues and spending both rising,
(ii)~spending still ahead of revenue (there's a deficit),
and (as a direct result)
(iii)~ratio of debt to GDP rising a little
(more on that to come).

Grading:  5 points for noting (i) and (ii).

\part Remember that interest payments in year $t$ are $ i_t B_{t-1}/Y_t$,
in which the timing of the components doesn't match.
We get what we want from:
\begin{eqnarray*}
    i_t B_{t-1}/ Y_{t} &=& i_t (B_{t-1}/ Y_{t-1}) (Y_{t-1}/ Y_{t})
            \;\;=\;\; i_t (B_{t-1}/Y_{t-1}) /(1+g_t + \pi_t) .
\end{eqnarray*}
That gives us interest payments in 2011 of 5.7\% of GDP and
a primary deficit of $-1.5$\% (that is, a surplus).

We can't compute the numbers for 2012 until we have the debt to GDP ratio
for 2011, so that's no longer part of the question.

Grading:  10 points for correct calculations.

\part The key relation is this one:
\begin{eqnarray*}
    \Delta ({B_{t}}/{Y_{t}})
            &=&
                (i-\pi) ({B_{t-1}}/{Y_{t-1}})
                - g ({B_{t-1}}/{Y_{t-1}})
             +    ({D_{t}}/{Y_{t}})  .
\end{eqnarray*}
We refer to the components on the right as A, B, and C.
Calculations in the spreadsheet give us

\begin{center}
\begin{tabular}{lrrr}
\toprule
        &  2010 & 2011 & 2012  \\
\midrule
Interest payments  &  &  4.0 & 3.9 \\
Component A (interest)  &  &  4.0 & 3.9 \\
Component B (growth)            &   & --4.6 &--2.4 \\
Component C (primary deficit)   &   & --1.5 & 1.4 \\
Total change in $B/Y$       &       &--2.1 & 2.4 \\
Public debt (\% of GDP)     &  33.9 & 31.8 & 33.2 \\
\bottomrule
\end{tabular}
\end{center}

Over this period, the ratio of debt to GDP fell by 0.7\%.
The components contributed:
interest +7.9, growth --7.0, and the primary deficit --1.7.

One thing you might note is that impact of the enormous GDP growth
rate in 2011.

Grading:  10 points for correct calculations,
including the components.

\part This is a call to look at the checklist:
\begin{itemize}
\item Government debt and deficits:  we have deficits, but there's not
much sign yet of a growth debt to GDP ratio.
One future concern might be the possibility of borrowing now against future oil
revenue.  Will any debts incurred be spent wisely?
Will the oil revenue show up?
\item Banking system.  No information on that provided.
\item Exchange rate and reserves.  Reserves are modest,
but with the exchange rate floating there shouldn't be much
concern about that.
\item Politics.  Always an issue,
especially with a contentious election coming
and the promise of money from oil revenue.
It's an odd fact but a true one that revenue
from natural resources is more likely to cause problems than solve them.
\end{itemize}

Grading:  10 points for this list of issues and a sensible
discussion of each one.



\end{parts}
\end{solution}


%\pagebreak \phantom{xx} \pagebreak \phantom{xx} \pagebreak
% ======================================================================
\question {\it Aggregate implications of employer-provided health insurance.\/}
By an accident of history, health insurance in the US is generally
provided by employers.
Suppose a sharp rise in healthcare costs leads firms to hire fewer workers.
\begin{parts}
\part How would you represent this in an aggregate supply and demand diagram?
Which curve shifts?  In which direction?
(10~points)
\part What is the new short-run equilibrium?  Long-run equilibrium?
What happens to inflation and output?
(10~points)
\part How should the central bank respond?
Be specific about its goals and how it would accomplish them.
(10~points)
\end{parts}


\begin{solution}
\begin{parts}
\part Since we're talking about firms and production,
this must involve the supply side of the model.
We shift AS and AS$^*$ to the left, both by the same amount.
See the figure below.

\begin{center}
\setlength{\unitlength}{0.075em}
\begin{picture}(250,200)(0,0)
%\footnotesize
\thicklines

% horizontal axis
\put(-30,0){\vector(1,0){300}}
\put(255,-16){$Y$}
\put(142,-16){$Y^*$}
\put(102,-16){$Y^{*\prime}$}

% vertical axis
\put(0,-20){\vector(0,1){200}}
\put(-15,155){$P$}

% demand
\put(25,165){\line(4,-3){200}}\put(230,10){AD}
%\put(65,165){\line(4,-3){200}}\put(270,10){AD$'$}

% supply
\put(65,13){\line(4,3){200}} \put(270,160){AS}
\put(25,13){\line(4,3){200}} \put(230,160){AS$'$}
\put(146.4,0){\line(0,1){170}} \put(138,175){AS$^*$}
\put(106.4,0){\line(0,1){170}} \put(98,175){AS$^{*\prime}$}

% equilibrium labels
\put(150,55){\footnotesize A}
\put(122,75){\footnotesize B}
\put(95,94){\footnotesize C}
\put(95,76){\footnotesize D}
%\put(95,54){\footnotesize D}
% dotted lines
%\qbezier[31]{(133,0)(133,46)(133,92)}
%\qbezier[45]{(0,92)(67,92)(133,92)}
%\qbezier[45]{(0,72)(67,72)(133,72)}

\end{picture}
\end{center}
\bigskip\bigskip

Grading:  10 points for noting that it's a supply shift
and reproducing the figure above.

\part We started at A.
After the shift, we move to a new short-run equilibrium at B,
where the new AS crosses AD.
Evidently output falls and prices rise.

Eventually we move to a new long-run equilibrium at C,
where AD crosses the new AS$^*$.
At this point, output has fallen more and prices have risen more.

Grading:  7 points for telling us which curves determine
 short-run and long-run equilibria and 3 points for reporting
 the implications for output and prices.

\part The central bank has two goals:  stable prices and output
at its long-run equilibrium.
Here we've moved from A to C.
We're ok at C on the second goal:  output fell,
but that's the long-run equilibrium so there's nothing monetary policy
can do about that.
(We could consider other policies, but they're not the job of the central bank.)

Where C is bad is with respect to price stability:  prices are higher.
So the central bank could shift AD to the left, giving us the same
long-run output but lower prices.
The central bank would accomplish this by reducing the money supply,
which it might do by targeting a higher interest rate.

Grading:  5 points for listing the goals and 5 for showing
what they imply in this situation.

\end{parts}
\end{solution}


%\pagebreak \phantom{xx} \pagebreak %\phantom{xx} \pagebreak
% ======================================================================
\question {\it True/false/uncertain.\/}
Please explain why each statement is true, false, or uncertain.
The explanation is essential.
%
\begin{parts}
\part The New York Times reports that Apple paid a worldwide corporate
tax rate of about 10\%, while Walmart paid about 24\%.
This difference in tax rates is good for the global economy,
because Apple is part of the high-growth technology sector.
(10~points)

\part The unemployment rate is a leading countercyclical indicator
of economic growth.
(10~points)

\part If a central bank is purchasing foreign currency,
sterilization would consist of selling bonds.
(10~points)

\part Purchases of household furniture are more cyclical
than purchases of toothpaste.
(10~points)

\part If year-on-year GDP growth rises 1 percent, you
would expect the fed funds rate to go up by roughly 1.5 percent.
(10~points)

\part Since 1983, Hong Kong has run a hard peg, with
the Hong Kong Dollar trading between 7.75 and 7.85 per US dollar.
As a result, Hong Kong inherits US monetary policy.
(10~points)
\end{parts}

\begin{solution}
\begin{parts}
\part False.  A sound principle of tax policy is to apply the same rate
to (in this case) both firms:  this is what maximizes the surplus,
minimizes the cost of distortions.
And in this case, it's arguable that Walmart has done more for aggregate
productivity than Apple.

Grading:  7 points for saying that the two firms should be taxed at the same rate,
3 for noting that this comes from some kind of welfare analysis.

\part True.  Furniture is a durable good, hence likely to be more cyclical 
than a nondurable good like toothpaste.  

Grading:  10 points for noting the durability issue.  

\part Half true.  The unemployment rate is countercyclical,
but it's a lagging indicator.

Grading:  5 points for each part.

\part True.  Here's how that might look.
The central bank starts out with the balance sheet

\begin{center}
\begin{tabular}{lr|lr}
Assets  &&  Liabilities \\
\midrule
Bonds &  10    & Money & 20 \\
FX reserves & 10
\end{tabular}
\end{center}

If it purchases foreign currency (say 5 worth) with domestic currency,
that changes to

\begin{center}
\begin{tabular}{lr|lr}
Assets  &&  Liabilities \\
\midrule
Bonds &  10    & Money & 25 \\
FX reserves & 15
\end{tabular}
\end{center}

So the money supply has gone up.
To undo that, they would sell 5 worth of bonds, taking back money in return:

\begin{center}
\begin{tabular}{lr|lr}
Assets  &&  Liabilities \\
\midrule
Bonds &  5    & Money & 20 \\
FX reserves & 15
\end{tabular}
\end{center}

Grading:  5 points for a verbal argument,
5 more if they include balance sheets like those above.

\part False.
It's an allusion to the Taylor rule,
which says (in the form we used it in class) that an increase in GDP growth
of 1\% would lead to an increase in the fed funds rate of 0.5\%.

Grading:  5 points if they note this refers to the Taylor rule,
5 more for noting the correct answer.

\part True.
It's a reference to the trilemma, where you get to choose no more than two of:
(i) fixed exchange rate, (ii)~free flows of capital, and (iii)~independent monetary policy. Hong Kong has chosen (i) and (ii), given up on (iii).

Grading:  7 points for noting the connection to the trilemma,
3 more for raising the issue of capital mobility,
which wasn't stated in the question.
It's ok if they don't know whether HK has free capital mobility,
but they need to state it as an issue.

\end{parts}
\end{solution}

\end{questions}

%\pagebreak \phantom{xx} %\pagebreak \phantom{xx}

\vfill \centerline{\it \copyright \ \number\year \ NYU Stern
School of Business}

\end{document} 