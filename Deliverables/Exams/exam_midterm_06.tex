\documentclass[letterpaper,12pt]{article}

\RequirePackage{comment}
\RequirePackage{hyperref}
\RequirePackage{GE05}
% this inputs graphicx, too

\newcommand{\POP}{\mbox{\em POP\/}}
\newcommand{\GDP}{\mbox{\em GDP\/}}
\newcommand{\GNP}{\mbox{\em GNP\/}}
\newcommand{\NX}{\mbox{\em NX\/}}
\newcommand{\NY}{\mbox{\em NY\/}}
\newcommand{\CA}{\mbox{\em CA\/}}
\newcommand{\NFA}{\mbox{\em NFA\/}}
\newcommand{\Def}{\mbox{\em Def\/}}
\newcommand{\CPI}{\mbox{\em CPI\/}}
\newcommand{\phm}{\phantom{--}}

\def\ClassName{The Global Economy}
\def\Category{Professor David Backus}
\def\HeadName{Midterm Exam}

\begin{document}
\parindent = 0.0in
\parskip = \bigskipamount
\thispagestyle{empty}%
\Head

\centerline{\large\bf \HeadName}%
\centerline{March 9, 2006; Version:  \today}
%\centerline{Revised:  \today}

\bigskip
You have 75 minutes to complete this exam.  Please answer each
question in the space provided. You may consult one page of notes
and a calculator, but devices capable of wireless transmission are
prohibited.

I understand that the honor code applies: I will not lie, cheat,
or steal to gain an academic advantage, or tolerate those who do.

\begin{flushright}
\rule{4in}{0.5pt} \\ (Name and Signature)
\end{flushright}




\begin{enumerate}
% ======================================================================
\item {\it Argentina and Chile.\/}  
Argentina and Chile are two neighboring South American countries, 
but their recent experience has been much different. 
Your task is to provide a quantitative analysis of this difference, 
using:  
%
\begin{center}
\tabcolsep=0.15in
\begin{tabular}{lcccccc}
\hline\hline
Country &  Year  &   $Y/L$ &  $K/L$  &  $L$ & \POP &   $H$    \\
\hline\hline
Argentina   &  1990   & 20,564  &  58,666  &  11.4  & 32.5 & 8.1     \\
Argentina   &  2000   & 25,670   &  50,514 &  15.9  & 37.0& 8.8     \\
Chile      &  1990   & 16,984   &  22,950  &  4.8   & 13.1& 7.0     \\
Chile      &  2000   & 25,084   &  38,488  &  6.0   & 15.2  & 7.6    \\
\hline\hline
\end{tabular}
\end{center}
%
Here $Y$ is PPP-adjusted
real GDP, $L$ is employment, $K$ is the stock of physical capital, 
$\POP$ is population, and $H$ is average years of school 
for the adult population. 
We refer to $Y/L$ as output per worker and $K/L$ as capital per worker.  
$ Y $ and $K$ are measured in 2000 US dollars and $\POP$ in millions.   

%\end{enumerate}
%\end{document} 

\begin{enumerate}

\item Which country had higher output per worker in 2000?  Output per capita?  Total factor productivity?  
(10~points)

%\end{enumerate}

\item Which country's output per worker grew more rapidly between
1990 and 2000?  (10~points)

%\begin{comment}
\item Which factors are most responsible for the difference in
growth rates of output per worker? (10~points)
%\end{comment}

\item An analyst at Deutsche Bank argues that the major policy 
difference between 
the two countries is Chile's private pension system, 
which has been associated with higher saving and investment rates.  
In the 1990s, for example, investment rates were 16\% in Argentina and 
21\% in Chile.  
Given the evidence, as well as your experience with other countries, 
do you find this argument persuasive?  (10~points)  
\end{enumerate}



%\begin{comment}
Answer.
%
\begin{enumerate}

\item The relevant numbers are 
%
\begin{center}
\tabcolsep=0.15in
\begin{tabular}{lccc}
\hline\hline
Country &   $Y/L$ &  $Y/\POP$  &  $A$  \\
\hline\hline
Argentina    & 25,670   &  110,312 &  162.9     \\
Chile        & 25,084   &  99,016  &  192.2    \\
\hline\hline
\end{tabular}
\end{center}
%
TFP is computed from $ A = (Y/L)/[(K/L)^\alpha H^{1-\alpha} ] $
using $\alpha = 1/3$.  
Argentina has somewhat higher GDP per worker and GDP per capita.
Chile has higher TFP.  
It's not part of the question, but you might ask:  Why the difference?  
Because the level of education is higher in Argentina.  
(This is a little tricky, you'll have to think through how the calculation 
works.) 

\item Average annual growth rates of GDP per worker were  2.22\% in Argentina 
and 3.90\% in Chile.  
The first number is computed like this:  
\[
    \gamma_{Y/L} \;=\; \frac{\log(25670/20564)}{2000-1990} 
            \;=\; 0.0222  \;=\; 2.22\%.
\]
The second applies the same method to data for Chile.

\item The work is mainly computing growth rates for the various components.  Once we do this, we
can compute the various contributions to growth from the growth accounting relation for output per
worker:
\[
     \gamma_{Y/L} \;=\; \gamma_A + \alpha \gamma_{K/L} + (1-\alpha) \gamma_H .
\]
In our example, this is
\begin{eqnarray*}
    \mbox{Argentina:} &&  2.22 \;=\; 2.16 \mbox{ (A)} -  0.50 \mbox{ (K/L)} 
            + 0.55 \mbox{ (H)}  \\
    \mbox{Chile:} &&  3.90 \;=\; 1.63 \mbox{ (A)} +  1.72 \mbox{ (K/L)} + 0.55 \mbox{ (H)} .
\end{eqnarray*}
The biggest difference is clearly in growth of capital per worker:  
this actually fell in Argentina.  

\item Could it be saving and investment?  
Clearly there's a big difference here in capital accumulation, 
so there's something to the story.
But we also know that such differences don't effect growth indefinitely:
diminishing returns to capital kick in. 
A more subtle is that $K$ didn't fall:  
the fall in $K/L$ is the result of an increase in $L$, 
which we'd argue is the consequence of the stabilization of the economy
after the hyperinflation of the later 1980s 
(something you're not expected to know).  

\end{enumerate}

%\end{enumerate}
%\end{comment}
%\end{enumerate}
%\end{document}
%\end[comment}


%\pagebreak \phantom{xx} \pagebreak \phantom{xx} \pagebreak
% ======================================================================
\item {\it Vietnam.\/} 
Since 1991, Vietnam's per capita GDP has
been growing at an average rate of 6.75\%, positioning the
Southeast Asian economy among the fastest growing in the world. 
According to analysts, an obstacle to even speedier
growth will be removed by mid--2006, when Vietnam is expected to
join the World Trade Organization.

Some facts.  
In 2005, manufacturing and construction accounted for 41\% of GDP (up
from 38\% in 2001) and services for 38\% (from 27\%), whereas agriculture dropped to 21\% (from 23\%).
Employment is divided among the three sectors in the following
proportions: 17\%, 25\%, and 57\%, respectively (the service figure
includes 10\% in state employment). Female labor force participation
is high by world standards, with women accounting for 49\% of the
labor force. According to the World Bank, school enrollment rates
are 100\% in primary school, 70\% in secondary school, and 10\% in
university.  

The most notable immediate effect of
WTO membership will be the scrapping of US quotas on garment imports
from Vietnam. While these quotas limit Vietnamese exports to the
US, some analysts warn that their removal may have a relatively
modest effect: the EU's decision to waive its import quotas this year 
had only a small effect on shipments, probably because of strong
competition from other producers, particularly China. 
A further difficulty is the spate of strikes in foreign-owned factories, 
which led the government to raise the minimum wage paid by foreign firms
by 40\%.  See the attached article in {\it The Economist\/}.  

\begin{enumerate}

\item 
In which broad sectors is Vietnam likely to shift its
production in the next five years?  20 years?  
Why? (10~points)

\item To comply with the conditions for WTO membership, the
Vietnamese National Assembly recently passed two pieces of
legislation that may have a substantial impact on the
economy. The Anti-corruption Law, which comes into effect June
1st, is intended to improve the detection and prevention of public
official corruption.  The Law on Investment, also
expected to come into effect in mid-2006, allows investment projects worth
less than \$1m to proceed without registration, and projects
valued at less than \$19m  will need to be registered but will not
require licenses.   

Describe --- concretely --- how each of these changes are likely to effect 
the economy's total factor productivity.
(10~points)

\item In your view, what government policies are likely to be 
most effective in raising the living standard of the Vietnamese people 
over the next 25 years?  Why? (10~points)

\end{enumerate}

Answer.  
\begin{enumerate}
\item 
In recent times, agriculture has fallen as a fraction of GDP, 
and the other two sectors have increased their shares.  
We'd expect that to continue for two reasons.
One is that this is a standard pattern for developing countries, 
including the US 150 years ago:  people move off farms.  
Another is that this reflects the (admittedly crude) productivities 
implied by the GDP and employment shares:  
17\% of the workforce produces 41\% of GDP in manufacturing 
and construction, 
whereas 57\% of the workforce produces only 21\% of GDP in agriculture.  
Some of this might reflect differences in skill level 
and capital per worker across industries, but it's likely that 
productivity is lower in agriculture, 
and that market forces will reallocate people from agriculture 
into manufacturing --- and maybe services, too.    

\item Each of these changes should reduce the cost of running a business, 
which will increase TFP directly.
The investment law, in particular, 
should make it easier to start new businesses,  
which should facilitate the reallocation described above. 
As the economy shifts resources to more productive sectors, 
overall TFP will increase.  
It's the same argument we used with international trade.   


\item There are lots of sensible answers.  
We'd stress flexible labor markets; 
competitive product markets, including competition from abroad; 
and a legal system that enforces property rights.  

\end{enumerate}

%\pagebreak \phantom{xx} \pagebreak \phantom{xx} \pagebreak
% ======================================================================
\item {\it Miscellany.\/}
\begin{enumerate}

\item {\it Innovation and prices.\/}  
Many analysts have argued that inflation is overstated, 
because some price increases reflect improvements in product quality.  
They argue further that this leads us to understate the growth 
of real output and TFP.  
Do you find this argument persuasive?  Why or why not?  (10~points)

\item {\it Competition.\/}
The Mexican Business Association states:  
``Competition is a problem in Mexico, since local firms often have difficulty 
competing with large foreign firms.  
We think restrictions on foreign competition would be good for 
the Mexican economy and workers.''
Do you find this argument persuasive?  Why or why not?  (10~points)

\item {\it Trade deficits.\/}
A Financial Times columnist reports:  ``A trade deficit is a sign that 
national investment is greater than national saving.  Since investment 
is a good thing, that must be good for the country.''
What could the logic be for these statements?  Do you find them persuasive?  
(10~points)

\end{enumerate}

%\begin{comment}
Answer.  
\begin{enumerate}

\item We find the argument persuasive:  lots of products are 
better than they were, and this is likely not adequately 
taken into account in measures of prices used (for example) to deflate GDP.  
Since GDP at current prices is relatively well measured, 
and is the product of real GDP and a price index, 
overstatement of the price index leads naturally to understatement
of real GDP.  

\item We think they're wrong:   competition would be good for Mexican 
consumers, and likely good for workers, too, although there may
be some initial disruption.  
These kinds of statements are almost always self-serving: 
it's the managers and owners the policy is protecting.  
More than that, lots of evidence suggests that local firms 
have a substantial advantage in competing with foreign 
firms.  

\item 
The first sentence is correct as a matter of accounting:
Since $ S = I + \NX$, a trade deficit ($\NX<0$) means $ I>S$.  
The second there are different opinions about.  
Certainly investment will raise capital and GDP.
But since the deficit is financed by selling claims on 
the economy to the rest of the world (borrowing, in other words), 
the question is whether the return on the investment (the extra output) 
is enough to cover the cost of borrowing.  

PS:  The FT typically says the reverse, as some have pointed out.   

\end{enumerate}
%\end{comment}


\end{enumerate}

%\pagebreak \phantom{xx} \pagebreak \phantom{xx} 

%\end{document}

\newpage
Trouble at the mill \\
Jan 26th 2006 |  HANOI \\
From The Economist print edition 

Strikes and pay rises afflict the new South-East Asian tiger

FACTORY workers in Vietnam have an extra reason to celebrate Tet, the lunar new year holiday that begins this weekend. Their government recently decided to raise the minimum wage in foreign-owned factories by up to 40\%, starting on February 1st. Pay packets in Hanoi and Ho Chi Minh City will now start at \$45 a month, the first mandated rise in several years. Experienced workers can expect an extra 7\% increase on top of that.

What is especially unsettling for investors is how the workers got their extra dough. Since late December, wildcat strikes have swept through the industrial zones surrounding Ho Chi Minh City. Tens of thousands of workers joined the protests over wages and conditions. Some of these turned violent, and machines were wrecked at one Taiwanese-owned plant. Bosses claim that outside agitators stoked the protests, distributing notes at factory gates while police stood idly by.

Apparently caught off-guard, the government issued a decree earlier this month raising the minimum wage in foreign-owned factories. Most strikers have now returned to work, but some have not, and investors are fuming over production stoppages and a higher wage bill. The European Chamber of Commerce has gone so far as to write a tart letter to the prime minister, Phan Van Khai, reminding him that investors set up shop in Vietnam precisely because ``the workforce is not prone to industrial action�.

At least, not until now. Workers in Vietnam have staged walkouts before, particularly over alleged mistreatment by foreign managers, but the scale and co-ordinated nature of the latest strikes are, well, striking. Some observers find it implausible that they could occur without the prior knowledge of the ruling party, which forbids independent trade unions. As in China, workers are allowed to join only a pliant, party-affiliated union.

Most of the affected factories are owned by East Asian companies, the biggest investors in Vietnam. At Song Than industrial zone, on the outskirts of Ho Chi Minh City, 80\% of the factories are owned by Taiwanese, producing clothing, shoes, furniture and bicycles for export. They grumble that higher wages will drive away foreign investment, running at \$5.8 billion last year, and give warning that Vietnam needs to stay competitive. ``Chinese wages are higher. But the quality and efficiency are also higher,� said Chen Chi Young, an official at Taiwan's de facto embassy.

So why didn't Vietnam crush the illegal strikes? One reason, say observers, may be internal jockeying ahead of the party congress in April, a five-yearly affair. The aim could have been to embarrass the provincial officials where the unrest began, or to burnish the leadership's credentials, or both. The factories most affected may also be a clue: Vietnam and Taiwan both claim ownership of the Spratly Islands, along with several other countries. On December 15th, Taiwan said it was building a landing strip on one of the islands.

Or perhaps the workers were simply fed up with low pay and stingy bosses, and were too numerous to repress. Vietnam has one of the world's fastest growing economies. Now it is learning that higher output means higher expectations.


\end{document} 