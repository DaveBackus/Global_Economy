\documentclass[letterpaper,12pt]{article}

%\usepackage{amsmath}
\usepackage{comment}
\usepackage{booktabs}
\RequirePackage[hypertex]{hyperref}
    \hypersetup{colorlinks=true,urlcolor=blue,linkcolor=red}
\RequirePackage{GE05}

\newcommand{\GDP}{\mbox{\em GDP\/}}
\newcommand{\NDP}{\mbox{\em NDP\/}}
\newcommand{\GNP}{\mbox{\em GNP\/}}
\newcommand{\NX}{\mbox{\em NX\/}}
\newcommand{\NY}{\mbox{\em NY\/}}
\newcommand{\CA}{\mbox{\em CA\/}}
\newcommand{\NFA}{\mbox{\em NFA\/}}
\newcommand{\Def}{\mbox{\em Def\/}}
\newcommand{\CPI}{\mbox{\em CPI\/}}

\def\ClassName{The Global Economy}
\def\Category{Class Notes}
\def\HeadName{Excel Basics}

\begin{document}
\thispagestyle{empty}%
\Head

\centerline{\large \bf \HeadName}%
\centerline{Revised: \today}

\bigskip
In this class and in the business world you're often asked to do 
routine data analysis.  
Excel and other spreadsheet programs (the one in Open Office, for example) 
are often the easiest way to do simple computations and graphics.  
Here's a quick overview of the basics in you're a little rusty.  

\begin{enumerate}

\item Entering numbers and expressions.  
To create a series of numbers: type a starting value --- 
say 10 --- in cell B2.  In cell B3 type =B2+10.  The cell B3 should now read 20.  To easily extend this series down this column, select column B3 and copy it by choosing CTL+C.  Now highlight cells B4 through B50 and paste by typing CTL+V.  Click on cell B40.  See how it reads =B39+10?  
When you copy and paste, Excel automatically changes 
the relative cell references.

\item Relative cell references.  
What if you do not want a relative cell reference?   In cell B1 enter the value 0.5.  Now suppose we would like to raise each element of the vector we created in part 1. to the power 0.5 and put the result in column C.  In cell C2 type =B2**\$B\$1.  Now copy and paste C2 into C3 through C50, as in part 1.  Click on cell C40.  Notice that it reads =B40**\$B\$1.  The relative part of the formula (B40) has changed, but the absolute reference (\$B\$1) has not.  The nice part about this method is that you can change the parameter B1 (from say 0.5 to 0.75) and column C will be automatically updated.

\item Miscellaneous ``operators'' and functions:    

\begin{center}
\begin{tabular}{lc} 
 multiplication & *  \\
 division       & /  \\
 to raise x to the power y  &  $x\widehat{\phantom{g}}y$  \\
 natural log of x &  ln(x) or LN(x)
\end{tabular} 
\end{center} 

?? add mean, std ??

\item  Graphing.  To create a graph, from the menu bar choose Insert>Chart, which will open the chart wizard.  
\begin{enumerate}    
\item  Choose XY (Scatter) and the scatter with data points connected by lines without (or with) markers.  Choose next.
\item  Choose the tab labeled "series" and click on the button to the right of the text box labeled X values.  This will close the wizard and return control to the spread sheet. 
\item Highlight the data we would like to plot on the X-axis, B2 through B50.  Then click on the button to the right of the text box that is floating above your worksheet.  This should take you back to the chart wizard.  
\item  Repeat b. and c. using the Y values text box and choose the data you would like plotted on the Y-axis, C2 through C50.
\item Choose next and enter the titles and axis labels for your chart.
\item Choose next and select "As new sheet:" then choose finish.
\item You should now have a graph of the data in a new sheet labeled "Chart 1."
\item To modify your chart, choose Chart>Chart Type... from the menubar to restart the chart wizard.
\item Things in Excel are very customizable.  Try right clicking on a part of your graph to see the options.
\end{enumerate}
\end{enumerate}

\vfill \centerline{\it \copyright \ \number\year \ NYU Stern
School of Business}

\end{document}
