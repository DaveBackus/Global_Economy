\documentclass[letterpaper,12pt]{article}

\usepackage[hypertex]{hyperref}
    \hypersetup{colorlinks=true,urlcolor=blue,linkcolor=red}
\usepackage{comment}
\RequirePackage{GE05}
\RequirePackage{booktabs}

\newcommand{\GDP}{\mbox{\em GDP\/}}
\newcommand{\NDP}{\mbox{\em NDP\/}}
\newcommand{\GNP}{\mbox{\em GNP\/}}
\newcommand{\NX}{\mbox{\em NX\/}}
\newcommand{\NY}{\mbox{\em NY\/}}
\newcommand{\CA}{\mbox{\em CA\/}}
\newcommand{\NFA}{\mbox{\em NFA\/}}
\newcommand{\Def}{\mbox{\em Def\/}}
\newcommand{\CPI}{\mbox{\em CPI\/}}

\def\ClassName{The Global Economy}
\def\Category{Class Notes}
\def\HeadName{International Capital Flows}

\begin{document}
\thispagestyle{empty}%
\Head

\centerline{\large \bf \HeadName}%
\centerline{Revised: \today}

\bigskip
Capital moves around the world these days at lightning speed, 
with investors in one country buying assets in another 
as a routine matter.  
Individuals, pension funds, government funds:  they all do it.  
This set of notes concerns the measurement system used 
to track the total quantity of cross-border trade in assets
and related indicators of impending crises.  


\subsubsection*{Trade in goods, services, and income}

The relevant measurement system is the balance of (international) 
payments (BOP), 
a close relative of the National Income and Product Accounts (NIPA) that focuses on international transactions.  
This is simply accounting, in the sense that we're counting things in a consistent way and not applying any particular theoretical framework.  Nevertheless, an important idea emerges: 
countries that run trade deficits can also
be thought of as attracting foreign investment or borrowing from abroad.
Trade in assets mirrors trade in goods.  

The balance of payments starts with trade in 
goods and services and related flows of income.  
We saw  one such measure earlier:  the NIPA's {\it net exports\/}
of goods and services (exports minus imports). 
By this measure, the US ran a balance of --\$498b
in 2003, a deficit of roughly 5\% of its \$11 trillion GDP. 
See Table~\ref{tab:usbop}.

Two closely-related measures are commonly reported.  
The {\it merchandise trade balance\/} is similar to net exports, 
but includes only trade in goods (``merchandise'').  
It is reported monthly, and so is more readily available than the 
quarterly NIPA.  
Service trade includes such things as foreign tourists visiting the US (hotels, restaurants), 
consulting services provided by US firms for foreign
clients, and foreign students attending US universities. Since the
US currently runs a modest surplus in service trade, the merchandise
trade deficit (slightly) overstates the deficit for trade in goods
and services (net exports). 
The {\it current account balance\/} is a broader concept than net exports;
it consists of net exports plus net receipts of capital income, labor
income, taxes, and transfers from abroad 
({\it net foreign income\/} for short).  
Mathematically, 
\[
    \CA \;=\; \NX + \mbox{Net Foreign Income} .
\]
Net foreign income includes such items as payment of interest on US government bonds owned by foreign central banks (a negative entry), 
salaries received by American consultants working in Tokyo 
(a positive entry), 
and salaries paid to Russian hockey players in the US 
(a negative entry).  We see in Table~\ref{tab:usbop} that in 2003
the US was a (modest) net recipient of capital income and net payer
of labor income.


\begin{table}[h!] 
\centering
\begin{tabular}{lr}
\hline\hline
% Net exports of goods   &  \\
% Net Exports of services & \\
 Net exports of goods and services & --498.1 \\
 Net labor income from ROW  & --5.5 \\
 Net capital income from ROW &   38.8  \\
 Net taxes and transfers from ROW & --67.4 \\
 {\bf Current account}  & {\bf --530.7} \\
 \\
 Net direct investment in US &  133.9 \\
 Net purchase of private securities  &  251.0  \\
 Net purchase of US govt securities  &  248.6  \\
 Net loans and other  & --102.8  \\
 {\bf Capital and financial account (�inflows�)} &  {\bf 542.7}  \\
 \\
 {\bf Statistical discrepancy} &  {\bf --12.0} \\
 \hline\hline
\end{tabular}
\caption{US Balance of Payments, 2003, billions of dollars. ROW means ``rest of world.''} \label{tab:usbop}
\end{table}

The current account balance is thus the broadest measure of a country's flow
of ``current'' payments to and from the rest of the world. In the
US, there's little difference between net exports and the current
account: in 2003 the current account deficit was only
slightly larger than the trade deficit. In other countries the flows
of labor and capital income may play a larger role.  


\subsubsection*{Trade in assets}

There are also flows related to capital and financial
transactions.  
You can see in Table~\ref{tab:usbop} that the US in 2003 was the net recipient of \$542.7b of capital and financial ``inflows,'' meaning that
foreigners' purchases of US assets were greater than US nationals'
purchases of foreign assets by this amount. 
By convention this is reported as a
positive entry, even though it corresponds to an accumulation of
liabilities with respect to the rest of the world.  Foreigner's purchases
of domestic assets consisted of direct investment (a controlling
interest in a US business), purchases of equity and bonds issued by
US corporations, purchases of US government and agency issues, and
some other minor items we won't bother to enumerate.

The central insight of the balance of payments 
is that these financial transactions must match the current transactions:
\[
    \mbox{Current Account} + \mbox{Capital and Financial Account} \;=\; 0.
\]
The logic is that the financial transactions finance any 
imbalance in ``current'' transactions.  
It's not quite true in the data, because the numbers are not
entirely accurate, so we add a balancing item 
(``statistical discrepancy'' or ``errors and omissions'') to make up the
difference. The point is that any deficit in the current account
must be financed by selling assets or accumulating liabilities with
respect to the rest of the world.  The same accounting truism
applies to a firm or individual: if your expenditures exceed
your receipts, you need to sell assets or borrow to finance the
difference. Firms do this regularly when they make major additions
to plant and equipment. And households often do the same when they
buy houses.  

The interesting thing about this accounting identity is that it
gives us a different perspective on current account deficits.  If we
run a current account deficit as a reflection of a trade deficit, as
in the US right now, we're tempted to look at imports and exports as
the reason. Perhaps foreign countries are keeping our goods out of
their home markets, or pushing down their exchange rates to encourage
exports.  That's the first reaction most people have.  But now we
know that a current account deficit must correspond to a capital and
financial inflow:  foreign investors are buying our assets.  This
perspective leads us to think about the investment opportunities in
the US and elsewhere in the world that might lead to this. Are US
assets particularly attractive? Or are foreign assets unattractive?
Both perspectives are right, in the sense that they're true as a
matter of accounting arithmetic, but the second one captures more
clearly the dynamic aspect of investment decisions.  


\subsubsection*{Net foreign assets}

The capital and financial account measures net flows of financial
claims: changes in the net asset position, in other words. 
The balance sheet position of an economy is referred to as its net international
investment position (NIIP) or simply net foreign assets (NFA).  
If a country's claims on the rest of the world exceed 
their claims on it, 
then it has positive net foreign assets and is said to be a net
creditor. If negative, a net debtor. The position changes over time
as indicated by the capital and financial account.  
Mathematically we would say
%
\begin{eqnarray}
        \mbox{\it NFA\/}_{t}  &=& 
        \mbox{\em NFA\/}_{t-1}  +   \mbox{\em NX\/}_t %\nonumber \\
              + \mbox{Net Foreign Income}
                                    + \mbox{Asset Revaluations} .
        \label{eq:niip}
\end{eqnarray}
As in most accounting frameworks, there's a connection 
between the income statement (the ``flows'' in economics parlance) 
and the balance sheet (the ``stocks'') 
The timing convention is similar to financial accounting:  
$\mbox{\it NFA\/}_{t}$ is the net foreign asset position at the end 
of period $t$.  


An analogous relation for an individual might go something like this:  
Suppose you
start with no assets or liabilities, then borrow 50,000 for the
first year of your MBA. 
You spend the entire 50,000 and have no other source of funds, 
so you have a cash flow deficit of --50,000 for the year.
At the end of the year, you have a net asset position of
--50,000.  
The bookkeeping is analogous to equation (\ref{eq:niip}),
with $\NFA$ analogous to your net worth, 
$\NX$ analogous to your annual cash flow surplus or deficit, 
and the last two terms ignored to keep things manageable.  
If we added interest on the debt, that would show up in 
Net Foreign Income.


Why do we need asset revaluations?  By tradition, we (try to) 
measure international investments at market value, so if the value
of an asset changes we need to account for it in NFA.
In international investments, 
asset revaluations occur both through the usual change in 
prices of equity and bonds and through changes in exchange rates 
for instruments denominated in foreign currencies.  


\begin{table}
\centering
\begin{tabular}{lr}
\toprule

{\bf US-owned assets abroad}  &  7,864.0 \\
Direct investment &  2,730.3 \\
Corporate equity &  1,972.2 \\
Bonds  & 502.1 \\
Loans and other &  58.4 \\
Reserves \& govt &  268.3 \\
{\bf Foreign-owned assets in the US} & 10,515.0 \\
Direct investment  & 2,435.5 \\
Corporate equity  & 1,538.1  \\
Corporate bonds   &  1,853.0 \\
US govt (treasuries, currency, official) &  2,334.6  \\
Loans and other &  2,353.8      \\
{\bf Net international investment position} &  --2,651.0  \\
\bottomrule 
\end{tabular}
\caption{US Net International Investment Position, 2003 Yearend, billions of dollars.}
\label{tab:usniip}
\end{table}

At the end of 2003, the US had a net financial asset position of
--\$2,651b, meaning that foreign claims on the US exceeded US claims
on the rest of the world by this amount.  A complete accounting of
assets and liabilities is reported in Table~\ref{tab:usniip}.


\subsubsection*{Sources of external deficits}

We'll talk more about the difference between the trade balance and
the current account shortly,
but for now let's ignore the difference and consider a trade deficit.  
If we have a large deficit, should we be worried?  
Is it a sign that the economy is in trouble?  
In this and many other cases, it's helpful to consider an analogous situation
for a firm.  
Suppose a firm is accumulating liabilities. 
Is that a bad sign?   
The answer is that it depends how the liabilities are used.  
If they finance productive investments, 
then there should be no difficulty servicing the liabilities.  
In fact, the ability to finance them suggests that someone
thinks the investments will pay off.  
But if the money is wasted (surely you can think of examples!), 
then investors might be concerned.  
The same is true of countries:  it depends where the funds go.    

Consider the flow identity that we saw at the start of the course:  
\[
    S  \;\equiv\; Y - C - G \;=\; I + \NX.
\]
Typically this is expressed as a ratio to GDP, 
with everything measured at current prices:  
\[
    \frac{S}{Y}  \;\equiv\; \left( 1 - \frac{C+G}{Y} \right) 
        \;=\; \frac{I}{Y} + \frac{\NX}{Y}.
\]
If we run a trade deficit ($\NX < 0$), 
it must (as a matter of accounting) 
reflect some combination of low saving 
and high investment (high $I$).  
High investment is typically not a concern.  
If we borrow from abroad to
finance new plant and equipment, and the plant and equipment lead to
higher output, we can use the extra output to cover the liabilities.  
We might guess that the ability to borrow from abroad is useful in the
same way that a firm's ability to borrow allows it to pursue
positive NPV projects.  But what if we finance household consumption 
or government purchases?  
We have to answer the same question:  was the
expenditure worthwhile?  Here there is room for concern, 
but a serious answer would depend on the nature of the expenditures.  

The Lawson Doctrine, named after British government official Nigel Lawson, 
makes a distinction between public and private sources of deficits.
Recall that we can divide saving into private  and government components, 
so that  
\[
    S_p + S_g  \;=\; I + \NX.
\]
In Lawson's view, a trade (or current account) deficit that financed
a difference between private saving and investment 
is fine.
But if the external deficit (trade or current account) 
stems from a government deficit, 
it's worth a more careful look.
In practice, emerging markets crises often stem from government 
deficits that are financed abroad, which would raise concern
in Lawson's view. 


\subsubsection*{Net foreign asset dynamics}

There's a natural source of dynamics in international borrowing, 
just as there was with government debt:  
since debt accrues interest, it tends to grow over time unless
something is done to counteract it.  
In an international context, NFA plays the role of debt.
This is something of a misnomer, 
because in many countries the claims of countries on others 
include equity and direct investment as well as debt.  
File that away for later.   
We generally look at NFA relative to GDP, so the question 
is which is growing more rapidly.  
If the current situation leads the ratio of NFA to GDP to explode, 
we say the situation is not sustainable.  


How do NFA and GDP change through time?  
We've seen that NFA changes like this:  
\begin{eqnarray*}
    \mbox{\it NFA\/}_{t} &=& \mbox{\it NX\/}_t + \mbox{\it NFA\/}_{t-1} + 
                \mbox{Net Foreign Interest Income}  \\
                &=& \mbox{\it NX\/}_t + (1+i) \mbox{\it NFA\/}_{t-1}  .
\end{eqnarray*}
Note that everything here is nominal, including the 
interest rate $i$ on the net foreign asset position.  
Here we're skipping asset revaluations and 
the non-interest component of net foreign income, 
but could add them back in later if we thought they were relevant.  
If the growth rate of (nominal) GDP is $g$, we can write 
\[
    Y_{t} \;=\; (1+g) Y_{t-1} .
\]
With these inputs, we see that $\NX/Y$ evolves like this: 
\begin{equation}
    \frac{\NFA_{t}}{Y_{t}} \;=\; 
                \left( \frac{1+i}{1+g} \right)  \frac{\NFA_{t-1}}{Y_{t-1}} 
             +  \frac{\NX_{t}}{Y_{t}} .
\end{equation}
This equation is the basis of our analysis.  


How does the ratio of NFA to GDP change over time?  
The first issue is whether $g$ is larger or smaller than $i$.  
If $g>i$, then a country can run deficits forever without 
$\NFA/Y$ exploding.  
More commonly, $ g<i$, in which case a negative net foreign 
asset position will continue to grow, even if we set $\NX = 0$.
In this more common case, the trade deficit is said to be unsustainable.  
But if it's unsustainable, what happens?  
The theory doesn't say, but we can imagine some possibilities:  
the trade deficit turns to surplus, the country defaults on 
some or all of its foreign liabilities, and so on.
More commonly, this is used to project the growth 
of NFA over the next few years.
If this leads to a large ratio of NFA to GDP, then 
investors start to wonder whether they'll be repaid.  
How large does it have to be to generate concern?
It depends on the country.  


\subsubsection*{Capital controls}

??



\subsubsection*{Signs of trouble?} 

The bottom line is that the current account deficit 
and net foreign asset position are important indicators 
of the state of an economy.  
Important, but what do we make of them?  
Take a current account deficit.
Is it bad (the word deficit sounds bad!) 
or good (look, people want to invest in our country!)?  
You need to look at the overall picture and come up with a judgement.
The same for net foreign assets.
It's another piece of the puzzle to consider when deciding 
whether a country is a good opportunity. 

Here are some things analysts look at:  
%
\begin{itemize}

\item Extent of foreign borrowing.  Generally anything over 50\% of GDP 
would get a closer look.  
Why?  Because the larger the debt, the more attractive default is.  
Developed countries are typically given more leeway than developing countries, on the theory that their political 
institutions are more stable.

\item Origin and nature of foreign borrowing.  
If it's the government, that generally creates more concern than 
private borrowing:  it can be difficult to enforce a claim against a sovereign government and investors know that.   
If it's a bank, that generates indirect concern, because a failing bank 
may be bailed out by the government, 
which converts it to a government claim.  
There may be other private debts that have implicit government guarantees.
Finally, equity is the least worrisome, because the value of the claim 
adjusts to the ability to pay.  

\item Denomination of foreign borrowing.  
Borrowing in foreign currency generates more concern than borrowing in 
local currency, because a drop in the value of the currency increases
the debt in local terms.  

\item Maturity of borrowing.  Short-term borrowing is more worrisome, 
since it needs to be rolled over frequently.  
Mexico in 1994 had both of these problems:  a large part of the government
debt was short-term and denominated in dollars.
When investors refused to buy new debt, the government was unable to 
pay off the maturing debt.  The peso fell 50\%, 
which effectively doubled the dollar debt in peso terms.  
\end{itemize}
%
If you think this looks like corporate finance, you're right.
Many of the same principles apply to countries.  
The difference is that there is no obvious way to enforce a claim
on a foreign government:  if it chooses default, 
you have limited ability to argue.  


\subsubsection*{Example:  Risk and reward in Romania, May 2008}

Here's an example that shows how you might use information
about international capital flows to assess country risk.
It also uses information discussed in prior classes, 
including the government debt and deficit.  

Romania has had a chaotic history since unification 
under the Ottoman Empire in 1859 and independence in 1878.
In 1940, it lost territory to Hungary and the Soviet Union; 
after World War II, the remainder became part of the Soviet bloc.  
Since December 1989, when the communist regime fell, 
Romania has embarked on a cautious path of reform 
and has been an official member of the European Union since January 2007.
Romania remains a poor country, 
with GDP per capita below Poland and Hungary 
but above Bulgaria and Serbia.  
The current minority government, which faces an election in November, 
has been notably slow to accelerate reform, 
failing specifically to address concerns expressed by the EU and IMF 
about fiscal policy, the current account, judicial reform, and corruption. 
Current economic indicators include:
%
%\tabcolsep=0.25in
\begin{center}
\begin{tabular}{lcr}
\hline
  {\it Indicator}   &&  {\it Value}     \\
\hline 
  GDP growth (real) &&  6\%       \\
  Inflation         &&  5\%       \\
  Short-term interest rate &&  8\%   \\
  Fiscal balance:  total  (ratio to GDP)    &&  --2.6\%  \\
  Fiscal balance:  primary  (ratio to GDP)  &&  --0.7\%   \\
  Government debt (ratio to GDP)            &&  18\%       \\
  Current account balance (ratio to GDP)    &&  --11.3\%   \\
  Current account balance (US dollars) &&   --23b   \\
  Net foreign assets (ratio to GDP)         &&   --19\%   \\
  Foreign reserves (US dollars)           &&  39b  \\
\hline 
\end{tabular}
\end{center}
%
Text and numbers adapted from Economist Intelligence Unit's 
{\it Country Profile\/} and {\it Country Risk Service\/}.  

Overall, what do you see as the major risks to a short-term investor in government or private sector debt?  
Would you recommend such an investment?  Why or why not?  

Here's a list of issues and indicators:   
%
\begin{itemize}

\item Government deficits.  
The terminology comes from the source.
It tells us that the government deficit is 2.7\% of GDP  
and the primary deficit is 0.7\%.
The difference (1.9\%) represents interest payments on government debt.  

\item Government debt. The government debt is a modest 18\% of GDP.  
Debt dynamics work like this:  
\[
    \frac{\mbox{\em B}_{t}}{Y_{t}} \;=\; 
            \left( \frac{1+i}{1+g} \right)
            \frac{\mbox{\em B}_{t-1}}{Y_{t-1}} +  
            \frac{\mbox{\em D}_{t}}{Y_{t}} .
\]
With $ i = 8\%$ and $ g = 6 + 5 = 11\%$, 
the numbers tell us that next year's ratio of government debt ($B$) 
to GDP ($Y$) will be:
\[
    (1.08/1.11) \times 0.18 +  0.007 \;=\; 0.182 .
\]
That is:  next year's debt to GDP ratio is not expected to be much 
different from this year's.  
The main reason is that the primary deficit is small.    

What could go wrong?  (a) Growth could slow down, which would lower $d$ 
and also (probably) increase the primary deficit.  
(b)~The interest rate could rise sharply if investors become concerned.
Neither of these seems likely given the numbers 
(modest debt and deficits).  

[Check:  these numbers imply interest on the debt of 
$ i B/Y = 0.08 \times 0.18 = 0.014$ (1.4\% of GDP).
This is below our number, but not by a lot.]  

You would also want to look into hidden government liabilities.  
Eg, how stable is the banking system?  
Is it borrowing a lot?  In what form?  
Ditto pensions.  

\item International borrowing. 
The net foreign asset position is negative, and about the same size as 
the government debt. 
Is it mostly government debt?  Or private borrowing? 
Is the borrowing debt or equity?  
If debt, is it denominated in local or foreign currency?  
This information will help us decide what the risks are. 
If government debt, then there's more risk.
If foreign-denominated, also more risk, because a fall of 
the local currency could increase the debt quickly in local terms.  
If equity, less risk:  the risk is shifted (by design) to the investor.  


Net foreign asset dynamics work like this:  
\begin{eqnarray*}
    \frac{\mbox{\em NFA}_{t}}{Y_{t}} &=& 
            \left( \frac{1+i}{1+g} \right)
            \frac{\mbox{\em NFA}_{t-1}}{Y_{t-1}} +  
            \frac{\mbox{\em NX}_{t}}{Y_{t}} \\
           &=& (1.08/1.11) \times (-0.19) -0.113 
               \;=\; -0.298 .
\end{eqnarray*}
The trade deficit is pretty big, which is making net foreign assets 
increasingly negative.  
Again, you'd like to know who's is doing the borrowing, and in what form.  
The numbers tell you it can't be the government, since the government 
deficit is so much smaller than the trade deficit.  

\item Reserves.  
The central bank has what seems to be a lot of foreign currency reserves. 
That gives us some assurance that it will be able to withstand pressure
on the currency.
That said, international financial markets are huge:  \$39b may not be enough. 

\item Other thoughts.  EU membership makes it unlikely (we'll see!) that
Romania would default on government debt.  
Corruption might be a concern for private issues  --- could a borrower
simply decide not to pay you and bribe the judge to go along?  
You'd want to look more closely at that.  
But the concern with the highest probability is the possibility of a large
decline in the value of the currency.  

\end{itemize}



\subsubsection*{Executive summary}

\begin{enumerate}
%\item There are several measures of current transactions with other countries:
%\begin{itemize}
%\item The merchandise trade balance measures exports minus imports of goods. 
%\item Net exports includes trade in services as well. 
%\item The current account includes net international factor
%income, taxes, and transfers.
%\end{itemize}
%We refer to them collectively as ``external'' balances (or deficits or %surpluses) and disregard any differences.  

\item The current account is mirrored by an equal and opposite capital and financial account
measuring net asset transactions: a trade deficit is also a capital inflow
(foreign borrowing). 

\item The net international investment position measures our current net claims on the rest of the world.

%\item The flow identity tells us that the external deficit 
%reflects some combination of personal saving, government saving, 
%and investment.

\item Current account deficits and net foreign borrowing are 
often treated a signs of pending trouble.
Whether or not they are depends on how the borrowing is used.  

\end{enumerate}


\begin{comment}
\subsubsection*{Review questions}

\begin{enumerate}


\end{enumerate}
\end{comment}


\subsubsection*{Further reading}

For more information:  
%
\begin{itemize}
\item In the US, international transactions are reported along with the National Income and
Product Accounts by the Bureau of Economic Analysis.  See their
\href{http://www.bea.gov/bea/di1.htm}{International Economic Accounts}.

\item The International Monetary Fund's
\href{http://ifs.apdi.net/imf/ifsbrowser.aspx?branch=ROOT}{International Financial Statistics} is
the best single source of balance of payments and international investment data.

\item International standards for BOP data are set by a working committee of the International
Monetary Fund.  Their \href{http://www.imf.org/external/np/sta/bop/bop.htm}{web site} includes
discussions of both conceptual and measurement issues. The
\href{http://www.imf.org/external/bopage/arindex.htm}{annual reports} are a good overview.  One of
the recent highlights:  in 2002, the world trade balance was \$44b, meaning that countries
reported \$44b more exports than imports.  Since every export must be someone else's import, this
can't really be true, but it points to some of the difficult measurement issues faced by the
people putting these accounts together.

\end{itemize}


\vfill \centerline{\it \copyright \ \number\year \ NYU Stern
School of Business}

\end{document}
