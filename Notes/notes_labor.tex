\documentclass[letterpaper,12pt]{article}

\usepackage{ge05}
\usepackage{comment}
\usepackage{booktabs}
\usepackage{amsmath}
\usepackage[dvipdfm]{hyperref}
\urlstyle{rm}   % change fonts for url's (from Chad Jones)
\hypersetup{
    colorlinks=true,        % kills boxes
    allcolors=blue,
    pdfsubject={NYU Stern course GB 2303, Global Economy},
    pdfauthor={Dave Backus @ NYU},
    pdfstartview={FitH},
    pdfpagemode={UseNone},
%    pdfnewwindow=true,      % links in new window
%    linkcolor=blue,         % color of internal links
%    citecolor=blue,         % color of links to bibliography
%    filecolor=blue,         % color of file links
%    urlcolor=blue           % color of external links
% see:  http://www.tug.org/applications/hyperref/manual.html
}


\def\ClassName{The Global Economy}
\def\Category{Class Notes}
\def\HeadName{Labor Markets}

\begin{document}
\thispagestyle{empty}%
\Head

\centerline{\large \bf \HeadName}%
\centerline{Revised: \today}

\bigskip
Some of the most important markets for aggregate economic performance
are those for labor and (financial) capital,
which affect every industry and product.
%They are also among the most complex markets ---
%and among the most heavily regulated.
Countries differ markedly in their treatment of both markets,
with (evidently) different outcomes as a result.

Our focus here is on labor markets.
We describe in broad terms what well-functioning markets might look like
and compare them to the kinds of institutions and regulations we see around the world.


\subsubsection*{Labor market indicators}

Most countries collect extensive labor market data.
This includes measures of employment, unemployment,
and sometimes detailed information about flows or workers
in and out of jobs.

We'll start with the {\it population\/},
a count of the total number of people in a given geographic area.
Strangely enough, numbers like this are estimates:
we don't know exactly how many people there are in
the US, Canada, or China, but statistical agencies come up
with estimates by a number of methods, typically based on surveys.
For most labor market statistics,
the starting point is the
{\it adult population\/}.
Countries differ in their definitions of an adult,
but think of people between the ages of 25 and 64.

The next step is to identify labor market status of everyone
in the adult population:
(i)~employed (has a job), (ii)~unemployed (not working but would like to),
and (iii)~inactive or not in the labor force (everyone else).
%Categories (i) and (ii) constitute the {\it labor force\/}.
Category (ii)~is a little fuzzy:  how do we know whether or not
you would like to work? In some countries the answer comes from a
survey: we ask the person whether he or she is actively looking
for a job. In others only people claiming unemployment benefits
are classified as such.

Information on the labor market status of individuals leads to
statistics on the {\it labor force\/},
the {\it employment rate\/},
the {\it participation rate\/},
and the {\it unemployment rate\/}.
The employment rate is the ratio of employment to the adult population.
The labor force is the number of people in categories (i) and
(ii): either working or unemployed.
The participation rate is the ratio of the labor force to the adult population.
The unemployment rate is the ratio of the number of people who are unemployed to the labor force.

The details here are important.
We see large differences, for example, in employment rates across countries.
In the US, for example,
the employment rate has been about 60\% over the last decade,
and in France about 52\%.
The source of this difference lies primarily in the number of inactive
people, not the number of unemployed.
For that reason, and because it's easier to measure,
many analysts focus on employment rather than unemployment.
Newspapers, of course, tend to do the opposite.


In the US, employment data are collected and reported monthly by
the Bureau of Labor Statistics. The BLS releases its
closely-watched monthly report, ``The Employment Situation,'' at
8:30 am on the first Friday of the month.  It includes such
indicators as employment (``nonfarm payroll''),
the unemployment rate, and the size of the labor force.
This release is based on two surveys, the Current
Employment Statistics (CES) survey of payroll records of firms and
the Current Population Survey (CPS) of households. The CES covers
300,000+ businesses and provides detailed industry data on
employment, hours, and earnings of workers on nonfarm payrolls.
The most closely watched number in the US is probably the monthly
estimate of nonfarm employment, which comes from this survey.
The CPS covers 60,000+ households and is conducted for the BLS by
the Bureau of the Census. It provides a comprehensive body of data
on labor force status:  employment, unemployment, and so on.  Both
sets of data are updated periodically as more information comes
in.  Since the two sources are radically different,
they occasionally result in conflicting information about such
basic indicators as the number of people employed.

Other countries collect similar data, but the sources and definitions vary.

%Many countries also collect data on changes in labor market status
%from one month to the next, generally termed {\it labor market
%flows\/}.
\begin{comment}
In a typical case, the categories are expanded as
follows: (ia)~employed in same job as last period, (ib)~employed
in different job, (ii)~unemployed, and (iii)~not in labor force.
Related information is collected from firms.  A firm is said to
create jobs if it lists more positions (filled or not) than in the
previous period and destroy jobs if there are fewer.  Total {\it
job creation\/} for the economy as a whole is the total number of
positions added by firms that create jobs.  Similarly, total {\it
job destruction\/} is the total number of positions lost by firms
that destroy jobs.  The astonishing feature of flow data is how
much churning goes on: millions of people gain and lose jobs every
month, and millions of jobs are created and destroyed, whether the
economy is growing or shrinking.
\end{comment}
%More on this shortly.



\subsubsection*{Supply and demand for labor}

Differences in the abilities and skills of individuals
make labor markets incredibly complex,
but we can get a sense of the impact of government regulation
in a model in which there is a single market for a single kind of labor.

%\bigskip
{\it Demand for labor.\/}
The demand for labor comes from (typically) firms,
whose decision-making process works something like this.
Consider a firm that produces output using the production function
$Y= AF(K,L)=AK^{\alpha}L^{1-\alpha}$ (``Cobb-Douglas'').
Let us say,
in the interest of keeping things as simple as possible, that the
capital stock is some fixed number $K$. How does the firm's profit
depend on the choice of labor input $L$? If $p$ is the price of
one unit of output and $w$ is the price (wage) of one unit of
labor (here the unit may be one hour), the firm's profit is
%
\begin{eqnarray*}
    \mbox{Profit} &=& p Y - w L - \mbox{Fixed Costs} \\
                  &=& p A F(K,L) - w L - \mbox{Fixed Costs} .
\end{eqnarray*}
%
How does profit vary with $L$? The fixed costs might be attributed
to capital or other factors, but they do not vary with $L$.
For this reason they won't affect the demand for labor.  What is
affected by $L$ is output (which in turn affects sales revenue)
and cost (the wage bill).

\begin{comment}
If we increase labor by an amount
$\Delta L$, the increase in output is
%
\[
    \Delta Y  \;=\; F(K,L+\Delta L)-F(K,L)
        \;=\; \frac{F(K,L+\Delta L)-F(K,L)}{\Delta L}\Delta L.
\]
%
When $\Delta L$ is small, this is a derivative:
%
\begin{equation*}
    \Delta Y  \;\approx\; \frac{\partial F(K,L)}{\partial L}\Delta L.
\end{equation*}
%
(See the ``Mathematics Review'' if this seems overly mysterious.)
The increase in revenue is therefore $p\times \Delta Y = p \times
[\partial F(K,L)/\partial L] \Delta L$. What is the increase in
cost from the same increase $\Delta L$? Simply $w\Delta L$.
\end{comment}

If the firm maximizes its profit, it will add labor as
long as the marginal benefit is greater than the marginal cost.
We learned earlier that we can characterize the solution
to the firm's maximization problem by computing the derivative
of the profit function and setting it equal to zero.
After arranging terms, we see that maximum profit occurs
when the marginal benefit of an additional unit of labor (more revenue)
equals its marginal cost (the wage):
\[
    p A \; \frac{\partial F(K,L)}{\partial L} \;=\; w .
\]
%
This is an equation we can solve for $L$ and represents the amount
of labor the firm will hire (demand) for any given values of $p$,
$w$, and $K$.  In the Cobb-Douglas case, $ Y = A K^\alpha
L^{1-\alpha}$ and the demand function follows from
%
\begin{equation*}
    p (1-\alpha) A K^{\alpha}L^{-\alpha} \;=\; w ,
\end{equation*}
%
which implies (solve for $L$)
%
\begin{equation}
    L \;=\; K\left[\frac{ p (1-\alpha) A}{w}\right]^{{1}/{\alpha}}.
    \label{eq:ld}
\end{equation}
%
The aggregate (total) demand for labor is the sum of demands
across all firms. Its important feature, for our purposes, is that
it falls when the wage rises, just as we assume most demand
functions do.
%That is:  demand curves slope down.
(This property follows from the diminishing marginal
productivity of labor, one of the conditions we imposed on the
production function.)

%\bigskip
{\it Supply of labor.\/}
Now let's turn to the supply of labor. The short version: supply
increases with the wage.

A more complete version goes like this.
When selling their labor services, individuals
make two decisions: whether to work at all and, if so, how much.
Obviously, many jobs offer limited flexibility on the
second dimension.  For this reason, and also because it simplifies
the analysis, we will ignore the second choice and consider the
decision of an individual deciding whether or not to work a given
number of hours $h$.  What is the gain from working? The pecuniary
gain is $wh$. However, what matters for the individual is the
increase in happiness, or satisfaction, that such pay induces.
This may vary across individuals.  For example, the same pay may
be more valuable to a poor individual than to a rich one. What is
the loss from working? Simple!  He has less time to dedicate to
other activities, including playing bridge or football, cooking,
reading economics books, or spending time with family.
The value of these
activities is what economists call the \textit{reservation wage}.
For any wage less than this, the benefits of not working are
greater than the benefits of working. For the economy as a whole,
the higher the wage, the greater the number of people
who decide to work.
Thus aggregate labor supply $L^{s}$ increases with the wage rate
$w$.

% ?? real and nominal ??

%\bigskip
{\it Labor market equilibrium.\/}
The equilibrium wage rate is a value $w^{*}$ at which demand and
supply are equal, as illustrated in Figure~\ref{fig:eq}.
The number of workers employed at this wage defines the level of
employment $L^{*}$. This kind of analysis should be familiar to
anyone who has taken an economics course.


%%%%%%%%%%%%%%%%%%%%%%%%%%%%%%%%%%%%%%%%%%%%%%%%%%%%%%%%%%%%%%%%%%%%%%%%%%%%
%  Supply and demand diagram
\begin{figure}[h!]
%
\begin{center}
\setlength{\unitlength}{0.075em}
\begin{picture}(250,200)(0,-10)
%\footnotesize
\thicklines

% horizontal axis
\put(-30,0){\vector(1,0){300}}
\put(255,-16){$L$}

% vertical axis
\put(0,-20){\vector(0,1){200}}
\put(-15,155){$w$}

% demand
\put(35,165){\line(4,-3){200}}\put(240,10){$L^d$}

% supply
%\put(35,13){\line(5,3){200}} \put(240,130){$L^s$}
\put(35,33){\line(5,3){200}} \put(240,150){$L^s$}
%\put(35,53){\line(5,3){200}} \put(240,170){$S''$}

% equilibrium labels
%\put(157,77){\footnotesize A}
%\put(142,88){\footnotesize A}
%\put(121,54){\footnotesize C}
%\put(138,64){\footnotesize C}
%\put(133,92){\line(0,-1){92}}
% dotted line
\qbezier[31]{(133,0)(133,46)(133,92)}
\qbezier[45]{(0,92)(67,92)(133,92)}
%\qbezier[45]{(0,72)(67,72)(133,72)}

\put(131,-16){$L^*$}
\put(-15,90){$w^*$}


\end{picture}
\end{center}
\caption{Equilibrium in the labor market.}
\label{fig:eq}
\end{figure}
%%%%%%%%%%%%%%%%%%%%%%%%%%%%%%%%%%%%%%%%%%%%%%%%%%%%%%%%%%%%%%%%%%%%%%%%%%%%

%{\it Unemployment and distortions.\/}
Where is unemployment in Figure~\ref{fig:eq}?
We might think of
$L^*$ as representing both employment (the number of people hired
by firms, since the point is on their demand curve) and the labor
force (the number of people who want to work at the going wage).
Unemployment is the difference between these two numbers, so
evidently there is none.  There is an almost perfect analogy
between this market and a dealer-type securities market such as
the Nasdaq.  At the Nasdaq, dealers post bid and ask prices and
are committed to executing transactions at those prices. If you
want to sell a given security, you just contact one of them (or
ask your broker to do this for you).  If you'd like to sell at
less than the bid, you have a deal.  There are no ``unemployed''
securities.

%If the labor market worked this way, individuals would post their
%reservation wage.  If it were low enough, some firm would hit
%itThere could not be unemployment, simply because any person
%willing to work would be able to do so, simply by lowering the
%reservation wage enough.


\subsubsection*{Supply and demand with a minimum wage}

There are many reasons labor markets don't work quite
like our frictionless market,
but we'll look at one:  government policies and institutions.
Examples include minimum wages, restrictions on wage adjustments, labor unions, limits on hiring and firing, and so on.
We don't argue that these things are bad --- they may very
well have benefits --- but they do have an impact on how the labor
market works.

%%%%%%%%%%%%%%%%%%%%%%%%%%%%%%%%%%%%%%%%%%%%%%%%%%%%%%%%%%%%%%%%%%%%%%%%%%%%
%  Supply and demand diagram
\begin{figure}[h!]
%
\begin{center}
\setlength{\unitlength}{0.075em}
\begin{picture}(250,200)(0,-10)
%\footnotesize
\thicklines

% horizontal axis
\put(-30,0){\vector(1,0){300}}
\put(255,-16){$L$}

% vertical axis
\put(0,-20){\vector(0,1){200}}
\put(-15,155){$w$}

% demand
\put(35,165){\line(4,-3){200}}\put(240,10){$L^d$}

% supply
%\put(35,13){\line(5,3){200}} \put(240,130){$L^s$}
\put(35,33){\line(5,3){200}} \put(240,150){$L^s$}
%\put(35,53){\line(5,3){200}} \put(240,170){$S''$}

% dotted line
\qbezier[30]{(133,0)(133,46)(133,92)}
\qbezier[45]{(0,92)(67,92)(133,92)}
%
\qbezier[38]{(165,0)(165,46)(165,112)}
\qbezier[38]{(101,0)(101,46)(101,112)}
\qbezier[55]{(0,112)(67,112)(165,112)}

% labels
\put(104,120){\footnotesize A}
\put(160,120){\footnotesize B}
\put(143,87){\footnotesize C}
\put(107,62){\footnotesize D}
\put(6,120){\footnotesize E}
\put(89,80){\footnotesize F}

\put(131,-16){$L^*$}
\put(-19,90){$w^*$}
\put(-19,110){$w_m$}

\end{picture}
\end{center}
\caption{The labor market with a minimum wage.}
\label{fig:minimum}
\end{figure}
%%%%%%%%%%%%%%%%%%%%%%%%%%%%%%%%%%%%%%%%%%%%%%%%%%%%%%%%%%%%%%%%%%%%%%%%%%%%


We'll look at the minimum wage, since it's the simplest example.
(Remember:  simple is a good thing.)
If we prohibit purchases of labor services below a given wage,
how does the market work?
Figure~\ref{fig:minimum} shows how the supply-and-demand analysis
might be applied in this case.  In the figure, the minimum wage
($w_{m}$) is higher than the equilibrium wage ($w^{*}$).
At the minimum wage, the supply of labor
(the line segment EB) is greater than the demand for labor
(the line segment EA).
Minimum wage laws don't force firms to hire,
so employment is given by their demand function.
The difference between supply and demand is
unemployment (the line segment AB).
In the figure, unemployment is represented by the line segment AB.
The unemployment {\it rate\/} is the ratio of unemployment (AB)
to the supply of labor (EB).
As a practical matter, it's an open question whether the impact of the
minimum wage on unemployment is large or small, but few economists
doubt that it's there.


Who are the winners and losers from the minimum wage?
The beneficiaries are the employed,
who get a higher wage than they would otherwise.
Who loses?  The unemployed (who would prefer to work)
and firms (who must pay more for labor).
Taken as a whole, the policy reduces welfare by an amount
represented by the triangle ACD.
This will seem clear if you recall how welfare triangles work,
mysterious otherwise,
but it's true in this setting that a minimum wage reduces overall welfare.

\subsubsection*{Labor market institutions}

The minimum wage is a useful illustration,
but labor market institutions and regulations
differ dramatically across countries along many dimensions.
Some of them have been collected and reported by the
World Bank's Doing Business.
They report, for most countries in the world,
the flexibility an employer has over the terms of employment,
including the employer's ability to vary hours,
the cost and difficulty of dismissal,
and whether workers can be hired under fixed-term (as opposed to permanent)
contracts.
Some countries also limit hours (no more than 25 hours a week!) 
or give unions more power over the terms of employment contracts.


The evidence suggests that many kinds of labor market regulation
have the ultimate effect of reducing employment.
There is, of course, a useful role for laws that enforce fair dealing.
Paying workers for fewer hours than they worked,
for example, is impossible to justify.
But many labor market regulations seem to discourage firms from hiring,
although their stated purpose is ``protect jobs.''
Restrictions on layoffs and mandatory severance payments
seem to do this, since they raise the long-term cost
of hiring workers.
In many countries, we've seen growth in part-time and unofficial
workers precisely to get around this kind of regulation.


\subsubsection*{Labor market flow indicators}

The supply and demand diagram is indicative,
but it can't address one of the most striking features of
modern economies:
the huge amount of turnover.
Economies create and destroy jobs at truly amazing rates.
Even MBA graduates often find themselves changing jobs
frequently, either by choice (they find jobs they like better)
or misfortune (economic downturns, mergers, and so on).


In the aggregate,
we see this in data from both firms and households.
At firms, new positions are created either because existing firms
expand their operations or because new firms hire workers. We
refer to this phenomenon as \textit{job creation}. At the same
time, existing firms sometimes eliminate positions.  We refer to
this process as \textit{job destruction}.  An establishment (a
production unit, like a factory or store) is said to create
jobs if it increases the number of positions between one period
and the next.  The economy's overall job creation is the sum of
the increases in positions across all the establishments that
created jobs.  Similarly, an establishment is said to destroy jobs
if it decreases its number of positions from one period to the
next. The economy's overall job destruction is the sum of the
decreases in jobs across all establishments that destroyed jobs.
The sum of job creation and job destruction is referred to as
\textit{job reallocation} or \textit{job turnover}. Job creation,
destruction, and reallocation \textit{rates} are ratios of
job creation, destruction, and reallocation, respectively, to the
total number of jobs in the economy. These rates vary greatly
across countries, but their magnitudes are impressive everywhere.
%In the US in 1988 (the last available data),
%the annual job creation rate in manufacturing
%was 8.3\%. The job destruction rate was about the same. Therefore
%job reallocation amounted to a staggering 16.7\% a year. For the
%US economy as a whole, job creation and destruction rates were
%13.0\% and 10.4\%, respectively, over the same period.

We see similar turnover in household data,
with individuals reporting frequent changes in jobs
or employment status.
An employed individual can either stay in her job, move to another
job, become unemployed (leave one job but look for another), or
leave the labor force (leave one job without wanting another).
Similarly, an unemployed person can become employed (get a job),
stay unemployed (look for a job without taking one), or leave the
labor force (stop looking for a job).  Someone out of the labor
force can become employed (take a job), become unemployed (look
for a job without taking one), or remain out of the labor force.
We have information from surveys on each of these possible
changes.

Some summary measures will give you a sense of how this looks for
a typical individual.  {\it Accessions\/} occur when individuals
take new jobs, regardless of their current status. {\it
Separations\/} occur when individuals leave jobs, regardless of
reason. The accession and separation {\it rates\/} are ratios of
accessions and separations, respectively, to total jobs.  The sum
of the two rates is referred to as the \textit{worker reallocation
rate}. In the late 1980s, the annual accession rate in the US was
45.2\% and the annual separation rate was 46.0\%.  The worker
reallocation rate was a staggering 91.2\%! These numbers do not
say why workers change jobs, merely that they do. The numbers for
the US are unusually high.  Worker reallocation rates are
generally lower in other countries, including France (59.6\%) and
Italy (68.1\%). Apparently in the US it is more common both to
leave a job and to take a new job.
You might ask yourself whether the factors that make the separation rate
high also make the accession rate high, or whether they are
influenced by different factors.

\begin{comment}
Fine points (feel free to skip). Part of worker reallocation is induced by job
reallocation. That is, when a job is created, there is always an
accession and when a job is destroyed there is always a
separation. The opposite, however is not true. Accessions can
occur without a new job being created and separations can occur
without a job being destroyed. This happens whenever a company
substitutes a worker for another one in exactly the same position.
The excess of worker reallocation over job reallocation is called
\textit{excess worker reallocation}.
\end{comment}

\subsubsection*{Virtues of flexible input markets}


There's a big-picture point hidden here somewhere.
In a modern economy, the mix of products is changing constantly.
Many of the products and services we use today
simply weren't available twenty years ago:
high-speed internet access, iPods, databases, and so on.
A successful economy needs a mechanism to shift
capital and labor to these new uses and away from others.
The evidence suggests that countries with labor and capital markets
that do this well tend to perform well in the aggregate.
The ability of an economy to reallocate jobs across
firms, industries, and geographical areas is
perhaps even more important than capital.
Sometimes labor market institutions are an obstacle to this
reallocation process,
reducing aggregate productivity and output.


Most developed economies have gone through at least one
major reallocation in their history,
as most people shifted from agriculture to industry.
In the US, there's been a further shift from manufacturing to services.
Services seem less vivid than physical goods to most of us,
but the fact is that goods represent a smaller part of modern
developed economies than services.
You might ask yourself what sector you're likely to go into.
Past experience suggests that few of our students are targeting jobs
in manufacturing, much less agriculture.

If reallocation is important,
then labor market institutions are central to aggregate performance.
If people cannot be shifted quickly to more productive sectors
and firms, aggregate productivity will be lower.
Consider the impact of an increase in productivity in a single industry.
What is the impact on employment?
The immediate effect is probably to throw some people out of work, since
a smaller number of people can produce the same output.
It depends on the elasticity of demand:  if the elasticity is low,
the increase in productivity reduces employment.
The long-term effect is to reallocate these workers
to other sectors.
We would argue that this is good for the economy as a whole,
even if it's painful for the people who had reallocation forced on them.
Nevertheless, there's a long history of concern about reallocation,
dating back at least to the machine-smashing ``Luddites.''


Deregulation is similar, in the sense that it often leads to reallocation of capital and labor.  Consider an example from Europe. In recent years, the European
airline market has gone through a deregulation process that
resembles the one that characterized the US in the 1980s. Before
deregulation, every European country had its own national airline.
The airline was typically a monopolist on internal routes and a
(collusive) duopolist (with the other country's airline) on
international routes.  Like many monopolists, these airlines
charged high prices and operated inefficiently.  Now any company
can fly any route in the European Union. The entry of new carriers
has meant big trouble
for many of the incumbents. Sabena and Swissair went bust. Others
(Alitalia?) may follow suit. It is clear that at the end
of the reorganization of the industry, fares will be much lower
(they are already, as a matter of fact).
As a result, people will fly more often,
 and there may even be more jobs in the industry.
 Along the way, job destruction by the incumbents will be
 accompanied by job creation by the new entrants.


Who gains? Consumers and workers in new firms. Who loses?
The workers and shareholders of incumbent firms.
Standard analysis of the costs of monopoly tells us
that the gains are greater than the losses,
but the political process does not necessarily weight them
equally.  In fact, the costs are often concentrated while the benefits are thinly spread over many individual consumers and firms,
so the political process will tend to give greater weight to the former.
Nevertheless it's likely that such a reallocation
is a net benefit to the society as a whole.

%\begin{comment}
\subsubsection*{A model of unemployment dynamics (optional)}

These flows suggest a more dynamic labor market than our supply-and-demand
analysis.
Where do they come from?
You might consider your own case:
what factors might cause you to leave a job?   take a new one?
We'll look at a simple model that helps us to understand the overall
impact of individual decisions like this on economy-wide
employment and unemployment rates.
To keep things simple, we will ignore transitions in and out of the labor
force and focus on employment and unemployment.

To start, individuals may be either unemployed, $U$, or employed, $E$, with $U+E={L}$ and $L$ a fixed number.
The unemployment and employment rates are then $u=U/{L}$ and $e=E/{L}$.
Since the only states are $U$ and $E$,
the employment and unemployment rates sum to one: $ u+e=1$.

The unemployment rate can change over time if individuals
change their labor market status.
Let us say that a constant fraction $0<s<1$ ($s$ for separation)
of employed people lose
their jobs and become unemployed.
Why might this be?  Perhaps because their firms are doing poorly,
and either shrink or go out of business altogether.
Let us also say that a constant fraction $0<a<1$ ($a$ for accession) of
unemployed people find new jobs, as they find employers who need their skills.
We've seen that this kind of reallocation is the norm for modern economies.
With these two ingredients, the unemployment rate changes like this
from one period to the next:
\begin{equation*}
    u_{t+1} \;=\; u_{t}+se_t-au_t
\end{equation*}
or
\[
    \Delta u_{t+1} \;=\; se_t-au_t .
\]
Unemployment rises over time if more employed people lose their jobs than unemployed people find new jobs and falls over time if the reverse is true. (Formally, this should remind you of the dynamics of capital accumulation in the Solow growth model).
Since $u_t+e_t=1$, we can write this as
\begin{equation}
    \Delta u_{t+1} \;=\; s (1-u_{t})-a u_{t}
    \label{eq:u-dyns}
\end{equation}
This is useful because we have summarized the dynamics of unemployment in a single equation. If you were told that the unemployment rate was now (say)  $u_{0}$, then repeated application of (\ref{eq:u-dyns}) would enable you to calculate the unemployment rate in future periods. It would be a very simple matter to do these calculations in Excel.

Using equation (\ref{eq:u-dyns}) we can compute
both the period-by-period dynamics and the steady state unemployment rate.
We find steady state unemployment ($\overline{u}$, say) by setting $\Delta u_{t+1}=0$ so that
\[0=s(1-\overline{u})-a\overline{u}\]
Solving for $\overline{u}$ gives
\[
    \overline{u} \;=\; \frac{s}{s+a} ,
\]
a number between zero and one. Over time, the unemployment rate ``gravitates'' to this value, so we might also refer to it as the ``long run'' or ``natural'' unemployment rate.
The idea is that we want a measure of the underlying unemployment rate that would prevail if it were not for temporary shocks that make $u_0\ne\overline{u}$.
Clearly, steady state unemployment is higher when $s$, the separation rate, is higher and is lower when $a$, the accession rate, is higher. That is, if people lose jobs easily and have difficulty finding them then the unemployment rate will be higher.

You should think of the parameters of this model, $a$ and $s$, as standing in for various country-specific labor market policies and institutional arrangements that affect the willingness of firms to hire and fire workers, the willingness of workers to take and quit jobs, and so on.
Examples of such policies include: minimum wages,
taxes on labor income, regulations that control the length of the working week (such as the 35 hour week in France), the size of severance payments for layoffs, the size of unemployment benefits, and so on. Which of these policies decrease $a$? Which of these policies decrease $s$?

Here is a different way to look at the steady state calculation. With a little bit of statistics, one can show that an individual's average duration of unemployment in this model is $1/a$ and an individual's average duration of employment is $1/s$. Steady state unemployment is given by
\[\overline{u} = \frac{s}{s+a} = \frac{\frac{1}{a}}{\frac{1}{a}+\frac{1}{s}}=\frac{\textmd{duration of  $U$}}{\textmd{duration of $U$ + duration of $E$}}
\]
So, to think about why European unemployment is relatively high when compared to the US, we might begin with factors that either increase the duration of unemployment or reduce the duration of employment. Which of these factors do you think is more important for European unemployment?
%\end{comment}


\subsubsection*{Institutions and labor market dynamics (optional)}

We can get a sense of the dynamics of labor market reallocation
 by putting our earlier labor market analysis to work.
In our model, the accession and separation rates also govern the response of the unemployment rate to a shock that pushes unemployment above (or below) its steady state value. In particular, these rates govern the {\it speed of adjustment} back to steady state. To see this, write the equation governing the dynamics of the unemployment rate as
\[
    u_{t+1} \;=\; s+ [1-(s+a)] u_{t} .
\]
In order for the model to be empirically realistic, we need a worker reallocation rate, $s+a$, that satisfies
\[0<s+a<1\]
With a little bit of algebra, the unemployment rate can be written in terms of deviations from its steady state value
\[
    u_{t+1}-\overline{u} \;=\; \lambda (u_t-\overline{u}) ,
\]
where $\lambda=1-(s+a)$ is the speed of adjustment of the unemployment rate to its steady state value. The reason for this name is that with a little bit more algebra one can show that if the unemployment rate is hit by a temporary shock (that does not affect $a$ or $s$) so that it takes on the value $u_0$, the adjustment back to steady state is given by the formula
\[
    u_t-\overline{u} \;=\; \lambda^{t}(u_0-\overline{u})
\]
or
\[
    u_t \;=\; (1-\lambda^{t})\overline{u}+\lambda^{t}u_0 .
\]
What's important is not the algebra but the idea:
in economies with low values of $s$ and $a$,
worker reallocation is slow.
Here we see that this results in slow adjustment of the unemployment rate.
But you can imagine, as well, slow reaction of producers to consumer
preferences.
The churning of the labor market implied by high values of $s$ and $a$
thus serves a purpose:  it shifts workers to jobs where their value is the highest.



\begin{comment}
Notice that a high separation rate $s$ has two important effects on the economy: it (i)~raises the steady state unemployment rate, but (ii) helps the economy respond quickly to temporary shocks to unemployment. On the other hand, a high accession rate $a$ has two unambiguously good effects: it (i) reduces the steady state unemployment rate, and (ii) helps the economy respond quickly to temporary shocks. This suggests that one way to think about the European unemployment problem is as an economy with an $s$ that is moderate and an $a$ that is low.
\end{comment}

Numerical example.
%Because we have ignored transitions out of the labor force, these numbers %can't be realistic.
Suppose in both Europe and the US, $s=0.01$, and that in the US accession is $a_{US}=0.19$ but lower in Europe, $a_{EU}=0.09$. Then not only is European unemployment 10\% as opposed to 5\% in the US, but European unemployment takes considerably longer to respond to shocks than does the US ($\lambda_{EU}=0.90,\lambda_{US}=0.80$). Based on this example, can you imagine the economic consequences if, as recent evidence hints, the U.S. labor market has become less flexible?




\subsubsection*{Executive summary}

\begin{enumerate}
\item Labor market indicators include employment (the number of
people working), unemployment (the number not working who would
like to), the labor force (employment plus unemployment), and the
unemployment rate (the ratio of unemployment to the labor force).

\item Measures to protect workers often have the opposite effect:
by making labor more expensive to firms, they may reduce employment.

\item Many developed economies are characterized by high rates of job
and worker reallocation: people change jobs and firms change
workers --- frequently.

\item The efficiency of work reallocation affects the performance
of the economy as a whole. In particular, flexible labor markets help an economy respond quickly to the inevitable changes that occur.

%\item Labor-saving productivity is good:
%it allows us to choose between less work and more output.
%In practice, the main effect has been more output,
%which we get by reallocating labor across sectors.

\end{enumerate}

\subsubsection*{Review questions}

\begin{enumerate}

\item Labor market indicators.
You are the mayor of a small village, whose
labor market data are: population (100), employment (55), and
unemployment (5).
\begin{enumerate}
\item Draw a circle on a piece of paper corresponding to the
population of your village. Divide it into sections corresponding
to employment, unemployment, labor force, and not in the labor
force.

\item Compute the participation rate and the unemployment rate.
\end{enumerate}

Answer.
\begin{enumerate}
\item For you to do.

\item The participation rate is 60\% [$= (55+5)/100$] and the
unemployment rate is 8.3\% [$= 5/60$].
\end{enumerate}

\item Supply and demand. Consider an economy with one firm (which
is nevertheless a price taker).  It produces widgets according to
the production function $Y = A K^{1/2}L^{1/2}$ and sells them for
2 dollars each.
[Yes, we know we said $\alpha$ would always be one-third,
but apparently we lied.  Sorry.]
For simplicity, assume that $A = K = 1$.  The
supply of labor in this economy is $L^{s}=w^{3/2}$.
%
\begin{enumerate}
\item What are the equilibrium values for the labor force, the
wage, and employment? What is the unemployment rate?

\item Suppose the government imposes a minimum wage of \$1.1. What
are the equilibrium values for the labor force, the wage, and
employment? What is the unemployment rate?
\end{enumerate}

Answer.
%
\begin{enumerate}
\item The demand for labor is given by $L^{d}(w) = 1 [ 2 \times
(1/2)]/{w}]^{2} = w^{-2}$.  [This follows from the labor demand
function, equation (\ref{eq:ld})].  Equating demand and supply
yields $L^{*}=w^{*}=1$. The labor force is one and the
unemployment rate is zero.

\item With a minimum wage of \$1.1, the supply of labor (the labor
force) is $1.1^{{3}/{2}} = 1.15$. The demand for labor is
$1.1^{-2}=0.826$. Therefore employment is 0.826 and the
unemployment rate is $(1.15-0.826)/{1.15}= .2817$ or 28.17\%.
\end{enumerate}


\item Layoffs.
From the New York Times, March 6, 2009 (rough paraphrase):
The WARN Act requires employers to give 60 days' notice
if a plant is closed or 500 or more people are laid off at one location.
Some wonder whether notice should be required for other job losses.
A Berkeley professor says
it's a matter of ``transparency and decency.''
An IBM executive disagrees, noting that it's routine for the company to lay off some employees while hiring others.
``This business is in a constant state of transformation.''
If you were asked to advise President Obama,
how would you describe the costs and benefits of
wider application of the WARN Act?

Answer.
This is another cost of firing workers,
and we know that higher firing costs tend to discourage firms
from hiring workers in the first place.
The question is how high the cost is.
If we figure workers will keep producing, it may be small,
but I'd guess it was closer to adding 60 days to severance.
And there's probably red tape associated with such a law.

I asked to two of my favorite management professors
the same question.
Professor Wiesenfeld said:
``I think giving people advanced notice (not 60 days, but not escorting them to the door immediately) is generally beneficial,
because it helps to retain the commitment and motivation of the person's former colleagues.''
Professor Freedman said:
``It depends on the situation.  It has the potential to be extremely serious. For example, I have encountered situations where people who are given notice act in a very destructive manner.''
He adds:
``In bad times the flexibility [to fire some people and hire others]
is much more important than in good times because organizations have less slack.''

\item Income support systems.
In the US and UK, flexible labor markets are accompanied by
income support systems in which (for example) people with low
incomes pay negative income tax; that is, they receive money.
In what ways is this better than direct intervention in labor markets?  Worse?

Answer.  Many see this as an effective way to
reconcile the benefits of a flexible labor market
with the safety net that protects workers from some
of the challenges of losing one's job.
Restrictions on firing, for example, help people with jobs keep them,
but discourage firms from hiring more people.


\item Labor market dynamics.
In our model of unemployment dynamics, suppose the accession rate is
$a = 0.2$ and the separation rate is $s = 0.01$.
\begin{enumerate}
\item Compute the steady state unemployment rate and the speed
of adjustment parameter.
\item Suppose a restriction on firing makes workers less attractive to firms.
What effects might this have on long-run unemployment
and the response of the economy to an increase in the unemployment rate?
\end{enumerate}

Answer.
\begin{enumerate}
\item The steady state unemployment rate is $\bar{u} = s/(s+a) = 0.048$
or 4.8\%.
The speed of adjustment parameter is $\lambda = 1-(s+a) = 0.79$.
\item This restriction is likely to lower the accession rate $a$,
thereby increasing long-run unemployment and making periods of high unemployment
last longer (ie, making $\lambda$ closer to one).
\end{enumerate}



\end{enumerate}

\subsubsection*{If you're looking for more}


Labor data is less standardized across countries than national
income and product accounts.
The US Bureau of Labor Statistics has a nice collection
of
\href{http://www.bls.gov/fls/}{international data}
constructed in roughly comparable ways across countries.
Ditto the
\href{http://www.oecd.org/topicstatsportal/0,2647,en_2825_495670_1_1_1_1_1,00.html}
{OECD}.
The World Bank's Doing Business
has an extensive collection of
\href{http://www.doingbusiness.org/data/exploretopics/employing-workers}
{indicators}
of labor market institutions.

Finally, Denmark is a fascinating example when it comes to labor markets, 
since it combines flexibility (it's easy to fire people, for example)
with generous social support (unemployment insurance
and training, for example).
A nice overview is
Jianping Zhou's
\href{http://www.imf.org/external/pubs/ft/wp/2007/wp0736.pdf}
{IMF working paper},
``Danish for All? Balancing Flexibility with Security,'' 2007.


\vfill \centerline{\it \copyright \ \number\year \ NYU Stern
School of Business}


\end{document}

