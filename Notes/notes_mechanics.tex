\documentclass[letterpaper,12pt]{article}

\usepackage{amsmath}
\usepackage[dvips]{graphicx}
\usepackage{hyperref}
%    \hypersetup{colorlinks=true,urlcolor=blue,linkcolor=red}
\RequirePackage{GE05}

\newcommand{\GDP}{\mbox{\em GDP\/}}
\newcommand{\NDP}{\mbox{\em NDP\/}}
\newcommand{\GNP}{\mbox{\em GNP\/}}
\newcommand{\NX}{\mbox{\em NX\/}}
\newcommand{\NY}{\mbox{\em NY\/}}
\newcommand{\CA}{\mbox{\em CA\/}}
\newcommand{\NFA}{\mbox{\em NFA\/}}
\newcommand{\Def}{\mbox{\em Def\/}}
\newcommand{\CPI}{\mbox{\em CPI\/}}
\newcommand{\CU}{\mbox{\em CU\/}}
\newcommand{\RE}{\mbox{\em RE\/}}
\newcommand{\MB}{\mbox{\em MB\/}}

\def\ClassName{The Global Economy}
\def\Category{Class Notes}
\def\HeadName{Mechanics of Monetary Policy}

\begin{document}
\thispagestyle{empty}%
\Head

\centerline{\large \bf \HeadName}%
\centerline{Revised: \today}

\bigskip
This is really about the mechanics of monetary policy...
Money as it's commonly used consists primarily of the liabilities of private financial institutions.
But if they're private, how does the central affect them?  ..


\subsubsection*{Commercial Banking in the US}

     Financial intermediaries are ``middle-men" who funnel funds from sources
of funds (savers, foreign investors) to users (business, government).  In principle, savers could
purchase assets directly from users, as when an individual buys a treasury bill or share of stock.
But in practice, most funds flow through intermediaries of various types:  commercial banks,
thrift institutions, pension plans, insurance companies, and so on.  Investment banks play a role
here, too, and also help to match borrowers and lenders. To get a rough idea where these
institutions stand in the US, here are some numbers for 1991:
%
\begin{center}
\begin{tabular}{lrr}
               Institution          &   Assets   & Percent    \\
                                                 &            \\
               Commercial banks     &   3442.5   &  29.1      \\
               Thrift institutions  &   1414.1   &  11.9      \\
               Insurance companies  &   2096.9   &  17.7      \\
               Pension funds        &   2729.7   &  23.0      \\
               Finance Companies    &    812.4   &   6.9      \\
               Mutual Funds         &   1352.0   &  11.4      \\
                                                 &            \\
               Total                &  11847.6   & 100.0      \\
\end{tabular}
\end{center}
%
[The numbers are billions of dollars, from the Federal Reserve's {\em Flow of Funds Accounts
Financial Assets and Liabilities Year-End}, Z.1.]  These figures should make it clear that banks
are not, by a long shot, the only financial institutions in the US.  In fact, their market share
has fallen from about 39 percent in 1970 to 29 percent in the numbers above.  Thrifts have done
even worse, falling from 20 to 12 percent over the same period (and for obvious reasons). The big
gainers have been pension funds and mutual funds.

     While many financial trends are global, there are nonetheless
substantial cross-country differences in financial institutions.  The most obvious of these
concern banks.  The US banking system differs from many countries both in the range of services
supplied by commercial banks and in the sheer number of them---roughly 12,000 at last count.  This
compares with 10 in Canada, of which 4 or 5 have almost all of the business.  Or about a hundred
in the UK, where 6-8 banks control about 80 percent of the market. We could make similar
statements about France, Germany, and Japan:  they have nowhere near as many banks as the US.
[Again: I am referring here to commercial banks or their equivalent in other countries, not thrift
institutions, investment banks, or anything else that sounds like a bank.] These banks differ, as
well, in their range of activities.  US banks have traditionally accepted deposits from customers
(individuals and businesses) and used the proceeds to finance loans to businesses and individuals
and investments in corporate and government securities.  This is often allied with related
businesses like credit cards, foreign currency transactions, and so on.  But until recently the
reach of commercial banks did not extend to investment banking activities, like underwriting.  In
many countries, however, banks provide a more complete range of financial services.   In Germany,
for example, ``universal" banks provide investment banking and insurance services.

     In the US, the banking and insurance industries have been separately
regulated almost from the start, and commercial and investment banking were formally torn apart by
the Glass-Steagall Act of 1933---hence the separate identities now of Morgan Stanley and JP
Morgan.  In recent years, the Act has been weakened, and many of the leading commercial banks now
offer some range of investment banking services through so-called Section 20 subs.  Bankers Trust
has moved so far along this path that while it's still officially a commercial bank, it no longer
has a retail business.  Conversely, Merrill Lynch and other retail brokers now offer money market
accounts that serve much the same purpose as checking accounts at commercial banks.  In short,
changes in regulations have had an enormous impact on the financial services industry in general,
and commercial banking in particular, over the last twenty years or so.

     Along with the reduction in the market share of commercial banking, we
have seen, and will likely continue to see, a sharp reduction in the number of banks.  The huge
number of commercial banks in the US reflects a long-standing political impasse.  On the one hand,
Alexander Hamilton, resting in the Trinity Church cemetery near the head of Wall Street, argued
that a strong banking system was essential to the economic viability of the country.  On the
other, Thomas Jefferson and his Virginia compatriots thought that large banks would place too much
power in the hands of a few individuals.  The result was a compromise that placed much of the
oversight and control of banks in the hands of states.  Thus we have, in some ways, not a single
national set of banking legislation but fifty different sets.  This has come up in talks with the
European Community concerning reciprocal foreign banking regulations:  the Europeans complain that
access to US markets requires the approval of individual states as well as the federal government.
As the regulations supporting the excessive segmentation of the banking industry are relaxed, we
can expect to see a steady decline in the number of banks.


\subsubsection*{Monetary Aggregates}

     One of the ways in which commercial banks show up in this course is
through their deposits, which are part of what we generally call ``money". Money, as the word is
understood by economists and business-people, consists of both currency issued by the government
and deposits at banks (and now at other deposit-taking institutions, as well).  The narrowest
monetary aggregate is M1, which consists largely of currency and checking accounts. This aggregate
is an attempt to mimic our theory, in which we viewed money as something that could be used to
make transaction.  We defined this as currency, but in practice it's often easier to write a check
than to pay in cash, so we might want to include checking accounts in our measure of money. In
November 1990 M1 consisted of
%
\begin{center}
\begin{tabular}{lr}
                        Currency             &   244.8   \\
                        Checkable deposits   &   569.4   \\
                        Travellers checks    &     8.4   \\
                                             &           \\
                        M1 (total)           &   822.6   \\
\end{tabular}
\end{center}
%
[These numbers are billions of dollars; see Table A12 of the {\em Federal Reserve Bulletin}.]
[Note that we have lots of currency:  about \$1000 for every man, woman, and child in the US.  Who
has all this?  And why?]  In this aggregate and others, ``deposits" includes money at institutions
other than commercial banks:  savings and loans, credit unions, brokerage firms, and so on.  This
wasn't always the case, but since these institutions now have deposits that are like bank accounts
in everything but name, they are now included in the data and are treated the same by regulators
(reserve requirements, for example).

     M1 is the aggregate that conforms most closely with our macroeconomic
theory.  We argued that ``money" was useful for transactions, but paid no interest.  [That led to
the money demand relation, $L(i,Y)$, that underlies the LM curve.]  But as we often find, the
world is more complicated than our theory.  Particularly during the 1970s and 80s, the line
dividing accounts used for transactions and those that were pure investments became extremely
fuzzy.  Banks, in competing for deposits, offered accounts that paid interest like time deposits
yet had some of the check writing capabilities of checking accounts.  In fact only a small part of
the so-called checkable deposits above consists of pure demand deposits.  They also offered,
especially to large business customers, demand deposits that could be converted to time deposits
or other interest-bearing asset at the end of the day to get higher interest.

     As a result, economists who studied this found that M1 seemed to vary
erratically over time as demand deposits were relabeled as time deposits and so on, and that the
definition of M1 itself was becoming highly arbitrary. In the last decade, emphasis has switched
instead to broader definitions of money that include both demand and time deposits and are thus
less prone to variation as names and features of deposits change.  The disadvantage of the broader
definitions is that we are no longer talking about assets held purely for transactions purposes.
In any case, the numbers for M2 and M3 are:
%
\begin{center}
\begin{tabular}{lcr}
                              M2      & &    3316.4   \\
                              M3      & &    4085.8   \\
\end{tabular}
\end{center}
%
M2 contains, in addition to M1, overnight RPs and Eurodollars, money market accounts, and small
denomination time and saving accounts.  M3 contains, in addition to M2, large denomination time
deposits and Eurodollar accounts, and some other things.  [See the notes to Table A13 of the {\em
Federal Reserve Bulletin} for details.]  You can see that the major difference is between M1 and
M2:  the latter is about 4 times as big as the former.  The Federal Reserve currently emphasizes
M2 in its policy statements.  Our graphs of ``money" in earlier lectures were also M2.

     It's clear, then, that most of what we call ``money" consists of bank
deposits and is therefore not under the direct control of the Fed.  To see how the Fed might
influence M2 indirectly, we need to think a little about how the banking system interacts with
monetary policy.  That's what we do next.


\section{A Theoretical Model of a Banking System}

     Many aspects of economic theory have been around for decades, even
centuries.  Past and future changes in the global financial system, however, are likely to make
some of what we're about to do obsolete before long.  The tradition in theory has been to
emphasize the role of banks over other financial intermediaries and focus, in particular, on
banks' role as suppliers of assets that are used in making transactions---checking accounts and
their close relatives.  But as the line between banks and other institutions gets fuzzier, and
alternative means of payments arise, these two distinctions may turn out to be less useful than
they have been in the past. Nevertheless, this line of study gives us a start toward understanding
how the financial system operates.

     The objective of this section is to provide a link between the money
between the monetary aggregates used in our theory (think of this as M2) and the part of ``money"
that is under the direct control of the Federal Reserve (which we call the {\em monetary base},
MB).  We try to spell out the link between Fed policy and monetary aggregates, and the role of the
banking system in this process.

     {\em A bankless economy.}
To get ourselves warmed up, as it were, let's look at the balance sheets of the Fed and the
Private Sector in a stylized economy that has no banking system, and the effect on these balance
sheets of an open market operation.  Then we'll go on to see how a banking system changes the
analysis.  Let us say, then, that the Private Sector (excluding banks) has, among other assets,
500 of treasury bills, 100 of currency, and some equity.  Its balance sheet might then be
something like
%
\begin{center}
\begin{tabular}{lrclr}
\multicolumn{5}{c}{Private Sector Balance Sheet}                             \\
                                                                &       \\
            Assets                &&&
                        \multicolumn{2}{l}{Liabilities and Net Worth}   \\
%                                                               &       \\
            Currency        &   100   &&   Net worth  &   8600          \\
            Treasury bills  &   500   &&                                \\
            Equity          &  8000   &&                                \\
\end{tabular}
\end{center}
%
[In real life, this would be much more complicated, but since this is theory we can go easy on
ourselves.]

     The Fed might have, say, an inventory of 100 in treasury bills and a
liability of the same 100 in currency, since currency in the US is Federal Reserve Notes:  in
effect, interest free loans from the public to the Fed. (Read one sometime to see for yourself.)
Thus the Fed's balance sheet is
%
\begin{center}
\begin{tabular}{lrclr}
\multicolumn{5}{c}{Federal Reserve Balance Sheet}               \\
                                                        &       \\
                    Assets      &&&          Liabilities        \\
%                                                       &       \\
                    Treas bills & 100    &&  Currency   &100    \\
\end{tabular}
\end{center}
%
(The convention is that the Fed has no net worth:  earnings accrue to the Treasury.)

     In this economy, like the one I had in mind when we talked about the
Keynesian model, the money supply is the supply of currency:  100.  We can change this with an
open market operation.  If the Fed wants to increase the money supply by 10, it simply buys 10
worth of treasury bills from the public.  [Work through this on the balance sheets for practice.]
This changes the composition of the balance sheets of both the public sector and the Fed, but not
their net worths.  That's what was going on behind the scenes in our discussion of monetary policy
in the Keynesian model:  an increase in the money supply made the composition of the private
sector balance sheets more liquid, in the sense that it included more money after the open market
purchase than before.

     {\em A banking system.}
That was practice, now we develop the same idea for an economy with a banking system.  We add bank
deposits (and the corresponding loans) to the private sector's balance sheet and bring banks into
the picture.  A possible configuration is:
%
\begin{center}
\begin{tabular}{lrclr}
\multicolumn{5}{c}{Private Sector Balance Sheet}                             \\
                                                                &       \\
            Assets                &&&
                        \multicolumn{2}{l}{Liabilities and Net Worth}   \\
%                                                               &       \\
            Currency        &    50   &&   Bank Loans &    150          \\
            Bank Deposits   &   200   &&   Net worth  &   8600          \\
            Treasury bills  &   500   &&                                \\
            Equity          &  8000   &&                                \\
\end{tabular}
\end{center}
%
\begin{center}
\begin{tabular}{lrclr}
\multicolumn{5}{c}{Federal Reserve Balance Sheet}               \\
                                                        &       \\
                    Assets      &&&          Liabilities        \\
%                                                       &       \\
                    Treas bills & 100    &&  Currency   & 50    \\
                                &        &&  Reserves   & 50    \\
\end{tabular}
\end{center}
%
\begin{center}
\begin{tabular}{lrclr}
\multicolumn{5}{c}{Commercial Banks' Balance Sheet}             \\
                                                        &       \\
                    Assets      &&&          Liabilities        \\
%                                                       &       \\
                    Reserves    &  50    &&  Deposits   &200    \\
                    Loans       & 150    &&             &       \\
\end{tabular}
\end{center}
%
You'll note that net worth is zero for the Fed (it's ``owned" by the Treasury) and Commercial
Banks (they're owned by shareholders).

     A useful example of a monetary aggregate in this economy is $M = \CU$
(Currency) $+ D $(Bank Deposits).  [This is simpler than we saw in the real world, since we only
have one type of deposit.  With more than one type of deposit we have more than one type of money
and a more complicated theoretical setup.]  The Fed, on the other hand, controls the amount of
currency held by the private sector (as cash) and banks (as reserves).  We call this quantity the
monetary base, $\MB = \CU + \RE $ (Reserves).

     The question is how an open market operation that changes the monetary
base MB influences the monetary aggregate $M$---whether, that is, we can talk about the Fed
influencing a monetary aggregate, when policy involves the narrower monetary base.  We can derive
the relation between the monetary base $MB$ and the monetary aggregate $M$ if we make some
assumptions about behavior. Let us say, first, that private agents like to hold cash and bank
deposits in some strict proportion:
$$
                                 \CU/D  =  \gamma,
$$
where $\gamma$ is some number that we might expect to be roughly constant.  The idea is that we
make some transactions with cash, others with checks, and the proportions of the two doesn't
change much.  Let us also assume that banks hold a constant fraction of their deposits as
reserves:
$$
                                 \RE/D  =  \rho .
$$
This latter assumption is pretty good, since the Fed requires them to hold reserves proportional
to their deposits (we'll see the details shortly). From a bank's point of view this acts as a tax
on their deposits, since reserves earn no interest.  Even if there were no minimum reserves, banks
might be expected to hold some fraction of deposits in cash as part of their day to day business.
From this, we can derive a relation between the monetary aggregate and the monetary base.  We
know:
$$
               \MB  =  \RE + \CU    \eqno{\mbox{(equilibrium condition)}}
$$
$$
                   M  =  \CU + D    \eqno{\mbox{(definition of money)}}
$$
This leads (after some relatively simple algebra) to
$$
             M  = \left( \frac{1+\gamma}{\gamma+\rho} \right) \MB.
$$
The expression in brackets is referred to as the money multiplier, since we generally see that the
stock of money is a multiple of the monetary base.  In the US, for example, the multiple is about
3 for M1 and over 10 for M2 and M3.

     We now have an answer to our question:  if the ratios
$ \rho $ and $ \gamma $ are approximately constant, then by controlling the monetary base the Fed
exerts indirect control over the broader monetary aggregates.  In that sense, we can speak loosely
about the Fed ``controlling" M2 and other aggregates.

     But are the ratios constant?  We can get some idea by plotting the data.
In Figures 1 and 2 we see how monetary aggregates and related variables have behaved over time.
In Figure 1 I've graphed MB, M1, and M2 for the last thirty years (each is scaled to equal 0.0 in
the first quarter of 1959).  The trends are somewhat different, with M2 and M3 growing faster than
MB and M1. In Figure 2 we see the money multipliers for the three aggregates.  Again there has
been some variation over time (as there must be since the aggregates have grown at different
rates).  You can see the same thing in different form Figure 3, where the growth rates of MB and
M2 are drawn.  In short, the money multipliers are another case of a reasonable approximation, but
in the short run we see some variation which is reflected in different growth rates across
aggregates.  Thus the money multiplier theory is only a rough guide and in the short run, at
least, the Fed may have a difficult time affecting monetary aggregates.


\subsubsection*{Application:  Contrary Movements in Monetary Aggregates}

     Over the last few years we've seen, as we did in the Depression, a
divergence between the movements in the monetary base (MB) and monetary aggregates (like M2), with
the base growing more rapidly than the aggregates. See Figure 3.  Apparently the increase in the
monetary base has been offset by declines in the money multipliers.

     The story has some similarity to the Depression.  For whatever reasons,
there has been, since 1986, a sharp rise in the ratio of currency to deposits (this includes all
the deposits counted in M2); see Figure 4.  This implies, as we've seen, a fall in the M2
multiplier.  Thus M2 over this period has grown less rapidly than the base.  But why?  Three
possibilities cross my mind, maybe you can think of others:  (i) Lack of confidence in the banking
system led people to put less of their wealth in banks.  Given deposit insurance this is probably
a misplaced concern, but maybe it affected peoples' behavior.  (ii) Growth in the underground
economy (drugs?) led people to use more cash than before.  (iii) Banks made less effort than
before to attract deposits, since they had no desire to to make additional loans, when past loans
were turning out so badly.  Or a minor variation: alternatives to banks (mutual funds, brokers and
dealers, etc.) attracted some of the funds that were previously invested in commercial banks, and
thus led to a decline in the D part of the currency-deposit ratio.  In other words, the decline in
commercial banking's market share shows up here as a rise in the currency-deposit ratio.

     Whatever the reason, it gives you some idea of the difficulties of
``controlling" monetary aggregates.  Some critics have argued that Greenspan has starved the
banking system of funds; he replied, in essence, that the funds were there (base growth was
reasonable) but that the banking system wasn't attracting deposits and (the other thing banks do)
loaning them out. As the saying goes:  ``You can bring a horse to water, but you can't make him
drink."  Given the enormous changes we've seen in the financial system in the last fifteen years,
it may simply be that broad aggregates like M2, which emphasize bank liabilities, are no longer
good indicators of how well the financial system is meeting the needs of the economy.

     We could tell a similar story about Japan:  monetary aggregates have
been growing more slowly than the monetary base, as people take money out of banks and invest it
elsewhere, including the government's postal saving system.  This shows up as a drop in the money
multiplier.  It's an open question, given the conflicting evidence, whether we view monetary
policy in Japan as loose, tight, or in between.


\section{The Federal Reserve and US Monetary Policy}

     The Federal Reserve System is a curious mixture of public and private,
federal and regional.  The Board of Governors, in Washington DC, consists of 7 individuals
appointed by the President and approved by the Senate.  Each serves a fourteen year term (unless a
better offer comes along).  One of its members is appointed chair for a four-year term by the same
procedure. Greenspan, for example, was appointed in 1988 and reappointed in 1992.  The length of
term and the timing of chair appointments (before presidential elections) is intended to provide
the Fed some short-run independence from the political process.  Although the institutions vary,
this independence is a feature of central banks in many developed countries.

     The system also includes 12 regional Federal Reserve Banks, whose
location tells you something about the US and its politics in 1913, when the Federal Reserve Act
was passed.  These regional banks do a lot of the bank supervision, check-clearing, and other
day-to-day operations of the system.

     With regard to macroeconomic policy, the Fed has three distinct
instruments.  By far the most important is open market operations, but they also control reserve
requirements on deposits and the discount rate on borrowing from the Fed.  Each of these policies
is determined by a somewhat different combination of players.  Open market operations are
determined by the Federal Open Market Committee (FOMC), which meets eight times a year and can
change policy between meetings in conference calls.  The FOMC consists of the seven members of the
Board of Governors and five presidents of regional banks.  Of the five, the president of the NY
Fed is always a member, and the other 11 members rotate (Chicago gets the odd share).  Thus the
president of the Minneapolis Fed is a voting member every third year.  In off years, regional
presidents are ex officio members.

     Day-to-day open market operations are carried out with repurchase
agreements, or repos, on US government securities with sanctioned US security dealers (which
includes many of the big financial institutions of various types, including commercial banks).  In
a typical ``system RP" (system distinguishing this from private sector repos) the Fed purchases
government securities and sells them back again a few days later.  The terminology means that the
customer has ``repurchased" the securities it temporarily sold to the Fed.  The difference between
the sale and repurchase prices indicates the interest rate on the transaction.  The reverse
transaction is called a reverse repo by the market, and a matched sale purchase (or MSP) by the
Fed. Both of these instruments allow the Fed to affect the quantity of reserves and the monetary
base.

     The most popular indicator of the Fed's open market operations is the
federal funds rate.  In the course of satisfying their reserve requirements, banks often borrow
and lend reserves.  A bank with more reserves than it needs will loan them to a bank that needs
more, and charge interest on the loan.  This market for reserves is referred to as the federal
funds market, and the rate the federal funds rate.  With reserve requirements imposed weekly, the
loans are generally of very short duration, often only a day.

     The other two policy instruments are changed less frequently.  As I
mentioned, one of the tasks of the Fed is to provide short-term loans to banks.  These loans are
secured with government securities.  In the old days this took the form of banks selling
securities to the Fed at a discount, with the result that this arm of the Fed is known as the
discount window.  The interest rate on such borrowing is called the discount rate.  The discount
rates are suggested by the regional banks subject to the Board of Governors and in recent times
have been below market rates.  The loans are made at the discretion of the Fed, and banks that are
seen as abusing their borrowing privileges may lose them.  In practice, the discount rate is not
viewed as an important aspect of monetary policy, but it is often used to signal the Fed's
intentions.  A rate cut, for example, may indicate that the Fed foresees lower rates, and possibly
looser monetary policy.

     The final policy instrument of policy is reserve requirements.  As we've
seen, banks (and other depository institutions) must hold reserves against deposits in the form of
cash or deposits at federal reserve banks.  These reserves do not pay interest.  The Board sets
these requirements within limits set by the Monetary Control Act of 1980.  These requirement are
changed much less often than the discount rate.

     I should add one final policy instrument, foreign exchange intervention,
which in the US is the joint responsibility of the Fed and the Treasury.

\section{Application:  Recent Changes in Reserve Requirements}

     There has been concern recently that the dire straits of the banking
system has led to excessively tight loan requirements which in turn might prolong the recession:
what might be termed a credit crunch.  There's some question what this term means, but most have
in mind tighter credit than would be indicated by the amount of money available to be loaned out.

     The Fed has responded to these concerns by changing some of the subtler
aspects of monetary control in ways that should make funds more readily available to borrowers.
One policy was to eliminate reserve requirements on some types of deposits.  As of early December
of 1990, required reserve ratios were
%
\begin{center}
\begin{tabular}{lrl}
          Transaction accounts   &  12\%  & (3 on first 41 million)  \\
          Time deposits            \\
  \hspace{0.25in} Personal       &   0                                \\
  \hspace{0.25in} Nonpersonal    &   3\%   & (if shorter than 18 months) \\
          Eurodeposits           &   3\%   \\
\end{tabular}
\end{center}
%
Other deposits were free of reserve requirements.  As of December 27, 1990 the last two categories
were reduced to 0.  And on April 2, 1992 the rate on transaction accounts was reduced to 10
percent.  The effect is to provide banks with more money to loan out and thus enable them to earn
more (they avoid, in effect, the ``reserve tax" on these deposits).  The idea was to make banks
healthier and get loans into the system (or, in our model, to increase the money multiplier).
There has also been talk of easing the use of the discount window for bank borrowing from the Fed
and of relaxing accounting rules on bad loans.  All of these suggest that Fed policy consists of
more than open market operations.

     On the whole, the US has been moving closer to the British system, where
reserve requirements are minuscule, thereby reducing the competitive disadvantage of noninterest
bearing reserves for banks.  Germany has been moving in a similar direction in an effort to put
their banks on a more equal footing with those of the Benelux countries.  A good guess is that
reserve requirements will be much less important worldwide ten years from now than they were ten
years ago.


\section{Monetary Policy in Other Countries}

     Although our theoretical discussion of monetary policy has been generic,
the institutional details differ widely across countries.  If you're in this business, there's no
substitute for learning the rules of the country you're in.  An executive summary for a few major
countries follows.  Most of this comes from the JP Morgan publication cited at the end.

     {\em Germany.}
The formal structure of the Bundesbank mirrors the US Federal Reserve System, although in practice
the Bundesbank is widely thought to have greater autonomy.  The Directorate consists of
presidential appointments and state (Lander) nominees, who serve for eight years each.  Price
stability is its legally mandated primary objective.  Open market operations generally work
through repurchase agreements.  The repo rate is generally bracketed by two rates on bank
borrowing from the Bundesbank.  A limited quota of credit is available at the discount rate, which
puts a floor on the repo rate.  The lombard rate on overnight loans to banks acts as a ceiling.
Both are used, as is the discount rate in the US, to signal changes in interest rates. Reserve
requirements have been reduced over the last few years.  Foreign exchange intervention is used to
fulfill obligations of the Exchange Rate Mechanism of the European Monetary System.

     {\em Japan.}
The policy making bodies of the Bank of Japan (BoJ) consist of both political appointees with
five-year terms and career civil servants, and operate under the guidance of the Ministry of
Finance (MOF).  For this reason, most observers regard the BoJ as having little independence, but
since their debacle with high inflation following the 1973-75 OPEC price increase, they have been
very successful in maintaining price stability. Open market operations are executed with purchases
of government securities in the long-term, repurchase agreements on a variety of securities in the
short term; the Gensaki rate on bond repos is the most widely cited.  Reserve requirements are
changed infrequently.

     {\em United Kingdom.}
Authority over the Bank of England rests with the Chancellor of the Exchequer (the finance
minister of the government), which gives the Bank more limited control over monetary policy than
either the Fed or the Bundesbank.  Policy operates through open market operations in Treasury
securities (``gilts").  A ``minimum lending rate" for Bank loans to discount houses (security
dealers) is sometimes used to signal policy-induced changes in interest rates.  Reserve
requirements are minimal.  Policy is circumscribed by its (wavering) commitment to the Exchange
Rate Mechanism of the European Monetary System.


\section*{Summary}

\begin{enumerate}
\item We've seen various concepts of money:  the monetary base and three monetary aggregates, M1,
M2, and M3.

\item Open market operations have a direct influence on the monetary base. Monetary aggregates are
influenced by private sector decisions as well as monetary policy.

\item The monetary base and monetary aggregates are related through ``multipliers," which in
practice are only roughly constant.  In this sense, the Fed cannot claim to have close control
over the broader aggregates (``the loose steering wheel"), although its actions do influence them.

\item The Federal Reserve System in general, and the Federal Open Market Committee in particular,
are responsible for monetary policy in the US.  Its tools are open market operations, reserve
requirements, and the discount rate.

\item The mechanics of monetary policy differ across countries.
\end{enumerate}





\subsubsection*{Executive summary}

Business cycles around the world share these properties:
\begin{enumerate}
\item Economies do not growth smoothly, they exhibit a lot of
short-run volatility.

\end{enumerate}


\subsubsection*{Review questions}

You might think about each of the following issues, which we'll
address at greater length shortly.
\begin{enumerate}

\item Why do you think employment lags the economy?

\end{enumerate}


\subsubsection*{Further reading}

For more information ...
%
\begin{itemize}
\item Fed Governor Ben Bernanke,
``\href{http://www.federalreserve.gov/BoardDocs/Speeches/2005/20050330/default.htm}{Implementing monetary policy},''
speech made MArch 30, 2005, outlining the mechanics of US monetary policy.  Terrific summary.

\end{itemize}


\vfill \centerline{\it \copyright \ \number\year \ NYU Stern
School of Business}

\end{document}
