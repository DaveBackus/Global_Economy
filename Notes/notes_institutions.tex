\documentclass[letterpaper,12pt]{article}

\usepackage{ge05}
\usepackage{comment}
\usepackage{booktabs}
\usepackage[dvipdfm]{hyperref}
\urlstyle{rm}   % change fonts for url's (from Chad Jones)
\hypersetup{
    colorlinks=true,        % kills boxes
    allcolors=blue,
    pdfsubject={NYU Stern course GB 2303, Global Economy},
    pdfauthor={Dave Backus @ NYU},
    pdfstartview={FitH},
    pdfpagemode={UseNone},
%    pdfnewwindow=true,      % links in new window
%    linkcolor=blue,         % color of internal links
%    citecolor=blue,         % color of links to bibliography
%    filecolor=blue,         % color of file links
%    urlcolor=blue           % color of external links
% see:  http://www.tug.org/applications/hyperref/manual.html
}

\newcommand{\GDP}{\mbox{\em GDP\/}}
\newcommand{\NDP}{\mbox{\em NDP\/}}
\newcommand{\GNP}{\mbox{\em GNP\/}}
\newcommand{\NX}{\mbox{\em NX\/}}
\newcommand{\NY}{\mbox{\em NY\/}}
\newcommand{\CA}{\mbox{\em CA\/}}
\newcommand{\NFA}{\mbox{\em NFA\/}}
\newcommand{\Def}{\mbox{\em Def\/}}
\newcommand{\POP}{\mbox{\em POP\/}}

\def\ClassName{The Global Economy}
\def\Category{Class Notes}
\def\HeadName{Institutions}

\begin{document}
\thispagestyle{empty}%
\Head

\centerline{\large \bf Institutions \& Policies}%
\centerline{Revised:  \today}

\bigskip
The enormous international differences in GDP per person
reflect, in large part, enormous differences in
productivity.
But where do these differences in productivity come from?
It's tempting to attribute them to the ability and dedication
of the people who live there,
but (on second thought) there are smart dedicated people everywhere.
We now believe that productivity reflects
the quality of local institutions and policies.
Stated more concretely:
it's not Steve Jobs that makes an economy productive,
it's the institutions and policies that allow and encourage someone like Jobs to
operate effectively.
Some countries have environments that encourage productive
activity, and others do not.
What's striking is not that this could be true,
but how big a difference it seems to make.


\subsubsection*{Good institutions}

So what do we mean by good institutions?
The world's a complicated place, and it doesn't come
with any simple recipes.
But countries with good economic performance
 share some features.
We would say good institutions are social mechanisms
that facilitate good economic performance.
What might those be?
Here's a short list.

{\it Good governance.\/}
It's essential that the government be strong enough
to guarantee the security and safety of the country
and people,
but not so strong that those in power abuse others for their own benefit.
It's a delicate balance,
but most productive economies have both strong governments
and clear limits to the government's power.
It's also helpful for there to be consistency in policy over time,
so that the business environment doesn't change every time
there's a change of government.
This reduction in uncertainty is itself an important
contributor to economic activity.

{\it Rule of law.\/}
It's also important that the legal system enforce the law:
that the police and judiciary are honest and
enforce the laws of the land.

{\it Property rights.\/}
We sometimes take this for granted,
but the laws should be clear about who owns what.
Without that, effective economic activity is impossible.
How can you sell something you don't own?
Imaginative people may be able to do just that,
but it's not the basis for a productive economy.
How can you get a mortgage if you can't establish
that you own real estate?
Why would anyone lend on those terms?

{\it Open and competitive markets.\/}
You often hear about ``free markets,''
but what seems to work best are honest, open, flexible, competitive markets
for products as well as capital and labor.
That's different from what you might term business-friendly government:
markets that protect sellers from competition or fraud.
The idea is not to protect producers,
but to allow them to compete honestly.

We'll give examples of each in class, but you might try to think of your own.


\subsubsection*{Institutions or policies?}

Institutions bring to mind the difference between North and South Korea.
The two countries have the same culture ---
and the same history until 1950.
At that time, living standards were similar,
probably a little higher in the North.
Today, best estimates indicate that GDP per capita in the South
is more than 10 times that of the North.
The huge difference in performance surely reflects the huge difference
in institutions between the countries:
the form of government and the nature of economic activity.


In other cases, policies may play a more important role.
We think of policies as less fundamental aspects of
the economic environment than institutions.
An honest judicial system is an institution,
but tax rates and government spending are policies.
There's a fuzzy line between the two,
but the idea is that policies are more easily
changed than institutions.

Peter Henry and Conrad Miller (link below) illustrate
the role of policies in a comparison
of Barbados and Jamaica.
We'll draw liberally from their paper.
They note that the two countries have similar backgrounds and institutions:
\begin{quote}
Both [are] former British colonies,
small island economies,
and predominantly inhabited by the descendants of Africans. ...
As former British colonies, Barbados and
Jamaica inherited almost identical political,
economic, and legal institutions: Westminster
Parliamentary democracy, constitutional protection of property rights,
and legal systems rooted in English common law.
\end{quote}
Nevertheless,
Barbados grew by 1.3\% a year faster between 1960 and 2002,
giving it a substantially higher standard of living.
(This is a bigger difference than it looks --- the power
of compound interest and all that.)


One clear difference between the two countries was their
macroeconomic policies.
In the 1970s, Jamaica increased government spending
on job creation programs, housing, food subsidies, and many other things.
When tax revenue failed to keep up, the government found itself
with large persistent budget deficits, which they financed by
borrowing from the central bank.
This, in turn, led to inflation rates of 20\% and higher.
A fixed exchange rate raised the price of Jamaican goods relative to imports,
which led to restrictions on imports and wage and price controls.

Barbados also had a fixed exchange rate,
but combined it with fiscal discipline, monetary restraint,
and openness to trade.
The result was much different macroeconomic outcomes.
It's possible other factors played a role, too,
but this seems to be a case where policies were more
important than institutions.


\subsubsection*{Other factors}

[??]

Democracy, resources, education,....

Summarize evidence.  Mention Barro, Easterly...

Cause or effect?  

\subsubsection*{Executive summary}

\begin{enumerate}
\item Good institutions are the primary source of good economic performance.
\item A short list would include:  governance, rule of law,
property rights, and open competitive markets.
\item Stable macroeconomic policies matter, too.
\end{enumerate}


\begin{comment}
\subsubsection*{Review questions}

\begin{enumerate}
\item ...
\end{enumerate}
\end{comment}

\subsubsection*{If you're looking for more}

The comparison of Barbados and Jamaica comes from Peter Henry and Conrad Miller's
``\href{http://web-docs.stern.nyu.edu/old_web/aboutstern/publications/Tale_of_Two_Islands-AER.pdf}
{A tale of two islands}.''

Here are some other good reads, in order of increasing length:
\begin{itemize}
\item Ben Bernanke,
``\href{http://www.federalreserve.gov/newsevents/speech/bernanke20110928a.htm}
{Lessons from emerging markets}.''
Nice short summary of what good institutions and policies look like.

\item Bill Easterly,
\href{http://www.amazon.com/Elusive-Quest-Growth-Economists-Misadventures/dp/0262550423}
{\it The Elusive Quest for Growth}.
Essentially a collection of essays on topics related to helping poor countries,
unusually witty for an economist.

\item David Landes,
\href{http://www.amazon.com/Wealth-Poverty-Nations-Some-Rich/dp/0393318885}
{\it The Wealth and Poverty of Nations}.
Less witty than Easterly, but he gives us an interesting historical
perspective on the major countries of the world:  Europe, India, China, etc.
\end{itemize}


\vfill \centerline{\it \copyright \  \number\year \  NYU Stern
School of Business}


\end{document}
