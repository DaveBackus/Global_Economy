\documentclass[letterpaper,12pt]{article}

\RequirePackage{comment}
\RequirePackage[hypertex]{hyperref}
    \hypersetup{colorlinks=true,urlcolor=blue,linkcolor=red}
\RequirePackage{GE05}

\newcommand{\GDP}{\mbox{\em GDP\/}}
\newcommand{\NDP}{\mbox{\em NDP\/}}
\newcommand{\GNP}{\mbox{\em GNP\/}}
\newcommand{\NX}{\mbox{\em NX\/}}
\newcommand{\NY}{\mbox{\em NY\/}}
\newcommand{\CA}{\mbox{\em CA\/}}
\newcommand{\NFA}{\mbox{\em NFA\/}}
\newcommand{\Def}{\mbox{\em Def\/}}
\newcommand{\CPI}{\mbox{\em CPI\/}}
\newcommand{\IP}{\mbox{\em IP\/}}
\newcommand{\PD}{\mbox{\em PD\/}}
\newcommand{\hsp}{\hspace{0.25in}}

\def\ClassName{The Global Economy}
\def\Category{Class Notes}
\def\HeadName{Government Debt and Deficits}

\begin{document}
\thispagestyle{empty}%
\Head

\centerline{\large \bf \HeadName}%
\centerline{Revised: \today}

\bigskip
Fiscal policy refers to government decisions to spend, tax, and issue debt.  Summary measures of fiscal policy, such as the government deficit or the government debt, are often used to assess the
likely performance of the economy, including  
the likelihood that the government will honor its debt.  
We examine each of these aspects of fiscal policy, starting with
a quick overview of how governments report revenues and expenses.  


\subsubsection*{Government revenues, expenses, and debt}

The starting point of any analysis of fiscal policy is a summary 
of revenues and expenses.  
Countries differ in the size of government relative to the economy, 
in the sources of tax revenues, 
and in their expenditures.  
Governments everywhere collect revenue (largely through taxes), 
purchase goods and services (schools, police, courts, military, roads), 
and transfer money to people and corporations through a variety of
programs (social insurance, health care).  
The distinction between purchases and transfers is important:  
only purchases show up in the expenditure identity, 
but transfers are a large part of total expenses in most economies.  
Governments also pay interest on outstanding government debt, 
an expense worth tracking separately.  


In Table~\ref{tab:usdef}
we report some numbers for the US in 2007 (the most recent available), 
For comparison, US GDP was \$14.3 trillion.
We can compute various summary measures from the entries in the
table.  {\it Government purchases\/} is the sum of (government)
consumption and investment expenditures (2,212.0 + 462.8). 
This is $G$ in the expenditure identity. 
{\it Transfers\/} are another category of expenditures (1,332.9); 
we treat them separately because
they consist of transfers of purchasing power, not direct purchases
of goods or services. They include Social Security, medicare and medicaid, 
and so on. 
The {\it government deficit\/} (+399.4!) is the difference between total
total expenditures and total receipts. It's a measure of the borrowing requirement of the government.  
The negative of the government deficit might be called 
{\it government saving\/}.  
The {\it primary deficit\/} equals
government deficit minus interest payments.  


\begin{table}[h]
\centering \tabcolsep=0.2in
\begin{tabular}{lr}
\hline\hline Category  &  Amount \\
\hline\hline

{Total receipts}        &      4,209.3 \\
\hsp Tax receipts              &      2,948.5     \\
\hsp Contributions for social insurance        &   996.1     \\
\hsp Other              &          264.7      \\

\hline

{Total expenditures}    &    4,608.7    \\
\hsp Government consumption expenditures    &   2,212.0          \\
\hsp Government investment expenditures (gross)   &    462.8     \\
\hsp Transfers            &       1,721.3        \\
\hsp Interest payments             &    411.1            \\
\hsp Balancing item              &       --198.5     \\

\hline

Net lending or borrowing   &     --399.4  \\

\hline\hline
\end{tabular}
\caption{US government budget (federal, state, and local), 
billions of US dollars.  Source:  BEA website.}
\label{tab:usdef}
\end{table}

We can put these elements together in a relation we'll call 
the {\it government budget constraint\/}.
On the expense side, governments purchase goods and services 
(label this $G$), make transfer payments ($V$) to (primarily) households, 
and pay interest on accumulated debt.
If the debt is $B$ and the interest rate is $i$, then this payment 
is $iB$.
On the revenue side, they collect taxes ($T$).
Note that $T$ is tax revenue, not the tax rate.
By convention, all of these things are nominal:
they're measured in current prices.  
The government budget constraint is then 
\begin{equation}
    G_t + V_t + i_t B_{t-1} - T_t  \;=\; (B_{t} - B_{t-1} ) .
    \label{eq:gbc}
\end{equation}
Here $B_{t-1}$ is the amount of debt outstanding at the end of
period $t-1$ and the start of period $t$.  
Equation (\ref{eq:gbc}) says, in essence, that any surplus or deficit must be matched
by a change in the quantity of debt.
The left-hand side of (\ref{eq:gbc}) 
is the government deficit, 
the right the change in the quantity of debt.


The elements of equation (\ref{eq:gbc}) 
are often used to generate summary measures 
of fiscal policy. 
The most common are ratios  of 
the government deficit and government debt to GDP.  
We'll look at both, as well as the connection between them.  


\subsubsection*{Pay me now or pay me later}  


Governments need to finance their spending with taxes.
It's not quite true --- governments have other sources of revenue ---
but it's close enough to be worth remembering.  
Issuing debt allows a government to postpone taxes, 
just as a credit card allows an individual to postpone paying 
for purchases, 
but does not eliminate the obligation.  
Delay, in fact, comes with a cost:  we need to pay the original 
obligation --- plus interest.  
In the rest of this section we make the same point more formally.  

We're going to take our budget constraint, equation (\ref{eq:gbc}), 
and use it to express debt as a present value of future surpluses.  
To make things simple, define the primary deficit $D$ as the deficit 
net of interest payments (sort of an ``EBITDA'' number):
\[
    D_t \;=\;  G_t + V_t - T_t .
\]
The primary surplus is the same object with a minus sign. 
Then (\ref{eq:gbc}) can be expressed as 
\[
    B_{t}   \;=\; (1+i) B_{t-1} + D_t 
   \label{eq:gbc-bdynamics1}
\]
or 
\begin{equation}
    B_{t-1}   \;=\;  B_{t}/(1+i) - D_t/(1+i) .
    \label{eq:gbc-bdynamics2}
\end{equation}
All three versions contain the same information, but we'll 
focus on (\ref{eq:gbc-bdynamics2}).  
If we use it to substitute repeatedly, we find 
\begin{eqnarray*}
    B_{t-1}   &=&  B_{t+1}/(1+i)^2 - [D_t/(1+i)+ D_{t+1}/(1+i)^2] \\
            &=&  B_{t+2}/(1+i)^3 - 
                [D_t/(1+i)+ D_{t+1}/(1+i)^2 + D_{t+2}/(1+i)^3] \\
            &=&  B_{t+n-1}/(1+i)^n - 
                [D_t/(1+i)+ D_{t+1}/(1+i)^2 + \cdots + D_{t+n-1}/(1+i)^n] .
%    \label{eq:gbc-bdynamics-rec}
\end{eqnarray*}
If we assume that debt can't grow faster than the interest rate forever, 
then as we continue to substitute, the first term goes to zero.
[The technical condition is $B_{t+n}/(1+i)^n$ goes to zero as $n$ 
approaches infinity.
It amounts to not allowing the government top run a Ponzi scheme.]
The assumption seems reasonable, since our payments 
aren't growing fast enough in this case even to cover the interest.  
The relation then becomes  
\begin{eqnarray*}
    B_{t-1}   &=&  -[D_t/(1+i)+ D_{t+1}/(1+i)^2 + 
                   D_{t+2}/(1+i)^3] +   \cdots ]  \\
            &=& - \mbox{Present Discounted Value of Primary Deficits} \\
            &=& \mbox{Present Discounted Value of Primary Surpluses} .
\end{eqnarray*}
In words:   the current government debt must be matched 
by the present discounted value of future primary surpluses.  
As we said at the start, all spending must be financed by tax revenue.  
It's not enough to shrink the deficit:  eventually we have 
to run surpluses, measured net of interest payments.  
There's a limit, in other words, to deficit spending. 

%  ?? steady state D = - i B.  


Although (primary) government deficits must eventually be reversed, 
they can affect the economy while they last.  
One route is distribution.  
If taxes are delayed long enough, the tax burden 
will be shifted from current to future generations.
Thus the US contribution to World War II was financed largely with debt, 
shifting some of the burden from those alive at the time to those born later.


Is deficit financing a good or bad thing?  
One approach to this question is based on tax smoothing.
As we mentioned in an earlier class,
the economic disincentives built into taxes are minimized
by having relatively fixed tax rates even if tax revenues
vary over time.
On average this is likely to lead to governments running (modest) 
surpluses in booms and deficits in recessions, 
as tax revenues go up and down with the economy.  

Whether you think deficits are good or bad, 
they are a piece of data that investors watch closely.
Especially in emerging markets, current deficits
may be interpreted as a sign that the government 
may not honor its debts.
Even a hint of this changes the perspective of investors.  


\subsubsection*{Debt dynamics} 

One common measure of fiscal discipline (or lack thereof)
is the ratio of government debt to GDP.  
Since the debt accrues interest, it has a natural tendency to grow.  
If fiscal policy leads the ratio of debt to GDP to grow, 
then if nothing changes --- ever --- it will eventually become too large 
for the government to finance:  
interest payments alone will eat up the entire budget.  
We would say the situation is not sustainable.  
Something has to change to keep this from happening.  


How do debt and the debt-to-GDP ratio change through time?  
Recall that debt evolves according to  
\begin{eqnarray*}
    B_{t} &=& D_t + (1+i) B_{t-1}  ,
\end{eqnarray*}
where $D$ is (again) the primary deficit.  
Note that everything is nominal, including the interest rate $i$.  
If the growth rate of (nominal) GDP is $g$, then  
\[
    Y_{t} \;=\; (1+g) Y_{t-1} 
\]
and $B/Y$ follows 
\begin{equation}
    \frac{B_{t}}{Y_{t}} \;=\; 
                \left( \frac{1+i}{1+g} \right)  \frac{B_{t-1}}{Y_{t-1}} 
             +    \frac{D_{t}}{Y_{t}} .
    \label{eq:debtdynamics}             
\end{equation}
This equation tells us how the debt-to-GDP ratio changes 
from period to period.   


How does the debt-to-GDP ratio evolve?  
The first issue is whether $g$ is larger or smaller than $i$.  
If $g>i$, then a government can run deficits forever without 
the ratio exploding --- although it could nevertheless become large.  
More commonly, $ g<i$, as above, 
in which case the ratio 
will continue to grow, even if we have a zero primary deficit ($D = 0$).
In this more common case, any deficit is unsustainable, 
in the sense that if the deficit isn't eliminated, 
the ratio of debt to GDP will eventually grow without limit.  
Similarly, interest payments will grow without bound,
which doesn't sound good. 


If interest on the debt becomes large enough, something must change.
But what?
One candidate is the deficit itself:  the government 
does this analysis, realizes that interest will eat up 
all its revenue, and decides to reduce the deficit.  
This could come from a reduction in spending or an increase in taxes; 
both work the same way in this analysis, although  
in other respects the two may differ.  
Another candidate --- one we'd generally prefer not to consider --- 
is default.
Bond investors can do this analysis, too, 
and if they find that the debt is more than the government 
is likely to pay back, they will stop lending. 
This tends to reduce the deficit, too, 
by restricting the government's ability to finance it.  


This is a mechanical analysis, but a useful one. 
By looking at the components of equation (\ref{eq:debtdynamics}) 
we can get a sense of how fast the debt ratio is changing, 
and therefore whether or not we're headed for trouble in the near future.
How much debt is too much?  
It depends on the situation.
(We know this is a copout, but it's true.)
If investors get the idea that they may not be paid back, 
then it's too high.  
When that happens depends on the country and circumstances. 


Here's an example.  Peru in 2002 had debt equal to 47\% of GDP.  
In 2007, the debt to GDP ratio had fallen to 27\% through a combination
of fiscal discipline and rapid growth.  
The 2008 numbers were forecast to be:  primary deficit --1.4\%, 
nominal interest rate 7.5\%, 
nominal GDP growth 8.0\%.  
Given these numbers, how much would you expect the debt to GDP 
ratio to fall over the next year?  

This is a direct application of (\ref{eq:debtdynamics}).
The numbers imply that $B/Y$ falls from 27\% to 25.5\%.  
For extra credit:  how does your answer change if growth 
falls to 5\%?  if the interest rate rises to 10\%?  


\subsubsection*{What else?}

The idea behind our analysis of debt dynamics 
is that current fiscal policy (the primary deficit) 
determines how the debt evolves.  
In fact, current decisions often involve commitments 
for future expenditures that don't show up in the 
current government budget.  


Here's a list of things that deserve a closer look:  
%
\begin{itemize}
\item Social security and pensions.  
Many countries have implicit commitments to pay 
money to retired people in the future, 
either through social security systems or pensions.
In developed countries, aging populations have made these looming payments
a serious concern.  
Health-care payments are similar.  

\item Financial bailouts.  
We tend to treat these as one-offs, but in fact they happen all the time.
And they're invariably expensive.  
Roubini, Sargent, and others have argued that the cost of financial 
bailouts may have triggered both the Asian crisis 
and the Israeli hyperinflation of the 1980s.  

\item Maturity of debt.  
Even if the debt is stable, countries (or companies!) 
may find themselves in difficulty if they have to refinance
a large fraction of their debt over a short period of time.  
In late 1994, for example, Mexico had most of its debt in short-term 
securities.
When the failed to refinance, it triggered a financial collapse.  


\item Denomination of debt.  
Many developing countries issue debt in hard currencies (dollars, say, or euros).
If their currency falls, then their debt in local terms rises in value. 

\end{itemize}


\subsubsection*{Executive summary}

\begin{enumerate}

\item Countries differ in the magnitude and composition 
of government spending, taxes, and debt.  

\item Debt dynamics imply that a government deficit must eventually 
be reversed. 

\item The debt-to-GDP ratio falls if (a)~the primary deficit is negative 
(a surplus), (b)~the country grows rapidly, and/or (c)~the interest rate 
on debt is low.  

\end{enumerate}


%\begin{comment}
\subsubsection*{Review questions}

\begin{enumerate}

\item ?? add some problems 

\end{enumerate}
%\end{comment}


\subsubsection*{Further information}

It's too technical for this course, 
but the material on debt dynamics was adapted from 
Craig Burnside, ed., {\it Fiscal Sustainability in Theory and Practice\/}, 
World Bank, 2005.  


\vfill \centerline{\it \copyright \ \number\year \  
NYU Stern School of Business}

\end{document}


%***************************************************************************************
US data:  http://www.bea.gov/national/nipaweb/TableView.asp?SelectedTable=84&ViewSeries=NO&Java=no&Request3Place=N&3Place=N&FromView=YES&Freq=Year&FirstYear=2006&LastYear=2008&3Place=N&Update=Update&JavaBox=no#Mid 