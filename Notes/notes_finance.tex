\documentclass[letterpaper,12pt]{article}

\usepackage{ge05}
\usepackage{comment}
\usepackage{booktabs}
\usepackage{amsmath}
\usepackage[dvipdfm]{hyperref}
\urlstyle{rm}   % change fonts for url's (from Chad Jones)
\hypersetup{
    colorlinks=true,        % kills boxes
    allcolors=blue,
    pdfsubject={NYU Stern course GB 2303, Global Economy},
    pdfauthor={Dave Backus @ NYU},
    pdfstartview={FitH},
    pdfpagemode={UseNone},
%    pdfnewwindow=true,      % links in new window
%    linkcolor=blue,         % color of internal links
%    citecolor=blue,         % color of links to bibliography
%    filecolor=blue,         % color of file links
%    urlcolor=blue           % color of external links
% see:  http://www.tug.org/applications/hyperref/manual.html
}


\def\ClassName{The Global Economy}
\def\Category{Class Notes}
\def\HeadName{Financial Markets}

\begin{document}
\thispagestyle{empty}%
\Head

\centerline{\large \bf \HeadName}%
\centerline{Revised: \today}

\bigskip
Some of the most important markets for aggregate economic performance
are those for labor and (financial) capital,
which affect every industry and product.
Countries differ markedly in their treatment of both markets,
with (evidently) different outcomes as a result.

Our focus here is on financial markets,
which are perhaps the most difficult markets to manage effectively.


\subsubsection*{Financial markets}

Financial markets are central to economic performance,
because they facilitate (if they work well)
the allocation of resources to the most productive firms.
In the US today,
some firms borrow from banks,
others issue bonds and equity in capital markets,
and still others raise money through venture capitalists.
Countries differ widely in how they do this,
but they all have ways of channeling funds from
households (savers) to firms (borrowers).


The primary issue with financial markets is information,
and we know that markets sometimes handle information poorly.  
Here investors need to understand the risks faced by borrowers, 
but borrowers typically know more about themselves than others do.  
A bank, for example, needs to know enough about its borrowers to 
assess the risk of default, 
and its depositors need to know enough about the bank's ability to 
do this well to assess the risk to their deposits.  
A bank (or other financial institution) is therefore in the information business:
its goal is to process information efficiently so that it 
can assess and manage risk.  

None of this is easy to do.  
All of these financial arrangements 
require institutional support.
An effective financial system requires some version of
%
\begin{itemize}
\item Creditor protection.
If A loans money to B, it's essential that A's claim be honored.
That requires a legal system that makes
the creditor's rights clear and enforces them if necessary.
[You might be thinking about ``property rights'' and the ``rule of law''
about now.]
Without this, people will simply not make loans.
Or they make loans only to friends and relatives.
Weak ineffective financial systems follow naturally
when creditors are not protected.

\item Governance.
The laws of most countries give creditors some
say over the management of firms.
Equity investors, for example,
are represented (in principle) by boards of directors.
There's endless debate about how best to do this,
but there's no question doing it well is important.

\item Disclosure.
When people invest in securities, they need to understand what
they're buying.
In most countries with active securities markets,
the law dictates disclosure of relevant financial information.
Again, some countries do this better than others.

\item Central banks.  
Most countries have central banks.  
If run well, they play an important role in the economy, 
particularly as lenders of last resort during financial crises.  
\end{itemize}
%
Measures of these things are available from a number
of sources,
including the World Bank's 
\href{http://www.doingbusiness.org/}{Doing Business}
website.  


\subsubsection*{Financial regulation and crises}

From the perspective of a country, one of 
the challenge of managing financial markets is that they 
can cause enormous collateral damage.  
If a farmer goes bankrupt, you buy milk from someone else.
But if a large financial institution goes under, 
it can slow down the whole economy.  
The question is how to manage financial markets to minimize the damage.  

No one has come up with a perfect answer.
An unregulated financial system may work well most of the time, 
but will experience occasional crises.  
A more tightly regulated system may (it's not a sure thing)
have lower crisis risk, but the regulation may distort
the allocation of capital.  
Most approaches to financial regulation face a tradeoff of this sort.  

Consider deposit insurance.  
In the US, bank panics were a common occurrence
up through the 1930s.
During the Depression,
thousands of banks went under.
Some of them were insolvent.
Others closed because depositors demanded their money back
for fear that the bank would go under:
what we call bank runs.  
It's a consequence, in part, of people having imperfect 
information about the bank's soundness.  

The solution --- or rather, one solution --- was to provide deposit insurance.
Milton Friedman and Anna Schwartz called federal
deposit insurance ``the most important structural change''
made in the 1930s to deal with bank runs.
And it worked:  bank runs pretty much ended.

But like many solutions, it raised new problems.
The problem with deposit insurance is what economists call ``moral hazard''
and others might call the ``other people's money'' problem.
Since depositors don't face the risk of losing their money,
banks don't face the risk of withdrawal, 
and they have less reason to control the risk of their investments.
Or to put it differently:  their borrowing costs don't reflect the risk
of their loan portfolios.  
So they take excessive risk, which is hardly what we're looking for.  
We therefore add to deposit insurance some regulatory oversight intended
to limit their ability to take risks.  
We've seen that it's hard to get this right, 
and we're still trying.  


\subsubsection*{Executive summary}

\begin{enumerate}
\item Financial markets work well only when based on strong institutions.
\item But it's hard to get this exactly right.  
\end{enumerate}

\begin{comment}
\subsubsection*{Review questions}

\begin{enumerate}

\item ...
\end{enumerate}
\end{comment}

\subsubsection*{If you're looking for more}

The logic and operation of financial institutions is a huge subject
in its own right.  
Among the courses we have on the topic is Professor Schoenholtz's
``Money and Banking,'' course ECON-GB.2333.  
Or see his book:  
Stephen Cecchetti and Kermit Schoenholtz, 
\href{http://www.amazon.com/Banking-Financial-Markets-Stephen-Cecchetti/dp/007337590X/}
{Money, Banking and Financial Markets}.


\vfill \centerline{\it \copyright \ \number\year \ NYU Stern
School of Business}


\end{document}

