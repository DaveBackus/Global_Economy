\documentclass[letterpaper,12pt]{article}

\usepackage{ge05}
\usepackage{comment}
\usepackage{booktabs}
\usepackage[dvipdfm]{hyperref}
\urlstyle{rm}   % change fonts for url's (from Chad Jones)
\hypersetup{
    colorlinks=true,        % kills boxes
    allcolors=blue,
    pdfsubject={NYU Stern course GB 2303, Global Economy},
    pdfauthor={Dave Backus @ NYU},
    pdfstartview={FitH},
    pdfpagemode={UseNone},
%    pdfnewwindow=true,      % links in new window
%    linkcolor=blue,         % color of internal links
%    citecolor=blue,         % color of links to bibliography
%    filecolor=blue,         % color of file links
%    urlcolor=blue           % color of external links
% see:  http://www.tug.org/applications/hyperref/manual.html
}

\newcommand{\GDP}{\mbox{\em GDP\/}}
\newcommand{\NDP}{\mbox{\em NDP\/}}
\newcommand{\GNP}{\mbox{\em GNP\/}}
\newcommand{\NX}{\mbox{\em NX\/}}
\newcommand{\NY}{\mbox{\em NY\/}}
\newcommand{\CA}{\mbox{\em CA\/}}
\newcommand{\NFA}{\mbox{\em NFA\/}}
\newcommand{\Def}{\mbox{\em Def\/}}
\newcommand{\CPI}{\mbox{\em CPI\/}}

\def\ClassName{The Global Economy}
\def\Category{Class Notes}
\def\HeadName{Macroeconomic Crises}

\begin{document}
\thispagestyle{empty}%
\Head

\centerline{\large \bf \HeadName}%
\centerline{Revised: \today}

\bigskip
Economies periodically experience {\it crises\/}:
economic downturns that are not only larger than typical recessions
but qualitatively different.
The idea is now fresh in our minds,
but similar episodes have occurred throughout recorded history.
They're less common in modern developed countries,
but they can happen anywhere.
Like snowflakes and business cycles, no two are exactly the same,
but they share some common features.


\subsubsection*{Classic crisis triggers}

There are three classic triggers of macroeconomic crises:
sovereign debt, financial fragility, and a fixed exchange rate.

{\it Sovereign debt\/} problems date back as far as governments have borrowed.
If investors fear that a government may not repay its debt,
the market for debt collapses, often taking the economy with it.
In the old days, wars were the standard problem.
Wars are expensive, and if investors thought the expense was more
than the government was willing or able to bear,
they would stop buying the debt.
In modern times, governments spend money on many things,
but the possibility of default remains.
Argentina in 2002 and Greece today are recent examples.
This experience should remind us that sovereign debt
need not be riskfree.

One of the issues with government debts is sovereignty.
If a corporation defaults, the creditors take it to court
and claim the assets.
With governments there's no such mechanism and the process
is sloppier as a result.

{\it Financial fragility\/} is another classic trigger of crises.
We know from centuries of experience that when the
financial system freezes up,
economic activity slows down sharply.
We saw that in 2008,
but the same thing happened during the Bank of England Panic of 1825,
the Baring Crisis of 1890,
the US Panic of 1907,
Japan and Scandinavia in the 1990s,
and many other occasions.
It's a feature of even advanced financial systems that they
sometimes implode.

In most cases, these financial problems follow from
poor investments (real estate is a common example),
which put the solvency of financial institutions in question.
The problems tend to snowball:
worries about the viability of one firm may lead others to
reduce their lending, leading to a cycle of retrenchment
that puts even sound firms in trouble.
The word ``panic'' is apt here and stems
from the imperfect information investors have
when deciding where to put their money.


{\it Fixed exchange rates\/} are the third classic trigger.
For whatever reason, fixed exchange rates periodically
must be defended in ways that undermine the economy.
Recent examples include the UK in 1992,
Mexico in 1994,
Korea in 1997,
and Argentina in 2000.

Many crises combine several of these elements.
The Euro Zone today has all three.
In Ireland and Spain, bailing out their banks
landed the governments in financial peril.
In other countries, bank positions in government debt
put the banking system in peril.
A question for another time is what, if anything,
the single currency has to do with this.


\subsubsection*{Crisis indicators}

Crises are inherently hard to predict.
Why?
Think about cardiologists.
We understand that they can identify risk factors
(weight, high blood pressure) but cannot predict
the date of a heart attack with any precision.
Crises are worse.
Once people see a crisis on the way,
their actions tend to reinforce it,
either by avoiding government debt,
withdrawing their money from banks,
or investing their money in foreign currency.
But like the cardiologist, you can use what we learned earlier
to identify signs of trouble.


Analysts differ in the details, but most
would include the following in any assessment of crisis risk:
%
\begin{itemize}
\item {\it Government debt and deficits.\/}
Rules of thumb:  worry if government
deficit is more than 5\% of GDP or debt is more than 50\% of GDP.
Adjust upward for developed countries, downward for developing countries.
Watch out for hidden liabilities:  pensions, health care, bailouts, etc.
These are often much larger than official liabilities.

Fine points:  worry further if debt is short-term and/or denominated in foreign currency.
Short-term debt subjects the government to refinancing (``rollover'') risk:
 markets may demand better terms or refuse to refinance.  Foreign-denominated debt subjects government to risk
if currency falls in value, making the debt larger in local terms.

\item {\it Banking/financial system.\/}
This isn't something we've discussed,
but  analysts track leverage, duration mismatch, exposure concentration,
risk management processes,
and nonperforming loans.
The challenge is measuring them accurately from reported information.

\item {\it Exchange rate and reserves.\/}
Rule of thumb:  worry if the exchange rate is fixed, or close to it,
and the currency is significantly overvalued in PPP terms
(Big Macs cost 30\% more than in other currencies;
real exchange rate has risen more than 30\% in last 2-5 years).
Worry more if foreign exchange reserves
are low or have fallen significantly.
\end{itemize}

All of these things generate more concern
in countries with weak institutions.
It's not an accident that Greece is in worse trouble
than France or Germany.
Analysts often find that local politics are more important
to the outcome than the numbers.

\subsubsection*{Crisis responses}

What should a government do when faced with a crisis?
It depends on the trigger.
Standard advice includes:

{\it Sovereign debt crises.\/}
If the problem is that the government is borrowing too much,
the answer is to stop doing it:
run primary surpluses until the debt is manageable.
Default is also an option, and saves money in the short term,
but probably raises borrowing costs in the future.
And if you go through a default, it's helpful to resolve it as quickly as possible. 
IMF support is often used to cushion the blow:
contingent on progress with the deficit,
they loan the government money on more attractive terms
than the market would provide.
The ``conditionality,'' as it's called,
helps reduce moral hazard and can make it easier politically
for governments to undertake painful policy changes.
Such conditional lending can be critical in a crisis,
when high borrowing rates exacerbate the government's debt problems.

{\it Financial crises.\/}
If the financial system is fundamentally sound but illiquid,
the longstanding advice is for the central bank to lend
aggressively.
The standard quote comes from Walter Bagehot,
a 19th-century businessman and journalist:
``To avert panic, central banks should lend early and freely,
to solvent firms, against good collateral, and at high rates.''
Put simply,
the goals of crisis prevention and crisis management
are often at odds.


If the financial system is insolvent,
it's important to get it recapitalized and operating again.
This comes with more than a little irony,
but governments sometimes find themselves bailing out
precisely those banks that triggered the crisis.
The trick is to do it in ways that don't bankrupt the
government and inflict some pain on the bank's management and creditors
(incentives for the future).
These things happen fast, so it's hard to get everything right.

{\it Fixed exchange rates.\/}
Let them float.
More controversial:  impose capital controls to inhibit the
response of capital markets to a possible drop in the exchange rate.
(See:  trilemma.)
Capital controls are a dangerous tool,
because the fear of future capital controls
can generate a crisis on its own,
as investors rush to get their money out of the country.
Capital controls on inflows may, for that reason, be more attractive
than controls on outflows.


\subsubsection*{Executive summary}

\begin{enumerate}\itemsep=0.0in
\item The classic crisis triggers are
(i)~government debt and deficits,
(ii)~a fragile financial system,
and (iii)~a fixed exchange rate system.

\item Crises are hard to predict,
but we nevertheless have some useful indicators.

%\item Institutions and politics are central.
\end{enumerate}


\begin{comment}
\subsubsection*{Review questions}

\begin{enumerate}
\item {\it Country risk analysis.\/}
Make up your own:
collect data for any country of interest
and run through the same analysis we did
for Turkey.
\end{enumerate}
\end{comment}

\subsubsection*{If you're looking for more}

On the most recent crisis, Ben Bernanke's
\href{http://www.federalreserve.gov/newsevents/testimony/bernanke20100902a.htm}
{testimony} to the crisis commission is a nice overview (also mercifully short).
There are a number of well-written books about economic crises.
Among the best:
%
\begin{itemize}
\item Carmen Reinhart and Ken Rogoff,
\href{http://www.amazon.com/This-Time-Different-Centuries-ebook/dp/B004EYT932/}
{\it This Time is Different\/}.
The recent bestseller, covering 800 years and the whole world.

\item Robert Bruner and Sean Carr,
\href{http://www.amazon.com/Panic-1907-Lessons-Learned-Markets/dp/0470452587/}
{\it The Panic of 1907\/}.
Great read, short, you'll think it's about 2007.

\item David Wessel,
\href{http://www.amazon.com/Fed-We-Trust-Bernankes-Great/dp/0307459683/}
{\it In Fed We Trust\/}.
Terrific book from the Wall Street Journal writer.

%\item Paul Blustein,
%    {\it And the money rolled in (and out).\/}
%    Wonderful review of Argentina's 1999-2001 crisis.
%    He has another one, {\it The chastening\/}, about the Asian crisis
%    and the IMF.
%\item Andres Oppenheimer, {\it Bordering on chaos: Mexico's roller-coaster journey %toward prosperity.\/}
%    About Mexico in running up to the election and
%    subsequent crisis of 1994.
\end{itemize}


\vfill \centerline{\it \copyright \ \number\year \ NYU Stern
School of Business}

\end{document}
