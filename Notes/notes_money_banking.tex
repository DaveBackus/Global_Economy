\documentclass[letterpaper,12pt]{article}

\usepackage{amsmath}
\usepackage[dvips]{graphicx}
\usepackage{hyperref}
%    \hypersetup{colorlinks=true,urlcolor=blue,linkcolor=red}
\RequirePackage{GE05}

\newcommand{\GDP}{\mbox{\em GDP\/}}
\newcommand{\NDP}{\mbox{\em NDP\/}}
\newcommand{\GNP}{\mbox{\em GNP\/}}
\newcommand{\NX}{\mbox{\em NX\/}}
\newcommand{\NY}{\mbox{\em NY\/}}
\newcommand{\CA}{\mbox{\em CA\/}}
\newcommand{\NFA}{\mbox{\em NFA\/}}
\newcommand{\Def}{\mbox{\em Def\/}}
\newcommand{\CPI}{\mbox{\em CPI\/}}
\newcommand{\CU}{\mbox{\em CU\/}}
\newcommand{\RE}{\mbox{\em RE\/}}
\newcommand{\MB}{\mbox{\em MB\/}}

\def\ClassName{The Global Economy}
\def\Category{Class Notes}
\def\HeadName{Money and Banking}

\begin{document}
\thispagestyle{empty}%
\Head

\centerline{\large \bf \HeadName}%
\centerline{Revised: \today}

\bigskip

This note is about the nuts and bolts of monetary policy. It is an
updated version of chapter 10 of the manuscript entitled
``Macroeconomics: A Modern Perspective," by Tom Cooley and Lee
Ohanian.

\subsubsection*{What is money?}

A standard way to begin discussions of money is to try to define
what it is. This is somewhat difficult to do because historically
many things have been used as money - shells, beads, cigarettes,
pieces of paper. What characteristics make any of these suitable as
a form of money? One way to think about this is to define money in
terms of the services it provides. Money is an asset. An asset is
something that serves as a store of value. This means that it can be
used as a way of transferring consumption between periods. But, lots
of things, stocks, bonds, real estate, can and do fulfill that
function. Money is really quite different from other assets because
it provides another important service - it serves as a medium of
exchange. The medium of exchange role implies that it is freely
exchanged for goods and services and it has wide acceptance and
(generally) well understood value. Another service that money
provides is that it serves as a unit of account. The role of unit of
account means that when we talk about the value of other assets or
consumption goods we use monetary units as a standard way of
denominating them.

How does money come into being and why are people willing to accept
some forms of money? We know that different forms of money have
evolved naturally in many societies. One example that is often cited
is that cigarettes became used as a form of money in prisoner of war
camps during World War II. They are also often used as a form of
money in U.S. prisons. What problems does the existence of an
accepted form of money solve? The easiest way to understand this is
to imagine a simple economy in which individuals all specialize in
the production of a single good. Some grow wheat, some harvest wood,
some raise chickens and some educate the young. The educators and
wood harvesters have to eat, the food producers need wood and need
to educate their young and so on. People benefit from transacting
with one another. But if this were a pure barter world, then
transaction could only take place when we found someone who offered
in trade something we desire and who desired that which we produce.
This is called the \textit{double coincidence of wants}. The point
is that transacting in such a world would be very inefficient.
Suppose, instead that there were some accepted medium of exchange.
It need not be anything with intrinsic value. It could be stones of
a certain size and shape, or pieces of paper embossed with a picture
of long dead politicians. All that is required is that everyone
agree that it is the medium of exchange and agree on its relative
value. In this world, educators could now exchange education
services for the type of money and use the money to purchase wheat
and wood without worrying about whether the producers of wheat and
woods that he encountered had any need for education services. The
acceptance of a medium of exchange thus facilitates transactions in
the society because it removes an important impediment to economic
activity. That is why money arises naturally. Now, lets talk about
how we measure it.

Because the distinguishing characteristic of money is its use to
facilitate transactions, a definition of money would include assets
commonly used to facilitate transactions and would exclude assets
that are difficult or impossible to use for transactions. My house,
for example, is an asset and a store of value, but it would be
difficult to use for transactions because it is not divisible and
there may be a lot of uncertainty about its value. Unfortunately,
the distinction between these types of assets is not always very
sharp and it has changed over time because of technology and
improved access to information. Accordingly, there is not a unique
empirical definition of money. Rather, there are several measures
that we use and each of them tends to be useful for some purposes.
In what follows we will discuss the monetary measures that are
tracked in the United States. Although the discussion is specific to
the United States these definitions tend to be fairly universally
applied.

The narrowest measure of the money supply counts only
government-issued currency held by the non-bank public. This
aggregate is included in all broader definitions of money and is
called the currency component of the money supply.

A somewhat broader definition of the money supply includes the total
monetary liabilities of the federal government and is known either
as \textit{high-powered money} or the \textit{monetary base},
denoted MB. This broader definition includes currency held by the
non-bank public as well as reserves held by commercial banks as
backing for their deposit liabilities. Banks can hold reserves in
either of two forms, \textit{vault cash} held directly by the
commercial bank and \textit{reserve deposits} maintained by the
commercial bank at one of the twelve regional Federal Reserve Banks.
The Federal Reserve has tight short-term control over the monetary
base but not over broader monetary aggregates.

A somewhat broader definition of money is known as M1. It includes
currency held by the non-bank public, travelers checks, and
checkable deposits at commercial banks. (Note that bank reserves do
not appear directly in M1. Instead, they back deposits at commercial
banks.) A still broader definition of the money supply, known as M2,
includes M1 plus savings deposits, small time deposits (under
\$100,000), money market mutual fund shares (MMMFs) held by the
public, money market deposit accounts (MMDAs), overnight Eurodollar
deposits in foreign branches of U.S. banks, and overnight repurchase
agreements whereby a bank sells a security overnight to a non-bank
institution. A still broader definition, M3, includes M2 plus large
certificates of deposit and MMMFs held by institutions.

Table \ref{fig:money_measures} shows the components of the various
measures of money as of February 2005 (figures are in billions of
dollars).
%
\begin{table}
\begin{center}
\begin{tabular}{lr}
\hline%
Monetary Base                            & 764.5 \\
\\
M1 & 1,366.9 \\
\hspace{0.25in} Currency                 & 702.0 \\
\hspace{0.25in} Travelers Checks         & 7.5 \\
\hspace{0.25in} Demand Deposits          & 335.6 \\
\hspace{0.25in} Other Checkable Deposits & 321.8 \\
\\
M2 & 6,456.3 \\
\\
M3 & 9,502.5 \\
\hline
%
\end{tabular}
\end{center}
\caption{United States - Measures of Money Supply in February
2005.}\label{fig:money_measures}
\end{table}
%
For comparison, nominal $GDP$ in 2004 was about \$11,735 billion. It
is worth noting that what you might commonly think of as money,
currency, is only a small fraction of these broader measures. But it
is clear that all of these other components of money are available,
to some degree or other, for transactions.

For our purposes, the exact definition of the money supply is not
crucial. The important points to note are (a) deposits rather than
currency constitute the largest portion of the broader monetary
aggregates and (b) the distinction between these broader aggregates
and the monetary base, over which the Federal Reserve has close
control. From now on, we will speak only of the monetary base and
the money supply, with the latter being composed of currency and
deposits. For most purposes we need not be specific about which
deposits are included in the definition.

\subsubsection*{Managing the money supply}

In the United States and in most other countries, the supply of
money is controlled by a Central Bank. The U.S. central bank is
called the Federal Reserve. It was established in 1913 by an act of
congress and has responsibility for regulating banks and, most
importantly, for formulating and conducting monetary policy. The
Federal Reserve is an independent central bank. This means that it
can formulate and implement policy independently of the government
in power. This arrangement is not true of other countries and we
will discuss the importance of this independence later on. There are
12 regional Federal Reserve Banks and a Board of Governors of the
Federal Reserve System that resides in Washington D.C.. Since the
1930's power over monetary policy has been concentrated in the
Federal Reserve Board and a group called the Federal Open Market
Committee (FOMC). The FOMC consists of the seven Governors of the
Federal Reserve System, who are appointed by the president for
staggered 14 year terms, the Chairman, currently Alan Greenspan, who
serves for a four year term and five of the regional bank presidents
who serve on a rotating basis. The FOMC meets every six weeks and is
the group that sets monetary policy.

As noted above, the Federal Reserve directly controls the monetary
base. Through its control of the monetary base and other tools of
monetary policy, the Federal Reserve also influences the short-term
behavior of the broader monetary aggregates and can control their
long-run behavior. The Federal Reserve's short-run control over
aggregates broader than the monetary base is far from perfect, and
commercial banks and the non-bank public can have a substantial
effect on the short-term behavior of the broader monetary
aggregates. Historically banks were at center of the financial
system and the Federal Reserve had substantial control over the
provision of credit in the economy. In recent years, however, the
role of non-bank financial institutions such as insurance companies
and pension funds has increased and these institutions and the
credit they provide are not subject to the influence of the Fed.

We will illustrate in some detail how the Fed implements monetary
policy. Before we do that, it will be useful to describe the
features of a banking system where deposits must be backed by
reserves.

The United States has what is called a fractional-reserve banking
system, meaning that commercial banks have to hold reserves equal
only to a fraction of their deposit liabilities. Banks hold reserves
for two reasons. First, because they must be able to honor demands
for cash by their depositors. Second, in the U.S. and in many other
countries, they are required to maintain certain reserves. The
amount of such required reserves has varied tremendously over time
and it varies tremendously across countries. In Canada and the U.K.
for example, banks are not required to maintain any set fraction of
deposits as reserves. Of course this doesn't mean the banks in those
countries don't maintain reserves because they still do so for the
first reason cited. Variations in reserve requirements have at
various times been an important tool of monetary policy. How those
variations affect the economy is an important topic that we will
discuss further. The significance of the fact that banks only have
to keep fractional reserves is that banks can use a dollar of
reserves to back several dollars of deposits. We now examine how
commercial banks use additional reserves to create deposits.

The table below illustrates a highly simplified balance sheet of a
commercial bank. The major asset categories are reserves, earning
assets, and buildings and facilities. Reserves earn no interest. The
two main types of earning assets are loans made by the bank and
securities (mainly government debt) held by the bank. The bank's
major liability is deposits held for its customers.

\begin{center}
\begin{tabular}{|c|c|}
\hline \multicolumn{2}{|c|}{Bank Balance Sheet} \\
\hline\hline Assets & Liabilities \\ \hline\hline
\multicolumn{1}{|l}{Reserves} & \multicolumn{1}{|l|}{Deposits} \\
\multicolumn{1}{|l}{\thinspace\thinspace\thinspace\thinspace\thinspace
\thinspace\thinspace\thinspace\thinspace\thinspace\thinspace\thinspace
\thinspace Required} & \multicolumn{1}{|l|}{} \\
\multicolumn{1}{|l}{\thinspace\thinspace\thinspace\thinspace\thinspace
\thinspace\thinspace\thinspace\thinspace\thinspace\thinspace\thinspace
\thinspace\thinspace\thinspace Excess} & \multicolumn{1}{|l|}{} \\
\multicolumn{1}{|l}{Earning Assets} & \multicolumn{1}{|l|}{} \\
\multicolumn{1}{|l}{\thinspace\thinspace\thinspace\thinspace\thinspace
\thinspace\thinspace\thinspace\thinspace\thinspace\thinspace\thinspace
\thinspace\thinspace Securities} & \multicolumn{1}{|l|}{} \\
\multicolumn{1}{|l}{\thinspace\thinspace\thinspace\thinspace\thinspace
\thinspace\thinspace\thinspace\thinspace\thinspace\thinspace\thinspace
\thinspace\thinspace Loans} & \multicolumn{1}{|l|}{} \\
\multicolumn{1}{|l}{Buildings, etc.} & \multicolumn{1}{|l|}{Net Worth} \\
\hline
\end{tabular}
\end{center}

Suppose we decided to start a student bank, to be run by the
association of students. Initially all items on the balance sheet of
Student Bank are zero. (For simplicity, we do not explicitly show
net worth, securities, buildings and facilities, etc., on the
balance sheet.) Now suppose that our customers deposit \$1,000,000
of currency in the Bank (these are well-to-do students or there are
lots of them!). This currency (now vault cash) counts as a part of
Student Bank's reserves. Suppose that the only regulation this bank
faces is that it must maintain a certain fraction of its deposit as
reserve. \ If the required reserve ratio is 0.2, Student Bank now
has required reserves of \$200 and excess reserves of \$800. The
following table shows the bank's new balance sheet. Because these
reserves earn no interest, the bank wishes to convert its excess
reserves into earning assets.

\begin{center}
\begin{tabular}{|l|r|l|r|}
\hline
\multicolumn{4}{|c|}{Student Bank} \\
\hline \multicolumn{2}{|c|}{Assets(000's)} &
\multicolumn{2}{l|}{Liabilities (000's)}
\\ \hline\hline
\multicolumn{1}{|l|}{Reserves} & \multicolumn{1}{|r|}{1,000} &
\multicolumn{1}{|l|}{Deposits} & 1,000 \\
\multicolumn{1}{|c|}{Required} & 200 & \multicolumn{1}{|c}{} &
\multicolumn{1}{|c|}{} \\
\multicolumn{1}{|c|}{Excess} & 800 & \multicolumn{1}{|c}{} &
\multicolumn{1}{|c|}{} \\
\multicolumn{1}{|l|}{Loans} & 0 & \multicolumn{1}{|c|}{} &
\multicolumn{1}{|c|}{} \\ \hline
\end{tabular}
\end{center}

But, the bank was formed to make some money and provide services to
other students. Student Bank can convert its excess reserves into
earning assets by making a loan. Suppose that Hillary, a customer of
Student Bank, comes to the bank and applies for an \$800,000 loan to
invest in some vacation real estate in Arkansas. The bank's credit
review committee approves the loan and credits Hillary's checking
account with \$800,000. The bank's deposits increase to \$1,800,000,
its required reserves increase to \$360,000, and its excess reserves
fall to \$640,000. The table below shows the new balance sheet of
Student Bank.

\begin{center}
\begin{tabular}{|l|r|l|r|}
\hline
\multicolumn{4}{|c|}{Student Bank} \\
\hline%
\multicolumn{2}{|c|}{Assets (000's)} & \multicolumn{2}{l|}{Liabilities (000's) } \\%
\hline\hline%
\multicolumn{1}{|l|}{Reserves} & \multicolumn{1}{|r|}{1,000} & \multicolumn{1}{|l|}{Deposits} & 1,800 \\%
\multicolumn{1}{|l|}{\thinspace\thinspace\thinspace\thinspace Required} & \multicolumn{1}{|r|}{360} & \multicolumn{1}{|l|}{} &  \\%
\multicolumn{1}{|l|}{\thinspace\thinspace\thinspace\thinspace\thinspace
\thinspace Excess} & \multicolumn{1}{|r|}{640} &
\multicolumn{1}{|l|}{} &
\\
\multicolumn{1}{|l|}{Loans} & \multicolumn{1}{|r|}{800} &
\multicolumn{1}{|l|}{} &  \\ \hline
\end{tabular}
\end{center}

Hillary now goes to the real estate developer and writes a check for
\$800,000 to pay for her part of the investment. The developer
deposits the check in his account at Ozark Bank. When the check
clears, Student Bank must transfer \$800,000 in cash to Ozark Bank.
Student Bank loses deposits and reserves of \$800,000, while Ozark
Bank gains deposits and reserves in the same amount. The balance
sheets of the two banks are shown below. Notice that Student Bank
now has no excess reserves, while Ozark Bank has excess reserves of
\$640,000.

\begin{center}
\begin{tabular}{|l|r|l|r|c|l|r|l|r|}
\hline \multicolumn{4}{|c|}{Student Bank} & \multicolumn{1}{|c|}{} & \multicolumn{4}{|c|}{Ozark Bank} \\%
\hline \multicolumn{2}{|c|}{Assets (000's)} &
\multicolumn{2}{c|}{Liabilities (000's) } & \multicolumn{1}{|c|}{} & \multicolumn{2}{|c}{Assets (000's)} & \multicolumn{2}{|c|}{Liabilities (000's)} \\%
\hline\hline%
\multicolumn{1}{|l|}{Reserves} & \multicolumn{1}{|r|}{200} & \multicolumn{1}{|l|}{Deposits} & 1,000 & \multicolumn{1}{|l|}{} & Reserves & 800 & Deposits & 800 \\%
\multicolumn{1}{|l|}{\thinspace\thinspace\thinspace\thinspace\thinspace
\thinspace\thinspace Required} & \multicolumn{1}{|r|}{200} &
\multicolumn{1}{|l|}{} &  & \multicolumn{1}{|l|}{} &
\thinspace\thinspace
\thinspace\thinspace\thinspace\thinspace\thinspace Required & 160 &  &  \\
\multicolumn{1}{|l|}{\thinspace\thinspace\thinspace\thinspace\thinspace
\thinspace\thinspace Excess} & \multicolumn{1}{|r|}{0} &
\multicolumn{1}{|l|}{} &  & \multicolumn{1}{|l|}{} &
\thinspace\thinspace
\thinspace\thinspace\thinspace\thinspace\thinspace Excess & 640 &  &  \\
\multicolumn{1}{|l|}{Loans} & \multicolumn{1}{|l|}{800} &
\multicolumn{1}{|l|}{} &  & \multicolumn{1}{|l|}{} & Loans & 0 &  &  \\
\hline
\end{tabular}
\end{center}

Ozark Bank now converts its excess reserves into earning assets by
making a loan of \$640,000, to Bill, leaving it with the following
balance sheet:

\begin{center}
\begin{tabular}{|l|l|l|l|}
\hline
\multicolumn{4}{|c|}{Ozark Bank} \\%
\hline
\multicolumn{2}{|c|}{Assets (000's)} & \multicolumn{2}{l|}{Liabilities (000's) } \\%
\hline\hline
\multicolumn{1}{|l|}{Reserves} & \multicolumn{1}{|r|}{800} & Deposits & \multicolumn{1}{|r|} {1,400} \\%
\multicolumn{1}{|l|}{\thinspace\thinspace\thinspace\thinspace\thinspace \thinspace Required} & \multicolumn{1}{|r|} {288} &  &  \\%
\multicolumn{1}{|l|}{\thinspace\thinspace\thinspace\thinspace\thinspace \thinspace Excess} & \multicolumn{1}{|r|} {512} &  &  \\%
\multicolumn{1}{|l|}{Loans} & \multicolumn{1}{|r|} {640} & \multicolumn{1}{|l|}{} &  \\
\hline
\end{tabular}
\end{center}

The borrower, Bill spends this \$640,000 with various merchants who
deposits their funds in Arkansas Bank. When everything clears, these
two banks are left with the balance sheets shown below.

\begin{center}
\begin{tabular}{|ll|ll|cllll|}
\hline \cline{6-9} \multicolumn{4}{|c|}{Ozark Bank} &
\multicolumn{1}{|c|}{} &
\multicolumn{4}{|c|}{Arkansas Bank} \\
\hline \multicolumn{2}{|c|}{Assets (000's)} &
\multicolumn{2}{c|}{Liabilities (000's) } & \multicolumn{1}{|c|}{} &
\multicolumn{2}{|c|}{Assets (000's)} &
\multicolumn{2}{|c|}{Liabilities (000's)} \\ \hline\hline
\multicolumn{1}{|l|}{Reserves} & \multicolumn{1}{|r|} {160} &
\multicolumn{1}{|l|}{Deposits} & \multicolumn{1}{r|} {800} &
\multicolumn{1}{|l|}{} & \multicolumn{1}{l|}{Reserves} &
\multicolumn{1}{l|}{640} & \multicolumn{1}{l|}{Deposits} & \multicolumn{1}{|r|}{640} \\
\multicolumn{1}{|l|}{\thinspace\thinspace\thinspace\thinspace\thinspace
\thinspace Required} & \multicolumn{1}{|r|} {160} &
\multicolumn{1}{|l|}{} &  & \multicolumn{1}{|l|}{} &
\multicolumn{1}{l|}{\thinspace\thinspace
\thinspace\thinspace\thinspace\thinspace Required} &
\multicolumn{1}{l|}{128}
& \multicolumn{1}{l|}{} &  \\
\multicolumn{1}{|l|}{\thinspace\thinspace\thinspace\thinspace\thinspace
\thinspace Excess} & \multicolumn{1}{r|} {0} &
\multicolumn{1}{|l|}{} & & \multicolumn{1}{|l|}{} &
\multicolumn{1}{l|}{\thinspace\thinspace\thinspace\thinspace\thinspace
\thinspace\thinspace Excess} & \multicolumn{1}{l|}{512} &
\multicolumn{1}{l|}{} &  \\
\multicolumn{1}{|l|}{Loans} & \multicolumn{1}{|r|} {640} &
\multicolumn{1}{|l|}{} &  & \multicolumn{1}{|l|}{} &
\multicolumn{1}{l|}{Loans} &
\multicolumn{1}{r|}{0} & \multicolumn{1}{l|}{} &  \\
\hline
\end{tabular}
\end{center}

This process, which is repeated again and again, is known as the
bank credit expansion process because the supply of bank credit
increases at each step. Each increase in bank lending causes an
equal increase in deposits and the money supply. Banks are unique
among financial institutions because their lending activities
increase the money supply.

What is the end result of the credit expansion process? Deposits
increase at each stage of the process, and the total increase in
deposits is \$(1,000,000 + 800,000 + 640,000 + 512,000 + ...) =
\$5,000,000. Total bank reserves increase by \$1,000,000 in the
initial stage and then increase no further. The \$1,000,000 of
additional reserves is just redistributed among different banks. The
end result is that \$1,000,000 of additional bank reserves supports
\$5,000,000 of additional deposits. This result is due to the fact
that the reserve ratio is 0.2, so that each dollar of reserves can
support five dollars of deposits.

\subsubsection*{The Money Multiplier}

In the above example, we implicitly assumed that each bank customer
held all of his or her money in deposits. What would have happened
if each depositor had held a mix of deposits and currency? To be
specific, suppose the real estate developer had deposited only
\$600,000 in Ozark Bank and had held \$200,000 in currency (perhaps
to make political contributions!). Ozark Bank would have had 25
percent fewer excess reserves and would have created fewer deposits.
In addition, each subsequent bank would have created 25 percent less
deposits. The net result is more currency, fewer deposits, and less
total money in the system. How do we know that the decline in
deposits outweighs the increase in currency, resulting in a smaller
total money supply? The reason is that each dollar held by the
public in currency removes one dollar in reserves from the banking
system, and each dollar of reserves (under the assumption of a
reserve ratio of 0.2) can back five dollars of deposits.

We could illustrate this result by following an initial deposit of
\$1,000,000 through several different banks, as in our previous
example, but an example with both currency and deposits would get
very messy. This simplest illustration is algebraic. We will use the
following notation:

$M$ = the money supply

$C$ = the currency component of the money supply

$D$ = deposits

$H$ = high-powered money (the monetary base)

$R$ = bank reserves

$c$ = the ratio of currency to deposits

$rr$ = the ratio of reserves to deposits (the reserve ratio)

\smallskip\

The money supply is equal to deposits plus currency:

\begin{equation*}
M=D+C.
\end{equation*}

The monetary base is divided between currency and bank reserves:

\begin{equation*}
H=C+R.
\end{equation*}

Banks hold reserves equal to a fraction of their deposits:

\begin{equation*}
R=rrD.
\end{equation*}

The public holds currency equal to a fraction of deposits:

\begin{equation*}
C=cD.
\end{equation*}

Substituting for currency, the money supply is given by:

\begin{equation*}
M=D+cD=(1+c)D.
\end{equation*}

Substituting for currency and reserves, the monetary base is given
by:

\begin{equation*}
H=cD+rrD=(c+rr)D.
\end{equation*}

The ratio of the money supply to the monetary base is called the
\textit{money multiplier} and is equal to

\begin{equation*}
m=\frac MH=\frac{(1+c)}{(c+rr)}.
\end{equation*}

The money multiplier tells us how much each dollar of the monetary
base can support. In our first example, people hold no currency
$(c=0)$ and the reserve ratio is 0.2, so that the money multiplier
is five. Suppose that the Federal Reserve injects six dollars into
the financial system. The six dollars, when deposited in the banking
system, is multiplied through the credit expansion process into 30
dollars of deposits.

If people hold currency equal to 20 percent of deposits $(c=0.2)$
and the reserve ratio is still 0.2, then the money multiplier is
only three. If the Federal Reserve injects six dollars of
high-powered money into the system, the public initially holds one
dollar in currency and deposits the remaining five dollars in the
banking system. Through the credit expansion process, the banking
system multiplies this five dollars of reserves into a larger amount
of deposits. However, at each step of the process, the public
withdraws some money from the banking system so as to keep its
currency holdings equal to 20 percent of deposits. Each of these
currency withdrawals reduces bank reserves, so that the banking
system can create fewer deposits at each stage of the credit
expansion process. In the end, the money supply expands by only 18
dollars, three dollars of which is currency and 15 dollars of which
is deposits. Because high-powered money increases by six dollars and
currency holding increase by only three dollars, the remaining three
dollars find their way into the banking system as reserves. The
additional reserves are exactly equal to the amount banks must hold
to back the 15 dollars of new deposits.

\subsubsection*{The Balance Sheet of the Central Bank}

In the previous discussion we followed money through individual
banks without explicitly acknowledging the role of the central bank
- the Federal Reserve - in the process. Now, let us look at the
banking sector as a whole and the role of the central bank in
changing the money supply.

The following table illustrates the entries in the aggregate balance
sheet for the banking sector of a country:

\begin{center}
\begin{tabular}{|cl|}
\hline%
\multicolumn{2}{|c|}{\textbf{Banking Sector}} \\%
\hline\hline%
\multicolumn{1}{|c|}{Assets} & \multicolumn{1}{|c|}{Liabilities} \\%
\hline\hline%
\multicolumn{1}{|l}{Reserves} & \multicolumn{1}{|l|}{Deposits} \\%
%
\multicolumn{1}{|l}{\thinspace\thinspace\thinspace\thinspace\thinspace \ \ \thinspace Required} & \multicolumn{1}{|l|}{\ \ \ \ \ \ \ \ Demand Deposits}\\%
%
\multicolumn{1}{|l}{\thinspace\thinspace\thinspace\thinspace\thinspace \thinspace\ \ Excess} & \multicolumn{1}{|l|}{\ \ \ \ \ \ \ \ Time Deposits}\\%
%
\multicolumn{1}{|l}{} & \multicolumn{1}{|l|}{Borrowing from} \\%
%
\multicolumn{1}{|l}{Cash} & \multicolumn{1}{|l|}{\ \ \ \ \ \ \ \ \ Central Bank} \\%
%
\multicolumn{1}{|l}{Securities} & \multicolumn{1}{|l|}{\ \ \ \ \ \ \ \ \ Other}\\%
%
\multicolumn{1}{|l}{Loans} & \multicolumn{1}{|l|}{} \\%
%
\hline
\end{tabular}
\end{center}

This is just like the balance sheets considered above except that
now we have expanded the range of assets and liabilities that a bank
has. Bank reserves here represent the banking sector's deposits with
the central bank. The securities can make many forms but include
securities issued by the government. Now let us look at the balance
sheet of the central bank.

\bigskip

\begin{center}
\begin{tabular}{|cc|}
\hline \multicolumn{2}{|c|}{\textbf{Central Bank}} \\%
\hline\hline%
\multicolumn{1}{|c|}{Assets} & \multicolumn{1}{|c|}{Liabilities} \\%
\hline\hline%
Securities & \multicolumn{1}{|c|}{Currency} \\%
Loans to Banks & \multicolumn{1}{|c|}{Reserves} \\%
%
Foreign Exchange & \multicolumn{1}{|c|}{Other Deposits} \\%
%
\hline
\end{tabular}
\end{center}

The banks assets include government securities, loans to member
banks, and foreign exchange reserves. The latter includes deposits
abroad, foreign denominated securities, and deposits with other
central banks. The banks liabilities include currency issued,
reserves deposited by member banks and other deposits. In the U.S.,
the Federal Reserve by law cannot pay interest on reserves.

\subsubsection*{The Tools of Monetary Policy}

The Central Bank has three main tools with which to influence the
money supply. The most important is \textit{open market operations},
through which the Central Bank buys and sells federal government
debt securities on the secondary market. In the U.S. these
securities are issued by the U.S. Treasury and certain government
agencies, but the Federal Reserve is prohibited from purchasing
government debt directly when issued. The Central Bank uses open
market operations to control the monetary base. For example, let's
assume the Central Bank buys \$100 million of government bonds from
private banks. It pays for those bonds by crediting the private
banks accounts at the Central Bank. Such accounts are part of both
bank reserves and the monetary base. The effect is shown in the two
balance sheets below:

\begin{center}
\bigskip

\bigskip
\begin{tabular}{|crcr|}
\hline
\multicolumn{4}{|c|}{\textbf{Banking Sector}} \\
\hline\hline%
\multicolumn{2}{|c|}{Assets (000's)} & \multicolumn{2}{|c|} {Liabilities (000's)}\\%
\hline\hline%
\multicolumn{2}{|l}{Reserves} & \multicolumn{2}{|l|}{Deposits} \\%
\multicolumn{2}{|l}{\thinspace\thinspace\thinspace\thinspace\thinspace \thinspace Required} & \multicolumn{2}{|l|}{\ \ \ \ \ \ Demand Deposits} \\%
\multicolumn{1}{|l}{\thinspace\thinspace\thinspace\thinspace\thinspace \thinspace Excess} & \multicolumn{1}{r|} {+100,000} & \multicolumn{2}{|l|}{\ \ \ \ \ \ Time Deposits} \\%
\multicolumn{2}{|l}{} & \multicolumn{2}{|l|}{Borrowing from} \\
%
\multicolumn{2}{|l}{Cash} & \multicolumn{2}{|l|}{\ \ \ \ \ \ \ Central Bank}\\%
%
\multicolumn{1}{|l}{Securities} & \multicolumn{1}{r|} {-100,000} & \multicolumn{2}{|l|}{\ \ \ \ \ \ \ Other} \\%
%
\multicolumn{2}{|l}{Loans} & \multicolumn{2}{|l|}{} \\%
\hline \hline%
\end{tabular}

\bigskip

\begin{tabular}{|crcr|}
\hline \multicolumn{4}{|c|}{\textbf{Central Bank}} \\
\hline\hline%
\multicolumn{2}{|c|}{Assets (000's)} & \multicolumn{2}{|c|}{Liabilities (000's)}\\%
\hline\hline%
\multicolumn{1}{|l}{Securities} & \multicolumn{1}{r|}{+100,000} & \multicolumn{2}{l|}{Currency} \\%
%
\multicolumn{2}{|l|}{Loans to Banks} & \multicolumn{1}{|l}{Reserves} & \multicolumn{1}{l|}{+100,000} \\%
%
\multicolumn{1}{|l}{Foreign Exchange} & \multicolumn{1}{l|}{}  & \multicolumn{1}{|l}{Other Deposits} & \multicolumn{1}{l|}{}  \\%
\hline
\end{tabular}
\end{center}

The private banks can now use the additional \$100 million of
reserves to back additional deposits by the public. They have excess
reserves so they are free to make additional loans to customers,
thereby increasing the amount of money or ``liquidity'' in the
economy in the way we illustrated in our earlier simple example. The
net result is that an open market purchase expands the monetary base
and increases bank reserves.

Conversely, if the Central Bank sells government securities that it
holds to the private banking sector it has the effect of decreasing
the monetary base. An open market sale of securities by the Central
Bank causes a reduction in bank reserves and the monetary base.

The Central Bank's second tool of monetary control is
\textit{reserve requirements}. The Central Bank can require banks to
hold reserves equal to some fraction of their deposits. If it raises
the required reserve ratio (i.e., the ratio of reserves to
deposits), the Central Bank reduces the amount of deposits that
banks can support with a given stock of reserves. This has the
effect of reducing the amount of money that is available for lending
and thus reducing the money supply. \ In practice changes in reserve
requirements have not been used much as an instrument of monetary
policy for some time. In Canada and the United Kingdom required
reserves are zero. In the United States reserve requirements for
time and savings deposits were eliminated in 1990.

If a commercial bank's actual reserves are below the level required
by the Central Bank, the bank can borrow additional reserves either
from the Central Bank or from another commercial bank that has
excess reserves. In the U.S. such interbank loans are known as
\textit{federal funds}, and the interest rate on these loans is
known as the\textit{\ federal funds rate}. The U.S. Federal
Reserve's borrowing facility is known as the \textit{discount
window}, and the associated interest rate is the \textit{discount
rate}. Changes in the discount rate are the Federal Reserve's third
tool of monetary control. By raising the discount rate, the Federal
Reserve can discourage banks from borrowing at the discount window,
thus making banks less aggressive in creating deposits. In addition,
the Federal Reserve can informally discourage what it regards as
excessive commercial bank borrowing at the discount window.

How does the description just given fit with our understanding of
how monetary policy works in practice? In the United States the
policy decisions made at meetings of the Federal Open Market
Committee are usually stated in terms of interest rates. For
example, at the meeting held March 22nd, 2005, the FOMC decided to
raise the ``Target level'' for the Federal Funds Rate and the
Discount Rate each by 25 basis points. Thus, the stated policy is to
raise the Federal Funds rate by 25 basis points or one quarter of
one percent. Such a policy is implemented by conducting open market
operations that have the effect of withdrawing reserves (and hence
liquidity) from the system to the point at which the market rate for
federal funds - reserves borrowed from other banks or from the Fed,
rise by twenty five basis points. For that reason the specific goals
of monetary policy are stated in terms of an interest rate rather
than the size of a monetary aggregate or the amount of reserves in
the system. The Discount Rate is something that is just implemented
directly. The Fed raises the rate at which it is willing to lend
reserves to member banks.

These examples illustrate the role of the Central Bank, the banks,
and the public in the money supply process. The Central Bank
controls the monetary base and sets minimum required values for the
reserve ratio. As a matter of prudence, banks generally hold some
reserves above the minimum required, but they can vary the amount of
excess reserves they hold and the volume of new loans and deposits
they generate. These decisions affect the money multiplier through
their effect on the reserve ratio. Finally, the public can
significantly alter the money multiplier by changing the mix of
currency and deposits it holds.

\subsubsection*{The Management of Reserves}

In some ways this idea of the money multiplier depending on the
amount of required reserves is outdated. In the U.S. required
reserves are about ten percent. But, even these reserve requirements
may not really be binding for banks. And, as we noted above, in many
countries, Canada being one example, required reserves are zero.

It is interesting to think about the issue of reserves from the
point of view of the bank. As we discussed earlier, banks still hold
reserves so they can honor transactions made by customers. If I
write a check to pay a bill or use my debit card to purchase
something, the bank must honor it immediately. Clearly it would be
unacceptable for them to argue that I have to wait because they are
temporarily short of funds. So, they are motivated to hold reserves.
But, if the required level of reserves is greater than what they
need to cover transactions they are unhappy because those reserves
do not earn any interest.

How do banks respond to this mismatch between required reserves and
desired reserves? Recall, that banks are no longer required to hold
reserves for time deposits or for money market deposit accounts. In
response to the mismatch between required reserves and desired
reserves banks have created sweep accounts. At the end of the day
they can sweep excess funds from checking accounts, which are
subject to reserves requirements, into money market accounts that
are exempt from reserve requirements. The end result of this is that
reserves will be reduced toward the level that is needed to meet
transactions demand.

Because banks are able to operate closer to their margins, they are
likely to go into the markets for reserves more often. A bank that
inadvertently finishes a day with too few reserves can avoid penalty
by borrowing reserves from other banks in the \textit{Federal Funds
Market. }Or, the bank can borrow money from the Federal Reserve
through the Discount Window. Similarly, a bank which ends the day
with too many reserves is likely to be a net lender in the Federal
Funds market. The increased use of sweeps accounts has the effect of
causing more activity in the Federal Funds Market and potentially
causing the Federal Funds rate to be more volatile. Some observers
have worried that the use of sweeps complicates the implementation
of monetary policy by making specific targets for the funds rate
more difficult to hit. To counter this possibility that reserve
management might make the Federal Funds rate more volatile, the Fed
has changed the period over which banks calculate their reserves.

\subsubsection*{Other Functions of A Central Bank}

The discussion so far has focused mainly on the role of the Central
Bank in the money supply process. In the United States and in most
other countries the Central Bank has other important duties as well.
Typically Central Banks have some responsibility for the supervision
and regulation of banks. It has a responsibility to guarantee the
soundness of the banking system and with that role in mind it
regulates the actions of member banks to make sure that they remain
financially sound. Bank regulation usually takes three forms.

Banks are required to meet minimum \textit{capital requirements}.
These specify the ratio of capital to bank assets. There are
international standards for the minimum ratio of capital to risk
adjusted assets that are by the Bank for International Settlements.

\textit{Bank examination} is often an important function of the
central bank. Bank portfolios are reviewed periodically to make
certain that banks adequate reserves for losses on risky loans.

Finally, Central Banks usually provide \textit{lender of last
resort} facilities for banks that are solvent but may be illiquid
because of the fact that they tend borrow short term but make long
term loans. In many countries the difficulty is to decide when banks
are indeed solvent or whether they are simply illiquid, and then
there is problem of deciding what to do about it. That is a topic
worthy of extensive discussion in itself but it would take us too
far afield from our primary objective.

Finally, in some countries, notably the U.S., the Central Bank is
responsible for the regulation of banking structure. That is, it
oversees the acquisition of banks and non-banking businesses by bank
holding companies. For many years and for historical reasons there
was a fear of consolidation in the banking industry and of the
expansion of banks into related financial activities. In recent
years many of the barriers to these activities have been removed,
and this function of the Fed has changed.


%\subsubsection*{Executive summary}

%Business cycles around the world share these properties:
%\begin{enumerate}
%\item Economies do not growth smoothly, they exhibit a lot of
%short-run volatility.

%\end{enumerate}


%\subsubsection*{Review questions}

%You might think about each of the following issues, which we'll
%address at greater length shortly.
%\begin{enumerate}

%\item Why do you think employment lags the economy?

%\end{enumerate}


%\subsubsection*{Further reading}

%For more information ...
%
%\begin{itemize}
%\item National Bureau of Economic Research: US
%\href{http://www.nber.org/cycles.html/}{business cycle dates} and
%\href{http://www.nber.org/releases/}{calendar} of release dates
%for international economic indicators (you can sign up for email
%reminders).

%\end{itemize}


\vfill \centerline{\it \copyright \ \number\year \ NYU Stern
School of Business}

\end{document}
