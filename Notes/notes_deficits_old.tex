\documentclass[letterpaper,12pt]{article}

\RequirePackage{comment}
\RequirePackage[hypertex]{hyperref}
    \hypersetup{colorlinks=true,urlcolor=blue,linkcolor=red}
\RequirePackage{GE05}
%\usepackage[hypertex]{hyperref}
%    \hypersetup{colorlinks=true,urlcolor=blue,linkcolor=red}

\newcommand{\GDP}{\mbox{\em GDP\/}}
\newcommand{\NDP}{\mbox{\em NDP\/}}
\newcommand{\GNP}{\mbox{\em GNP\/}}
\newcommand{\NX}{\mbox{\em NX\/}}
\newcommand{\NY}{\mbox{\em NY\/}}
\newcommand{\CA}{\mbox{\em CA\/}}
\newcommand{\NFA}{\mbox{\em NFA\/}}
\newcommand{\Def}{\mbox{\em Def\/}}
\newcommand{\CPI}{\mbox{\em CPI\/}}
\newcommand{\IP}{\mbox{\em IP\/}}
\newcommand{\PD}{\mbox{\em PD\/}}
\newcommand{\hsp}{\hspace{0.25in}}

\def\ClassName{The Global Economy}
\def\Category{Class Notes}
\def\HeadName{Government Debt and Deficits}

\begin{document}
\thispagestyle{empty}%
\Head

\centerline{\large \bf \HeadName}%
\centerline{Revised: \today}

\bigskip
The phrase ``fiscal policy'' refers to government decisions 
to spend, tax, and issue debt.  
Since governments play a central role in most modern economies, 
summary measures of fiscal policy are used to assess the
performance of the economy, future interest rates, 
and the likelihood that the government will honor its debt.  
We examine each of these aspects of fiscal policy, starting with
a quick overview of how government finances are measured and reported in the National Income and Product Accounts.


\subsubsection*{The government's budget}

Governments everywhere collect revenue (largely through taxes), purchase goods and services (schools, police, courts, military), 
transfer money to people and corporations through a variety of
programs (social insurance, health care),   
and pay interest on outstanding government debt.  
You can get a sense of how this looks in Table~\ref{tab:usdef}, where we report the basic government finance numbers for the US in 2003.  (For comparison, US GDP in 2003 was \$11 trillion.)

\begin{table}[h]
\centering \tabcolsep=0.2in
\begin{tabular}{lr}
\hline\hline Category  &  Amount \\
\hline\hline

{Total receipts}        &      3,060.4 \\
\hsp Tax receipts              &      2,033.8     \\
\hsp Contributions for social insurance        &   773.2     \\
\hsp Other              &          759.7      \\

\hline

{Total expenditures}    &    3,566.7    \\
\hsp Government consumption expenditures             &      1,717.1          \\
\hsp Government investment expenditures (gross)             &    358.5            \\
\hsp Transfers            &       1,332.9         \\
\hsp Interest payments             &    303.0            \\
\hsp Balancing item              &           --144.8     \\

\hline

Net lending or borrowing   &     --506.3  \\

\hline\hline
\end{tabular}
\caption{US government budget in 2003. Billions of US dollars.}
\label{tab:usdef}
\end{table}

We can compute various summary measures from the entries in the
table.  {\it Government purchases\/} is the sum of (government)
consumption and investment expenditures (1,717.1 + 358.5). This is
the $G$ in the expenditure identity. {\it Transfers\/} are another
category of expenditures (1,332.9); we treat them separately because
they consist of transfers of purchasing power, not direct purchases
of goods or services. They include Social Security outlays. The {\it
government deficit\/} (506.3) is the difference between total
expenditures and total receipts. It's a measure of the borrowing
requirement of the government. The {\it primary deficit\/} equals
government deficit excluding interest payments (506.3 -- 303.0 = 203.3).
You might think of it as a rough analog to operating cash flow.  
{\it National saving\/} (--367.8) is minus the deficit plus net
investment (gross investment minus depreciation); it can't be
computed from the numbers in the table.


We can put these elements together in a relation we'll call 
the {\it government budget constraint\/}.
On the expense side, governments purchase goods and services 
(label this $G$), make transfer payments ($V$) to (primarily) households, 
and pay interest on accumulated debt.
If the debt is $B$ and the interest rate is $i$, then this payment 
is $iB$.
On the revenue side, they collect taxes ($T$).
Note that $T$ is tax revenue, not the tax rate.
By convention, all of these things are nominal:
they're measured in current prices.  
The government budget constraint is then 
\begin{equation}
    G_t + V_t + i_t B_t - T_t  \;=\; (B_{t+1} - B_t ) .
    \label{eq:gbc}
\end{equation}
It says, in essence, that any surplus or deficit must be matched
by a change in the quantity of debt.
The left-hand side of (\ref{eq:gbc}) 
is the government deficit, 
the right the change in the quantity of debt.
Typically all of these components are measured in 
units of the local currency. 


The elements of equation (\ref{eq:gbc}) 
are often used to generate summary measures 
of fiscal policy. 
The most common are ratios  of 
the government deficit and government debt to GDP.  
We'll look at both, as well as the connection between them.  


\subsubsection*{Pay me now or pay me later}  


Governments need to finance their spending with taxes.
It's not quite true --- governments have other sources of revenue ---
but it's close enough to be worth remembering.  
Issuing debt allows a government to postpone taxes, 
just as a credit card allows an individual to postpone paying 
for purchases, 
but does not eliminate the obligation.  
Delay, in fact, comes with a cost:  we need to pay the original 
obligation, plus interest.  
If this is obvious to you,  you can skim the rest of this section.


We're going to take our budget constraint, equation (\ref{eq:gbc}), 
and use it to express debt as a present value of future surpluses.  
To make things simple, define the primary deficit $D$ as the deficit 
net of interest payments:
\[
    D_t \;=\;  G_t + V_t - T_t .
\]
The primary surplus is the same object with a minus sign. 
Then (\ref{eq:gbc}) can be expressed as 
\[
    B_{t+1}   \;=\; (1+i) B_t + D_t 
\]
or 
\begin{equation}
    B_{t}   \;=\;  B_{t+1}/(1+i) - D_t/(1+i) .
    \label{eq:gbc-bdynamics}
\end{equation}
All three versions contain the same information, but we'll 
focus on (\ref{eq:gbc-bdynamics}).  
If we use it to substitute repeatedly, we find 
\begin{eqnarray*}
    B_{t}   &=&  B_{t+2}/(1+i)^2 - [D_t/(1+i)+ D_{t+1}/(1+i)^2] \\
            &=&  B_{t+3}/(1+i)^3 - 
                [D_t/(1+i)+ D_{t+1}/(1+i)^2 + D_{t+2}/(1+i)^3] \\
            &=&  B_{t+n}/(1+i)^n - 
                [D_t/(1+i)+ D_{t+1}/(1+i)^2 + \cdots + D_{t+n-1}/(1+i)^n] .
%    \label{eq:gbc-bdynamics-rec}
\end{eqnarray*}
If we assume that debt can't grow faster than the interest rate forever, 
then as we continue to substitute, the first term goes to zero.
[The technical condition is $B_{t+n}/(1+i)^n$ goes to zero as $n$ 
approaches infinity.]
The assumption seems reasonable, since our payments 
aren't growing fast enough in this case even to cover the interest.  
The relation then becomes  
\begin{eqnarray*}
    B_{t}   &=&  -[D_t/(1+i)+ D_{t+1}/(1+i)^2 + 
                   D_{t+2}/(1+i)^3] +   \cdots ]  \\
            &=& - \mbox{Present Discounted Value of Primary Deficits} \\
            &=& \mbox{Present Discounted Value of Primary Surpluses} .
\end{eqnarray*}
In words:   the current government debt (the result of past deficits) 
must be matched  by the present discounted value of future primary surpluses
(in which we pay our creditors back).  
As we said earlier, all spending must by financed by tax revenue.  
It's not enough to shrink the deficit:  eventually we have 
to run surpluses, measured net of interest payments.  
There's a limit, in other words, to deficit spending. 
None of this is any different from an individual borrowing 
from a bank.   


Although (primary) government deficits must eventually be reversed, 
they may have affect the economy while they last.  
One route is distribution.  
If taxes are delayed long enough, the tax burden 
will be shifted from current to future generations.
Thus the US contribution to World War II was financed largely with debt, 
shifting some of the tax burden from those alive at the time 
to those born later.

Is deficit financing a good or bad thing?  
One approach to this question is based on tax smoothing.
As we mentioned in an earlier class,
the economic disincentives built into taxes are minimized
by having relatively fixed tax {\it rates\/} 
even if tax revenues vary over time.
On average this is likely to lead to governments running (modest) 
surpluses in booms and deficits in recessions, 
as tax revenues go up and down with the economy.  


\subsubsection*{Debt dynamics and sustainability} 

One common measure of fiscal discipline (or lack thereof)
is the ratio of government debt to GDP.  
Since the debt accrues interest, it has a natural tendency to grow.  
If the current situation leads the ratio of debt to GDP to grow, 
then if nothing changes --- ever --- it will eventually become too large 
for the government to finance:  
interest payments alone will eat up the entire budget.  
We would say the situation is not sustainable.  
Something will change to keep this from happening.  
The question is what.  
We'll work through the possibilities.  


How do debt and the debt-to-GDP ratio change through time?  
Debt evolves according to  
\begin{eqnarray*}
    B_{t+1} &=& D_t + (1+i) B_t  ,
\end{eqnarray*}
where $D$ is (again) the primary deficit.  
Note that everything is nominal, including the interest rate $i$.  
If the growth rate of (nominal) GDP is $g$, then  
\[
    Y_{t+1} \;=\; (1+g) Y_t 
\]
and $B/Y$ follows 
\begin{equation}
    \frac{B_{t+1}}{Y_{t+1}} \;=\; 
                \left( \frac{1+i}{1+g} \right)  \frac{B_{t}}{Y_{t}} 
             +   \left( \frac{1}{1+g} \right)  \frac{D_{t}}{Y_{t}} .
    \label{eq:debtdynamics}             
\end{equation}
This equation tells us how the debt-to-GDP ratio changes 
from period to period.   


How does the debt-to-GDP ratio evolve?  
The first issue is whether $g$ is larger or smaller than $i$.  
If $g>i$, then a government can run deficits forever without 
the ratio exploding --- although it could nevertheless become large.  
More commonly, $ g<i$, as above, 
in which case the ratio 
will continue to grow, even if we have a zero primary deficit ($D = 0$).
In this more common case, any deficit is unsustainable, 
in the sense that if the deficit isn't eliminated, 
the ratio of debt to GDP will eventually become infinite.
Similarly, interest payments will grow without bound.


If interest on the debt becomes large enough, something must change.
But what?
One candidate is the deficit itself:  the government 
does this analysis, realizes that interest will eat up 
all its revenue, and decides to reduce the deficit before that happens.  
This could come from a reduction in spending or an increase in taxes; 
both work the same way in this analysis, 
although  in other respects the two may differ.  
Another candidate --- one we'd generally prefer not to consider 
--- is default.
Bond investors can do this analysis, too, 
and if they find that the debt is more than the government 
is likely to pay back, they will stop lending. 
This tends to reduce the deficit, too, 
by restricting the government's ability to finance it.  


This is a mechanical analysis, but a useful one. 
By looking at the components of equation (\ref{eq:debtdynamics}), 
we can get a sense of how fast the debt ratio is changing, 
and therefore whether or not we're headed into trouble in the near future.
How much debt is too much?  
It depends on the situation.
(We know this is a copout, but it's true.)
If investors get the idea that they may not be paid back, 
then it's too high.  
When that happens depends on the country and circumstances. 
Ultimately investors are betting on a country's ability and 
willingness to repay its debt, so we're back in the business 
of assessing the political aspects of default and repayment.  


What's missing?  What we're doing is extrapolating 
the current situation, but 
governments make commitments all the time about future 
spending that we should also consider. 
Right now, for example, most governments have made at least informal
commitments to finance (in part) their citizens' retirements.
But as the fraction of the population collecting such benefits
increases, it becomes increasingly difficult to finance.
What will happen?  Will benefits be cut?  
Will taxes in the working population go up?  
Probably some combination, but 
the point is that we have more information at hand than the 
current primary deficit.  
In principle, these future commitments should appear 
as liabilities now, but pension accounting remains insufficient 
for this task in both the public and private domains.  


\subsubsection*{Executive summary}

\begin{enumerate}

\item Countries differ widely in the magnitude of government spending.  

\item Deficits are promises to collect future taxes.  Any effect on the economy must be based on differences in timing of revenue rather than differences in its present discounted value.

\item Debt dynamics imply that a government deficit must eventually 
be reversed. 

\end{enumerate}


\begin{comment}
\subsubsection*{Review questions}

\begin{enumerate}

\item 

\end{enumerate}
\end{comment}

%\subsubsection*{Further information}


\vfill \centerline{\it \copyright \ \number\year \  
NYU Stern School of Business}

\end{document}


% **********************************************************************************************
Class outline



Class 1:  
govt budget constraint 
incentives...  


Class 2:  ...  





% old stuff
\subsubsection*{Government purchases}

Government purchases of goods and services use resources:  if the government purchases more, then
unless there is a comparable increase in production there will be less available for consumption
and investment.  The primary effect is almost certainly to reduce the quantity of resources
available for other purposes, but beyond that the impact will depend on the type of government
purchase and specific features of the economic system.

Government purchases cover a wide range.  They include spending on the judicial system, public
education, health and medical services, transportation and telecommunication infrastructure, and
national security. Some of these expenditures raise the productivity of firms (eg, highways),
others provide utility to consumers (health services), and still others may do neither or both.
Some purchases compete with privately produced services (education) while others do not (national
security).

Let us consider, in the interest of concreteness, an increase in government purchases of goods and
services that neither increase the productivity of firms nor substitute for private consumption.
Let us assume, in addition, that the increase is financed by a comparable increase in lump-sum
taxes.  What is the impact on output, employment, consumption, investment, and interest rates?

There are many answers to this question, but what follows is a summary of our current
understanding of the issues.
%
\begin{itemize}
\item Output rises, but by less than the increase in government purchases.  In the Keynesian
models that you many of you studied as undergraduates, output rises because the economy has excess
capacity, and government spending acts as a catalyst.  But those models often suggest
``multiplier'' effects that lead to larger increases in output than government purchases. Modern
dynamic models operate through labor supply (see below) and typically imply small output effects.

\item Consumption and investment fall.  You might have guessed this from the output effect: since
$G$ rises by more than $Y$, there is less left for $C$ and $I$ (see:  expenditure identity). We're
ignoring international trade here, but will come back to it later in the course.

\item Employment rises.  This is the reason output rises (more people are working), but why?  In
Keynesian models we're simply jump-starting the economy.  In modern dynamic models, the increase
in government purchases reduces household's lifetime income (remember, their taxes went up, too).
This income effect leads them to work more (they buy less leisure).  The longer the increase in
spending lasts, the larger the impact on lifetime income and the greater the impact on labor
supply.  In most models the impact is relatively small in any case.  If the increase in spending
lasts a long time, the impact on employment (and output) is reduced further by a reduction in the
capital stock, which reduces the demand for labor.

\item Interest rates.  If the increase is temporary, then we expect consumption to increase in the
future (rebound) and hence for the interest rate to rise now, before returning to its current
level when the spending stops.  If the increase is permanent, consumption drops immediately and
the interest rate doesn't change.

\end{itemize}

These are claims of theory, but are the conclusions reasonable?  A number of kinds of evidence
suggest they are.  First, we have periodic wars. Wars are natural experiments, because they
increase government purchases sharply. In the US, for example, all of these features apply to
World War I, World War II, and the Korean War.  (Later wars were too small relative to the size of
the economy to have a noticeable impact.) Note, in particular, the evidence for WW II, which many
observers have credited with getting the country out of the Depression. (Robert Barro's {\it
Macroeconomics\/} has a nice summary of this evidence if you're interested in the details.)
Second, we can compare differences in the performance of countries with different levels of
government purchases.  The evidence for a broad cross-section of countries is that those with high
spending have lower growth rates and (eventually) lower output. Whether this reflects the
mechanism we've described or something else is a matter for debate, but the simple Keynesian
prediction that high levels of spending will be associated with high levels of output is simply
not a feature of the data.



\subsubsection*{Debt and deficits}

We're going to treat debt and deficits as promises to collect taxes in the future. The question is
whether the timing of taxes affects the economy.

Let us consider a drop in current tax revenue that increases the current deficit.  The government
budget constraint implies
\[
    B_t \;=\; (T_t - G_t - V_t)/(1+r_t) + B_{t+1}/(1+r_t) .
\]
If we substitute for the last term using the same equation at $t+1$, we get
\begin{eqnarray*}
    B_t &=& (T_t - G_t - V_t)/(1+r_t)
        + (T_{t+1} - G_{t+1} - V_{t+1})/[(1+r_t)(1+r_{t+1})] \\
        && + \; B_{t+2}/[(1+r_t)(1+r_{t+1})] .
\end{eqnarray*}
If we do this over and over again, and assume that the debt doesn't explode (we're making a
technical assumption that guarantees convergence), we find that the value of the current debt
equals the present discounted value of future surpluses:
\begin{eqnarray*}
    B_t &=&  (T_t - G_t - V_t)/(1+r_t) + (T_{t+1} - G_{t+1} - V_{t+1})/[(1+r_t)(1+r_{t+1})] \cdots \\
        &=&  \sum_{s=0}^\infty R_{t,t+s} (T_{t+s} - G_{t+s} - V_{t+s}) ,
\end{eqnarray*}
where
\[
    R_{t,t+s} \;=\; 1/[(1+r_t) (1+r_{t+1}) \cdots (1+r_{t+s})]
\]
is the discount factor applied at date $t$ to cash flows at $t+s$.  In words:  the debt is the
present value of future (primary) surpluses.  If we do something that changes the current deficit,
we must increase a future surplus to match.
