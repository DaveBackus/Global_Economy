\documentclass[letterpaper,12pt]{article}

\usepackage{amsmath}
\usepackage{natbib}
\usepackage[dvips]{graphicx}
\usepackage{hyperref}
%    \hypersetup{colorlinks=true,urlcolor=blue,linkcolor=red}
\RequirePackage{GE05}

\newcommand{\GDP}{\mbox{\em GDP\/}}
\newcommand{\NDP}{\mbox{\em NDP\/}}
\newcommand{\GNP}{\mbox{\em GNP\/}}
\newcommand{\NX}{\mbox{\em NX\/}}
\newcommand{\NY}{\mbox{\em NY\/}}
\newcommand{\CA}{\mbox{\em CA\/}}
\newcommand{\NFA}{\mbox{\em NFA\/}}
\newcommand{\Def}{\mbox{\em Def\/}}
\newcommand{\CPI}{\mbox{\em CPI\/}}

\newtheorem{example}{Example}

\def\ClassName{The Global Economy}
\def\Category{Class Notes}
\def\HeadName{Modern Macroeconomic Theory}

\begin{document}
\thispagestyle{empty}%
\Head

\centerline{\large \bf \HeadName}%
\centerline{Revised: \today}

\bigskip

This note provides a brief introduction to modern macroeconomic
theory. The paradigm that we will discuss is the result of more than
three decades of research in macroeconomics. Two among the scholars
that gave the most notable contributions to this research program
are on the faculty at Stern: Thomas Cooley and Thomas Sargent. The
ideas that we introduce are mainstream in the best economics school
in the world. It is very unfortunate that many business leaders,
politicians, and journalists, still base their views on outdated and
logically flawed theories that were mainstream in academic circles
until the early '70s.

In the lecture note entitled ``Business Cycle Theory" we will use
the framework of modern macroeconomic theory in order to
illustrate what is today the prevalent theoretical approach to the
study of Business Cycles.

This note is based in part on the book ``Macroeconomics," by
Robert J. Barro of Harvard University, published by MIT Press. It
is a great book, but unfortunately it is not well suited for a
graduate business program.

\subsubsection*{Introduction}

Basically, modern macroeconomic theory describes the economy of a
country (any country) as a collection of firms and households,
which have well defined objectives (maximize their owners' wealth
in one case and maximize their life-time utility or satisfaction
in the other) and try to accomplish them by exchanging goods and
services with other firms and households.

We will make many simplifying assumptions. The most obvious are
that we will abstract from the existence of the rest of the world
(i.e. we will study an economy in isolation) and that we will not
consider the role of the government. Finally, you will notice that
we won't even mention \textit{money} and \textit{monetary policy}.
This is not to say that we believe the interaction with the rest
of the world and the policies of government and central bank to be
unimportant for macroeconomic outcomes. On the contrary. The point
is that the conclusions that we will reach in this note would
stand intact if we incorporated these elements in our discourse.
They would be much harder to derive, though. We will consider the
role of government policies later in the course. Next we will
introduce money, and at the end of the course we will study the
interaction with the rest of the world.

A further simplification is that in this note we will abstract
from uncertainty. We will re-introduce it later in the course,
when talking about business cycles.

The basic postulates that are at the basis of modern macroeconomic
theory are that \textit{people are smart} and \textit{react to
incentives.} People are \textit{smart} in the sense that, either
consciously or unconsciously, they understand the consequences of
major events (e.g. changes in government policy, changes in oil
prices, improvements in technology) on current and future levels
of wages and interest rates (among these rates are returns on
investment and borrowing rates such as mortgage rates). What do we
mean when we say that people \textit{react to incentives}? Well,
we simply mean that households, given current and future values of
wages and interest rates, are able to select those choices (how
much to consume? how much to save? how much to work?) that
maximize their level of satisfaction.

Here is a stylized view of the economy of a country in some year
$t$. We think of the economy as a collection of firms and
households. Firm use inputs such as capital ($K_{t}$) and labor
($L_{t}$) to produce commodities. Let's call the firms' output
$Y_{t}$. At the beginning of year $t$, the capital is given. This
is because we assume it takes one year to adjust the level of
capital. In order to produce, firms will also have to hire labor.
Then capital, labor, and the level of TFP ($A_{t}$) will imply a
level of output $Y_{t}$. Our first observation is that:

\begin{itemize}
\item firms express demand for labor services%
\item firms supply commodities%
\end{itemize}

What about households? At the beginning of year $t$, households
will have some financial assets, which we call $a_{t}$, yielding a
return $r_{t}$. We think of $a_{t}$ as the value of the
household's portfolio of securities: it can be as simple as the
money invested in a checking account and as exotic as the value of
a hedge fund or private equity fund holding. If an household's
assets are negative, it means that it is borrowing at the rate
$r_{t}$. The total resources available to an household will be the
sum of financial wealth, $a_{t}(1+r_{t})$, and labor income
$w_{t}L_{t}$, where $w_{t}$ is the wage and $L_{t}$ is the amount
of hours worked. The decisions of the household will be: how much
to consume of the commodities available on the market (we call
consumption expenditure $c_{t}$), how much to work ($L_{t}$), and
how much to re-invest in financial assets, assets that will yield
a return $r_{t+1}$. That is:

\begin{itemize}
\item households express demand for the various commodities (demand for consumption)%
\item households supply labor services%
\item households supply capital%
\end{itemize}

When a household has positive financial wealth, it is providing
resources either to other households (in form of direct or
intermediated loans) or to firms (in forms of equity or debt). If
we sum the holdings of financial assets across all households, we
obtain their total wealth. Notice that this must be equal to the
total value of firms. This is true because: 1) when we sum
financial assets across households, the claims toward other
households cancel out with the liabilities; 2) the households hold
the liabilities (debt and equity) issued by firms (the right-hand
side of their balance sheets), and 3) the value of the right-hand
side of the balance sheet equals the value of the left-hand side,
which is precisely the value of physical assets in place, mostly
plant, property, and equipment (capital stock $K_{t}$, in
economists' jargon). At year $t$, every firm will determine its
desired level of capital for the following year. Summing the
desired level of capital across firms gives the demand for capital
for the next year, $K_{t+1}$. If $K_{t+1}>K_{t}(1-\delta)$, there
will be investment in the economy. Therefore we can say that

\begin{itemize}
\item firms express a demand for capital%
\item if the capital demanded by all firms is higher than the capital in place, the aggregate of firms express a demand for investment%
\end{itemize}

Notice that in the aggregate, gross investment cannot be negative.
In fact a negative gross investment would imply either destroying
capital or transforming it into items which are good for
consumption. This is either very expensive, or infeasible.

The bottom line is that in pursuing their objectives, firms and
households demand and supply several goods and services. These
demands and supply meet in three markets: the market for labor,
the market for capital, and the commodity market. In the labor
market, firms and households exchange labor services. There will
be equilibrium in that market in year $t$ when the demand for
labor expressed by firms equals the supply expressed by
households. Similarly, the capital market will be in equilibrium
when the supply of capital from households (their total financial
wealth $a_{t+1}$) equals the demand for capital, $K_{t+1}$. In the
commodity market, the supply is given by the total value of the
commodities produced by firms. The market is in equilibrium when
such supply equals the demand for investment (expressed by the
firms themselves) and for consumption (expressed by households).

In the next section we will investigate the determinants of the
aggregate demand for labor, capital, and investment. Later on we
will consider the households' decision problem. In turn, this will
allow us to figure out the determinants of the demand for
consumption, of the supply of capital, and of the supply of labor.

\subsubsection*{Demand for labor, capital, and investment}\label{sec:firm}

The purpose of this section is to characterize the main
determinants of the aggregate (economy-wide) demand for labor,
capital, and investment. In order to accomplish this task, we will
study the decision problem common to all firms, i.e. the choice of
the capital and labor input, given values for the interest rate
and the wage rate. To determine the aggregate desired level of
capital and labor, we just need to add up the single decisions.
The desired level of capital next period, given the capital
currently in place, will determine the demand for investment.

We are going to characterize the firm problem using a very
familiar framework. That is, we are going to assume that firms
produce output using the Cobb-Douglas production function
$Y_{t}=A_{t}F(K_{t},L_{t})=A_{t}K_{t}^{\alpha}L_{t}^{1-\alpha}$.
As always, $K_{t}$ denotes physical assets in place and $L_{t}$
denotes the number of people on the payroll. The wage rate is
denoted by $w_{t}$. For the purpose of this section only, we are
going to assume the existence of a 1-year riskless security that
yields a return $r_{t}$. (This security could be the return on
loans to households, since we are assuming no uncertainty). If the
price of the product that the company produces is equal to 1, then
the value that it produces for the shareholders is given by
%
\begin{equation*}
A_{t}F(K_{t},L_{t})-r_{t}K_{t}-w_{t}L_{t}.
\end{equation*}
%
In fact, since there is no uncertainty, $r_{t}$ is the opportunity cost of capital for the firm:
the return its shareholders would gain if they invested in the riskless security rather than in
the company.

Here is the first question: for given capital stock $K_{t}$, what
is the optimal level of labor input for the firms? That is, how
many workers will it hire? We know the answer: the firm will keep
on hiring as long as the gain in revenues from increasing the
labor factor is larger than the cost of doing so. The gain in
revenues of adding $\Delta$ labor is simply

\begin{equation*}
A_{t}F(K_{t},L_{t}+\Delta)-A_{t}F(K_{t},L_{t})=\frac{A_{t}F(K_{t},L_{t}+\Delta)-A_{t}F(K_{t},L_{t})}{\Delta}\Delta.%
\end{equation*}

When $\Delta$ is small, we know that the gain in revenues is
approximately

\begin{equation*}
\frac{\partial A_{t}F(K_{t},L{t})}{\partial L_{t}}\Delta.
\end{equation*}

What is the cost to the firm of increasing the labor input by $\Delta $? Simply $w_{t}\Delta$.
Figure~\ref{fig:labor_dem} illustrates the point: the desired level of labor is the one that
maximizes profits.

The bottom line is that the firm will keep on increasing labor as
long as $\frac{\partial A_{t}F(K_{t},L_{t})}{\partial
L_{t}}\Delta>w_{t}\Delta$ or $\frac{\partial
A_{t}F(K_{t},L_{t})}{\partial L_{t}}>w_{t}$. It will stop once
$\frac{\partial A_{t}F(K_{t},L_{t})}{\partial L_{t}}=w_{t}$.


%%%%%%%%%%%%%%%%%%%%%%%%%%%%%%%%%%%%%%%%%%%%%%%%%%%%%%%%%%%%%%%%%%%%%%%%%%%%

\begin{figure}[h!]
\begin{center}
\begin{picture}
(250,200)(-30,-20)
\footnotesize%

% Horizontal axis
\put(-30,0){\vector(1,0){300}}%
\put(265,-14){$L_{t}$}%

% Vertical axis
\put(0,-20){\vector(0,1){200}}%
\put(-55,155){$A_{t}K_{t}^{\alpha}L_{t}^{1-\alpha}$}%

% Low curve
\qbezier[200](0,0)(40,106)(170,120)

% Vertical dotted line
\qbezier[50](65,0)(65,42.5)(65,85)

\put(0,0){\line(3,2){200}}%

\multiput(5,45)(12,8){9}{\line(3,2){10}}

\put(195,157){$w_{t}L_{t}$}%
\put(201,154){\vector(0,-1){10}}%

\put(135,132){$A_{t}K_{t}^{\alpha}L_{t}^{1-\alpha}$}%
\put(141,129){\vector(0,-1){10}}%


\end{picture}
\end{center}
\caption{Determining labor demand.} \label{fig:labor_dem}%
\end{figure}
%%%%%%%%%%%%%%%%%%%%%%%%%%%%%%%%%%%%%%%%%%%%%%%%%%%%%%%%%%%%%%%%%%%%%%%%%%%%

The latter condition tells us exactly how much labor the firm will want to hire, given a wage
$w_{t}$. In the case of Cobb-Douglas production function, the condition is

\begin{equation*}
(1-\alpha)A_{t}K_{t}^{\alpha}L_{t}^{-\alpha}=w_{t}.
\end{equation*}

Solving this equation for $L_{t}$ yields the firm's optimal level of labor, which is also labor
demand:

\begin{equation*}
L_{t}=K_{t}\left[\frac{A_{t}(1-\alpha)}{w_{t}}\right]^{\frac{1}{\alpha}}.
\end{equation*}

The aggregate demand for labor services is the sum of the demands across all firms. Notice that,
as common sense dictates, the aggregate demand for labor is decreasing in the wage level and
increasing in the TFP level, $A_{t}$.

Now we can ask the next question: for given employed labor
$L_{t+1}$, what is the desired amount of capital for the firm in
year $t+1$? We also know the answer to this one: the firm will
keep on accumulating capital as long as the gain in revenues from
increasing it is larger than the cost of doing so. The gain in
revenues of adding $\Delta$ capital is simply

\begin{equation*}
A_{t}F(K_{t+1}+\Delta,L_{t+1})-A_{t}F(K_{t+1},L_{t+1})=\frac{A_{t}F(K_{t+1}+\Delta,L_{t+1})-A_{t}F(K_{t+1},L_{t+1})}{\Delta}\Delta.
\end{equation*}

When $\Delta$ is small, we know that the gain in revenues is
approximated by

\begin{equation*}
\frac{\partial A_{t}F(K_{t+1},L_{t+1})}{\partial K_{t+1}}\Delta.
\end{equation*}

What is the cost to the shareholders of increasing the capital input by $\Delta$? Simply
$r_{t+1}\Delta$. In fact $r_{t+1}$ is the opportunity cost for the shareholders.

The bottom line is that the firm will keep on increasing capital as long as $\frac{\partial
F(K_{t+1},L_{t+1})}{\partial K_{t+1}}\Delta>r_{t+1}\Delta$ or $\frac{\partial
F(K_{t+1},L_{t+1})}{\partial K_{t+1}}>r_{t+1}$. It will stop once $\frac{\partial
F(K_{t+1},L_{t+1})}{\partial K_{t+1}}=r_{t+1}$. The latter condition tells us exactly how much
capital the firm will want to employ, given an interest rate $r_{t+1}$. In the case of
Cobb-Douglas production function, the condition is

\begin{equation*}
\alpha A_{t+1}K_{t+1}^{\alpha-1}L_{t+1}^{1-\alpha}=r_{t+1}.
\end{equation*}

Solving this equation for $K_{t+1}$ yields the firm's optimal capital stock:

\begin{equation*}
K_{t+1}=L_{t+1}\left[\frac{A_{t+1}\alpha}{r_{t+1}}\right]^{\frac{1}{1-\alpha}}.
\end{equation*}

The desired level of the capital stock at the aggregate level is the sum of the optimal capital
stocks across all firms. Notice that, as common sense dictates, the desired level of capital and
the demand for investment are decreasing in the interest rate and increasing in $A_{t+1}$.

\textit{Summary}

\vspace{-.4cm}

\begin{itemize}

\item The aggregate demand for labor is higher the higher TFP and the lower the wage rate

\item The aggregate desired level of capital is higher the higher TFP and the lower the interest
rate

\item The demand for investment is higher the higher TFP and the lower the interest rate

\end{itemize}

\subsubsection*{Choosing how much to consume and how much to work}\label{sec:work}%
%
In this section we will study how households choose their
consumption and work effort (i.e. how many hours to work). For
simplicity, we will disregard the future. That is, we will assume
that households live for only 1 year. We will re-introduce the
time dimension later.

Again for simplicity, we will assume that there is only one
consumption good available. Let $c_{t}$ denote the level of
consumption. In order to describe households' preferences, we will
use a handy construct known as a \textit{utility function}. Given
levels of consumption and work effort are going to imply a level
of utility, or satisfaction. The more we consume, the better off
we are. This means that the utility function must be increasing in
$c_{t}$. It turns out we also have preferences on how much time we
devote to work. Everything else equal, the less we work, the
better off we are. Therefore our utility is decreasing in work
effort. We define \textit{leisure time} as the fraction of
non-sleeping time (18 hours a day?) that we do not dedicate to
work. If we denote the time worked as $l_{t}$ and we normalize the
amount of non-sleeping hours to $1$, leisure turns out to be
$1-l_{t}$.

The trade-off that is induced by the choice problem at hand is very simple. We would love to stay
home from work all the time. In that case, however, we would not be able to consume anything. This
point is illustrated by the \textit{budget constraint}:
%
\begin{equation*}
c_{t}\leq w_{t}l_{t}+M_{t}.
\end{equation*}
%
The amount of consumption expenditure will be at most equal to
$M_{t}$ (any non-wage income) plus the wage compensation
$w_{t}l_{t}$.

\textit{Optional} \footnotesize We denote our utility function as
$u(c_{t},1-l_{t})$. Given that the more we consume the better off
we are, the budget constraint will always hold with equality. That
is, we will spend everything (remember that there is no future).
Therefore the decision problem can be cast in the following way:
%
\begin{align*}
\max_{c_{t},l_{t}} u(c_{t},1-l_{t})\\
\text{s.t. } c_{t}=w_{t}l_{t}+M_{t}
\end{align*}
%
This means that we pick consumption and work effort so to maximize
utility, subject to the budget constraint. It is very easy to draw
the budget constraint on the cartesian plane $(c_{t},1-l_{t})$. We
do this in Figure~\ref{fig:opt_cons_work} below.

\begin{figure}[h!]
\centering
\includegraphics[scale=.75]{utility.eps}\newline
\caption{Example of utility function.} \label{fig:utility}
\end{figure}

Figure~\ref{fig:utility} hints that the representation of the
utility function is not as easy, but with a little thought we can
do a good job at it. We use a device known as \textit{indifference
curve}. Most probably you have already seen it in ``Firms and
Markets" and in ``Foundations of Finance", when talking about
portfolio choice. An indifference curve is simply a slice of the
utility function. Along an indifference curve, the level of
utility is constant. As we travel along one of them from the
top-left corner of the graph towards the bottom-right, the level
of consumption decreases, inducing lower utility. However, the
level of work effort decreases as well, maintaining the level of
utility constant. Why are the indifference curves convex? Because
when we work very hard, we are willing to give up a lot of
consumption for a small increase in leisure. When we do not work
as hard, to compensate for the same decrease in work effort, we
are willing to give up less consumption. Obviously, the level of
utility is not the same across indifference curves. The farther
away an indifference curve is from the origin, the higher the
level of utility. The bottom line is that an optimizing individual
chooses the consumption and leisure pair that lies on the budget
constraint in correspondence of the indifference curve
characterized by the highest utility. It is easy to see that the
chosen pair lies at the point of tangency between the budget
constraint and the selected indifference curve.


%%%%%%%%%%%%%%%%%%%%%%%%%%%%%%%%%%%%%%%%%%%%%%%%%%%%%%%%%%%%%%%%%%%%%%%%%%%%

\begin{figure}[h!]
\begin{center}
\begin{picture}
(250,230)(-30,-20)
\footnotesize%

% Horizontal axis
\put(-30,0){\vector(1,0){300}}%
\put(265,-14){$1-l_{t}$}%

% Vertical axis
\put(0,-20){\vector(0,1){200}}%
\put(-15,175){$c_{t}$}%

% Tangent Indifference Curve
\qbezier[200](10,180)(60,55)(170,60)

% Low Indifference Curve
\qbezier[200](5,115)(55,35)(170,38)

% Medium Indifference Curve
\qbezier[200](5,155)(55,45)(170,48)

% High Indifference Curve
\qbezier[200](20,200)(70,60)(170,70)

\put(190,30){\vector(0,1){40}}%

\put(195,45){Higher levels of utility}

\put(170,-14){$1$}%

\end{picture}
\end{center}
\caption{Indifference Curves}\label{fig:indiff}%
\end{figure}
%%%%%%%%%%%%%%%%%%%%%%%%%%%%%%%%%%%%%%%%%%%%%%%%%%%%%%%%%%%%%%%%%%%%%%%%%%%%

You can verify that the graphic solution coincides with the
analytical solution. That is, the optimal choice of consumption
and work ($c^{*}_{t},l^{*}_{t}$) satisfies the following
condition:
%
\begin{equation*}
u_{1}(c_{t},1-l_{t})w_{t}-u_{2}(c_{t},1-l_{t})=0
\end{equation*}
%
or
%
\begin{equation*}
\frac{u_{2}(c_{t},1-l_{t})}{u_{1}(c_{t},1-l_{t})}=w_{t}.
\end{equation*}
%
Here $u_{1}(\cdot,\cdot)$ denotes the derivative with respect to consumption and
$u_{2}(\cdot,\cdot)$ the derivative with respect to leisure. The optimality condition will not
surprise you. It just says that in correspondence of ($c^{*}_{t},l^{*}_{t}$) the slope of the
budget constraint ($-w_{t}$) equals the slope of the indifference curve
$\left(-\frac{u_{2}(c_{t},1-l_{t})}{u_{1}(c_{t},1-l_{t})}\right)$.

\normalsize \setlength{\baselineskip}{15pt}

\vspace{1cm}

%%%%%%%%%%%%%%%%%%%%%%%%%%%%%%%%%%%%%%%%%%%%%%%%%%%%%%%%%%%%%%%%%%%%%%%%%%%%

\begin{figure}[h!]
\begin{center}
\begin{picture}
(250,200)(-30,-20)
\footnotesize%

% Horizontal axis
\put(-30,0){\vector(1,0){300}}%
\put(265,-14){$1-l_{t}$}%

% Vertical axis
\put(0,-20){\vector(0,1){200}}%
\put(-15,175){$c_{t}$}%

% Budget constraint
\put(0,120){\line(2,-1){170}}%
\put(170,0){\line(0,1){36}}%

% Tangent Indifference Curve
\qbezier[200](10,180)(60,55)(170,60)

% Low Indifference Curve
\qbezier[200](5,155)(55,45)(175,48)

% High Indifference Curve
\qbezier[200](20,200)(70,60)(180,70)

\put(170,-14){$1$}%

% Dashed Horizontal Line at the kink
\qbezier[70](0,34)(85,34)(170,34)

% Dashed Horizontal Line at the optimum
\qbezier[35](0,74)(45,74)(92,74)

% Dashed Vertical Line at the optimum
\qbezier[35](92,0)(92,37)(92,74)

\put(91,75){\circle*{3}}

\put(86,-14){$1-l^{*}_{t}$}%

\put(-15,72){$c^{*}_{t}$}%

\put(-15,31){$M_{t}$}%

\end{picture}
\end{center}
\caption{Optimal Choice of Consumption and Leisure}\label{fig:opt_cons_work}%
\end{figure}
%%%%%%%%%%%%%%%%%%%%%%%%%%%%%%%%%%%%%%%%%%%%%%%%%%%%%%%%%%%%%%%%%%%%%%%%%%%%

\textit{Effects of an increase in wealth}

Here we are interested in understanding how the households'
optimal choices change when it becomes wealthier. The idea is to
think of leisure as a good that we can purchase. The price of
leisure is the wage income that we forego by laying on the couch
(or training for a marathon) instead of going to work. Then it is
reasonable to think that as we become wealthier, we would like to
spend more money both in consumption and in leisure. This implies
that more wealth should lead to the choices of increasing
consumption and decreasing the number of hours worked. Are these
conclusions supported by the data? It is not surprising that the
wealth effect on consumption is positive. Around the world,
consumption per person has grown with the rise in income per
capita. The negative wealth effect on work effort is harder to
verify. Here are a couple of facts that seem to attest that the
wealth effect on work effort is indeed negative. In the United
States, the average hours worked per week for workers in
manufacturing declined from about 60 in 1890 to 42 in 1996. Other
developed countries show similar patterns. If we look across
countries at a point in time, we see that the average hours worked
are negatively correlated with income per capita.

\textit{Optional} \footnotesize An increase in wealth can be
thought of as an increase in non-wage income from $M_{t}$ to
$\hat{M}_{t}$. Figure~\ref{fig:wealth} illustrates the mechanism.
The increase in wealth implies a higher budget constraint. The new
optimal pair implies more consumption and more leisure (less
work). In economists' jargon, the wealth effect is positive for
consumption (i.e. more wealth implies more consumption) and
negative for work effort (i.e. more wealth implies less work).

%%%%%%%%%%%%%%%%%%%%%%%%%%%%%%%%%%%%%%%%%%%%%%%%%%%%%%%%%%%%%%%%%%%%%%%%%%%%

\begin{figure}[h!]
\begin{center}
\begin{picture}
(250,200)(-30,-20)
\footnotesize%

% Horizontal axis
\put(-30,0){\vector(1,0){300}}%
\put(265,-14){$1-l_{t}$}%

% Vertical axis
\put(0,-20){\vector(0,1){200}}%
\put(-15,175){$c_{t}$}%

% High Budget constraint
\put(0,120){\line(2,-1){170}}%
\put(170,0){\line(0,1){36}}%

% Low Budget constraint
\put(0,100){\line(2,-1){170}}%
%\put(170,0){\line(0,1){36}}%


% Tangent Indifference Curve
\qbezier[200](10,165)(60,57)(170,60)

% Low Indifference Curve
\qbezier[200](5,140)(55,42)(175,38)

\put(170,-14){$1$}%

% Top dot
\put(90,76){\circle*{3}}

% Bottom dot
\put(80,60){\circle*{3}}

\put(195,75){\vector(-1,-1){30}}%
\put(197,75){New budget constraint}%

% High Dashed Horizontal Line at the kink
\qbezier[70](0,34)(85,34)(170,34)

\put(-15,31){$\hat{M}_{t}$}%

% Low Dashed Horizontal Line at the kink
\qbezier[70](0,16)(85,16)(170,16)

\put(-15,15){$M_{t}$}%

\end{picture}
\end{center}
\caption{An increase in wealth.} \label{fig:wealth}
\end{figure}
%%%%%%%%%%%%%%%%%%%%%%%%%%%%%%%%%%%%%%%%%%%%%%%%%%%%%%%%%%%%%%%%%%%%%%%%%%%%

\normalsize

\newpage

\textit{Effects of an increase in wages}

Here we are interested in understanding the impact of changes in
the wage level on the optimal choices of consumption and leisure.
In a sense, when the wage increases, leisure becomes more
expensive. By not working, you forego a larger amount of
consumption. For this reason, you would like to work more (and
consume more). This is what economists call the substitution
effect: when a good becomes relatively cheaper, you would like to
substitute others in its favor. This is the case both with apples
Vs. bananas and with consumption Vs. leisure. On the other hand, a
higher wage makes your wealthier. The wealth effect kicks in. From
the above discussion, being wealthier makes you eager to buy more
leisure (i.e. work less).

The bottom line is that with respect to the consumption choice,
substitution and wealth effect reinforce each other: a higher wage
surely implies more consumption. Instead, with respect to the work
choice, they act in opposite ways. In reality, whether households
decide to work more or less when their wage increases is an
empirical question. The data seem to suggest that the answer
depends on the level of development. In the United States, for
example, in the last forty years there has been no strong trend in
average hours worked per week in manufacturing for the United
States. A likely conclusion is that with respect to the work
choice wealth and substitution effects roughly cancel out. There
was, however, a major decline in average hours worked at earlier
stages of economic development.

\textit{Optional} \footnotesize Figure \ref{fig:wage} provides an
example in which the substitution effect prevails. The increase in
wage induces an upward rotation of the budget constraint. The new
optimal pair lies on a higher indifference curve, in
correspondence of a higher consumption and lower leisure (more
work).

%%%%%%%%%%%%%%%%%%%%%%%%%%%%%%%%%%%%%%%%%%%%%%%%%%%%%%%%%%%%%%%%%%%%%%%%%%%%

\begin{figure}[h!]
\begin{center}
\begin{picture}
(250,200)(-30,-20)
\footnotesize%

% Horizontal axis
\put(-30,0){\vector(1,0){300}}%
\put(265,-14){$1-l_{t}$}%

% Vertical axis
\put(0,-20){\vector(0,1){200}}%
\put(-15,175){$c_{t}$}%

% High Budget constraint
\put(0,100){\line(2,-1){170}}%
\put(170,0){\line(0,1){16}}%

% Low Budget constraint
\put(0,143){\line(4,-3){170}}%

% High Indifference Curve
\qbezier[200](10,178)(60,57)(170,60)

% Low Indifference Curve
\qbezier[200](5,140)(55,42)(175,38)

\put(170,-14){$1$}%

\put(77,85){\circle*{3}}

\put(80,60){\circle*{3}}

\put(195,55){\vector(-1,-1){30}}%
\put(197,55){New budget constraint}%


% Low Dashed Horizontal Line at the kink
\qbezier[70](0,16)(85,16)(170,16)

\put(-15,15){$M_{t}$}%

\end{picture}
\end{center}
\caption{An increase in wage.} \label{fig:wage}
\end{figure}
%%%%%%%%%%%%%%%%%%%%%%%%%%%%%%%%%%%%%%%%%%%%%%%%%%%%%%%%%%%%%%%%%%%%%%%%%%%%

\begin{example}
Assume that the utility function is given by $u(c,1-l)=c^{\gamma}(1-l)^{1-\gamma}$,
$\gamma\in(0,1)$. Determine the optimal choices of consumption and work effort. How do such
choices change when the wage increases? We have that $u_{1}(c,1-l)=\gamma
c^{\gamma-1}(1-l)^{1-\gamma}$ and $u_{2}(c,1-l)=(1-\gamma)c^{\gamma}(1-l)^{-\gamma}$. Therefore
the optimality condition is given by
%
\begin{equation*}
\frac{1-\gamma}{\gamma}\frac{c}{1-l}=w.
\end{equation*}
%
This equation, along with the budget constraint, determines the optimal choices of work and
consumption:
%
\begin{align*}
c&=\gamma(w+M),\\
l&=\gamma-(1-\gamma)\frac{M}{w}.
\end{align*}
%
It turns out that when the wage increases, both consumption and work effort increase.

\end{example}

\normalsize

\textit{Summary}

\vspace{-.4cm}

\begin{itemize}

\item Households react to greater wealth by consuming more and by working less

\item Households react to higher wages by consuming more.

\item The response of work effort to higher wages depends on the relative magnitude of the wealth
and substitution effect. On one hand, higher wage makes leisure more costly (by not working, the
household gives up more). This is the substitution effect, and works in the direction of making
the household work more. On the other hand, a higher wage implies that the household is wealthier.
The wealth effect works in the direction of making the household work less.

\item The empirical evidence suggests that the wealth effect
prevails at low levels of development. For higher levels of
development, the two effects roughly cancel out, so that hours
worked do not respond very much to increases in wage.

\end{itemize}

\subsubsection*{Choosing how much to consume and how much to save}%

In this section we will study how households take their
consumption and saving decisions \textit{over time}. We will still
assume that there is only one consumption good, but we will
introduce the time dimension. For now, we will abstract from the
choice of work effort. We will reintroduce it later. For the time
being, the household's income at year $t$ is exogenously given,
and denoted by $y_{t}$. Households can borrow or save at some
interest rate $r_{t+1}$. Borrowing can consist of credit card debt
or a mortgage, for example. Saving means investing in assets that
can be as simple as a saving account or as complicated as a
portfolio of stock. In the simple world we are describing, all of
these assets yield the same return. The budget constraint at year
$t$ will look as follows:
%
\begin{equation*}
c_{t}+a_{t+1}=y_{t}+a_{t}(1+r_{t}),
\end{equation*}
%
where $a_{t+1}$ represents the wealth accumulated as of time $t$.
If we are currently in year $t=1$, then the budget constraint is
%
\begin{equation}
c_{1}+a_{2}=y_{1}+a_{1}(1+r_{1}).\label{budget_1}
\end{equation}
%
The value $a_{1}$ represents the household's initial financial
assets. The interest proceeds on the asset are $a_{1}r_{1}$.
Notice that $a_{1}$ is something that the household inherits from
the past and $y_{1}$ and $r_{1}$ are quantities that it takes as
given. What the household can choose is how much to consume
($c_{1}$) and how much to save (invest in financial assets)
($a_{2}$). If it decides to consume a lot, it may turn out that it
has to borrow. In such case, we will obtain that that $a_{2}<0$.

Next year (year 2), the budget constraint will be
%
\begin{equation}
c_{2}+a_{3}=y_{2}+a_{2}(1+r_{2}).\label{budget_2}
\end{equation}
%
This implies that the decisions taken now (year 1) will have an effect on year 2 as well. In fact
the value of $a_{2}$, which is decided now (year 1) will constitute the net wealth the household
will start with next year.

We can rewrite equation (\ref{budget_2}) as
%
\begin{equation*}
a_{2}=\frac{c_{2}+a_{3}-y_{2}}{1+r_{2}}.
\end{equation*}
%
Then we can substitute the latter expression in (\ref{budget_1})
to obtain
%
\begin{equation*}
c_{1}+\frac{c_{2}}{1+r_{2}}=y_{1}+\frac{y_{2}}{1+r_{2}}+a_{1}(1+r_{1})-\frac{a_{3}}{1+r_{2}}.
\end{equation*}
%
You will recognize the left-hand side as the present value of consumption and the right-hand side
as the total wealth, i.e. the year-1 value of all cash-flows.

A first, important consideration, is that the possibility of
borrowing and lending eliminates the link between current cash
inflows and current cash outflow (consumption expenditures). This
means that the problem households really face is to figure out the
present value of wealth (this is the right-hand side) and then
decide how to spend it across years 1 and 2. In the simple
scenario under consideration (two years only), the household needs
to figure out its incomes in periods 1 and 2 ($y_{1}$ and $y_{2}$,
respectively), the initial level of financial assets $a_{1}$, and
the end of period-2 assets $a_{3}$. These quantities and the
interest rates tell how much its wealth is. Let's call the wealth
$x_{1}$. Then the problem is to pick $c_{1}$ and $c_{2}$ in order
to maximize \textit{utility}. In this case, utility is defined
over $c_{1}$ and $c_{2}$.


\textit{Optional} \footnotesize The mathematical formulation of
the problem is

\begin{align*}
\max_{c_{1},c_{2}} u(c_{1},c_{2})\\
\text{s.t. } c_{1}+\frac{c_{2}}{1+r_{2}}=x_{1}.
\end{align*}


%%%%%%%%%%%%%%%%%%%%%%%%%%%%%%%%%%%%%%%%%%%%%%%%%%%%%%%%%%%%%%%%%%%%%%%%%%%%

\begin{figure}[h!]
\begin{center}
\begin{picture}
(250,200)(-30,-20)
\footnotesize%

% Horizontal axis
\put(-30,0){\vector(1,0){300}}%
\put(265,-14){$c_{1}$}%

% Vertical axis
\put(0,-20){\vector(0,1){200}}%
\put(-15,175){$c_{2}$}%


% Low Budget constraint
\put(0,100){\line(2,-1){200}}%
%\put(170,0){\line(0,1){36}}%


% Tangent Indifference Curve
\qbezier[200](10,165)(60,57)(170,60)

% Low Indifference Curve
\qbezier[200](5,140)(55,42)(175,38)

\put(200,-14){$x_{1}$}%

\put(80,60){\circle*{3}}

% Dashed Horizontal Line at the optimum
\qbezier[35](0,60)(40,60)(80,60)

% Dashed Vertical Line at the optimum
\qbezier[35](80,0)(80,30)(80,60)

\put(-45,98){$x_{1}(1+r_{2})$}%

\put(80,-14){$c_{1}^{*}$}%
\put(-15,60){$c_{2}^{*}$}%

\end{picture}
\end{center}
\caption{Consumption-saving decision.} \label{fig:cons_saving}
\end{figure}
%%%%%%%%%%%%%%%%%%%%%%%%%%%%%%%%%%%%%%%%%%%%%%%%%%%%%%%%%%%%%%%%%%%%%%%%%%%%

The solution is illustrated in Figure~\ref{fig:cons_saving}.
Combinations of high year-1 consumption and low year-2 consumption
will be achieved by running down financial assets (i.e. setting
$a_{2}$ low and eventually negative). On the other hand,
combinations of low year-1 consumption and high year-2 consumption
will be achieved by increasing financial assets (i.e. by saving a
lot, so to increase $a_{2}$).

Solving a simple example confirms our intuition. Let's consider
the case in which
%
\begin{equation*}
u(c_{1},c_{2})=log(c_{1}) + log(c_{2}).
\end{equation*}
%
Substituting the budget constraint into the maximization problem
yields
%
\begin{equation*}
\max_{c_{2}}\; log\left(x_{1}-\frac{c_{2}}{1+r_{2}}\right) + log (c_{2})%
\end{equation*}
%
The first-order condition for the problem is
%
\begin{equation*}
-\frac{1}{c_{1}(1+r_{2})}+\frac{1}{c_{2}},
\end{equation*}
%
which implies $c_{2}/c_{1}=(1+r_{2})$. This equation can be used
along with the budget constrained to solve explicitly for the
levels of consumption. We obtain that $c_{1}=x_{1}/2$ and
$c_{2}=x_{1}(1+r_{2})/2$. \normalsize

\newpage

\textit{Effects of an increase in wealth}

Here we ask what is the impact of an increase in wealth $x_{1}$ on
$c_{1}$ and $c_{2}$. The wealth can increase for a variety of
reasons. For example, current income $y_{1}$ might increase. This
will allow the household to spend more on consumption not only in
period 1, but also in period 2. In fact it can decide to save part
of the increase in $y_{1}$, thereby increasing disposable income
in year $2$. What if it is future income $y_{2}$ that increases?
This increases the spending capacity not only in year 2, but also
in year 1. In fact the household can borrow against that future
income. The important lesson is that it does not matter which of
the cash inflows increase. What is important is that the year-1
value of wealth increases. The increase in wealth will impact
consumption in all years. This is what we refer to as
\textit{consumption smoothing}.

\textit{Optional} \footnotesize We can use the usual graphical
analysis to figure out the answer. Figure~\ref{fig:higher_wealth}
says that consumption in both periods increases, following an
increase in wealth from $x_{1}$ to $\hat{x}_{1}$. Consider also
the simple example with logarithmic utility. The expressions for
$c_{1}$ and $c_{2}$ reveal that consumption in both periods rises
with the level of wealth.\normalsize

%%%%%%%%%%%%%%%%%%%%%%%%%%%%%%%%%%%%%%%%%%%%%%%%%%%%%%%%%%%%%%%%%%%%%%%%%%%%

\begin{figure}[h!]
\begin{center}
\begin{picture}
(250,200)(-30,-20)
\footnotesize%

% Horizontal axis
\put(-30,0){\vector(1,0){300}}%
\put(265,-14){$c_{1}$}%

% Vertical axis
\put(0,-20){\vector(0,1){200}}%
\put(-15,175){$c_{2}$}%


% Low Budget constraint
\put(0,100){\line(2,-1){200}}%
%\put(170,0){\line(0,1){36}}%

% High Budget constraint
\put(0,119){\line(2,-1){238}}%

% Tangent Indifference Curve
\qbezier[200](10,165)(60,57)(170,60)

% Low Indifference Curve
\qbezier[200](5,140)(55,42)(175,38)

\put(200,-14){$x_{1}$}%

\put(237,-14){$\hat{x}_{1}$}%


\put(80,60){\circle*{3}}%
\put(90,74){\circle*{3}}%

\put(230,55){\vector(-1,-1){30}}%
\put(232,55){New budget constraint}%

\put(-45,98){$x_{1}(1+r_{2})$}%

\end{picture}
\end{center}
\caption{An increase in wealth.} \label{fig:higher_wealth}
\end{figure}
%%%%%%%%%%%%%%%%%%%%%%%%%%%%%%%%%%%%%%%%%%%%%%%%%%%%%%%%%%%%%%%%%%%%%%%%%%%%

\textit{Effects of an increase in the interest rate}

Here the idea is that a rise in the interest rate $r_{2}$ lowers
the cost of year-2 consumption relative to that of year 1. This is
the case because a person can obtain more units of consumption
next period (period 2) for each unit of current consumption
forgone today (period 1). This change in relative costs motivates
people to substitute future goods ($c_{2}$) for current ones
($c_{1}$). You may recognize this as yet another form of
\textit{substitution effect}: when apples get cheaper with respect
to bananas, we tend to buy more apples. When year-2 consumption
becomes more convenient, we tend to buy more of it (i.e. save more
in year 1).

\textit{Optional} \footnotesize Consider once again the simple
example with logarithmic utility. The expressions for $c_{1}$ and
$c_{2}$ reveal that for given wealth $x_{1}$, the household
chooses to increase consumption in period 2. This is the
substitution effect at work. This analysis is not exhaustive
though. As it was the case when we considered the effects of a
wage increase, an increase in the interest rate induces a change
in the wealth $x_{1}$ as well. The impact on wealth of a higher
interest rate will be negative if $y_{2}-a_{3}>0$ and will be
positive otherwise. In general, higher interest rate increase the
life-time wealth of creditors (positive $a_{3}$) and decrease the
life-time wealth of debtors (negative $a_{3}$). \normalsize

\setlength{\baselineskip}{15pt}

%%%%%%%%%%%%%%%%%%%%%%%%%%%%%%%%%%%%%%%%%%%%%%%%%%%%%%%%%%%%%%%%%%%%%%%%%%%%

\begin{figure}[h!]
\begin{center}
\begin{picture}
(250,200)(-30,-20)
\footnotesize%

% Horizontal axis
\put(-30,0){\vector(1,0){300}}%
\put(265,-14){$c_{1}$}%

% Vertical axis
\put(0,-20){\vector(0,1){200}}%
\put(-15,175){$c_{2}$}%


% Low Budget constraint
\put(0,100){\line(2,-1){200}}%

% High Budget constraint
\qbezier[200](0,139)(65,74)(139,0)

% Tangent Indifference Curve
\qbezier[200](8,148)(60,46)(170,55)

% Low Indifference Curve
\qbezier[200](5,140)(55,42)(175,38)

\put(200,-14){$x_{1}$}%

\put(80,60){\circle*{3}}%
\put(43,98){\circle*{3}}%

\put(210,45){\vector(-2,-1){70}}%
\put(212,45){New budget constraint}%

\put(80,135){\vector(-1,-1){30}}%
\put(82,135){New optimal pair}%

\put(-45,98){$x_{1}(1+r_{2})$}%

\end{picture}
\end{center}
\caption{An increase in the interest rate.} \label{fig:higher_rate}%
\end{figure}
%%%%%%%%%%%%%%%%%%%%%%%%%%%%%%%%%%%%%%%%%%%%%%%%%%%%%%%%%%%%%%%%%%%%%%%%%%%%

\textit{Optional} \footnotesize In the case depicted in
Figure~\ref{fig:higher_rate}, it is assumed that when the interest
rate $r_{2}$ increases, the wealth $x_{1}$ decreases in such a way
that the pair initially chosen lies on the new budget constraint
and therefore is still available for choice. In such a case, it is
clear that $c_{1}$ decreases and $c_{2}$ increases. The individual
reacts by saving more. Again, this is the substitution effect at
work. What happens in general? A higher interest rate will always
tilt the consumer choice towards consuming more in the future.
That is: an higher interest rate always implies more savings.
However, if an higher rate implies higher wealth, both $c_{1}$ and
$c_{2}$ might increase. \normalsize

\textit{Reintroducing the choice of work effort}

It is now time to re-introduce the choice of how much to work. We
want to understand how households choose work effort over time.
What we need to do is to specify household's income as the sum of
its components: $y_{1}=M_{1}+w_{1}l_{1}$ and $y_{2}=M_{2}+wl_{2}$.
That is, income in year $t$ is given by the sum of non-wage income
$M_{t}$ and wage income $w_{t}l_{t}$. Now the household has to
choose the work efforts $l_{1}$ and $l_{2}$.

We know what happens if either $M_{1}$ or $M_{2}$ or both
increase. Since the wealth effect on the effort choice is
negative, the households will decide to work less in both periods.
That is, both $l_{1}$ and $l_{2}$ will decrease.

What is the impact of an increase in the interest rate $r_{2}$ on
the labor supply decisions? When $r_{2}$ rises, year-2 leisure
becomes cheaper relative to year-1 leisure. Therefore, an increase
in the interest rate motivates people to substitute toward year-2
leisure and away from year-1's. In other words, year-1 work effort
rises relative to year-2's. Also, recall that the substitution
effect induces a decrease in $c_{1}$ that has the effect of
increasing savings. The increase in $l_{1}$, by raising current
income, contributes to further increase savings.

\textit{Optimal decisions over the household's planning horizon}

So far in this section we have used a simple framework with two
years only: 1 and 2. Households, however, make plans for much
longer horizons. How long? As long as people do not care about the
well-being of their descendants, it seems sensible to talk about
life-time horizon. If however, as it is seems to be the case,
people care about their offsprings, the offsprings of their
offsprings, and so on, then we are led to conclude that the
planning horizon actually extends to infinity. The good news is
that for our analysis the length of the planning horizon does not
matter very much. Sure, if we were to perform a quantitative
analysis (i.e. put numbers instead of letters), there would be
differences. But we are only interested in a qualitative analysis.
That is: we are interested in understanding whether current work
effort increases or decreases because of a change in the interest
rate, but we are not interested in knowing by how much it changes.

The qualitative analysis of households' choices is the same, whether the planning horizon is
$T=2$, as above, or $T=60$, or $T=\infty$. For a planning horizon $T$, the budget constraint will
read
%
\begin{align*}
c_{1}+\frac{c_{2}}{1+r_{2}}&+\frac{c_{3}}{(1+r_{2})(1+r_{3})}+...+\frac{c_{T}}{\prod_{t=2}^{T}(1+r_{t})}=\\
&y_{1}+\frac{y_{2}}{1+r_{2}}+\frac{y_{3}}{(1+r_{2})(1+r_{3})}+...+\frac{y_{T}}{\prod_{t=2}^{T}(1+r_{t})}+a_{1}(1+r_{1})-\frac{a_{T+1}}{\prod_{t=2}^{T}(1+r_{t})}.
\end{align*}
%
Given the year-1 value of life-time wealth, an individual can still substitute between $c_{1}$ and
$c_{2}$. Just as before, for each unit of $c_{1}$ forgone, a household can obtain $1+r_{2}$
additional units of $c_{2}$. But there is nothing special about years 1 and 2. People can
substitute in a similar manner between $c_{2}$ and $c_{3}$, $c_{3}$ and $c_{4}$, and so on.

The above implies that an increase in $r_{t+1}$ always lowers $c_{t}$ relative to $c_{t+1}$. Also,
an increase in $r_{t+1}$ will imply that $l_{t}$ rises relative to $l_{t+1}$.

\newpage

\textit{A permanent increase in income}

Recall that for every year $t$, $y_{t}=M_{t}+w_{t}l_{t}$. What
happens if the non-wage income $M_{t}$ increases in all years? (We
call this a permanent increase in income). The answer is a simple
generalization of the answer we have given in the case in which
$T=2$. Consumption in all years will increase, and work effort in
all years will decrease. This is the wealth effect at work.
Suppose that $M_{t}$ increases by one dollar in every period. By
how much will the value of consumption increase? It turns out that
it will approximately increase by one dollar. That is, the
propensity to consume will be about 1, or the propensity to save
will be about 0. Again this is the result of consumption
smoothing. People like smooth (flat) consumption patterns over
their lifetime. Therefore, a 1 dollar permanent increase in income
is likely to trigger a 1 dollar permanent increase in consumption.
There are plenty of examples of permanent increases in income.
Here's one: if you have a government job, any increase in salary
is guaranteed until retirement and will be reflected in your
Social Security check.

\textit{A temporary increase in income}

Now suppose that the increase in income is limited to the current
period. That is, suppose only $M_{1}$ increases. (Economists refer
to such an event as a temporary increase in income) Households
would like to spread the extra income over consumption in all
periods. In order to raise future consumption, they have to raise
current saving. Therefore, the propensity to consume will be very
small (close to zero if the planning horizon is long enough) and
the propensity to save will be very close to 1. There are also
very good examples of temporary increases in income: a one-time
tax rebate, an inheritance, a bonus...


\textit{An increase in the wage rate}

Once again, the effects of an increase in the wage rate will
depend on whether the change is temporary or permanent.

A permanent increase in the wage rate (i.e. an increase in $w_{t}$) in all years $t$ will imply a
higher work effort at all times and an higher income. In turn, this will imply an increase in
consumption. The propensity to consume will be about 1. Hence, there will be no effects on
savings.

A temporary increase in the wage rate is an increase in compensation limited to one or few years.
For example, consider an increase in $w_{1}$. The work effort will increase in the current period,
but consumption will increase very little. The reason is that households like to spread the
windfall over their lifetime. The propensity to consume will be close to zero. Savings instead
will increase: the propensity to save will be close to 1.

\newpage

\textit{Summary}

\vspace{-.4cm}

\begin{itemize}

\item The possibility of borrowing and lending eliminates the link between current cash flows and
current cash outflows (consumption expenditures). What matters for consumption decisions is the
present value of lifetime wealth.

\item Households like to smooth consumption over their lifetime.

\item A permanent increase in income implies a propensity to consume of about one (households
increase consumption by an amount roughly equal to the increase in income) and therefore a
propensity to save about equal to zero. The wealth effect implies that work effort drops.

\item A temporary increase in income implies a propensity to consume close to zero and a
propensity to save close to one. This is the case because people like to smooth consumption. The
effect on work effort is negative but negligible.

\item An increase in the interest rate increases savings, lowers current consumption, and
increases current work effort.

\end{itemize}

\subsubsection*{General Equilibrium}

Economists use the term \textit{general equilibrium} to define a
situation in which all markets are in equilibrium at all times. In
our simplified description of the economy, there are only three
markets: the \textit{commodity market}, the \textit{capital
market} and the \textit{labor market}.

\textit{The labor market}

As we know, in this market the demand for labor coming from the
firms meets the supply coming from households. The aggregate
demand for labor is simply the sum across all firms' demands and
it depends negatively on the wage rate $w_{t}$ and positively on
the level of TFP. The supply of labor, instead, comes from the
households. Recall from above that it depends on their wealth (the
richer I am, the less I work), on the wage rate (the higher the
wage, the more I work) and also on the interest rate (the higher
the interest rate, the more I work). For the labor market to be in
equilibrium, the wage rate $w_{t}$ must be such that labor demand
and labor supply are equal.

\textit{The capital market}

In this market, the demand for capital coming from the firms meets
the supply of capital coming from the households. The aggregate
demand of capital is simply the sum across all firms' capital
stock. We know that it depends positively on TFP and negatively on
the interest rate. The supply of capital is the sum across
households' financial assets. It depends positively on the rate of
interest. For the capital market to be in equilibrium, the return
on capital $r_{t+1}$ must be such that firms's desired capital
stock ($K_{t+1}$) and the households desired total ($a_{t+1}$)
wealth are equal.

\textit{The commodity market}

From now on, for simplicity (things are already too complicated,
you will agree), we assume there is only one commodity produced.
Therefore the supply of this commodity in year $t$ coincides with
$\GDP$ in the same year. We denote it as $Y_{t}$, as usual.
$Y_{t}$ is the supply in the commodity market. On what does it
depend? Well, as we know, it depends on the total capital stock in
place, $K_{t}$, the total labor employed $L_{t}$, and the level of
$TFP$, i.e. $A_{t}$. The level of capital $K_{t}$ is
predetermined, while the labor employed $L_{t}$ is determined by
the equilibrium in the labor market. The demand side of the
commodity market is the sum of the demand for consumption and the
demand for investment. The demand for consumption comes from the
households. We have seen that it depends on a bunch of items,
among which their income (which in part is determined in the labor
market) and the level of the interest rate. The demand for
investment comes from the firms themselves. Firms will invest if
the desired level of capital, $K_{t+1}$, is higher than the
capital already in place, $K_{t}$, net of depreciation. In turn,
as we have seen, the desired level of capital will depend on the
level of TFP and on the interest rate.

\vfill \centerline{\it \copyright \ \number\year \ NYU Stern
School of Business}

\end{document}
