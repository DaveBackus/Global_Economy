\documentclass[letterpaper,12pt]{article}

\usepackage{ge05}
\usepackage{comment}
\usepackage{booktabs}
\usepackage[dvipdfm]{hyperref}
\urlstyle{rm}   % change fonts for url's (from Chad Jones)
\hypersetup{
    colorlinks=true,        % kills boxes
    allcolors=blue,
    pdfsubject={NYU Stern course GB 2303, Global Economy},
    pdfauthor={Dave Backus @ NYU},
    pdfstartview={FitH},
    pdfpagemode={UseNone},
%    pdfnewwindow=true,      % links in new window
%    linkcolor=blue,         % color of internal links
%    citecolor=blue,         % color of links to bibliography
%    filecolor=blue,         % color of file links
%    urlcolor=blue           % color of external links
% see:  http://www.tug.org/applications/hyperref/manual.html
}

\newcommand{\GDP}{\mbox{\em GDP\/}}
\newcommand{\NDP}{\mbox{\em NDP\/}}
\newcommand{\GNP}{\mbox{\em GNP\/}}
\newcommand{\NX}{\mbox{\em NX\/}}
\newcommand{\NY}{\mbox{\em NY\/}}
\newcommand{\CA}{\mbox{\em CA\/}}
\newcommand{\NFA}{\mbox{\em NFA\/}}
\newcommand{\Def}{\mbox{\em Def\/}}
\newcommand{\CPI}{\mbox{\em CPI\/}}
\newcommand{\IP}{\mbox{\em IP\/}}
\newcommand{\PD}{\mbox{\em PD\/}}
\newcommand{\hsp}{\hspace{0.25in}}

\def\ClassName{The Global Economy}
\def\Category{Class Notes}
\def\HeadName{Government Debt and Deficits}

\begin{document}
\thispagestyle{empty}%
\Head

\centerline{\large \bf \HeadName}%
\centerline{Revised: \today}

\bigskip
The fundamental principle of government finance is that
governments must finance spending with taxes.
Issues of debt postpone this obligation but do not eliminate it.
If a government doesn't collect enough tax revenue now,
it must collect it later --- or face default.
Since investors like to be repaid, they pay close attention
to government debt and deficits.
If they're too high, investors may demand higher yields
or even stop buying government securities altogether.
The consequences are never pretty.


\subsubsection*{Government revenues, expenses, and debt}

We start with a quick overview of
government spending and revenue decisions ---
what we call {\it fiscal policy\/}.

Countries differ in the size of government relative to the economy,
in the sources of tax revenues,
and in their expenditures.
Governments everywhere
purchase goods and services (schools, police, courts, roads, military),
transfer money to individuals (social insurance, health care),
and collect revenue (largely through taxes).
The distinction between purchases and transfers is important:
only purchases show up in the expenditure identity,
but transfers are a large part of total expenses in many economies,
particularly developed economies.
Governments also pay interest on outstanding government debt,
an expense we track separately.

We'll look at data for each of these in class.
As a rule, government spending and revenue is a larger fraction of GDP
in rich countries than poor ones.
Rich countries also spend more on transfers.
There is, however, a lot of variation at all levels of development.

We can put these elements together in a relation we'll call
the {\it government budget constraint\/}.
On the expense side, we label government purchases of goods and services $G$,
transfers $V$, and interest payments $iB$
(the product of the government debt $B$ and whatever interest rate $i$
the government pays on it).
On the revenue side, we label tax revenue $T$.
(Note: $T$ is tax revenue, not the tax rate.)
By convention, all of these things are nominal:
they're measured in local currency units.
The government budget constraint is then
\begin{eqnarray}
    G_t + V_t + i_t B_{t-1} - T_t  &=& B_{t} - B_{t-1}  .
    \label{eq:gbc}
\end{eqnarray}
Here $B_{t-1}$ is the amount of debt outstanding at the end of
period $t-1$.
The left-hand side of (\ref{eq:gbc})
is the government deficit,
the right the change in the quantity of debt.
The equation says, in essence, that any surplus or deficit is matched
by a change in the quantity of debt.
A government deficit, for example, is financed by issuing more debt.


The elements of equation (\ref{eq:gbc})
are often used to generate summary measures
of fiscal policy.
The most common are ratios  of
the government deficit and government debt to GDP.
We'll look at both, as well as the connection between them.


%\end{document}
\subsubsection*{Governments must finance spending with tax revenue --- eventually}


Governments need to finance their spending with taxes.
It's not quite true --- governments have other sources of revenue ---
but it's close enough to be worth remembering.
Issuing debt allows a government to postpone taxes,
just as a credit card allows an individual to postpone paying
for purchases,
but does not eliminate the obligation.
Delay, in fact, comes with a cost:  we need to pay the original
obligation, plus interest.
In the rest of this section we make the same point more formally.

We're going to take our budget constraint, equation (\ref{eq:gbc}),
and use it to relate debt to past and future deficits.
To make things a little simpler,
define the {\it primary deficit\/} $D$ as the deficit
net of interest payments (sort of an ``EBITDA'' number):
\[
    D_t \;=\;  G_t + V_t - T_t .
\]
The primary deficit is sometimes reported with the opposite
sign and called the primary budget balance or surplus.
With this simplification, (\ref{eq:gbc}) can be expressed as
\begin{eqnarray}
    B_{t}   &=& (1+i) B_{t-1} + D_t
   \label{eq:gbc-bdynamics1}
\end{eqnarray}
or
\begin{eqnarray}
    B_{t-1}  &=&  B_{t}/(1+i) - D_t/(1+i) .
    \label{eq:gbc-bdynamics2}
\end{eqnarray}
They're the same equation, but the first one looks backward from $t$ to $t-1$,
and the second forward from $t-1$ to $t$.
We'll put both to work.

Equation (\ref{eq:gbc-bdynamics1}) tells us where the debt came from.
If we substitute over and over again, back to some period $t-n$,
we have
\begin{eqnarray*}
    B_{t}  &=& D_t + (1+i) B_{t-1} \\
            &=& D_t + (1+i) [(1+i) B_{t-2} + D_{t-1}] \\
            &=& D_t + (1+i) D_{t-1} + (1+i)^2 D_{t-2} + \cdots + (1+i)^n B_{t-n}.
\end{eqnarray*}
In words:  the current debt is the debt we started with $n$ periods ago
plus the current value of past deficits plus accumulated interest.
It's like your credit card bill:
your current balance consists of past shortfalls plus accumulated interest.


Equation (\ref{eq:gbc-bdynamics2}) tells us what we need to do to service
the debt.
If we substitute repeatedly, we find
\begin{eqnarray*}
    B_{t-1}   &=&  B_{t+1}/(1+i)^2 - [D_t/(1+i)+ D_{t+1}/(1+i)^2] \\
            &=&  B_{t+2}/(1+i)^3 -
                [D_t/(1+i)+ D_{t+1}/(1+i)^2 + D_{t+2}/(1+i)^3] \\
            &=&  B_{t+n-1}/(1+i)^n -
                [D_t/(1+i)+ D_{t+1}/(1+i)^2 + \cdots + D_{t+n-1}/(1+i)^n] .
%    \label{eq:gbc-bdynamics-rec}
\end{eqnarray*}
If we assume that debt can't grow faster than the interest rate forever,
then as we continue to substitute, the first term goes to zero.
[The technical condition is $B_{t+n}/(1+i)^n$ goes to zero as $n$
approaches infinity.
It amounts to not allowing the government to run a Ponzi scheme,
paying off old debt by issuing new debt.]
The relation then becomes
\begin{eqnarray*}
    B_{t-1}   &=&  -[D_t/(1+i)+ D_{t+1}/(1+i)^2 +
                   D_{t+2}/(1+i)^3] +   \cdots ]  \\
            &=& - \mbox{Present Discounted Value of Primary Deficits} \\
            &=& \mbox{Present Discounted Value of Primary Surpluses} .
\end{eqnarray*}
In words:   the current government debt must be matched
by the present discounted value of future primary surpluses.
As we said at the start, all spending must be financed by tax revenue --- eventually.
It's not enough to shrink the deficit:  eventually we have
to run surpluses, measured net of interest payments.
%There's a limit, in other words, to deficit spending.

%  ?? steady state D = - i B.


\begin{comment}
Although (primary) government deficits must eventually be reversed,
they can affect the economy while they last.
One route is distribution.
If taxes are delayed long enough, the tax burden
will be shifted from current to future generations.
Thus the US contribution to World War II was financed largely with debt,
shifting some of the burden from those alive at the time to those born later.


Is deficit financing a good or bad thing?
One approach to this question is based on tax smoothing.
As we mentioned in ``Notes on taxes,''
the economic disincentives built into taxes are minimized
by having relatively constant tax rates even if tax revenues
vary over time.
On average this is likely to lead to governments running (modest)
surpluses in booms and deficits in recessions,
as tax revenues go up and down with the economy.
\end{comment}

%\end{document}
\subsubsection*{Debt dynamics}

Investors watch government debt and deficits
for signs that a government may not honor its debts.
Even a hint of this can change the rate at which the government
borrows or even its ability to access capital markets.
In practice, it's common to look at them as ratios to GDP.
In such ratios, we measure both in currency units,
so we have (for example)
the ratio of the nominal debt to nominal GDP.


How does the debt-to-GDP ratio change through time?
There's a useful decomposition of changes in the debt-to-GDP ratio
into components due to the interest rate, GDP growth, and the
primary deficit.
We'll work our way up to it.
Recall that debt evolves according to
\begin{eqnarray*}
    B_{t} &=& D_t + (1+i) B_{t-1}  ,
\end{eqnarray*}
where $D$ is the primary deficit.
You should recognize this as equation (\ref{eq:gbc-bdynamics1}).
Note that everything is nominal here, including the interest rate $i$.
If the growth rate of real GDP is $g$, then the growth rate
of nominal GDP is $g + \pi$, where $\pi$ is the inflation rate.
Therefore
\[
    Y_{t} \;=\; (1+g+\pi) Y_{t-1}
\]
and $B/Y$ follows
\begin{eqnarray}
    \frac{B_{t}}{Y_{t}}
            &=&
                \left( \frac{1+i}{1+g+\pi} \right)  \frac{B_{t-1}}{Y_{t-1}}
             +    \frac{D_{t}}{Y_{t}}  \nonumber \\
            &\approx&
                \left[ 1 + i - (g+\pi) \right]  \frac{B_{t-1}}{Y_{t-1}}
             +    \frac{D_{t}}{Y_{t}}   \nonumber \\
            &\approx&
                \frac{B_{t-1}}{Y_{t-1}} + (i-\pi) \frac{B_{t-1}}{Y_{t-1}}
                - g \frac{B_{t-1}}{Y_{t-1}}
             +    \frac{D_{t}}{Y_{t}}  .
    \label{eq:debtdynamics}
\end{eqnarray}
There's a lot here, so let's go through it carefully.
The second equation is based on the approximation
\begin{eqnarray*}
    \frac{1+i}{1+g+\pi} &\approx& 1 + i - g - \pi ,
\end{eqnarray*}
good for small values of $i$, $g$, and $\pi$.
All you need to know is that the right side is simpler than the left.
The third rearranges terms, combining $i-\pi$ into a real
interest rate term.
That gives us three sources
of change in the ratio of debt to GDP.
The first is the real interest on the debt, which accumulates
as long as the debt is positive.
The second is the growth of the economy,
which reduces the ratio by increasing the denominator.
The third is the primary surplus.

\begin{comment}
If interest on the debt becomes large enough, something must change.
But what?
One candidate is the deficit itself:  the government
does this analysis, realizes that interest will eat up
all its revenue, and decides to reduce the deficit.
This could come from a reduction in spending or an increase in taxes;
both work the same way in this analysis, although
in other respects the two may differ.
Another candidate --- one we'd generally prefer not to consider ---
is default.
Bond investors can do this analysis, too,
and if they find that the debt is more than the government
is likely to pay back, they will stop lending.
This tends to reduce the deficit, too,
by restricting the government's ability to finance it.
\end{comment}

This is a mechanical analysis, but a useful one.
By looking at the components of equation (\ref{eq:debtdynamics})
we can get a sense of the origins of past changes in the debt
and the potential sources of future changes.

{\it Example.\/}
Consider the US economy in 2012.
At the end of 2011, the debt-to-GDP ratio was
$B/Y = 73.8\%$.
(These and other numbers come from the tables at the end of
the OECD's {\it Economic Outlook\/}.)
% ?? Dummied up so the numbers sum
% See also:  http://www.oecd.org/document/25/0,3746,en_2649_34109_33702745_1_1_1_1,00.html
In 2012, they expect growth to be $g = 2.0\%$,
inflation to be $\pi = 1.9\%$,
interest payments on the debt to be $iB/Y = 2.2\%$,
and the primary deficit to be $D/Y = 5.6\%$.
If these numbers turn out to be right,
what is the implied nominal interest rate on debt?
What will the debt-to-GDP ratio be at the end
of the year?
Which components account for most of the change in the ratio?

Answer.  The interest rate follows from
\begin{eqnarray*}
    i &=& i (B/Y) /(B/Y)
            \;\;=\;\; 2.2/73.8 \;\;=\;\; 3.0\% ,
\end{eqnarray*}
which corresponds to a real interest rate of $ r = i-\pi = 3.0 - 1.9 = 1.1$.
The three terms in (\ref{eq:debtdynamics}) are
\begin{eqnarray*}
    (i-\pi) B/Y &=& 0.8\% \\
    -g B/Y      &=& -1.5\% \\
    D/Y         &=& 5.6\% .
\end{eqnarray*}
The change in $B/Y$ is the sum, 4.8\%,
so the end-of-2012 debt-to-GDP ratio should be 78.6\%.
Right now the largest term is the primary deficit.
If interest rates rise (they're very low right now),
that term will grow.

%\begin{comment}
{\it Example.\/}  Between 2003 and 2007,
Peru's debt fell from 47\% of GDP to 25\%.
Using the numbers in the tables below, what happened?
What was the primary source of this decline?

\begin{center}
%\begin{table}
\begin{tabular}{lcccc}
\toprule
        & Debt/GDP   & Interest &  Growth & Primary Deficit \\
        &  $B_{t-1}/Y_{t-1}$ &  $(i-\pi)B_{t-1}/Y_{t-1}$  &  $-g B_{t-1}/Y_{t-1}$
                & $D_t/Y_t$ \\
\midrule
2003 \hspace*{0.4in}
            & 47.1 &  \\
2004        & 44.3 & 0.2 & $-2.4$ & $-0.6$  \\
2005        & 37.7 & 1.1 & $-3.0$ & $-4.6$  \\
2006        & 33.1 & 1.0 & $-2.9$ & $-2.7$  \\
2007        & 20.9 & 1.1 & $-2.9$ & $-0.4$  \\
2008        & 25.0 & $-0.3\phantom{-}$ & $-3.0$ & $-2.5$ \\
\midrule
Sum         & 22.1 & 3.1 & $-14.3$ & $-10.9$ \\
\bottomrule
\end{tabular}
\end{center}
% Data:  WEO, Sep 2011.  Numbers don't add right, so I made the primary deficit
% the residual -- ie, it's dummied up
%\end{comment}

Answer.  You can see from the final row that
growth and a primary deficit both played large roles.
In fact, as we'll see shortly,
growth probably gets some credit even for the latter term,
since it tends to increase government revenue.


\subsubsection*{What's missing?}

Our summary of debt dynamics,
caputured in equation (\ref{eq:debtdynamics}) is true,
but there are some
issues that are buried beneath the mathematics.

One issue is the link between the interest rate
and the fiscal situation (the debt and deficit).
If investors start to worry about a government's
willingness to honor its debt,
they may demand higher interest rates,
which in turn raises future debt --- and so on.
%It's not a cycle you want to get caught in.
Such credit spreads can rise sharply
if the government has not shown sufficient
fiscal discipline.
(Here ``sufficient'' means whatever is needed to reassure
investors.)
It can also rise because global financial markets
place a higher premium on risk,
as they did during the 2008-09 financial crisis.
Over this period, spreads on emerging market debt of all kinds
widened, even in countries with fundamentally sound
fiscal positions.
%For that reason, it's often useful to consider scenarios
%in which the interest rate on government debt rises sharply.

Another issue is the link between growth and deficits.
If growth rises, as in Peru (example, above),
that reduces the debt-to-GDP ratio through
the impact on $g$ in equation (\ref{eq:debtdynamics}).
But it also generates higher tax revenues and hence
a lower primary deficit, even if tax {\it rates\/}
do not change.
This, in turn, reduces future debt further.
That's one reason Peru's fiscal situation looks so good:
the economy is booming.
Growth, then, is the cure for many problems.

A third issue is hidden government liabilities.
The idea behind our analysis
is that current fiscal policy (the primary deficit)
determines how the debt evolves.
In fact, current decisions often involve commitments
for future expenditures that don't show up in the
current government budget but are nevertheless important.
Here are some common examples:
%
\begin{itemize}
\item Social security and pensions.
Many countries have implicit commitments to pay
money to retired people in the future
that are not accounted for properly,
either through social security systems or pensions.
In the US, for example, we simply look at the current cash flow of
social security receipts and payments.
In principle, there should be an entry for unfunded pension
liabilities,
as there is for firms.
In many countries, aging populations have made these looming payments
a serious concern.
Health-care payments are similar.

\item Financial bailouts.
We tend to treat these as one-offs, but in fact they happen all the time and they're invariably expensive.
A country that bails out its banks may find its debt rise sharply.

\item Regional governments.
Relations between central governments and local authorities
differ widely around the world.
In the US, the precedent was set in the 1840s for state and local
governments to finance their own activities without help from
the central government.
In other countries, debt problems of regional governments are often
passed to the central government.
These implicit liabilities of the central government are a concern
in Europe right now.
\end{itemize}

A serious analysis of fiscal policy should therefore
start with equation (\ref{eq:debtdynamics})
but go on to consider changes in all of its
components.


\subsubsection*{How much debt is too much?}

How much debt is too much?
At what point should investors be concerned?
There is, unfortunately, no clear answer.
Argentina defaulted with debt of about 40\% of GDP,
but the UK had debt well over 100\% of GDP after World War II
and didn't generate undue concern.
Part of this reflects the quality of governance and
institutions in general,
but in many cases you're stuck trying to
guess how the local politics will play out.
There's no shortcut for that.

With that warning, here are some rules of thumb:
%
\begin{itemize}
\item Worry if the government debt is above 50\% of GDP.
This is very rough, but it's a start.

\item Worry if the deficit is above 5\% of GDP ---
and is expected to stay that way.
The issue is not so much any particular deficit,
but the long-term posture.

\item Use higher numbers for countries with strong institutions,
lower numbers for countries with weak institutions.

\item Worry if the debt is primarily short-term or denominated in
foreign currency.
Even if the debt is stable, countries
may find themselves in difficulty if they have to refinance
a large fraction of their debt over a short period of time.
(Companies are no different:  think Lehman Brothers.)
In late 1994, for example, Mexico had much of its debt in short-term
securities.
When investors refused to buy new issues, it triggered a crisis.
This despite relatively modest debt and deficits.

Foreign debt has similar risks.
Many developing countries issue debt denominated in hard currency
(dollars, say, or euros), but it's a mixed blessing.
Investors use it to avoid the risk of a currency collapse,
but if the currency collapses,
a country's debt can rise sharply,
perhaps increasing the odds of default.

\end{itemize}



\subsubsection*{Executive summary}

\begin{enumerate}

\item Countries differ enormously in the magnitude and composition
of government spending, taxes, and debt.

\item Government spending must be paid for, either now through
taxes or in the future by running primary surpluses.

\item The following factors govern changes in the debt-to-GDP ratio:
(a)~interest on the debt,
(b)~GDP growth,
and (c)~the primary deficit.

\end{enumerate}

%\begin{comment}
\subsubsection*{Review questions}

\begin{enumerate}

\item Deficits down under.
Consider this data for Australia:
%
\begin{center}
\begin{tabular}{lrr}
\toprule
        & 2010 & 2011 \\
\midrule
Real GDP growth (annual percent) & 2.6 & 2.9 \\
Inflation  (annual percent)      & 5.2 & 4.8 \\
Interest rate  (annual percent)  & 5.3 & 4.9 \\
Government deficit (primary, percent of GDP)    & 2.9 & 0.7 \\
Government deficit (total, percent of GDP)      & 4.1 & 2.2 \\
Government debt (end of period, percent of GDP) & 25.3 \\
\bottomrule
\end{tabular}
\end{center}
\begin{enumerate}
\item Why are the primary and total government deficits different?
\item What is the implied interest rate on government debt in 2011?
\item What is the government debt ratio at the end of 2011?
\end{enumerate}

Answer.
\begin{enumerate}
\item The difference is interest payments on government debt.
Apparently in 2011 they amounted to 1.5\% of GDP.
\item Evidently 5.9\% = 1.5/25.3.
\item We use the debt dynamics equation:
\begin{eqnarray*}
    \frac{B_{t}}{Y_t}
            &=&
                \frac{B_{t-1}}{Y_{t-1}} + (i-\pi) \frac{B_{t-1}}{Y_{t-1}}
                - g \frac{B_{t-1}}{Y_{t-1}}
             +    \frac{D_{t}}{Y_{t}}  \\
             &=& 25.3 + 0.3 [=(0.059-0.048)*25.3] - 0.7 [=0.029*25.3] + 0.7 \\
             &=&  25.6 .
\end{eqnarray*}

\end{enumerate}


\item How the US financed World War II.
Short answer:  they issued debt.
But how did they pay off the debt?
Between 1945 and 1974, the ratio of debt to GDP fell from 66\% to 11\%.
What led to the change?
George Hall and Thomas Sargent (source below) computed the following:

\begin{center}
%\begin{table}
\begin{tabular}{lccc}
\toprule
       & Interest &  Growth & Primary Deficit \\
       &  $(i-\pi)B_{t-1}/Y_{t-1}$  &  $-g B_{t-1}/Y_{t-1}$
                & $D_t/Y_t$ \\
\midrule
1945-1974 \hspace*{0.4in} & --12.5  & --21.6  &  --20.8 \\
\bottomrule
\end{tabular}
\end{center}
All numbers are percentages.

Answer.  If you look at the numbers, they tell you that
growth (the same debt looked smaller when GDP grew)
and primary surpluses account for most of this.
You also see a negative contribution from real interest
payments.
What does this tell us?
With hindsight,
we would say that investors lost money in real terms
because inflation was higher than they expected when they
purchased government debt.
If the government had paid (say) a one percent real return,
this contribution would have been positive
and the 1974 debt level would have been higher.
In that sense, the US used inflation to reduce the debt burden.
\end{enumerate}
%\end{comment}


\subsubsection*{If you're looking for more}

It's too technical for this course,
but some of the material on debt dynamics was adapted from
Craig Burnside, ed., {\it Fiscal Sustainability in Theory and Practice\/},
World Bank, 2005.

More user-friendly (and very good)
is George Hall's summary of his work with Thomas Sargent,
``\href{http://www.brandeis.edu/global/rosenberg/briefs/hall_brief.html}
{How will we pay down the debt?}''
They describe sources of changes in the US debt position
from World War II to 2008.

If you're looking for international data on government debt over 200 years or more,
you might want to look at the IMF's
\href{http://www.imf.org/external/pubs/cat/longres.cfm?sk=24332.0}{historical database}.


\vfill \centerline{\it \copyright \ \number\year \
NYU Stern School of Business}

\end{document}

